\documentclass[12pt]{article}

\pagestyle{headings}
\markboth{DNSSEC-Tools Operations Manual -- Installation Guide}{DNSSEC-Tools Operations Manual -- Installation Guide}

\newenvironment{packed}{\begin{list}{$\bullet$}{\parsep0in\itemsep0in}}{\end{list}}


%
% Macros for specific types of entities.
%
\newcommand{\cmd}[1]{{\em #1}}
\newcommand{\const}[1]{{\bf #1}}
\newcommand{\func}[1]{{\bf #1}}
\newcommand{\lib}[1]{{\em #1}}
\newcommand{\perlmod}[1]{{\bf #1}}
\newcommand{\path}[1]{{\bf #1}}
\newcommand{\sys}[1]{{\bf #1}}
\newcommand{\url}[1]{{\bf #1}}
\newcommand{\var}[1]{{\em #1}}
\newcommand{\xqt}[1]{{\bf #1}}

\textheight=8.50in
\voffset=-.55in
\voffset=-.75in

\setlength{\textwidth}{6.5in}
\setlength{\oddsidemargin}{0in}

\setcounter{secnumdepth}{4}
\setcounter{tocdepth}{2}

\parindent0em
\parskip0.65em

\hyphenation{dns-sec DNS-SEC}
\hyphenation{dns-sec--tools}
\hyphenation{key-rec}
\hyphenation{log-watch}


%%%%%%%%%%%%%%%%%%%%%%%%%%%%%%%%%%%%%%%%%%%%%%%%%%%%%%%%%%%%%%%%%%%%%%%%%%%%%%

\begin{document}
\markboth{DNSSEC-Tools Operations Manual -- Installation Guide}{DNSSEC-Tools Operations Manual -- Installation Guide}

\begin{titlepage}

\vspace{.5in}

\begin{center}
\LARGE{\bf
SOFTWARE USER MANUAL (SUM):
TRAINING, PROCEDURAL, AND
DEVELOPMENT DOCUMENTATION
}
\vspace{1in}

\Large{
DNSSEC-Tools Operations Manual

Installation Guide
\vspace{0.5in}

{\bf Contract: FA8750-04-C-0229
\vspace{0.125in}

CDRL A006}\\
}
31 August 2005
\end{center}

\vspace{.5in}
% \vspace{1.5in}
% \vspace{2.5in}

SUBMITTED BY

Sparta, Inc\\
7075 Samuel Morse Dr.\\
Columbia, MD 21046-3401
\vspace{0.25in}

\begin{table}[hb]
\begin{tabular}{lll}
	& Principal Investigator	& Contract/Financial Contact\\
Name	& George R. Mundy		& Kim Morrill\\
Phone	& (410) 872-1515		& (410) 872-1515\\
Fax	& (410) 872-8079		& (410) 872-8079\\
Email	& Russ.Mundy@sparta.com		& Kim.Morrill@sparta.com\\
\end{tabular}
\end{table}

\end{titlepage}
\markboth{DNSSEC-Tools Operations Manual -- Installation Guide}{DNSSEC-Tools Operations Manual -- Installation Guide}


%%%%%%%%%%%%%%%%%%%%%%%%%%%%%%%%%%%%%%%%%%%%%%%%%%%%%%%%%%%%%%%%%%%%%%%%%%%%%%

\clearpage

\begin{center}
{\Large
{\bf DNSSEC-Tools\\
Is your domain secure?}}
\end{center}
\markboth{DNSSEC-Tools Operations Manual -- Installation Guide}{DNSSEC-Tools Operations Manual -- Installation Guide}
\tableofcontents
\markboth{DNSSEC-Tools Operations Manual -- Installation Guide}{DNSSEC-Tools Operations Manual -- Installation Guide}

%%%%%%%%%%%%%%%%%%%%%%%%%%%%%%%%%%%%%%%%%%%%%%%%%%%%%%%%%%%%%%%%%%%%%%%%%%%%%%

\clearpage

\markboth{DNSSEC-Tools Operations Manual -- Installation Guide}{DNSSEC-Tools Operations Manual -- Installation Guide}
\section{About This Document}
\markboth{DNSSEC-Tools Operations Manual -- Installation Guide}{DNSSEC-Tools Operations Manual -- Installation Guide}
\label{about}

The goal of the DNSSEC-Tools project is to create a set of tools, patches,
applications, wrappers, extensions, and plugins that will help ease the
deployment of DNSSEC-related technologies. This document provides information
about the installation and configuration of the DNSSEC-enhanced tools by the
DNSSEC-Tools project.

For more information about this project and the tools that are being developed
and provided, please see the project web page at:

\url{http://www.dnssec-tools.org}

\subsection{Comments}

Please send any comments and corrections to developers@dnssec-tools.org.

\vspace{1in}

\subsection{Conventions}

The following typographical conventions are used in this document.

\begin{table}[hb]
\begin{tabular}{lll}
\cmd{command}		& & Command names\\
\const{constant}	& & Code constants\\
\func{call()}		& & System and function calls\\
\lib{library}		& & Library names\\
\perlmod{module}	& & Perl Modules\\
\path{path}		& & File and path names\\
\url{URL}		& & Web URLs\\
\var{variable}		& & Variables\\
\xqt{execution}		& & Simple command executions\\
\end{tabular}
\end{table}

Longer sets of command sequences are given in this format:
\begin{verbatim}
        # cd /tmp
        # ls
        # rm -fr *
\end{verbatim}
In most cases, output will not be displayed for given command sequences.

%%%%%%%%%%%%%%%%%%%%%%%%%%%%%%%%%%%%%%%%%%%%%%%%%%%%%%%%%%%%%%%%%%%%%%%%%%%%%%

\clearpage

\markboth{DNSSEC-Tools Operations Manual -- Installation Guide}{DNSSEC-Tools Operations Manual -- Installation Guide}
\section{DNSSEC-Tools Distribution Directory Structure}
\markboth{DNSSEC-Tools Operations Manual -- Installation Guide}{DNSSEC-Tools Operations Manual -- Installation Guide}
\label{directorystructure}

The various components of the DNSSEC-Tools project are spread across several
directories.  These include patches to existing software, new libraries, and
new programs.  These components are briefly described here.

%%%%%%%%%%%%%%%%%%%%%%%%%%%%%%%%%%%%%%%%%%%%%%%%

\subsection{Pre-existing Applications}

\begin{itemize}

\item{\path{apps/libspf2-1.{\it x.y}\_dnssec}}

The DNSSEC extension to \lib{libspf2} provides DNSSEC validation to DNS
queries in \lib{libspf2}.
Two versions are supported \lib{libspf2 1.0.4} and \lib{libspf2 1.2.5}.

\begin{itemize}
\item \path{libspf2-1.0.4\_dnssec\_patch.txt} contains a patch to
\lib{libspf2-1.0.4} that provides DNSSEC validation to DNS queries in
\lib{libspf2-1.0.4}.

\item \path{libspf2-1.0.4\_dnssec\_guide.txt} contains documentation
of the DNSSEC validation patch to \lib{libspf2-1.0.4} to aid developers in
using the new DNSSEC validation functionality in their applications.

\item \path{libspf2\_1.0.4\_dnssec\_howto.txt} contains instructions for
installing \lib{libspf2-1.0.4} with the DNSSEC validation patch.

\item \path{libspf2-1.2.5\_dnssec\_patch.txt} contains a patch to
\lib{libspf2-1.2.5} that provides DNSSEC validation to DNS queries in
\lib{libspf2-1.2.5}.

\item \path{libspf2-1.2.5\_dnssec\_guide.txt} contains documentation
of the DNSSEC validation patch to \lib{libspf2-1.2.5} to aid developers in
using the new DNSSEC validation functionality in their applications.

\item \path{libspf2-1.2.5\_dnssec\_howto.txt} contains instructions for
installing \lib{libspf2-1.2.5} with the DNSSEC validation patch.

\end{itemize}

\item{\path{apps/mozilla}}

This directory contains a software patch for the \cmd{mozilla} code, which
enables DNSSEC checking of URL DNS names.  It is in {\it beta} form.

\item{\path{apps/sendmail}}

This directory contains files that provide DNSSEC validation to
the \cmd{sendmail} Mail Transfer Agent.  There are patches for the
\cmd{sendmail} (version 8.13.3) and \cmd{spfmilter} (version 1.0.8) commands.

\begin{itemize}
\item \path{spfmilter-1.0.8\_dnssec\_patch.txt} contains a patch to
\cmd{spfmilter-1.0.8} that provides DNSSEC validation.  It uses the API from
\lib{libspf2} with the dnssec\_patch applied (see \path{../README}).

\item \path{spfmilter-1.0.8\_dnssec\_howto.txt} contains instructions
for adding DNSSEC validation to \cmd{spfmilter}.

\item \path{sendmail-8.13.3\_dnssec\_patch.txt} contains a patch to
\cmd{sendmail-8.13.3} that provides DNSSEC validation for DNS queries within
the \cmd{sendmail} Mail Transfer Agent (MTA).

\item \path{sendmail-8.13.3\_dnssec\_howto.txt} contains instructions
for adding DNSSEC validation to the \cmd{sendmail} MTA.

\item The \path{obsolete} directory contains files that provide a
\cmd{dnssec-milter} which validates the sending MTA.  These files
are obsolete and may be deleted in the future.

\end{itemize}

\item{\path{apps/thunderbird}}

The \path{thunderbird} directory contains an extension for displaying
the Received-SPF header and the result of DNSSEC validation in the
\cmd{thunderbird} mail client.

\end{itemize}

%%%%%%%%%%%%%%%%%%%%%%%%%%%%%%%%%%%%%%%%%%%%%%%%

\subsection{Libraries}

\begin{itemize}
\item{\path{lib/libsres}}

Provides basic functionality for name resolution, including the resolver
component of the ``validating resolver''.

\item{\path{lib/libvalida}}

Provides basic functionality for resource-record validation.

\item{\path{lib/val\_stub}}

\end{itemize}

%%%%%%%%%%%%%%%%%%%%%%%%%%%%%%%%%%%%%%%%%%%%%%%%

\subsection{New Tools and Modules}

\begin{itemize}
\item{\path{tools/donuts}}

Tools to analyze DNS zone files for errors.

\item{\path{tools/etc}}

Data required by DNSSEC-Tools programs.

\item{\path{tools/linux}}

Linux-specific software.

\item{\path{tools/logwatch}}

\cmd{logwatch} is a customizable log analyzer.  It has been customized to
analyze DNSSEC logs, and this directory contains \cmd{logwatch} configuration
files and scripts.

\item{\path{tools/mapper}}

Tool to create graphical maps of DNS zone data.

\item{\path{tools/modules}}

DNSSEC-Tools Perl modules.  These modules provide interfaces for such
things as reading configuration files and manipulating DNSSEC-Tools-specific
data.

\item{\path{tools/scripts}}

Perl scripts for signing DNSSEC zones and maintaining those signed zones.

\end{itemize}

%%%%%%%%%%%%%%%%%%%%%%%%%%%%%%%%%%%%%%%%%%%%%%%%%%%%%%%%%%%%%%%%%%%%%%%%%%%%%%

\clearpage

\markboth{DNSSEC-Tools Operations Manual -- Installation Guide}{DNSSEC-Tools Operations Manual -- Installation Guide}
\section{General Build Instructions}
\markboth{DNSSEC-Tools Operations Manual -- Installation Guide}{DNSSEC-Tools Operations Manual -- Installation Guide}
\label{buildinstructions}

Most of the tools, Perl modules, and libraries described on
\url{http://www.dnssec-tools.org} are easily installed by following the
instructions in the \path{INSTALL} file.  However, some of the results of
this package are patches to external programs.  It is hoped that the
DNSSEC-Tools modifications will be incorporated into those projects in the
future.  In the meantime, there are patches included in this source tree that
can be applied to those other projects.  Instructions for installing the
DNSSEC-Tools software follow.

%%%%%%%%%%%%%%%%%%%%%%%%%%%%%%%%%%%%%%%%%%%%%%%%

\subsection{Install Perl Modules}

Install the \perlmod{Net::DNS} and \perlmod{Net::DNS::SEC} Perl modules.
They can be installed using the Perl CPAN shell.  For example:

\begin{verbatim}
        # perl -MCPAN -e shell
        [...]

        cpan>  install Net::DNS
        [ CPAN will install it here ]

        cpan>  install Net::DNS::SEC
        [...]
        [... continue with other modules ...]
\end{verbatim}

%%%%%%%%%%%%%%%%%%%%%%%%%%%%%%%%%%%%%%%%%%%%%%%%

\subsection{Install Optional Software and Perl Modules}

A number of Perl modules and other software packages should be installed, but
aren't required.

\begin{table}[hb]
\begin{tabular}{lll}
Package	& Purpose & Comments\\
\hline

	& & Requires version 0.5 and the \\
\perlmod{Net::DNS::ZoneFile::Fast}		&
Zone file parsing		& patch in the DNSSEC-Tools \\
	& & the DNSSEC-Tools patch directory	\\

%	\path{tools/patches/Net-DNS-ZoneFile-Fast.patch}

\hline

\perlmod{Text::Wrap}				&
Simple formatter				&
						\\

\hline

\cmd{graphviz}					&
Displays zone maps				&
Available from					\\
	& & http://www.graphviz.org.		\\

\hline

\perlmod{Gtk2} {\bf}  \perlmod{Tk}	& Provides a GUI 	 & \\
\perlmod{Getopt::Long::GUI}		& interface for tools	 & \\
\perlmod{QWizard}			&		 & \\

\hline
\end{tabular}
\end{table}

\eject

%%%%%%%%%%%%%%%%%%%%%%%%%%%%%%%%%%%%%%%%%%%%%%%%

\subsection{Run the Configure Script}

The \cmd{configure} shell script attempts to guess correct values for various
system-dependent variables used during compilation.  It uses those values to
create a \path{Makefile} in each directory of the package. It may also create
one or more \path{.h} files containing system-dependent definitions.  Finally,
it creates the \path{config.status} shell script, which can be run in the future
to recreate the current configuration; the \path{config.cache} file, which
saves the results of its tests to speed up reconfiguring; the \path{config.log}
file, which contains compiler output (useful mainly for debugging
\cmd{configure}; and the \path{configure-summary} file, which contains the
summary displayed at the end of the \cmd{configure} run.

The \cmd{configure} invocation often gets lengthy and difficult to type, and
it is possible that you may have several different ways you want to configure
a system.  Consequently, you may want to create a shell script containing
your invocation.

If you need to do unusual things to compile the package, please try to figure
out how \cmd{configure} could check whether to do them, and mail diffs or
instructions to the address given in the \path{README} so they can be
considered for the next release.  If at some point \path{config.cache}
contains results you don't want to keep, you may remove or edit it.

The file \path{configure.in} is used to create \cmd{configure} by the
\cmd{autoconf} program.  You only need \path{configure.in} if you want to
change it or regenerate \cmd{configure} using a newer version of \cmd{autoconf}.

The simplest way to compile this package is to \cmd{cd} to the directory
containing the package's source code and type \xqt{./configure} to configure
the package for your system.  If you're  using \cmd{csh} on an old version
of System V, you might need to type \xqt{sh ./configure} instead to prevent
\cmd{csh} from trying to execute a different version of \cmd{configure}.

Running \cmd{configure} may take awhile.  While running, it prints some
messages telling which features it is checking for.  When it completes it
prints a short message (also available in \path{configure-summary})
indicating what functionality will be available when compiled.

%%%%%%%%%%%%%%%%%%%%%%%%%%%%%%%%%%%%%%%%%%%%%%%%

\subsection{Building, Installing, and Cleaning}

To build the package, type \cmd{make}.

To install the package, type \xqt{make install}.  This will install the
programs, data files, and documentation.
This must be done as {\it root}.

To uninstall the package, type \xqt{make clean}.  This will remove the program
binaries and object files from the source code directory.

To remove the files that \cmd{configure} created type \xqt{make distclean}.
This will allow you to compile the package for a different computer.

%%%%%%%%%%%%%%%%%%%%%%%%%%%%%%%%%%%%%%%%%%%%%%%%%%%%%%%%%%%%%%%%%%%%%%%%%%%%%%

\clearpage

\markboth{DNSSEC-Tools Operations Manual -- Installation Guide}{DNSSEC-Tools Operations Manual -- Installation Guide}
\section{DNSSEC Libraries}
\markboth{DNSSEC-Tools Operations Manual -- Installation Guide}{DNSSEC-Tools Operations Manual -- Installation Guide}
\label{libraries}

The DNSSEC-Tools distribution includes two libraries:  one that is capable of
sending queries to and receiving answers from a DNSSEC-aware name server,
and the other that provides DNSSEC resource-record validation functionality.

The \lib{libsres} library provides the resolver component of the ``validating
resolver''.  It is capable of recursively obtaining answers for an application
(validator) from a DNSSEC-aware name server.  Resolver policy will eventually
be used to control the query flags (CD, RD, etc.) and other parameters
sent to the name servers.

The \lib{libsres} library provides very basic functionality for name
resolution.  The data structures and interfaces exported to applications
have not been finalized and are expected to change. Many corner cases are
still not supported.

The \lib{libval} library provides basic functionality for resource-record
validation.  It relies on the \cmd{resolver} component to fetch answers
from a DNSSEC-aware name server.

Currently, there is no functionality to traverse the chain-of-trust while
performing record validation.  There is also no support for validation
policies. As such, the interfaces defined herein are in a state of flux
and are expected to change.

%%%%%%%%%%%%%%%%%%%%%%%%%%%%%%%%%%%%%%%%%%%%%%%%

\subsection{Installation}

Configure, build, and install the \lib{libsres} and \lib{val\_stub} libraries
using the following commands:

\begin{verbatim}
        # ./configure
        # make clean; make
        # make install
\end{verbatim}

%%%%%%%%%%%%%%%%%%%%%%%%%%%%%%%%%%%%%%%%%%%%%%%%

\subsection{Testing}

The validator library (\lib{val\_stub}) can be tested using its command-line
utilities.

To run the \cmd{resolver-driver}, execute \xqt{./driver}.

To run the \cmd{verifier}, execute \xqt{./verify $[[$CLASS$]$ TYPE$]$ DOMAIN\_NAME}.

If not provided, CLASS is assumed to be IN and TYPE is assumed to be A.

The \cmd{verify} program will not do chain-validation.  It will only check
that the DNSKEY can correctly verify the signature on a given resource record.

To run \cmd{gethost}, execute \xqt{./gethost $<$hostname$>$}.

%%%%%%%%%%%%%%%%%%%%%%%%%%%%%%%%%%%%%%%%%%%%%%%%%%%%%%%%%%%%%%%%%%%%%%%%%%%%%%

\clearpage

\markboth{DNSSEC-Tools Operations Manual -- Installation Guide}{DNSSEC-Tools Operations Manual -- Installation Guide}
\section{Modifications to Support DNSSEC in Existing Applications}
\markboth{DNSSEC-Tools Operations Manual -- Installation Guide}{DNSSEC-Tools Operations Manual -- Installation Guide}
\label{modifications}

A number of software packages have been modified to support DNSSEC.  These
packages include user programs and system programs:

\begin{table}[hb]
\begin{center}
\begin{tabular}{lll}
\lib{libspf2-1.2.5}	& \hspace{1in} & \cmd{sendmail}		\\
\cmd{logwatch}		& \hspace{1in} & \cmd{spfmilter-1.0.8}	\\
\cmd{mozilla}		& \hspace{1in} & \cmd{thunderbird}	\\
\end{tabular}
\end{center}
\end{table}

This section describes installation instructions, testing methods, and other
information for these packages.


%%%%%%%%%%%%%%%%%%%%%%%%%%%%%%%%%%%%%%%%%%%%%%%%

\subsection{\lib{libspf2-1.2.5}}

The DNSSEC extension to \lib{libspf2} provides DNSSEC validation to DNS
queries in \lib{libspf2}.

%%%%%%%%%%%%%%%%%%%%%%%%%%%%%%%%%%%%%%

\subsubsection{Installation}

To install the DNSSEC-enhanced \lib{libspf2-1.2.5}, \path{libspf2-1.2.5.tar.gz}
and \path{libspf2-1.2.5\-\_\-dns\-sec\_\-patch} must be downloaded from
\url{http://www.libspf2.org/download.html}.

Before compiling the patched \lib{libspf2}, you will need to install the
\lib{libsres} and the \lib{libval} libraries from
\url{http://dnssec-tools.sourceforge.net}.

Unpack and patch the \lib{libspf} distribution with the following commands:

\begin{verbatim}
    # tar -xvzf libspf2-1.2.5.tar.gz
    # cd libspf2-1.2.5
    # patch -p 0 -b -z .orig </path/libspf2-1.2.5_dnssec_patch.txt
\end{verbatim}

This will apply the patch and store the original files with a \path{.orig}
suffix.

To compile \lib{libspf2}, the following sequence of commands must be executed
in the main \path{libspf2-1.2.5} directory.  It will generate the new
\cmd{configure} command, as well as other files needed for compilation:

\begin{verbatim}
        # aclocal
        # autoheader
        # libtoolize --automake --force
        # automake
        # autoconf
        # ./configure --enable-dnssec-support
\end{verbatim}

Finally, compile and install \lib{libspf2} as per the instructions given
in the \lib{libspf2-1.2.5} distribution.

%%%%%%%%%%%%%%%%%%%%%%%%%%%%%%%%%%%%%%

\subsubsection{Usage}

The layered DNS architecture in \lib{libspf2} is amenable to adding DNSSEC
validation.  The {\it spf\_\-dns\_\-resolv} layer has been modified to use
the {\it val\_query()} function from the \lib{libval} library for making
DNS queries.  This function returns a validator status along with the answer
to the DNS query.

The following files have been modified:
\begin{description}
\item{\path{configure.ac}}:
Added configuration checks for the DNSSEC validator library.

\item{\path{config.h.in}}:
New {\it \#defines} for DNSSEC validator.

\item{\path{src/include/spf\_server.h}}:
New constants for adding the DNSSEC validation layer.

\item{\path{src/include/spf\_dns\_resolv.h}}:
New function declaration to support DNSSEC validation.

\item{\path{src/include/spf\_response.h}}:
Additional error codes for DNSSEC validation failure.

\item{\path{src/include/spf\_dns.h}}:
Additional error codes for DNSSEC validation failure.

\item{\path{src/libspf2/spf\_strerror.c}}:
Map DNSSEC validation failure error code to an appropriate string.

\item{\path{src/libspf2/Makefile.am}}:
Added new source files for compilation.

\item{\path{src/libspf2/spf\_interpret.c}}:
Return appropriate error codes for DNSSEC validation failure.

\item{\path{src/libspf2/spf\_dns\_resolv.c}}:
Return appropriate error codes for DNSSEC validation failure.

\item{\path{src/libspf2/spf\_get\_exp.c}}:
Return appropriate error codes for DNSSEC validation failure.

\item{\path{src/libspf2/spf\_server.c}}:
Return appropriate error codes for DNSSEC validation failure.

\end{description}

In addition, the \func{SPF\_dns\_lookup()} function can now return
the DNS\-SEC\_\-FAILURE error value.

The error code SPF\_E\_DNSSEC\_FAILURE is added to the list of error codes
in the response structure.  It is returned when a DNSSEC validation failure
occurs during SPF processing.

%%%%%%%%%%%%%%%%%%%%%%%%%%%%%%%%%%%%%%

\subsubsection{API for Applications}

The enumerated type {\it SPF\_server\_dnstype\_t} has been extended with two
values:  SPF\_\-DNS\_\-DNS\-SEC\_\-RESOLV and SPF\_\-DNS\_\-DNS\-SEC\_\-CACHE.
This allows applications to add a DNSSEC validation layer on top of either
the resolve layer or the cache layer.

Thus, if an application wants DNSSEC validation with caching, it must create
the SPF server instance using:

\begin{verbatim}
    SPF_server_t *spf_server = SPF_server_new (SPF_DNS_DNSSEC_CACHE, debug)
\end{verbatim}

The \var{debug} variable specifies the desired debug level.  After this, the
application can perform SPF processing with the SPF server instance as usual.

After SPF processing, the \var{SPF\_response} object/variable is available
to the application.  The application can check the error codes within this
variable to see if any of them matches SPF\_\-E\_\-DNS\-SEC\_\-FAILURE, and
determine if there was DNSSEC validation failure during SPF processing.

For example, it can do the following:

\begin{verbatim}
        SPF_request_t *requestp;
        SPF_response_t *responsep;

        /* initialize and configure requestp ... */

        /* SPF-checks */
        SPF_request_query_mailfrom(requestp, &responsep);

        /* Check for DNSSEC validation failure */
        do {
            int i, num_errs;
            SPF_error_t *err;
            SPF_errcode_t errcode;
            char *errmsg;

            num_errs = SPF_response_warnings (responsep);

            for (i=0; i<num_errs; i++) {
                err = SPF_response_message (responsep, i);
                if (err) {
        	    errcode = SPF_error_code (err);
        	    if (errcode == SPF_E_DNSSEC_FAILURE) {
        	        errmsg = (char *) SPF_error_message (err);
        	        /* Take appropriate action */
        	    }
                }
            }
        } while (0);
\end{verbatim}


%%%%%%%%%%%%%%%%%%%%%%%%%%%%%%%%%%%%%%

\subsubsection{Example}

The \path{spfmilter-1.0.8\_dnssec\_patch} patch for \cmd{spfmilter-1.0.8}
provides three modes of operation for DNSSEC:  {\it ignore}, {\it warn},
{\it reject}.  If the mode is {\it ignore}, the SPF server is initialized
with the \const{SPF\_\-DNS\_\-CACHE} flag; otherwise, it is initialized
with the \const{SPF\_\-DNS\_\-DNS\-SEC\_\-RESOLV} flag:

\begin{verbatim}
        if (dnssec_policy == SPFMILTER_DNSSEC_POLICY_IGNORE) {
            spf_server = SPF_server_new(SPF_DNS_CACHE, debug);
        }
        else {
            spf_server = SPF_server_new(SPF_DNS_DNSSEC_RESOLV, debug);
        }
\end{verbatim}

After SPF processing is done, the \path{spfmilter-1.0.8\_dnssec\_patch} looks
for the \const{SPF\_\-E\_\-DNS\-SEC\_\-FAILURE} error code in the response if
it is operating in the {\it reject} mode.  If it detects this error code, it
will return \const{SPF\-MIL\-TER\_\-RESULT\_\-FAIL}, which causes the mail to
be rejected or marked according to the \cmd{spfmilter} configuration.

\begin{verbatim}
        /* Check if there was a DNSSEC validation failure */
        if (dnssec_policy == SPFMILTER_DNSSEC_POLICY_REJECT) {
            int i;
            int num_errs;

            printf("spfmilter: DNSSEC reject policy is in effect\n");
            num_errs = lib_get_num_errors(ld);

            for (i=0; i<num_errs; i++) {
                SPF_error_t *err;

                err = SPF_response_message (ld->responsep, i);
                if (err) {
                    if (SPF_error_code(err) == SPF_E_DNSSEC_FAILURE) {
                        printf("spfmilter: DNSSEC validation failure.
Rejecting mail.\n");
                        retval = SPFMILTER_RESULT_FAIL;
                        break;
                    }
                }
            }
        }
\end{verbatim}

In the above code, \var{ld->responsep} points to the response from SPF
processing.  The number of error codes in the response is returned by the
\func{lib\_get\_num\_errors(ld)} function.

If \cmd{spfmilter} is operating in the {\it warn} mode, the
\const{SPF\_\-E\_\-DNS\-SEC\_\-FAILURE} error code will be present in the
response and will be added to the Received-SPF mail-header as an {\it x-dnssec}
field:
\begin{verbatim}
        x-dnssec="fail (DNSSEC validation failed.)";
\end{verbatim}

If \cmd{spfmilter} is operating in the {\it warn} or {\it reject} mode and
DNSSEC validation succeeds, the Received-SPF mail-header will contain the
following:
\begin{verbatim}
        x-dnssec="pass";
\end{verbatim}

The same result will be given if there was no SPF processing.

If \cmd{spfmilter} is operating in the {\it ignore} mode, DNSSEC validation
is not performed.  The Received- SPF mail-header will show:
\begin{verbatim}
        x-dnssec="none";
\end{verbatim}

%%%%%%%%%%%%%%%%%%%%%%%%%%%%%%%%%%%%%%%%%%%%%%%%

\subsection{\cmd{sendmail}}

This section describes the installation, configuration and execution steps
for adding DNSSEC validation to \cmd{sendmail-8.13.3}.  This DNSSEC
validation is for outbound email.

%%%%%%%%%%%%%%%%%%%%%%%%%%%%%%%%%%%%%%

\subsubsection{Installation}

Download \path{sendmail-8.13.3.tar.gz} from
\url{ftp://ftp.sendmail.org/pub/sendmail/}.

This distribution must be unpacked and patched with the DNSSEC validation
patch.  These commands will apply the patch and store the original files with
a \path{.orig} suffix:

\begin{verbatim}
        # tar -xvzf sendmail-8.13.3.tar.gz
        # cd sendmail-8.13.3
        # patch -p 0 -b -z .orig < /path/sendmail-8.13.3_dnssec_patch.txt
\end{verbatim}

This patch requires the \lib{libval} library for DNSSEC validation.  This
library can be found at \url{http://dnssec-tools.sourceforge.net}.  You must
install this library before compiling the patched \cmd{sendmail} source.

Add the following line to the
\path{sendmail-8.13.3/devtools/Site/site.config.m4} file,
if it is not already present:

\begin{verbatim}
        APPENDDEF(`confLIBS', `-lsres -lval -lcrypto')
\end{verbatim}

Build and install \cmd{sendmail} as per the instructions given in the
\path{README} and \path{INSTALL} files with the \cmd{sendmail} distribution.

To enable DNSSEC validation, the \var{RequireDNSSEC} option must be added to
the \var{ResolverOptions} in the \cmd{sendmail} configuration.  This can be
done by adding the following configuration line to the \path{sendmail.mc}
file:

\begin{verbatim}
        define(`confBIND_OPTS', `+RequireDNSSEC')
\end{verbatim}

You will need to generate the \path{sendmail.cf} file from \path{sendmail.mc}
and then place it in your system's \cmd{sendmail} system configuration
directory.

To disable DNSSEC validation, remove this option from \const{BIND\_OPTS}.

Start \cmd{sendmail} as usual with the \path{sendmail.cf} file appropriately
configured with the \var{ResolverOptions}.

%%%%%%%%%%%%%%%%%%%%%%%%%%%%%%%%%%%%%%

\subsubsection{Testing}

Take the following steps to ensure that DNSSEC validation is working properly
for your \cmd{sendmail}. 

\begin{enumerate}

\item Start \cmd{sendmail} with the above patch applied,
and the \var{RequireDNSSEC} option added to ResolverOptions.

\item Send email to a user at a domain whose MX records can be DNSSEC
validated.  Verify that the mail goes through properly.

\item Send email to a user at a domain whose MX records cannot be DNSSEC
validated.  Verify that an SMTP error message is returned back to the sender.

\item Remove the \var{RequireDNSSEC} option from \var{ResolverOptions} and
restart \cmd{sendmail}.   Run steps 2 and 3 above, and verify that mail goes
through in both cases, since there is no DNSSEC validation.

\end{enumerate}

%%%%%%%%%%%%%%%%%%%%%%%%%%%%%%%%%%%%%%%%%%%%%%%%

\subsection{\cmd{spfmilter-1.0.8}}

This section describes the installation, configuration and execution steps
for adding DNSSEC validation to \cmd{sendmail} using \cmd{spfmilter-1.0.8}.

%%%%%%%%%%%%%%%%%%%%%%%%%%%%%%%%%%%%%%

\subsubsection{Installation}

Download \path{spfmilter-1.0.8.tar.gz} from
\url{http://www.libspf2.org/download.html}.

Unpack and patch the \cmd{spfmilter} distribution with the following commands:

\begin{verbatim}

        # tar -xvzf spfmilter-1.0.8.tar.gz
        # cd spfmilter-1.0.8
        # patch -p 0 -b -z .orig </path/spfmilter-1.0.8_dnssec_patch.txt
\end{verbatim}

This will apply the patch and store the original files with a \path{.orig}
suffix.

Before compiling, run \cmd{autoconf} from the main \path{spfmilter-1.0.8}
directory to generate a new \path{configure} file.  Then, run \cmd{configure}
with the {\it --enable-dnssec-support} option:

\begin{verbatim}
        # ./configure --enable-dnssec-support
\end{verbatim}

Finally, compile and install \cmd{spfmilter} as per the instructions given
in the \cmd{spfmilter-1.0.8} distribution.  The only change is that instead
of installing the original \lib{libspf2}, you need to patch it with the
\path{libspf2-1.x.y\_dnssec\_patch} (where {\it x.y} is either 0.4 or 2.5)
given in \path{../libspf2-1.x.y\_dnssec\_patch.txt}, and then install
\lib{libspf2}.  For instructions on how to install \lib{libspf2} with the
DNSSEC patches, see \path{../libspf2-1.x.y\_dnssec\_howto.txt}.

Configure \cmd{sendmail} as per the instructions given in the \cmd{spfmilter}
distribution.

%%%%%%%%%%%%%%%%%%%%%%%%%%%%%%%%%%%%%%

\subsubsection{Usage}

The DNSSEC patch adds a new command line option to \cmd{spfmilter}:
{\it -s $<$ignore$|$warn$|$reject$>$} or
{\it --dnssec\_policy=$<$ignore$|$warn$|$reject$>$}.
This option specifies the action \cmd{spfmilter} should take in the
event of DNSSEC validation failure.

If the option is {\it --dnssec\_policy=ignore} then DNSSEC validation will not
take place.  This is the default if the {\it --dnssec\_policy} option (or the
{\it -s} option) is not specified on the command line.

If the option is {\it --dnssec\_policy=reject} then a DNSSEC validation
failure will cause \cmd{spfmilter} to abort the processing of the current
message, and send an error message back to the sender.  The recipient will
not get the message.  The \cmd{sendmail} \path{maillog} file will show an
error message similar to the following:
\begin{verbatim}
        "Error: DNSSEC validation of SPF record failed."
\end{verbatim}

If the option is {\it --dnssec\_policy=warn} then the mail will be delivered
to the recipient, even in the event of a DNSSEC validation failure.  However,
in this case an error message will be added to the Received-SPF mail header.
This message will be similar to the following:
\begin{verbatim}
        x-dnssec="fail (DNSSEC validation of SPF record failed.)";
\end{verbatim}

The \cmd{sendmail} \path{maillog} file will also show a similar warning message.

In case of the {\it reject} and {\it warn} policies, if DNSSEC validation
succeeds, the Received-SPF mail header will contain the following status
message:
\begin{verbatim}
        x-dnssec="pass";
\end{verbatim}

In case of the {\it ignore} policy, the Received-SPF mail header will contain
the following status message:

\begin{verbatim}
        x-dnssec="none";
\end{verbatim}

which means that there was no DNSSEC validation.  This error code is also
added if there is no SPF record for the domain.

The DNSSEC policy is applied to all messages, irrespective of the sender or the
recipient.  Future versions may allow a more granular policy based on the
sender's domain, sender's email address, and recipient's email address.

Use other options to \cmd{spfmilter} as per the instructions given in the
\cmd{spfmilter} distribution.

%%%%%%%%%%%%%%%%%%%%%%%%%%%%%%%%%%%%%%

\subsubsection{Testing}

Start \cmd{spfmilter} with the above DNSSEC patch applied and start
\cmd{sendmail} with the appropriate configuration for \cmd{spfmilter}.

%%%%%%%%%%%%%%%%%%%%%%%%%%

\paragraph{Basic Scenarios: DNSSEC validation of the SPF Records \\}\verb" "

The following table gives a summary of the basic scenarios.  The policy columns show the successful completion result.

\begin{table}[hb]
\begin{center}
\begin{tabular}{|l|c|c|c|}
\hline
Domain		& \multicolumn{3}{c|}{Policy}	\\
\cline{2-4}
Characteristics	& Ignore & Warn & Reject \\
\hline

No SPF Records & No DNSSEC  & No DNSSEC		& No DNSSEC	\\
	       & Validation & Validation	& Validation	\\
\hline

SPF Records present & No DNSSEC	 & Error message in mail & Error message in mail log. \\
but not a DNSSEC    & Validation & header and mail log   & Mail dropped.  Error \\
signed zone	    &	         & mail log.	         & message sent to sender. \\
\hline

SPF Records present & No DNSSEC  & Success message & Success message \\
and a DNSSEC-	    & Validation & in mail header  & in mail header \\
signed zone	    &		 & and mail log.   & and mail log. \\
\hline

\end{tabular}
\caption{Basic Scenario Matrix}
\label{basicmatrix}
\end{center}
\end{table}

\subparagraph{Messages sent from a domain that does not have SPF records}

In this case, there is no SPF processing.  Hence, all mail will go through,
irrespective of the DNSSEC policy in \cmd{spfmilter}.  The Received-SPF header
will contain the following field:
\begin{verbatim}
    x-dnssec="pass";
\end{verbatim}

\subparagraph{Messages sent from a domain that has SPF records, but does not have DNSSEC-signed zones.}

\begin{itemize}
\item {\it ignore policy}:  DNSSEC validation will not be performed.
The Received-SPF header will contain {\it x-dnssec=``none'';}

\item {\it warn policy}:  An error message will be added to the Received-SPF
mail header and the \cmd{sendmail} \path{maillog}.
This message will be:
\begin{verbatim}
        x-dnssec="fail (DNSSEC validation of SPF record failed.)";
\end{verbatim}

\item {\it reject policy}:  An error message will be added to the
\cmd{sendmail} \path{maillog}.  The mail will be dropped and an error
message will be returned to the sender.  This message will be:
\begin{verbatim}
        x-dnssec="fail (DNSSEC validation of SPF record failed.)";
\end{verbatim}
\end{itemize}

\subparagraph{Messages sent from a domain that has SPF records as well as DNSSEC-signed zones.}

\begin{itemize}
\item {\it ignore policy}:  DNSSEC validation will not be performed.  The
Received-SPF header will contain {\it x-dnssec=``none'';}.

\item {\it warn policy}:  A DNSSEC validation success message will be added
to the Received-SPF mail header and the \cmd{sendmail} \path{maillog}:
{\it x-dnssec=``pass'';}.

\item {\it reject policy}:  A DNSSEC validation success message will be added
to the Received-SPF mail header and the \cmd{sendmail} \path{maillog}:
{\it x-dnssec=``pass'';}

\end{itemize}

%%%%%%%%%%%%%%%%%%%%%%%%%%

\paragraph{Intermediate Scenarios:  DNSSEC validation of the SPF Mechanisms \\}\verb" "

These scenarios test various SPF mechanisms.  In this case, the primary SPF
record is signed.  However, the further records that it points to may or may
not be signed.  It is assumed that the mail is sent from an IP address that
is ultimately referenced by the DNS records; i.e., in the absence of DNSSEC
validation, the SPF result will be a PASS.  However, one or more intermediate
DNS records may or may not be signed using DNSSEC.  This gives rise to many
interesting scenarios, some of which are enumerated in the following table
along with their expected results.

The policy columns show the successful completion result,
except for the PTR records.

\eject

\begin{table}[ht]
\begin{center}
\begin{tabular}{|c|c|c|c|c|}
\hline
Mechanism & Record Conditions	& \multicolumn{3}{c|}{Policy}	\\
\cline{3-5}
	  & 			& Ignore & Warn & Reject \\
\hline

A & Signed A		& pass	& pass		& pass \\
\cline{2-5}
  & Unsigned A		& pass	& mail/warning	& no mail/abort \\
\hline

MX & Signed MX		& pass	& pass		& pass \\
\cline{2-5}
   & Unsigned MX	& pass	& mail/warning	& no mail/abort \\
\hline

EXISTS & Signed EXISTS		& pass	& pass		& pass \\
\cline{2-5}
       & Unsigned EXISTS	& pass	& mail/warning	& no mail/abort \\
\hline

	& Signed INCLUDE	& pass	& pass		& pass		\\
	& Signed A		&	&		&		\\
\cline{2-5}
	& Signed INCLUDE	& pass	& mail/warning	& no mail/abort \\
INCLUDE	& Unsigned MX		&	&		&		\\
\cline{2-5}
	& Unsigned INCLUDE & pass & mail/warning	& no mail/abort \\
	& Signed A	   &	  &			&		\\
\cline{2-5}
	& Unsigned INCLUDE & pass & mail/warning	& no mail/abort \\
	& Unsigned MX	   &	  &			&		\\
\hline

	 & Signed REDIRECT & pass & pass		& pass		\\
	 & Signed A	   &	  & 			&		\\
\cline{2-5}
	 & Signed REDIRECT & pass & mail/warning	& no mail/abort \\
REDIRECT & Unsigned MX	   &	  &			&		\\
\cline{2-5}
	 & Unsigned REDIRECT & pass & mail/warning & no mail/abort \\
	 & Signed A	     &	    & 		  	&		  \\
\cline{2-5}
	 & Unsigned REDIRECT & pass & mail/warning & no mail/abort \\
	 & Unsigned MX	     &	    &		  	&		  \\
\hline

PTR	&	&	&	&	\\
\cline{2-5}
	&	&	&	&	\\
\hline
\end{tabular}
\caption{Intermediate Scenario Matrix}
\label{intermediatematrix}
\end{center}
\end{table}

%%%%%%%%%%%%%%%%%%%%%%%%%%%%%%%%%%%%%%%%%%%%%%%%

\subsection{\cmd{thunderbird}}

The DNSSEC-Tools distribution contains files for a \cmd{thunderbird}
extension, called {\it spfdnssec}, that displays the Received-SPF mail header
and its various fields.  In particular, it displays the {\it x-dnssec} field
that shows the result of DNSSEC validation during the SPF processing.

To compile, just run \cmd{make}.  This will create a file named
\path{spfdnssec.xpi}.  This can be used to install the extension in
\cmd{thunderbird} from the {\it Tools-$>$Extensions} menu.


%%%%%%%%%%%%%%%%%%%%%%%%%%%%%%%%%%%%%%%%%%%%%%%%

\subsection{\cmd{mozilla}}

This DNSSEC-Tools distribution contains a patch to apply to the \cmd{mozilla}
code for enabling DNSSEC checking of URL DNS names.  It is in {\it beta} form.

%%%%%%%%%%%%%%%%%%%%%%%%%%%%%%%%%%%%%%

\subsubsection{Installation}

To install the DNSSEC-Tools patch, perform the following steps:

\begin{enumerate}

\item Install the \path{dnssec-tools} package

{\it (We need a library.)}

\item Apply the patch found in this directory to the \cmd{mozilla} source tree.
This should use \cmd{mozilla} version 1.7.3 or higher.

\item If the configure portion of the patch fails, run \cmd{autoconf}
after you apply the patch.  Version 2.13 of \cmd{autoconf} should be use,
not 2.5x.

\item Add the {\it --with-system-val} flag to the \cmd{configure} script in the
\cmd{mozilla} hierarchy.

\item Build \cmd{mozilla}.  If you're using a \path{~/.mozconfig} file,
then add this line to it:
\begin{verbatim}
        ac_add_options --with-system-val
\end{verbatim}

\end{enumerate}

%%%%%%%%%%%%%%%%%%%%%%%%%%%%%%%%%%%%%%%%%%%%%%%%

\subsection{\cmd{logwatch}}

\cmd{logwatch} is a customizable log analyzer.  It has been customized to
analyze DNSSEC logs.  This section describes how to install and configure
\cmd{logwatch} for DNSSEC logging.

%%%%%%%%%%%%%%%%%%%%%%%%%%%%%%%%%%%%%%

\subsubsection{Installation}

Install \cmd{logwatch}, the current version of which is 6.0.2.
\cmd{logwatch} is available from\\
\url{http://www2.logwatch.org:81/tabs/download}.

When \cmd{logwatch} is installed, the files in the \path{tools/logwatch}
directory must be installed.

The following assumptions are made about the current installation set-up:
\begin{itemize}
\item Log files for BIND are located in \path{/var/log} and are called
\path{dnssec} and \path{resolver}.  The location and name of these files are
configured in the BIND config file, often called \path{named.conf}.  See
below for a few tips on configuring BIND to log security messages.

\item \cmd{logwatch} is installed in \$LOGWATCH\_DIR (by default this would be
\path{/etc/log.d}).
\end{itemize}

You may edit the \cmd{logwatch} config files and scripts to change these names
if you have used something else.

Copy the files in \path{conf/logfiles}, \path{conf/services},
\path{scripts/shared}, and \path{scripts/services} into the same
directory structure in \path{\$LOGWATCH\_DIR}.  For example:

\begin{verbatim}
        # cp ./conf/logfiles/* /etc/log.d/conf/logfiles/.
        # cp ./conf/services/* /etc/log.d/conf/services/.
        # cp ./scripts/services/* /etc/log.d/scripts/services/.
        # cp ./scripts/shared/* /etc/log.d/scripts/shared/.
\end{verbatim}

This is all that is necessary to get \cmd{logwatch} to monitor BIND's
security log files.  Run \cmd{logwatch} to see the DNSSEC and RESOLVER
sections for output.

%%%%%%%%%%%%%%%%%%%%%%%%%%%%%%%%%%%%%%

\subsubsection{Configuring BIND for Security Logging}

In your BIND configuration file (e.g., \path{named.conf})
you need to have a logging section.  It will look something like this:

\begin{verbatim}
        logging {
                channel resolver {
                        file "/var/log/resolver" versions 10 size 300k;
                        print-time yes;
                        print-category no;
                        print-severity yes;
                        severity debug 3;
                };
                channel dnssec {
                        file "/var/log/dnssec" versions 10 size 300k;
                        print-time yes;
                        print-category no;
                        print-severity yes;
                        severity debug 9;
                };
                category dnssec { dnssec; };
                category resolver { resolver; };
        };
\end{verbatim}

This allows you to send log messages to separate log files. This configuration
fits the \cmd{logwatch} configuration files provided here.  The ``channel'' is
a name of your own choosing.  The file name can be whatever you want, but if
you use something other than \path{/var/log/dnssec} or
\path{/var/log/resolver}, you will need to modify \path{dnssec.conf} and/or
\path{resolver.conf} in the \path{conf/logfiles} directory to match the file
name.

``Categories'' as used in \path{named.conf} are defined as follows for
\cmd{BIND 9.x}:

\begin{itemize}
\item dnssec: processing of DNSSEC-signed responses

\item resolver: Name resolution, including the processing of recursive
queries from resolvers
\end{itemize}

For more detail on configuring BIND, a good reference is $[$DNS and BIND$]$.

%%%%%%%%%%%%%%%%%%%%%%%%%%%%%%%%%%%%%%%%%%%%%%%%%%%%%%%%%%%%%%%%%%%%%%%%%%%%%%

\clearpage
\appendix

\markboth{DNSSEC-Tools Operations Manual -- Installation Guide}{DNSSEC-Tools Operations Manual -- Installation Guide}
\section{References}
\markboth{DNSSEC-Tools Operations Manual -- Installation Guide}{DNSSEC-Tools Operations Manual -- Installation Guide}
\label{app-refs}

\begin{description}
\item $[$DNS and BIND$]$ Albitz, Liu, Loukides; O'Reilly, 1998. 
\end{description}

%%%%%%%%%%%%%%%%%%%%%%%%%%%%%%%%%%%%%%%%%%%%%%%%%%%%%%%%%%%%%%%%%%%%%%%%%%%%%%

\end{document}

