
\clearpage
\subsection{ZSK Roll-Over --- Current ZSK Compromise}
\label{roll-emergency-curzsk}

{\bf If the KSK is also compromised, perform the emergency KSK roll-over first.}

As long as there is a valid KSK signature over the ZSK, the KSK can continue
to be used to inject false zone data.  If both keys are compromised, clients
are exposed to attacks\footnote{These attacks include signatures over false
data, replay attacks of the old KSK, and replay attacks of the old DS.} on
that data until the maximum of the expiration of the KSK's RRSIG (created by
the ZSK) and the parent's signature over the DS of that KSK.  Short TTLs allow
recursive servers to more quickly recover from key-compromise situations,
allowing them to get new keys more quickly.  Key compromise exposes the secure
recursive server to replays of the old key until the signature expires.

The emergency procedures described for key roll-over use the rationale that
injection of valid but false data (which can be generated using the
compromised key) is more serious than discontinuity in the ability to validate
true data. Thus, during emergency ZSK roll-over, there will be a period (up
to twice the maximum zone TTL) where the cached zone data may not validate
against the new ZSK. Also, the steps below are only useful if the Published
and Current keys are kept separate from each other and if the Published ZSK
has not also been compromised. If both ZSKs are compromised follow the steps
in Section~ref{roll-emergency-zsks}.  If only the Published key is compromised
follow the steps in Section~\ref{roll-emergency-pubzsk}.

See Figure~\ref{fig:zskroll-emerg}.

%%%%%%%%%%%%%%%%%%%%%%%%%%%%%%%%%%%%%%


\subsubsection{Generate a New ZSK}

Generate a new ZSK, as described in section~\ref{genzsk}.
Update the key-tag table with the new ZSK, and set its status to New (N).

\begin{center}
\begin{tabular}{|c|c|c|c|c|c|c|c|c|c|}
\hline
{\bf Zone} &
\multicolumn{4}{c|}{{\bf ZSK}} &
\multicolumn{4}{c|}{{\bf KSK}} &
{\bf Exp} \\
\cline{2-9}

 & Tag & Size & Creat & S & Tag & Size & Creat & S & \\
\hline

			&
\underline{zsktag-cur}	&
512			&
\underline{date}	&
C			&
\underline{ksktag}	&
1024			&
\underline{date}	&
C			&
			\\

\cline{2-9}

\underline{zone.name}	&
\underline{zsktag-pub}	&
512			&
\underline{date}	&
P			&
& & & & 
\underline{date}	\\

\cline{2-9}

				&
{\bf \underline{zsktag-new}}	&
{\bf 512}			&
{\bf \underline{date}}		&
{\bf N}				&
& & & &	\\

\hline
\end{tabular}
\end{center}



\subsubsection{Modify the Zone File}

The zone file must be modified to account for the key changes.  The Current
ZSK must be deleted and the New ZSK must be added.  Also, the SOA serial
number must be changed so that the zone file's new contents will be recognized.

\begin{tabbing}
\hspace{0.5in} \= 12345678 \= 12345678 \= 12345678\= 12345678 \= 12345678 \kill
\hspace{0.5in}\$ {\bf vi \underline{zonefile}} $[$ENTER$]$ \\
\hspace{0.5in}\underline{zone.name} \> \> \> IN \> SOA \> servername contact (\\
\hspace{3.5in}{\bf 2005092102} ; Increase current value by 1. \\
\hspace{4.4in};  This value may be different \\
\hspace{4.4in}; in your zone file. \\
\hspace{0.5in}\>           \>         ... \\
\hspace{0.5in}\>              ) \\
\hspace{0.5in}... \\
\hspace{0.5in};; ksk \\
\hspace{0.5in}\$INCLUDE ``/path/to/K\underline{zone.name}.+005+\underline{ksktag}.key'' \\
\hspace{0.5in}{\bf \sout{;; cur zsk}} \\
\hspace{0.5in}{\bf \sout{\$INCLUDE ``/path/to/K\underline{zone.name}.+005+\underline{zsktag-cur}.key''}} \\
\hspace{0.5in};; pub zsk \\
\hspace{0.5in}\$INCLUDE ``/path/to/K\underline{zone.name}.+005+\underline{zsktag-pub}.key'' \\
\hspace{0.5in}{\bf ;; new zsk} \\
\hspace{0.5in}{\bf \$INCLUDE ``/path/to/K\underline{zone.name}.+005+\underline{zsktag-new}.key}'' \\
\hspace{0.5in}... \\
\hspace{0.5in}\$ \\
\end{tabbing}


%%%%%%%%%%%%%%%%%%%%%%%%%%%%%%%%%%%%%%

\subsubsection{Sign Zone with the Published ZSK Only}

Follow steps \ref{check-zonefile-unsigned-nodeleg}~--~\ref{signzone-nodel-last}
if the zone does no delegation.  Follow steps
\ref{check-zonefile-unsigned-deleg}~--~\ref{signzone-deleg-last} if the
zone does delegation.  The ZSK used in the signing process in
Section~\ref{signzone-no-deleg} or~\ref{signzone-has-deleg} must be the key
marked as the Published key (P) in the key-tag table. The KSK used
as input to \cmd{dnssec-signzone} does not change, so the keyset does not
change and does not have to be re-sent to the parent.

%%%%%%%%%%%%%%%%%%%%%%%%%%%%%%%%%%%%%%


\subsubsection{Reload the Zone}

The \cmd{rndc} command will reload the name server configuration files and
the zone contents.  The name server process is assumed to be already running.

\begin{tabbing}
\hspace{0.5in}\$ {\bf rndc reload zone.name} $[$ENTER$]$ \\
\hspace{0.5in}\$ \\
\end{tabbing}


\subsubsection{Update the Key-Tags Table}

Update the key-tags table to reflect the changed key status.  Delete the old
Current ZSK. Change the status of the Published ZSK to Current.  Change the
status of the New ZSK to Published.

\begin{center}
\begin{tabular}{|c|c|c|c|c|c|c|c|c|c|}
\hline
{\bf Zone} &
\multicolumn{4}{c|}{{\bf ZSK}} &
\multicolumn{4}{c|}{{\bf KSK}} &
{\bf Exp} \\
\cline{2-9}

 & Tag & Size & Creat & S & Tag & Size & Creat & S & \\
\hline

					&
{\bf \sout{\underline{zsktag-cur}}}	&
{\bf \sout{512}}			&
{\bf \sout{\underline{date}}}		&
{\bf \sout{C}}				&
\underline{ksktag}			&
1024					&
\underline{date}			&
C					&
					\\

\cline{2-9}

\underline{zone.name}	&
\underline{zsktag-pub}	&
512			&
\underline{date}	&
{\bf \sout{P}}		&
& & & &
\underline{date}			\\

			&
			&
			&
			&
{\bf C}			&
& & & & \\

\cline{2-9}

			&
\underline{zsktag-new}	&
512			&
\underline{date}	&
{\bf \sout{N}}		&
& & & & \\

			&
			&
			&
			&
{\bf P}			&
& & & & \\

\hline
\end{tabular}
\end{center}



\subsubsection{Dispose of the Old Zone Key}

Delete the old ZSK's {\it .private} and {\it .key} files.

