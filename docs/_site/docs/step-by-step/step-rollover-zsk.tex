
\clearpage
\subsection{Current ZSK Roll-Over}
\label{roll-curzsk}

This section gives the steps necessary for the pre-publish scheme for ZSK
roll-over.  The alternative, the double-signature method, is used for rolling
over KSKs.  Double signatures for records signed by the ZSK can increase the
size of the zone many times.  The pre-publish scheme, although requiring more
steps for the roll-over, does not suffer from this problem. The size argument
does not apply during KSK roll-over since the DNSKEY RRset is the only record
doubly signed by the KSK.

See Figure~\ref{fig:zskroll}.

%%%%%%%%%%%%%%%%%%%%%%%%%%%%%%%%%%%%%%

\subsubsection{Ensure that Sufficient Time has Elapsed Since the Last Roll-Over}

The time between roll-overs has to be at least twice the maximum zone TTL
period.  This is the largest TTL in the entire zone file multiplied by two.

%%%%%%%%%%%%%%%%%%%%%%%%%%%%%%%%%%%%%%

\subsubsection{Sign Zone with the KSK and Published ZSK}

Follow steps \ref{check-zonefile-unsigned-nodeleg}~--~\ref{signzone-nodel-last}
if the zone does no delegation.  Follow steps
\ref{check-zonefile-unsigned-deleg}~--~\ref{signzone-deleg-last} if the
zone does delegation.  The ZSK used in the signing process in
Section~\ref{signzone-no-deleg} or~\ref{signzone-has-deleg} must be the key
that is marked as the Published key (P) in the key-tag table. The KSK used
as input to \cmd{dnssec-signzone} does not change, so the keyset does not
change and does not have to be re-sent to the parent.

Record the signature expiry date in the key-tag table.

\begin{center}
\begin{tabular}{|c|c|c|c|c|c|c|c|c|c|}
\hline
{\bf Zone} &
\multicolumn{4}{c|}{{\bf ZSK}} &
\multicolumn{4}{c|}{{\bf KSK}} &
{\bf Exp} \\
\cline{2-9}

 & Tag & Size & Creat & S & Tag & Size & Creat & S & \\
\hline

\underline{zone.name}	&
\underline{zsktag-cur}	&
512			&
\underline{date}	&
C			&
\underline{ksktag}	&
1024			&
\underline{date}	&
C			&
\underline{date}	\\

\cline{2-9}

			&
\underline{zsktag-pub}	&
512			&
\underline{date}	&
P			&
& & & & \\

\hline
\end{tabular}
\end{center}

%%%%%%%%%%%%%%%%%%%%%%%%%%%%%%%%%%%%%%


\subsubsection{Reload the Zone}

The \cmd{rndc} command will reload the name server configuration files and
the zone contents.  The name server process is assumed to be already running.

\begin{tabbing}
\hspace{0.5in}\$ {\bf rndc reload zone.name} $[$ENTER$]$ \\
\hspace{0.5in}\$ \\
\end{tabbing}


%%%%%%%%%%%%%%%%%%%%%%%%%%%%%%%%%%%%%%

\subsubsection{Wait for Old Zone Data to Expire from Caches}

Wait at least twice the maximum zone TTL period for the old zone data to
expire from name server caches.  This is the largest TTL in the entire zone
file multiplied by two.  This will also allow the new data to propagate.

%%%%%%%%%%%%%%%%%%%%%%%%%%%%%%%%%%%%%%


\subsubsection{Generate a New ZSK}

Generate a new ZSK, as described in section~\ref{genzsk}.
Update the key-tag table with the new ZSK, and set its status to New (N).

\begin{center}
\begin{tabular}{|c|c|c|c|c|c|c|c|c|c|}
\hline
{\bf Zone} &
\multicolumn{4}{c|}{{\bf ZSK}} &
\multicolumn{4}{c|}{{\bf KSK}} &
{\bf Exp} \\
\cline{2-9}

 & Tag & Size & Creat & S & Tag & Size & Creat & S & \\
\hline

			&
\underline{zsktag-cur}	&
512			&
\underline{date}	&
C			&
\underline{ksktag}	&
1024			&
\underline{date}	&
C			&
			\\

\cline{2-9}

\underline{zone.name}	&
\underline{zsktag-pub}	&
512			&
\underline{date}	&
P			&
& & & & 
\underline{date}	\\

\cline{2-9}

				&
{\bf \underline{zsktag-new}}	&
{\bf 512}			&
{\bf \underline{date}}		&
{\bf N}				&
& & & &	\\

\hline
\end{tabular}
\end{center}



\subsubsection{Modify the Zone File}

The zone file must be modified to account for the key changes.  The Current
ZSK must be deleted and the New ZSK must be added.  Also, the SOA serial
number must be changed so that the zone file's new contents will be recognized.

\begin{tabbing}
\hspace{0.5in} \= 12345678 \= 12345678 \= 12345678\= 12345678 \= 12345678 \kill
\hspace{0.5in}\$ {\bf vi \underline{zonefile}} $[$ENTER$]$ \\
\hspace{0.5in}\underline{zone.name} \> \> \> IN \> SOA \> servername contact (\\
\hspace{3.5in}{\bf 2005092102} ; Increase current value by 1. \\
\hspace{4.4in};  This value may be different \\
\hspace{4.4in}; in your zone file. \\
\hspace{0.5in}\>           \>         ... \\
\hspace{0.5in}\>              ) \\
\hspace{0.5in}... \\
\hspace{0.5in};; ksk \\
\hspace{0.5in}\$INCLUDE ``/path/to/K\underline{zone.name}.+005+\underline{ksktag}.key'' \\
\hspace{0.5in}{\bf \sout{;; cur zsk}} \\
\hspace{0.5in}{\bf \sout{\$INCLUDE ``/path/to/K\underline{zone.name}.+005+\underline{zsktag-cur}.key''}} \\
\hspace{0.5in};; pub zsk \\
\hspace{0.5in}\$INCLUDE ``/path/to/K\underline{zone.name}.+005+\underline{zsktag-pub}.key'' \\
\hspace{0.5in}{\bf ;; new zsk} \\
\hspace{0.5in}{\bf \$INCLUDE ``/path/to/K\underline{zone.name}.+005+\underline{zsktag-new}.key}'' \\
\hspace{0.5in}... \\
\hspace{0.5in}\$ \\
\end{tabbing}


\subsubsection{Update the Key-Tags Table}

Update the key-tags table to reflect the changed key status.  Delete the old
Current ZSK. Change the status of the Published ZSK to Current.  Change the
status of the New ZSK to Published.

\begin{center}
\begin{tabular}{|c|c|c|c|c|c|c|c|c|c|}
\hline
{\bf Zone} &
\multicolumn{4}{c|}{{\bf ZSK}} &
\multicolumn{4}{c|}{{\bf KSK}} &
{\bf Exp} \\
\cline{2-9}

 & Tag & Size & Creat & S & Tag & Size & Creat & S & \\
\hline

					&
{\bf \sout{\underline{zsktag-cur}}}	&
{\bf \sout{512}}			&
{\bf \sout{\underline{date}}}		&
{\bf \sout{C}}				&
\underline{ksktag}			&
1024					&
\underline{date}			&
C					&
					\\

\cline{2-9}

\underline{zone.name}	&
\underline{zsktag-pub}	&
512			&
\underline{date}	&
{\bf \sout{P}}		&
& & & &
\underline{date}			\\

			&
			&
			&
			&
{\bf C}			&
& & & & \\

\cline{2-9}

			&
\underline{zsktag-new}	&
512			&
\underline{date}	&
{\bf \sout{N}}		&
& & & & \\

			&
			&
			&
			&
{\bf P}			&
& & & & \\

\hline
\end{tabular}
\end{center}



%%%%%%%%%%%%%%%%%%%%%%%%%%%%%%%%%%%%%%

\subsubsection{Sign the Zone with the KSK and Current ZSK}

Follow the steps
\ref{check-zonefile-unsigned-nodeleg}~--~\ref{signzone-nodel-last}
if the zone does no delegation.  Follow the steps
\ref{check-zonefile-unsigned-deleg}~--~\ref{signzone-deleg-last}
if the zone does delegation.  The ZSK used in the signing process in
Section~\ref{signzone-no-deleg} or~\ref{signzone-has-deleg} must be the key
that is marked as the Current key (C) in the key-tag table (this was the older
Published key.) The KSK used as input to \cmd{dnssec-signzone} does not
change, so the keyset does not change and does not have to be re-sent to the
parent.


Record the signature expiry date in the key-tag table.

\begin{center}
\begin{tabular}{|c|c|c|c|c|c|c|c|c|c|}
\hline
{\bf Zone} &
\multicolumn{4}{c|}{{\bf ZSK}} &
\multicolumn{4}{c|}{{\bf KSK}} &
{\bf Exp} \\
\cline{2-9}

 & Tag & Size & Creat & S & Tag & Size & Creat & S & \\
\hline

\underline{zone.name}	&
\underline{zsktag-pub}	&
512			&
\underline{date}	&
C			&
\underline{ksktag}	&
1024			&
\underline{date}	&
C			&
\underline{date}	\\

\cline{2-9}

			&
\underline{zsktag-new}	&
512			&
\underline{date}	&
P			&
& & & & \\

\hline
\end{tabular}
\end{center}

%%%%%%%%%%%%%%%%%%%%%%%%%%%%%%%%%%%%%%

\subsubsection{Reload the Zone}

The \cmd{rndc} command will reload the name server configuration files and
the zone contents.  The name server process is assumed to be already running.

\begin{tabbing}
\hspace{0.5in}\$ {\bf rndc reload zone.name} $[$ENTER$]$ \\
\hspace{0.5in}\$ \\
\end{tabbing}


\subsubsection{Dispose of the Old Zone Key}

Delete the old ZSK's {\it .private} and {\it .key} files.

