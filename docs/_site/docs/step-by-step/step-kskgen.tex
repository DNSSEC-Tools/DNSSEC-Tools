
\clearpage
\subsection{Key-Signing Key (KSK) Generation}
\label{genksk}

This section provides the steps required to generate a new Key-Signing Key
(KSK).  See Figure~\ref{fig:keygen}.

%%%%%%%%%%%%%%%%%%%%%%%%%%%%%%%%%%%%%%

\subsubsection{Generate the Key}

Use the \cmd{dnssec-keygen} command to generate a key 1024 bits long.  Key
size is a rough measure of strength; KSKs are generally made stronger than
ZSKs.

\begin{tabbing}
\hspace{0.5in} \= 12345678 \= 12345678 \= 12345678 \= 12345678 \kill \\
\hspace{0.5in} \$ {\bf dnssec-keygen -a RSASHA1 -b 1024 -n ZONE -f KSK} {\bf \underline{zone.name}} \\
\hspace{0.5in} [ENTER] \\
\hspace{0.5in} K\underline{zone.name}.+005+\underline{ksktag} \\
\hspace{0.5in} \$ \\
\end{tabbing}

The process may take a few minutes to return its results. If the process
appears to have stalled, run the command using a pseudo-random number
generator as follows:

\begin{tabbing}
\hspace{0.5in} \= 12345678 \= 12345678 \= 12345678 \= 12345678 \kill \\
\hspace{0.5in} \$ {\bf dnssec-keygen -r /dev/urandom -a RSASHA1 -b 1024 -n ZONE -f KSK} \\
\hspace{0.5in} {\bf \underline{zone.name}} [ENTER] \\
\hspace{0.5in} K\underline{zone.name}.+005+\underline{ksktag} \\
\hspace{0.5in} \$ \\
\end{tabbing}

Two files are output by \cmd{dnssec-keygen}: \\
    - Private key contained in \path{K\underline{zone.name}.+005+\underline{ksktag}.private}\\
    - Public key contained in \path{K\underline{zone.name}.+005+\underline{ksktag}.key}

{\bf \underline{zone.name}} - the name of the zone (e.g., example.com)\\
{\bf \underline{ksktag}} - the key identifier (e.g., 24818) 

You must note this number in the key-tag table as you walk through this
document.  This number is automatically generated and should not be changed.
The key id will be the only field in the filename that changes as you rotate
keys, so it must be tracked.

This document uses RSASHA1 as the cryptographic algorithm, which is
represented by the ``005'' in the key name's algorithm field.

%%%%%%%%%%%%%%%%%%%%%%%%%%%%%%%%%%%%%%

\subsubsection{Update the Key-Tags Table}

Keep a record of the key-tags that currently refer to KSKs.

\begin{center}
\begin{tabular}{|c|c|c|c|c|c|c|c|c|c|}
\hline
{\bf Zone} &
\multicolumn{4}{c|}{{\bf ZSK}} &
\multicolumn{4}{c|}{{\bf KSK}} &
{\bf Exp} \\
\cline{2-9}

 & Tag & Size & Creat & S & Tag & Size & Creat & S & \\
\hline

\underline{zone.name} & & & & & \underline{ksktag} & 1024 & \underline{date} & & \\

\hline
\end{tabular}
\end{center}

Leave the status field (S) empty for now.

%%%%%%%%%%%%%%%%%%%%%%%%%%%%%%%%%%%%%%


\subsubsection{Store Separately the Private Keys for ZSKs and KSKs}

Storing the private keys for ZSKs and KSKs off-line is considered a good
security practice.  ZSK and KSK separation lessens the operational burden
during zone-key compromise, but only if the KSK is still considered safe for
use.  In cases where a signed zone is also updated via dynamic updates, the
ZSK will need to be on-line and available to the name server process.  To
avoid a common vulnerability point (such as compromise of the system on which
these keys reside), store the KSKs in a place that is considered safer.



