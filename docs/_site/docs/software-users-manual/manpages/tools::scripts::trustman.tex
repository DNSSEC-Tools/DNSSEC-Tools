\clearpage

\subsubsection{trustman}

{\bf NAME}

\cmd{trustman} - Manage keys used as trust anchors

{\bf SYNOPSIS}

trustman [options]

{\bf DESCRIPTION}

\cmd{trustman} manages keys used by DNSSEC as trust anchors.  It may be used
as a daemon for ongoing key verification or manually for initialization and
one-time key verification.

By default, \cmd{trustman} runs as a daemon to ensure that keys stored locally
in configuration files still match the same keys fetched from the zone where
they are defined.  (\path{named.conf} and \path{dnsval.conf} are the usual
configuration files.) These checks can be run once manually ({\it -S}) and in
the foreground ({\it -f}).

For each key mismatch check, if key mismatches are detected then \cmd{trustman}
performs the following operations:

\begin{itemize}
\item sets an add hold-down timer for new keys;
\item sets a remove hold-down timer for missing keys;
\item removes revoked keys from the configuration file.
\end{itemize}

On subsequent runs, the timers are checked.  If the timers have expired, keys
are added or removed from the configuration file, as appropriate.

{\bf CONFIGURATION}

\cmd{trustman} can also set up configuration data in the DNSSEC-Tools
configuration file for later use by the daemon.  This makes fewer command line
arguments necessary on subsequent executions.  (The configuration file is in
\path{dnssec-tools.conf}.)

Configuration data is stored in \path{dnssec-tools.conf}.  The current version
requires you to edit \path{dnssec-tools.conf} by hand and supply values for
the contact person's email address ({\it tacontact}) and the SMTP server ({\it
tasmtpserver}).  If necessary, edit the location of \path{named.conf} and
\path{dnsval.conf} in that file.

{\bf OPTIONS}

\cmd{trustman} takes a number of options, each of which is described in this
section.  Each option name may be shortened to the minimum number of unique
characters, but some options also have an alias (as noted.)  The single-letter
form of each option is denoted in parentheses, e.g.,
{\it --anchor\_data\_file datafile (-a)}.

\begin{description}

\item {\bf --anchor\_data\_file datafile (-a)}\verb" "

A persistent data file for storing new keys waiting to be added.

\item {\bf --config (-c)}\verb" "

Create a configure file for \cmd{trustman} from the command line options given.

\item {\bf --dnsval\_conf\_file conffile (-k)}\verb" "

A \path{dnsval.conf} file to read.

\item {\bf --zone zone (-z)}\verb" "

The zone to check.  Specifying this option supersedes the default
configuration file.

\item {\bf --foreground (-f)}\verb" "

Run in the foreground.

\item {\bf --hold\_time seconds (-w)}\verb" "

The value of the hold-down timer.

\item {\bf --mail\_contact\_addr email-address (-m)}\verb" "

Mail address for the contact person to whom reports should be sent.

\item {\bf --named\_conf\_file conffile (-n)}\verb" "

A \path{named.conf} file to read.

\item {\bf --no\_error (-N)}\verb" "

Send report when there are no errors.

\item {\bf --outfile output-file (-o)}\verb" "

Output file for configuration.

\item {\bf --print (-p)}\verb" "

Log/print messages to stdout.

\item {\bf --resolv\_conf\_file conffile (-r)}\verb" "

A \path{resolv.conf} file to read.  \path{/dev/null} can be specified to force
\lib{libval} to recursively answer the query rather than asking other name
servers).

\item {\bf --smtp\_server smtpservername (-s)}\verb" "

SMTP server that \cmd{trustman} should use to send reports by mail.

\item {\bf --single\_run (-S)}\verb" "

Run only once.

\item {\bf --syslog (-L)}\verb" "

Log messages to {\bf syslog}.

\item {\bf --sleeptime seconds (-t)}\verb" "

The number of seconds to sleep between checks. Default is 3600 (one hour.)

\item {\bf --test\_revoke}\verb" "

Use this option to test the REVOKE bit. No known implementation of
the REVOKE bit exists to date.

\item {\bf --help (-h)}\verb" "

Display a help message.

\item {\bf --verbose (-v)}\verb" "

Verbose output.

\item {\bf --version (-V)}\verb" "

Display version information.

\end{description}

{\bf SEE ALSO}

Net::DNS::SEC::Tools::conf.pm(3),
Net::DNS::SEC::Tools::defaults.pm(3),

dnssec-tools.conf(5)


