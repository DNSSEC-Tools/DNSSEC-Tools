\clearpage

\subsubsection{lsroll}

{\bf NAME}

\cmd{lsroll} - List the \struct{rollrec}s in a DNSSEC-Tools \struct{rollrec}
file

{\bf SYNOPSIS}

\begin{verbatim}

  lsroll [options] <rollrec-files>

\end{verbatim}

{\bf DESCRIPTION}

This script lists the contents of the specified \struct{rollrec} files.  All
\struct{rollrec} files are loaded before the output is displayed.  If any
\struct{rollrec}s have duplicated names, whether within one file or across
multiple files, the later \struct{rollrec} will be the one whose data are
displayed.

{\bf OUTPUT FORMATS}

The output displayed for each zone in a \struct{rollrec} file depends on the
selected records, the selected attributes, and the selected output format.
Each option in these option groups is described in detail in the next section.
The three base output formats, along with the default {\it -skip} format, are
described here.

The {\it -terse} option indicates that a minimal amount of output is desired;
the {\it -long} option indicates that a great deal of output is desired.  The
record-selection and attribute-selection options may be used in conjunction
with {\it -terse} to display exactly the set of \struct{rollrec} fields needed.

The default output format is that used when neither {\it -terse} nor {\it -long}
is given, and is a middle ground between terse and long output.

If the {\it -skip} option is given, then the default output format is a little
more restricted than the normal default.  Some \struct{rollrec} fields don't
make sense in the context of a skip records, and so are given as ``---''.
These fields are the KSK rollover phase, the ZSK rollover phase, the TTL
value, and the phase start.

\eject

The table below shows the fields displayed for each output format.

\begin{table}[ht]
\begin{center}
\begin{tabular}{|l|c|c|c|c|}
\hline
{\bf Rollrec Field} & {\bf Default} & {\bf Terse} & {\bf Long} & {\bf Skip} \\
\hline
rollrec name      & yes & yes & yes & yes \\
rollrec type      & no  & no  & yes & no  \\
zone name         & yes & no  & yes & yes \\
keyrec file       & yes & no  & yes & yes \\
KSK phase         & yes & no  & yes & no  \\
ZSK phase         & yes & no  & yes & no  \\
administrator     & no  & no  & yes & no  \\
directory         & no  & no  & yes & no  \\
logging level     & no  & no  & yes & no  \\
TTL value         & no  & no  & yes & no  \\
display flag      & no  & no  & yes & no  \\
phase start       & no  & no  & yes & no  \\
last KSK rollover & no  & no  & yes & no  \\
last ZSK rollover & no  & no  & yes & no  \\
\hline
\end{tabular}
\end{center}
\caption{\cmd{lsroll} Output Formats}
\end{table}

{\bf OPTIONS}

There are three types of options recognized by \cmd{lsroll}:  record-selection
options, attribute-selection options, and output-format options.  Each type
is described in the subsections below.

{\bf Record-selection Options}

These options select the records that will be displayed by \cmd{lsroll}.
By default, all records will be displayed; selecting one or the other of
these options will restrict the records shown.

In order to simplify the \cmd{lsroll} code and keep it easily understandable,
these options are mutually exclusive.

\begin{description}

\item {\bf -roll}\verb" "

List all ``roll'' records in the \struct{rollrec} file.

\item {\bf -skip}\verb" "

List all ``skip'' records in the \struct{rollrec} file.

\end{description}

{\bf Attribute-selection Options}

These options select the attributes of the records that will be displayed
by \cmd{lsroll}.

\begin{description}

\item {\bf -type}\verb" "

Include each \struct{rollrec} record's type in the output.  The type will be
either ``roll'' or ``skip''.

\item {\bf -zone}\verb" "

The record's zonefile is included in the output.  This field is part
of the default output.

\item {\bf -keyrec}\verb" "

The record's {\it keyrec} file is included in the output.
This field is part of the default output.

\item {\bf -kskphase}\verb" "

The record's KSK rollover phase are included in the output.
If this option is given with the {\it -zskphase} option, then the output will
follow the format described for the {\it -phases} option.
This field is part of the default output.

\item {\bf -zskphase}\verb" "

The record's ZSK rollover phase are included in the output.
If this option is given with the {\it -kskphase} option, then the output will
follow the format described for the {\it -phases} option.
This field is part of the default output.

\item {\bf -phases}\verb" "

The record's KSK and ZSK rollover phases are included in the output.
The listing is given with the KSK phase first, followed by the ZSK phase.

Examples of output from this option are:

\begin{table}[ht]
\begin{center}
\begin{tabular}{|c|c|c|}
\hline
{\bf KSK Phase} & {\bf ZSK Phase} & {\bf Output} \\
\hline
0 & 0 & 0/0 \\
3 & 0 & 3/0 \\
0 & 5 & 0/5 \\
\hline
\end{tabular}
\end{center}
\end{table}

\item {\bf -admin}\verb" "

The record's administrator value is included in the output.  If an
administrator value is not included in a \struct{rollrec}, then the value
``(defadmin)'' will be given.

\item {\bf -directory}\verb" "

The name of the directory that holds the zone's files is included in the
output.  If a zone directory is not included in a \struct{rollrec}, then the
value ``(defdir)'' will be given.

\item {\bf -loglevel}\verb" "

The \cmd{rollerd} logging level for this zone.  This value may be given in the
\struct{rollrec} file in either the textual or numeric form.  The textual form
of the logging level will be displayed, not the numeric.  If a logging level
value is not included in a \struct{rollrec}, then the value ``(deflog)'' will
be given.  If an undefined logging level value is included in a
\struct{rollrec}, then the value ``(unknownlog)'' will be given.

\item {\bf -ttl}\verb" "

The record's TTL value is included in the output.

\item {\bf -display}\verb" "

The record's display flag, used by \cmd{blinkenlights}, is included in the
output.

\item {\bf -phstart}\verb" "

The record's rollover phase is included in the output.
If no rollover has yet been performed for this zone, an empty date is given.

\item {\bf -lastksk}\verb" "

The record's last KSK rollover date is included in the output.
If no KSK rollover has yet been performed for this zone, an empty date is given.

\item {\bf -lastzsk}\verb" "

The record's last ZSK rollover date is included in the output.
If no ZSK rollover has yet been performed for this zone, an empty date is given.

\end{description}

{\bf Output-format Options}

These options select the type of output that will be given by \cmd{lsroll}.

\begin{description}

\item {\bf -count}\verb" "

Only a count of matching keyrecs in the \struct{rollrec} file is given.

\item {\bf -headers}\verb" "

Display explanatory column headers.

\item {\bf -terse}\verb" "

Terse output is given.  Only the record name and any other fields specifically
selected are included in the output.

\item {\bf -long}\verb" "

Long output is given.  All record fields are included.

\item {\bf -help}\verb" "

Display a usage message.

\end{description}

{\bf SEE ALSO}

blinkenlights(8),
rollchk(8),
rollinit(8),
rollerd(8)

Net::DNS::SEC::Tools::rollrec.pm(3)

Net::DNS::SEC::Tools::file-rollrec(5)

