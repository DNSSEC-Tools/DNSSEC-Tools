\clearpage

\subsubsection{val\_getaddrinfo()}

{\bf NAME}

\func{val\_getaddrinfo()}, \func{val\_freeaddrinfo()} -
get DNSSEC-validated network address and service translation

{\bf SYNOPSIS}

\begin{verbatim}
    #include <validator.h>

    int val_getaddrinfo(const struct val_context *ctx,
                        const char *nodename,
                        const char *servname,
                        const struct addrinfo *hints,
                        struct val_addrinfo **res,
                        val_status_t * val_status);

    void val_freeaddrinfo(struct val_addrinfo *ainfo);

    int val_getnameinfo(val_context_t * ctx,
                        const struct sockaddr *sa,
                        socklen_t salen,
                        char *host,
                        size_t hostlen,
                        char *serv,
                        size_t servlen,
                        int flags, 
                        val_status_t * val_status);

\end{verbatim}

{\bf DESCRIPTION}

\func{val\_getaddrinfo()} and \func{val\_getnameinfo} perform DNSSEC
validation of DNS queries.  They are intended to be DNSSEC-aware replacements
for \func{getaddrinfo(3)} and \func{getnameinfo(3)}.

\func{val\_getaddrinfo()} returns a network address value of type
\struct{val\_addrinfo} structure in the \var{res} parameter.
\var{val\_getnameinfo} is used to convert a \struct{sockaddr} structure
to a pair of host name and service strings.

\begin{verbatim}
    struct val_addrinfo
    {
            int ai_flags;
            int ai_family;
            int ai_socktype;
            int ai_protocol;
            size_t ai_addrlen;
            struct sockaddr *ai_addr;
            char * ai_canonname;
            struct val_addrinfo *ai_next;
            val_status_t  ai_val_status;
    };
\end{verbatim}

The \struct{val\_addrinfo} structure is similar to the \struct{addrinfo}
structure (described in \func{getaddrinfo(3)}).  \struct{val\_addrinfo} has an
additional field \var{ai\_val\_status} that represents the validation status
for that particular record.  \var{val\_status} gives the combined validation
status value for all answers returned by the each of the functions.
\func{val\_istrusted()} and \func{val\_isvalidated()} can be used to determine
the trustworthiness of data and \func{p\_val\_status()} can be used to display
the status value to the user in ASCII format (See \lib{libval(3)} more for
information).

The \var{ctx} parameter specifies the validation context, which can be set to
NULL for default values (see \lib{libval(3)} and \path{dnsval.conf} for more
details on validation contexts and validation policy).

The \var{nodename}, \var{servname}, and \var{hints} parameters have similar
syntax and semantics as the corresponding parameters for the original
\func{getaddrinfo()} function.  The \var{res} parameter is similar to the
\var{res} parameter for \func{getaddrinfo()}, except that it is of type struct
\struct{val\_addrinfo} instead of struct \struct{addrinfo}.  Please see the
manual page for \func{getaddrinfo(3)} for more details about these parameters.

\func{val\_freeaddrinfo()} frees the memory allocated for a
\struct{val\_addrinfo} linked list.

{\bf RETURN VALUES}

The \func{val\_getaddrinfo()} function returns 0 on success and a non-zero
error code on failure.  \var{*res} will point to a dynamically allocated
linked list of \struct{val\_addrinfo} structures on success and will be NULL
if no answer was available.

The \var{val\_status} parameter gives an indication for trustworthiness of
data. If \var{*res} is NULL, this value gives an indication of whether the
non-existence of data can be trusted or not.

{\bf EXAMPLE}

\begin{verbatim}
    #include <stdio.h>
    #include <stdlib.h>
    #include <validator.h>

    int main(int argc, char *argv[])
    {
             struct val_addrinfo *ainfo = NULL;
             int retval;

             if (argc < 2) {
                     printf("Usage: %s <hostname>\n", argv[0]);
                     exit(1);
             }

             retval = val_getaddrinfo(NULL, argv[1], NULL, NULL, &ainfo);

             if ((retval == 0) && (ainfo != NULL)) {

                     printf("Validation Status = %d [%s]\n",
                            ainfo->ai_val_status,
                            p_val_status(ainfo->ai_val_status));

                     val_freeaddrinfo(ainfo);
             }

             return 0;
    }
\end{verbatim}

{\bf SEE ALSO}

getaddrinfo(3) 

libval(3), val\_gethostbyname(3), val\_query(3)

