
\subsubsection{Generate Two ZSKs and One KSK}

Follow the steps in Section~\ref{genzsk} for generating the ZSKs and steps in
Section~\ref{genksk} for generating the KSK.

Designate one of the two ZSKs as the Current (C) zone signing key and use it
to sign the zone data; designate the other as the Published (P) key, which
is for future use following a ZSK roll-over.  Set the status of each of these
keys in the column marked S.

There is only one KSK and this is used to sign the zone apex keyset; mark its
status as Current (C).  The Published ZSK should be kept more
safely\footnote{It would be a good idea for an operator to apply increased
protection mechanisms (physical, file permissions and ownership, network,
etc.) to the Published ZSK than are used for the Current ZSK.}
than the Current ZSK.  The idea is that the Published ZSK can be easily rolled
in even if the Current ZSK is compromised (the Current ZSK may have to be
kept on-line in some circumstances).

If the KSK has been stored in a more secure location (off-line, more highly
protected directory, etc.) then it might be a good idea to store the
Published ZSK in the same secure location.

\begin{center}
\begin{tabular}{|c|c|c|c|c|c|c|c|c|c|}
\hline
{\bf Zone} &
\multicolumn{4}{c|}{{\bf ZSK}} &
\multicolumn{4}{c|}{{\bf KSK}} &
{\bf Exp} \\
\cline{2-9}

 & Tag & Size & Creat & S & Tag & Size & Creat & S & \\
\hline

\underline{zone.name}	&
\underline{zsktag-cur}	&
512			&
\underline{date}	&
C			&
\underline{ksktag}	&
1024			&
\underline{date}	&
C			& \\

\cline{2-9}

			&
\underline{zsktag-pub}	&
512			&
\underline{date}	&
P			&
& & & & \\

\hline
\end{tabular}
\end{center}
