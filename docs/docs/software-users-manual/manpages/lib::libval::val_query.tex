\clearpage

\subsection{\bf val\_query()}

{\bf NAME}

\func{val\_query()} - DNSSEC-validated resolution of DNS queries

{\bf SYNOPSIS}

\begin{verbatim}
    #include <validator.h>

    int val_query(const val_context_t *ctx,
                  const char *dname,
                  const u_int16_t class,
                  const u_int16_t type,
                  const u_int8_t flags,
                  struct val_response **resp);

    int val_free_response(struct val_response *resp);

    int val_res_query(const val_context_t *ctx,
                      const char *dname,
                      int class,
                      int type,
                      u_char *answer,
                      int anslen,
                      val_status_t *val_status);
\end{verbatim}

{\bf DESCRIPTION}

The \func{val\_query()} and \func{val\_res\_query()} functions perform DNSSEC
validation of DNS queries.  They are DNSSEC-aware substitutes for
\func{res\_query(3)}.

The \var{ctx} parameter is the validator context and can be set to NULL for
default settings.  (More information about this field can be found in
\lib{libval(3)}.)

The \var{dname} parameter specifies the domain name, \var{class} specifies the
DNS class and \var{type} specifies the DNS type.

The \func{val\_query()} function returns results in the \var{resp} linked-list
which encapsulates the results into the following structure:

\begin{verbatim}
    struct val_response
    {
        unsigned char       *vr_response;
        int                  vr_length;
        val_status_t         vr_val_status;
        struct val_response *vr_next;
    };
\end{verbatim}

The \var{vr\_response} and \var{vr\_length} fields are functionally similar to
the \var{answer} and \var{anslen} parameters in \func{res\_query(3)}.  Memory
for the \var{resp} linked-list is internally allocated and must be released
after a successful invocation of the function using the
\func{val\_free\_response()} function.  Each element in the \var{resp} linked
list will contain an answer corresponding to a single RRSet in the DNS reply.

If DNSSEC validation succeeds for a given RRSet, a value of
\const{VAL\_SUCCESS} is returned in the \var{vr\_val\_status} field of the
\var{val\_response} structure for that RRSet.  Other values are returned in
case of errors.  (See \path{val\_errors.h} for a listing of possible error
codes.)

\func{p\_val\_status()} returns a brief string description of the error
code.  \func{val\_istrusted()} determines if the error code indicates that the
response can be trusted.  (See \lib{libval(3)} for further information.)

The \var{flags} parameter controls the scope of validation and name
resolution, and the output format.  At present only the
\const{VAL\_QUERY\_MERGE\_RRSETS} flag is defined.  This flag is provided for
legacy applications that already use \func{res\_query(3)} and want to
transition to \func{val\_query()} with minimal change.  When this flag is
specified, all RRsets in the answer are merged into a single response and
returned in the first element of the \var{resp} array.  The response field of
this element will have a format similar to the answer returned by
\func{res\_query(3)}.  The validation status will be \const{VAL\_SUCCESS} only
if all the individual RRsets have been successfully validated.  Otherwise, the
validation status will be one of the other error codes.  If a value other than
\const{VAL\_SUCCESS} is returned and if multiple RRsets are present in the
answer, it is not possible to know which RRset resulted in the error status,
if this flag is used.

\func{val\_res\_query()} is provided as a closer substitute for
\func{res\_query(3)}.  It calls \func{val\_query()} internally with the
\const{VAL\_QUERY\_MERGE\_RRSETS} flag and returns the answers in the field
answer with length of \var{anslen}.  The validation status is returned in the
field \var{val\_status}.

{\bf RETURN VALUES}

The \func{val\_query()} function returns 0 on success.  Errors returned
by \func{resolve\_n\_check()} may be returned, as it is called internally
by \func{val\_query()}.

The \func{val\_res\_query()} function returns the number of bytes received on
success and -1 on failure.

{\bf EXAMPLES}

\begin{verbatim}
    #include <stdio.h>
    #include <stdlib.h>
    #include <strings.h>
    #include <arpa/nameser.h>
    #include <validator.h>

    #define BUFLEN 8096
    #define RESPCOUNT 3

    int main(int argc, char *argv[])
    {
             int retval;
                 int i;
             int class = ns_c_in;
             int type = ns_t_a;
             struct val_response *resp, *iter;

             if (argc < 2) {
                     printf("Usage: %s <domain-name>\n", argv[0]);
                     exit(1);
             }

             retval = val_query(NULL, argv[1], class, type, 0, &resp);

             if (retval == 0) {
                     for (iter=resp; iter; iter=iter->vr_next) {
                             printf("Validation Status = %d [%s]\n",
                                        iter->vr_val_status,
                                        p_val_status(iter->vr_val_status));
                     }
             }

             free_val_response(resp);

             return 0;
    }
\end{verbatim}

{\bf SEE ALSO}

\func{res\_query(3)}

\func{get\_context(3)}, \func{val\_getaddrinfo(3)},
\func{val\_gethostbyname(3)}

\lib{libval(3)}
