\clearpage

\subsubsection{lskrf}

{\bf NAME}

\cmd{lskrf} - List the \struct{keyrec}s in a DNSSEC-Tools \struct{keyrec} file

{\bf SYNOPSIS}

\begin{verbatim}
  lskrf [options] <keyrec-files>
\end{verbatim}

{\bf DESCRIPTION}

\cmd{lskrf} lists the contents of the specified \struct{keyrec} files.  All
\struct{keyrec} files are loaded before the output is displayed.  If any
\struct{keyrec}s have duplicated names, whether within one file or across
multiple files, the later \struct{keyrec} will be the one whose data are
displayed.

\cmd{lskrf} has three base output formats.  In ascending levels of detail,
these formats are terse output, default format, and long format.  Terse output
is given when the {\it -terse} option is specified; long output is given when
the {\it -long} option is specified.

The output displayed for each record in a \struct{keyrec} file depends on the
selected records, the selected attributes, and the selected output format.
Each option in these option groups is described in detail in the OPTIONS
section; the three basic output formats are described in the OUTPUT FORMATS
section.

{\bf OUTPUT FORMATS}

\struct{keyrec} files hold three types of \struct{keyrec} records:  zone
records, signing set records, and key records.  Each type of \struct{keyrec}
record contains \struct{keyrec} fields related to that type.  For example,
zone \struct{keyrec} records contain data about all the keys associated with a
particular zone; key \struct{keyrec} records contain key lengths and
algorithms for each particular key.  The data to be printed must be specified
by selecting some combination of the {\it -zone}, {\it -set}, {\it -key}, and
{\it -all} options.  There are also options for specifying specific types of
keys to be printed.

The three base output formats are the default format, the terse format, and
the long format.  The {\it -terse} option indicates that a minimal amount of
output is desired; the {\it -long} option indicates that a great deal of
output is desired.  The record-selection and attribute-selection options may
be used in conjunction with {\it -terse} to display exactly the set of
\struct{keyrec} fields needed.  The default output format is a middle ground
between terse and long output and is that used when neither {\it -terse} nor
$<$-long$>$ is given.

\eject

{\bf Zone \struct{keyrec} Output}

The table below shows the zone \struct{keyrec} fields displayed for each
output format.

\begin{table}[ht]
\begin{center}
\begin{tabular}{|l|c|c|c|}
\hline
{\bf Keyrec Field} & {\bf Default} & {\bf Terse} & {\bf Long} \\
\hline
keyrec type       & yes & no  & yes \\
zone name         & yes & yes & yes \\
zone file         & yes & no  & yes \\
signed zonefile   & yes & no  & yes \\
signing date      & yes & no  & yes \\
expiration date   & no  & no  & yes \\
archive directory & no  & no  & yes \\
KSK count         & no  & no  & yes \\
KSK directory     & no  & no  & yes \\
current KSK set   & no  & no  & yes \\
published KSK set & no  & no  & yes \\
ZSK count         & no  & no  & yes \\
ZSK directory     & no  & no  & yes \\
current ZSK set   & no  & no  & yes \\
published ZSK set & no  & no  & yes \\
new ZSK set       & no  & no  & yes \\
\hline
\end{tabular}
\end{center}
\caption{\cmd{lskrf} Zone Output Formats}
\end{table}

{\bf Set \struct{keyrec} Output}

The table below shows the signing set \struct{keyrec} fields displayed for each
output format.

\begin{table}[ht]
\begin{center}
\begin{tabular}{|l|c|c|c|}
\hline
{\bf Keyrec Field} & {\bf Default} & {\bf Terse} & {\bf Long} \\
\hline
keyrec type            & no  & no  & yes \\
set name               & yes & yes & yes \\
zone name              & yes & no  & yes \\
keys                   & yes & no  & yes \\
last modification date & no  & no  & yes \\
\hline
\end{tabular}
\end{center}
\caption{\cmd{lskrf} Signing Set Output Formats}
\end{table}

\eject

{\bf Key \struct{keyrec} Output}

The table below shows the key \struct{keyrec} fields displayed for each
output format.

\begin{table}[ht]
\begin{center}
\begin{tabular}{|l|c|c|c|}
\hline
{\bf Keyrec Field} & {\bf Default} & {\bf Terse} & {\bf Long} \\
\hline
keyrec type             &  yes & no  & yes \\
key name                &  yes & yes & yes \\
algorithm               &  no  & no  & yes \\
end date                &  no  & no  & yes \\
generation date         &  yes & no  & yes \\
key length              &  no  & no  & yes \\
key life                &  no  & no  & yes \\
key path                &  no  & no  & yes \\
keys                    &  no  & no  & yes \\
random number generator &  no  & no  & yes \\
zone name               &  yes & no  & yes \\
\hline
\end{tabular}
\end{center}
\caption{\cmd{lskrf} Key Output Formats}
\end{table}

{\bf OPTIONS}

\cmd{lskrf} takes three types of options:  record-selection options,
record-attribute options, and output-style options.  These option
sets are detailed below.

Record-selection options are required options; at least one record-selection
option {\bf must} be selected.  Record-attribute options and output-style
options are optional options; any number of these option {\it may} be
selected.

{\bf Record-Selection Options}

These options select the types of \struct{keyrec} that will be displayed.

\begin{description}

\item {\bf -all}\verb" "

This option displays all the records in a \struct{keyrec} file.

\item {\bf -zones}\verb" "

This option displays the zones in a \struct{keyrec} file.

\item {\bf -sets}\verb" "

This option displays the signing sets in a \struct{keyrec} file.

\item {\bf -keys}\verb" "

This option displays the keys in a \struct{keyrec} file.

The key data are sorted by key type in the following order:  Current KSKs,
Published KSKs, Current ZSKs, Published ZSKs, New ZSKs, Obsolete KSKs, and
Obsolete ZSKs.

\item {\bf -ksk}\verb" "

This option displays the KSK keys in a \struct{keyrec} file.

\item {\bf -kcur}\verb" "

This option displays the Current KSK keys in a \struct{keyrec} file.

\item {\bf -kpub}\verb" "

This option displays the Published KSK keys in a \struct{keyrec} file.

\item {\bf -kobs}\verb" "

This option displays the obsolete KSK keys in a \struct{keyrec} file.  This
option must be give if obsolete KSK keys are to be displayed.

\item {\bf -zsk}\verb" "

This option displays the ZSK keys in a \struct{keyrec} file.  It does not
include obsolete ZSK keys; the {\it -obs} option must be specified to display
obsolete keys.

\item {\bf -cur}\verb" "

This option displays the Current ZSK keys in a \struct{keyrec} file.

\item {\bf -new}\verb" "

This option displays the New ZSK keys in a \struct{keyrec} file.

\item {\bf -pub}\verb" "

This option displays the Published ZSK keys in a \struct{keyrec} file.

\item {\bf -zobs}\verb" "

This option displays the obsolete ZSK keys in a \struct{keyrec} file.  This
option must be give if obsolete ZSK keys are to be displayed.

\item {\bf -obs}\verb" "

This option displays the obsolete KSK and ZSK keys in a \struct{keyrec} file.
This option is a shorthand method specifying the {\it -kobs} and {\it -zobs}
options.

\end{description}

{\bf Record-Attribute Options}

These options select subsets of the \struct{keyrec}s chosen by the
record-selection options. 

\begin{description}

\item {\bf -valid}\verb" "

This option displays the valid zones in a \struct{keyrec} file.
It implies the {\it -zones} option.

\item {\bf -expired}\verb" "

This option displays the expired zones in a \struct{keyrec} file.
It implies the {\it -zones} option.

\item {\bf -ref}\verb" "

This option displays the referenced signing set \struct{keyrec}s and the
referenced key \struct{keyrec}s in a \struct{keyrec} file, depending upon
other selected options.

Referenced state depends on the following:

\begin{itemize}

\item Signing sets are considered to be referenced if they are listed in a
zone keyrec.

\item KSKs are considered to be referenced if they are listed in a signing set
keyrec that is listed in a zone keyrec.

\item ZSKs are considered to be referenced if they are listed in a signing set
keyrec that is listed in a zone keyrec.

\end{itemize}

This option may be used with either the {\it -sets} or {\it -keys} options.
If it isn't used with any record-selection options, then it is assumed that
both {\it -sets} and {\it -keys} have been specified.

\item {\bf -unref}\verb" "

This option displays the unreferenced signing set \struct{keyrec}s or the
unreferenced key \struct{keyrec}s in a \struct{keyrec} file, depending upon
other selected options.

Unreferenced state depends on the following:

\begin{itemize}

\item Signing sets are considered to be unreferenced if they are not listed in
a zone keyrec.

\item KSKs are considered to be unreferenced if they are not listed in a
signing set keyrec that is listed in a zone keyrec.

\item ZSKs are considered to be unreferenced if they are not listed in a
signing set keyrec that is listed in a zone keyrec.

\item Obsolete ZSKs are checked, whether or not the -obs flag was specified.

\end{itemize}

This option may be used with either the {\it -sets} or {\it -keys} options.
If it isn't used with any record-selection options, then it is assumed that
both {\it -sets} and {\it -keys} have been specified.

\end{description}

{\bf Zone-Attribute Options}

These options allow specific zone fields to be included in the output.  If
combined with the {\it -terse} option, only those fields specifically desired
will be printed.  These options must be used with the {\it -zone} option.

\begin{description}

\item {\bf -z-archdir}\verb" "

Display the zone's archive directory.  If an archive directory is not
explicitly set for the zone, the default directory will be listed.

\item {\bf -z-dates}\verb" "

Display the zone's time-stamps.  These are the signing date and the
expiration date.

\item {\bf -z-dirs}\verb" "

Display the zone's directories.  These directories are the KSK directory,
the ZSK directory, and the key archive directory.

\item {\bf -z-expdate}\verb" "

Display the zone's expiration date.

\item {\bf -z-ksk}\verb" "

Display the zone's KSK data.  This is the equivalent of specifying the {\it
-z-kskcount}, {\it -z-kskcur}, {\it -z-kskdir}, and {\it -z-kskpub} options.

\item {\bf -z-kskcount}\verb" "

Display the zone's KSK count.

\item {\bf -z-kskcur}\verb" "

Display the zone's Current KSK signing set.
If this is not defined, then ``$<$unset$>$'' will be given.

\item {\bf -z-kskdir}\verb" "

Display the zone's KSK directory.
If this is not defined, then ``.'' will be given.

\item {\bf -z-kskpub}\verb" "

Display the zone's Published KSK signing set.
If this is not defined, then ``$<$unset$>$'' will be given.

\item {\bf -z-sets}\verb" "

Display the zone's signing sets.  This is the equivalent of specifying the
{\it -z-kskcur}, {\it -z-kskpub}, {\it -z-zskcur}, {\it -z-zsknew}, and {\it
-z-zskpub} options.

\item {\bf -z-signdate}\verb" "

Display the zone's signing date.

\item {\bf -z-signfile}\verb" "

Display the zone's signed zonefile.

\item {\bf -z-zonefile}\verb" "

Display the zone's zonefile.

\item {\bf -z-zsk}\verb" "

Display the zone's ZSK data.  This is the equivalent of specifying the
{\it -z-zskcount}, {\it -z-zskcur}, {\it -z-zskdir}, {\it -z-zsknew},
and {\it -z-zskpub} options.

\item {\bf -z-zskcount}\verb" "

Display the zone's ZSK count.

\item {\bf -z-zskcur}\verb" "

Display the zone's Current ZSK signing set.
If this is not defined, then ``$<$unset$>$'' will be given.

\item {\bf -z-zskdir}\verb" "

Display the zone's ZSK directory.
If this is not defined, then ``.'' will be given.

\item {\bf -z-zsknew}\verb" "

Display the zone's New ZSK signing set.
If this is not defined, then ``$<$unset$>$'' will be given.

\item {\bf -z-zskpub}\verb" "

Display the zone's Published ZSK signing set.
If this is not defined, then ``$<$unset$>$'' will be given.

\end{description}

{\bf Set-Attribute Options}

These options allow specific set fields to be included in the output.  If
combined with the {\it -terse} option, only those fields specifically desired
will be printed.  These options must be used with the {\it -set} option.

\begin{description}

\item {\bf -s-keys}\verb" "

Display the set's keys.

\item {\bf -s-lastmod}\verb" "

Display the set's date of last modification.

\item {\bf -s-zone}\verb" "

Display the set's zone name.

\end{description}

{\bf Key-Attribute Options}

These options allow specific key fields to be included in the output.  If
combined with the {\it -terse} option, only those fields specifically desired
will be printed.  These options must be used with the {\it -key} option.

\begin{description}

\item {\bf -k-algorithm}\verb" "

Display the key's encryption algorithm.

\item {\bf -k-enddate}\verb" "

Display the key's end-date, calculated by adding the key's lifespan to its
signing date.

\item {\bf -k-length}\verb" "

Display the key's length.

\item {\bf -k-lifespan}\verb" "

Display the key's lifespan (in seconds.) This lifespan is {\bf only} related
to the time between key roll-over.  There is no other lifespan associated with
a key.

\item {\bf -k-path}\verb" "

Display the key's path.

\item {\bf -k-random}\verb" "

Display the key's random number generator.

\item {\bf -k-signdate}\verb" "

Display the key's signing date.

\item {\bf -k-zone}\verb" "

Display the key's zonefile.

\end{description}

{\bf Output-Format Options}

These options define how the \struct{keyrec} information will be displayed.

Without any of these options, the zone name, zone file, zone-signing date,
and a label will be displayed for zones.  For types, the key name, the key's
zone, the key's generation date, and a label will be displayed if these
options aren't given.

\begin{description}

\item {\bf -count}\verb" "

The count of matching records will be displayed, but the matching records
will not be.

\item {\bf -nodate}\verb" "

The key's generation date will not be printed if this flag is given.

\item {\bf -headers}\verb" "

Display explanatory column headers.  If this flag is given, then entry labels
will not be printed unless explicitly requested by use of the {\it -label}
option.

\item {\bf -label}\verb" "

A label for the \struct{keyrec}'s type will be given.

\item {\bf -long}\verb" "

The long form of output will be given.  See the OUTPUT FORMATS section for
details on data printed for each type of \struct{keyrec} record.

Long zone output can get {\it very} wide, depending on the data.

\item {\bf -terse}\verb" "

This options displays only the name of the zones or keys selected by other
options.

\item {\bf -help}\verb" "

Display a usage message and exit.

\item {\bf -h-zones}\verb" "

Display the zone-attribute options and exit.

\item {\bf -h-sets}\verb" "

Display the set-attribute options and exit.

\item {\bf -h-keys}\verb" "

Display the key-attribute options and exit.

\end{description}

{\bf SEE ALSO}

zonesigner(8)

Net::DNS::SEC::Tools::keyrec.pm(3)

file-keyrec(5)

