\clearpage

\subsubsection{rolllog.pm}

{\bf NAME}

\perlmod{Net::DNS::SEC::Tools::rolllog} - DNSSEC-Tools rollover logging
interfaces.

{\bf SYNOPSIS}

\begin{verbatim}
  use Net::DNS::SEC::Tools::rolllog;

  @levels = rolllog_levels();

  $curlevel = rolllog_level();
  $oldlevel = rolllog_level("info");
  $oldlevel = rolllog_level(LOG_ERR,1);

  $curlogfile = rolllog_file();
  $oldlogfile = rolllog_file("-");
  $oldlogfile = rolllog_file("/var/log/roll.log",1);

  $loglevelstr = rolllog_str(8)
  $loglevelstr = rolllog_str("info")

  $ret = rolllog_num("info");

  rolllog_log(LOG_INFO,"example.com","zone is valid");
\end{verbatim}

{\bf DESCRIPTION}

The \perlmod{Net::DNS::SEC::Tools::rolllog} module provides logging interfaces
for the rollover programs.  The logging interfaces allow log messages to be
recorded.  \cmd{rollerd} must be running, as it is responsible for updating
the log file.

Each log message is assigned a particular logging level.  The valid logging
levels are:

\begin{table}[h]
\begin{center}
\begin{tabular}{|l|c|l|}
\hline
{\bf Textual Level} & {\bf Numeric Level} & {\bf Meaning} \\
\hline
{\bf tmi}    & 1 & The highest level -- all log messages are saved.	\\
{\bf expire} & 3 & A verbose countdown of zone expiration is given.	\\
{\bf info}   & 4 & Many informational messages are recorded.		\\
{\bf phase}  & 6 & Each zone's current rollover phase is given.		\\
{\bf err}    & 8 & Errors are recorded.					\\
{\bf fatal}  & 9 & Fatal errors are saved.				\\
\hline
\end{tabular}
\end{center}
\caption{Logging Levels}
\end{table}

The logging levels include all numerically higher levels.  For example, if
the logging level is set to {\bf phase}, then {\bf err} and {\bf fatal}
messages will also be recorded.

{\bf LOGGING INTERFACES}

\begin{description}

\item \func{rolllog\_levels()}\verb" "

This routine returns an array holding the text forms of the user-settable
logging levels.  The levels are returned in order, from most verbose to least.

\item \func{rolllog\_level(newlevel,useflag)}\verb" "

This routine sets and retrieves the logging level for \cmd{rollerd}.
The \var{newlevel} argument specifies the new logging level to be set.
\var{newlevel} may be given in either text or numeric form.

The \var{useflag} argument is a boolean that indicates whether or not to give
a descriptive message and exit if an invalid logging level is given.  If
\var{useflag} is true, the message is given and the process exits; if false,
-1 is returned.

If given with no arguments, the current logging level is returned.  In fact,
the current level is always returned unless an error is found.  -1 is returned
on error.

\item \func{rolllog\_file(newfile,useflag)}\verb" "

This routine sets and retrieves the log file for \cmd{rollerd}.  The
\var{newfile} argument specifies the new log file to be set.  If \var{newfile}
exists, it must be a regular file.

The \var{useflag} argument is a boolean that indicates whether or not to give
a descriptive message if an invalid logging level is given.  If \var{useflag}
is true, the message is given and the process exits; if false, no message is
given.  For any error condition, an empty string is returned.

\item \func{rolllog\_num(loglevel)}\verb" "

This routine translates a text log level (given in \var{loglevel}) into the
associated numeric log level.  The numeric log level is returned to the
caller.

If \var{loglevel} is an invalid log level, -1 is returned.

\item \func{rolllog\_str(loglevel)}\verb" "

This routine translates a log level (given in \var{loglevel}) into the
associated text log level.  The text log level is returned to the caller.

If \var{loglevel} is a text string, it is checked to ensure it is a valid log
level.  Case is irrelevant when checking \var{loglevel}.

If \var{loglevel} is numeric, it is must be in the valid range of log levels.
\var{undef} is returned if \var{loglevel} is invalid.

\item \func{rolllog\_log(level,group,message)}\verb" "

The \func{rolllog\_log()} interface writes a message to the log file.  Log
messages have this format:

\begin{verbatim}
    timestamp: group: message
\end{verbatim}

The \var{level} argument is the message's logging level.  It will only be
written to the log file if the current log level is numerically equal to or
less than \var{level}.

\var{group} allows messages to be associated together.  It is currently used
by \cmd{rollerd} to group messages by the zone to which the message applies.

The \var{message} argument is the log message itself.  Trailing newlines are
removed.

\end{description}

{\bf SEE ALSO}

rollctl(1)

rollerd(8)

Net::DNS::SEC::Tools::rollmgr.pm(3)

