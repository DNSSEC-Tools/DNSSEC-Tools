\documentclass[12pt]{article}

\pagestyle{headings}
\markboth{DNSSEC-Tools Software User Manual (vers. 4) -- Manual Pages}{DNSSEC-Tools Software User Manual (vers. 4) -- Manual Pages}

\newenvironment{packed}{\begin{list}{$\bullet$}{\parsep0in\itemsep0in}}{\end{list}}


%
% Macros for specific types of entities.
%
\newcommand{\cmd}[1]{{\bf #1}}
\newcommand{\const}[1]{{\bf #1}}
\newcommand{\func}[1]{{\em #1}}
\newcommand{\lib}[1]{{\em #1}}
\newcommand{\perlmod}[1]{{\bf #1}}
\newcommand{\path}[1]{{\bf #1}}
\newcommand{\struct}[1]{{\em #1}}
\newcommand{\sys}[1]{{\em #1}}
\newcommand{\url}[1]{{\bf #1}}
\newcommand{\var}[1]{{\em #1}}
\newcommand{\xqt}[1]{{\bf #1}}

% \textheight=8.50in
\textheight=9.50in
\voffset=-.10in

\setlength{\textwidth}{6.5in}
\setlength{\oddsidemargin}{0in}

\setcounter{secnumdepth}{4}
\setcounter{tocdepth}{3}

\parindent0em
\parskip0.65em

\hyphenation{DNS-KEY}
\hyphenation{RR-SIG}
\hyphenation{RSA-SHA}
\hyphenation{dns-sec DNS-SEC}
\hyphenation{dns-sec--tools}
\hyphenation{get-host-by-name}
\hyphenation{key-rec}
\hyphenation{life-span}
\hyphenation{log-watch}
\hyphenation{set-time}
\hyphenation{tool-options}
\hyphenation{val-i-da-tor}


%%%%%%%%%%%%%%%%%%%%%%%%%%%%%%%%%%%%%%%%%%%%%%%%%%%%%%%%%%%%%%%%%%%%%%%%%%%%%%

\begin{document}
\markboth{DNSSEC-Tools Software User Manual (vers. 4) -- Manual Pages}{DNSSEC-Tools Software User Manual (vers. 4) -- Manual Pages}

\begin{titlepage}

\vspace{.5in}

\begin{center}
\LARGE{\bf
SOFTWARE USER MANUAL (SUM):
TRAINING, PROCEDURAL, AND
DEVELOPMENT DOCUMENTATION
}
\vspace{1in}

\Large{
DNSSEC-Tools Software User Manual

Manual Pages
}
\vspace{0.5in}

\vspace{3in}

14 September 2007
\end{center}

\vspace{.5in}

\begin{center}
Version 4
\end{center}

\end{titlepage}
\markboth{DNSSEC-Tools Software User Manual (vers. 4) -- Manual Pages}{DNSSEC-Tools Software User Manual (vers. 4) -- Manual Pages}


%%%%%%%%%%%%%%%%%%%%%%%%%%%%%%%%%%%%%%%%%%%%%%%%%%%%%%%%%%%%%%%%%%%%%%%%%%%%%%

\clearpage

\begin{center}
{\Large
{\bf DNSSEC-Tools\\
Is your domain secure?}}
\end{center}
\markboth{DNSSEC-Tools Software User Manual (vers. 4) -- Manual Pages}{DNSSEC-Tools Software User Manual (vers. 4) -- Manual Pages}
\tableofcontents
\markboth{DNSSEC-Tools Software User Manual (vers. 4) -- Manual Pages}{DNSSEC-Tools Software User Manual (vers. 4) -- Manual Pages}

%%%%%%%%%%%%%%%%%%%%%%%%%%%%%%%%%%%%%%%%%%%%%%%%%%%%%%%%%%%%%%%%%%%%%%%%%%%%%%

\clearpage

\markboth{DNSSEC-Tools Software User Manual (vers. 4) -- Manual Pages}{DNSSEC-Tools Software User Manual (vers. 4) -- Manual Pages}
\section{About This Document}
\markboth{DNSSEC-Tools Software User Manual (vers. 4) -- Manual Pages}{DNSSEC-Tools Software User Manual (vers. 4) -- Manual Pages}
\label{sect-about}

The goal of the DNSSEC-Tools project is to create a set of tools, libraries,
patches, applications, wrappers, extensions, and plugins that will help ease
the deployment and maintenance of DNSSEC-related technologies. This document contains manual
pages for the commands, libraries, Perl modules, and data files that are part
of the DNSSEC-Tools distribution.  The document organization is described
below.

\begin{description}

\item
Section~\ref{sect-about} describes the DNSSEC-Tools Software User Manual.

\item
Section~\ref{sect-commands} describes the DNSSEC-Tools commands.  These
commands include programs to analyze and manipulate zone files, maintain the
DNSSEC-Tools environment, and to assist with zone signing and key rollover.
The commands are divided into functional groups:
DNSSEC-Tools maintenance commands (Section~\ref{ssect-cmds-maint}),
DNS zone file commands (Section~\ref{ssect-cmds-zone}),
zone-signing commands (Section~\ref{ssect-cmds-sign}), and
zone-rollover commands (Section~\ref{ssect-cmds-roll}).

\item
Section~\ref{sect-libraries} describes two libraries developed for
DNSSEC-Tools.  The \lib{libsres} library provides secure address resolution
for applications.  The \lib{libval} library provides DNSSEC Resource Record
validation.

\item
Section~\ref{sect-modules} describes a number of Perl modules that were
written to support the DNSSEC-Tools commands.  These modules may be used in
development of additional commands to work with the existing DNSSEC-Tools.

\item
Section~\ref{sect-files} describes data files used by the DNSSEC-Tools
commands, libraries, and modules.

\end{description}

For more information about this project and the tools that are being developed
and provided, please see one of the project web pages at:

\url{http://www.dnssec-tools.org}  \\
\url{http://dnssec-tools.sourceforge.net}

%%%%%%%%%%%%%%%%%%%%%%%%%%%%%%%%%%%%%%%%%%%%%%%%%%%%%%%%%%

\clearpage

\subsection{\bf Conventions}

The following typographical conventions are used in this document.

\begin{table}[hb]
\begin{tabular}{lll}
\cmd{command}		& & Command names\\
\const{constant}	& & Code constants\\
\func{call()}		& & System and function calls\\
\lib{library}		& & Library names\\
\perlmod{module}	& & Perl Modules\\
\path{path}		& & File and path names\\
\url{URL}		& & Web URLs\\
\var{variable}		& & Variables\\
\xqt{execution}		& & Simple command executions\\
\end{tabular}
\end{table}

Longer sets of command sequences are given in this format:
\begin{verbatim}
        # cd /tmp
        # ls
        # rm -fr *
\end{verbatim}
In most cases, output will not be displayed for given command sequences.

\vspace{.25in}

\subsection{\bf Comments}

Please send any comments and corrections to developers@dnssec-tools.org.

%%%%%%%%%%%%%%%%%%%%%%%%%%%%%%%%%%%%%%%%%%%%%%%%%%%%%%%%%%%%%%%%%%%%%%%%%%%%%%

\clearpage

\markboth{DNSSEC-Tools Software User Manual (vers. 4) -- Manual Pages}{DNSSEC-Tools Software User Manual (vers. 4) -- Manual Pages}
\section{DNSSEC-Tools Commands}
\markboth{DNSSEC-Tools Software User Manual (vers. 4) -- Manual Pages}{DNSSEC-Tools Software User Manual (vers. 4) -- Manual Pages}
\label{sect-commands}


A number of commands have been developed to assist in maintaining
DNSSEC-secured domains.  These commands check zone files for errors,
assist in key generation and zone signing, perform key rollover, and
provide graphic information about zones.

The DNSSEC-Tools commands are divided in four functional groups:

\begin{itemize}

\item DNSSEC-Tools Maintenance Commands

\item DNS Zone-File Commands

\item Zone-Signing Commands

\item Key-Rollover Commands

\end{itemize}

The commands in each functional group are described in the following
subsections.

\clearpage
\subsection{\bf DNSSEC-Tools Maintenance Commands}
\label{ssect-cmds-maint}

The commands in this group primarily deal with DNSSEC-Tools configuration
file.  There is also a tool for time conversions, which may assist in
determining values for some fields for a zone.  The maintenance commands are:

\begin{table}[ht]
\begin{center}
\begin{tabular}{ll}
\cmd{dtinitconf} & create a new DNSSEC-Tools configuration file		\\
\cmd{dtdefs}	 & print the DNSSEC-Tools configuration values		\\
\cmd{dtconfchk}	 & checks a DNSSEC-Tools configuration file for errors	\\
\cmd{timetrans}	 & converts time units					\\
\end{tabular} 
\end{center}
\end{table}

\clearpage

\subsubsection{dtinitconf}

{\bf NAME}

\cmd{dtinitconf} - Creates a DNSSEC-Tools configuration file

{\bf SYNOPSIS}

\begin{verbatim}
  dtinitconf [options]
\end{verbatim}

{\bf DESCRIPTION}

The \cmd{dtinitconf} program initializes the DNSSEC-Tools configuration file.
By default, the actual configuration file will be created, though the created
file can be specified by the user.  Existing files, whether the default or one
specified by the user, will not be overwritten unless specifically directed
by the user.

Each configuration field can be individually specified on the command line.
The user will also be prompted for the fields, with default values taken from
the DNSSEC-Tools \perlmod{defaults.pm} module.  If the {\it -noprompt} option
is given, then a default configuration file (modulo command-line arguments)
will be created.

Configuration entries are created for several BIND programs.  Several
locations on the system are searched to find the locations of these programs. 
First, the directories in the path environment variable are checked; the
names of any directories that contain the BIND programs are saved.  Next,
several common locations for BIND programs are checked; again, the names of
directories that contain the BIND programs are saved.  After collecting these
directories, the user is presented with this list and may choose to use
whichever set is desired.  If no directories are found that contain the BIND
programs, the user is prompted for the proper location.

If the configuration file's parent directory does not exist, then an attempt
is made to create the directory.  The new directory's ownership will be set
to {\it root} for the owner and {\it dnssec} for the group, assuming the
{\it dnssec} group exists.

{\bf OPTIONS}

\cmd{dtinitconf} takes options that control the contents of the newly generated
DNSSEC-Tools configuration file.  Each configuration file entry has a
corresponding command-line option.  The options, described below, are ordered
in logical groups.

{\bf Key-related Options}

These options deal with different aspects of creating and managing
encryption keys.

\begin{description}

\item {\bf -algorithm algorithm}\verb" "

Selects the cryptographic algorithm. The value of algorithm must be one that
is recognized by \cmd{dnssec-keygen}.

\item {\bf -ksklength keylen}\verb" "

The default KSK key length to be passed to \cmd{dnssec-keygen}.

\item {\bf -ksklife lifespan}\verb" "

The default length of time between KSK roll-overs.  This is measured in   
seconds.

This value is {\bf only} used for key roll-over.  Keys do not have a life-time
in any other sense.

\item {\bf -zskcount ZSK-count}\verb" "

The default number of ZSK keys that will be created for a zone.

\item {\bf -zsklength keylen}\verb" "

The default ZSK key length to be passed to \cmd{dnssec-keygen}.

\item {\bf -zsklife lifespan}\verb" "

The default length of time between ZSK roll-overs.  This is measured in   
seconds.

This value is {\bf only} used for key roll-over.  Keys do not have a life-time
in any other sense.

\item {\bf -random randomdev}\verb" "

The random device generator to be passed to \cmd{dnssec-keygen}.

\end{description}

{\bf Zone-related Options}

These options deal with different aspects of zone signing.

\begin{description}

\item {\bf -endtime endtime}\verb" "

The zone default expiration time to be passed to \cmd{dnssec-signzone}.

\end{description}

{\bf DNSSEC-Tools Options}

These options deal specifically with functionality provided by DNSSEC-Tools.

\begin{description}

\item {\bf -admin email-address}\verb" "

{\bf admin} is the email address of the DNSSEC-Tools administrator.  This is the
default address used by the \func{dt\_adminmail()} routine.

\item {\bf -archivedir directory}\verb" "

{\bf directory} is the archived-key directory.  Old encryption keys are moved to
this directory, but only if they are to be saved and not deleted.

\item {\bf -binddir directory}\verb" "

{\bf directory} is the directory holding the BIND programs.

\item {\bf -entropy\_msg}\verb" "

A flag indicating that \cmd{zonesigner} should display a message about entropy
generation.  This is primarily dependent on the implementation of a system's
random number generation.

\item {\bf -noentropy\_msg}\verb" "

A flag indicating that \cmd{zonesigner} should not display a message about
entropy generation.  This is primarily dependent on the implementation of
a system's random number generation.

\item {\bf -roll-logfile logfile}\verb" "

{\bf logfile} is the logfile for the \cmd{rollerd} daemon.

\item {\bf -roll-loglevel loglevel}\verb" "

{\bf loglevel} is the logging level for the \cmd{rollerd} daemon.

\item {\bf -roll-sleep sleep-time}\verb" "

{\bf sleep-time} is the sleep-time for the \cmd{rollerd} daemon.

\item {\bf -savekeys}\verb" "

A flag indicating that old keys should be moved to the archive directory.

\item {\bf -nosavekeys}\verb" "

A flag indicating that old keys should not be moved to the archive directory
but will instead be left in place.

\item {\bf -usegui}\verb" "

A flag indicating that the GUI for specifying command options may be used.

\item {\bf -nousegui}\verb" "

A flag indicating that the GUI for specifying command options should not be
used.

\end{description}

{\bf dtinitconf Options}

These options deal specifically with \cmd{dtinitconf}.

\begin{description}

\item {\bf -outfile conffile}\verb" "

The configuration file will be written to {\bf conffile}.  If this is
not given, then the default configuration file (as returned by
\perlmod{Net::DNS::SEC::Tools::conf::getconffile()}) will be used.

If {\bf conffile} is given as {\bf -}, then the new configuration file will be
written to the standard output.

{\bf conffile} must be writable.

\item {\bf -overwrite}\verb" "

If {\it -overwrite} is specified, existing output files may be overwritten.
Without {\it -overwrite}, if the output file is found to exist then
\cmd{dtinitconf} will give an error message and exit.

\item {\bf -noprompt}\verb" "

If {\it -noprompt} is specified, the user will not be prompted for any input.
The configuration file will be created from command-line options and
DNSSEC-Tools defaults.  Guesses will be made for the BIND paths, based on
the PATH environment variable.

{\bf WARNING}:  After using the {\it -noprompt} option, the configuration file
{\bf must} be checked to ensure that the defaults are appropriate and acceptable
for the installation.

\item {\bf -edit}\verb" "

If {\it -edit} is specified, the output file will be edited after it has been
created.  The EDITOR environment variable is consulted for the editor to
use.  If the EDITOR environment variable isn't defined, then the \cmd{vi}
editor will be used.

\item {\bf -verbose}\verb" "

Provide verbose output.

\item {\bf -help}\verb" "

Display a usage message and exit.

\end{description}

{\bf SEE ALSO}

dnssec-keygen(8),
dnssec-signzone(8),
named-checkzone(8),
rollerd(8),
zonesigner(8)

Net::DNS::SEC::Tools::conf.pm(3),
Net::DNS::SEC::Tools::defaults.pm(3),\\
Net::DNS::SEC::Tools::dnssectools.pm(3),
Net::DNS::SEC::Tools::tooloptions.pm(3),\\
QWizard.pm(3)


\clearpage

\subsubsection{\bf dtdefs}

{\bf NAME}

\cmd{dtdefs} - Displays defaults defined for DNSSEC-Tools.

{\bf SYNOPSIS}

\begin{verbatim}
    dtdefs
\end{verbatim}

{\bf DESCRIPTION}

The \cmd{dtdefs} program displays defaults defined for DNSSEC-Tools.

{\bf SEE ALSO}

\perlmod{Net::DNS::SEC::Tools::defaults.pm(3)}

\clearpage

\subsubsection{\bf dtconfchk}

{\bf NAME}

\cmd{dtconfchk} - Check a DNSSEC-Tools configuration file for sanity.

{\bf SYNOPSIS}

\begin{verbatim}
    dtconfchk [options] [config_file]
\end{verbatim}

{\bf DESCRIPTION}

\cmd{dtconfchk} checks a DNSSEC-Tools configuration file to determine if the
entries are valid.

The {\it default\_keyrec} configuration entry is not checked.  This entry
specifies the default {\it keyrec} file name and isn't necessarily expected
to exist in any particular place.

{\bf Key-related Checks}

The following key-related checks are performed:

\begin{description}

\item {\it algorithm}\verb" "

Ensure the {\it algorithm} field is valid.  The acceptable values may be found
in the \cmd{dnssec-keygen} man page.

\item {\it ksklength}\verb" "

Ensure the {\it ksklength} field is valid.  The acceptable values may be found
in the \cmd{dnssec-keygen} man page.

\item {\it ksklife}\verb" "

Ensure the {\it ksklife} field is valid.  The acceptable values may be found
in the \perlmod{defaults.pm(3)} man page.

\item {\it zskcount}\verb" "

Ensure the {\it zskcount} field is valid.  The ZSK count must be positive.

\item {\it zsklength}\verb" "

Ensure the {\it zsklength} field is valid.  The acceptable values may be found
in the \cmd{dnssec-keygen} man page.

\item {\it zsklife}\verb" "

Ensure the {\it zsklife} field is valid.  The acceptable values may be found
in the \perlmod{defaults.pm(3)} man page.

\item {\it random}\verb" "

Ensure the {\it random} field is valid.  This file must be a character
device file.

\end{description}

{\bf Zone-related Checks}

The following zone-related checks are performed:

\begin{description}

\item {\it endtime}\verb" "

Ensure the {\it endtime} field is valid.  This value is assumed to be in the
"+NNNNNN" format.  There is a lower limit of two hours.  (This is an
artificial limit under which it {\it may} not make sense to have an end-time.)

\end{description}

{\bf Path Checks}

The following path checks are performed:

\begin{description}

\item {\it checkzone}\verb" "

Ensure the {\it checkzone} field is valid.  If the filename starts with a '/',
the file must be a regular executable file.

\item {\it keygen}\verb" "

Ensure the {\it keygen} field is valid.  If the filename starts with a '/',
the file must be a regular executable file.

\item {\it signzone}\verb" "

Ensure the {\it signzone} field is valid.  If the filename starts with a '/',
the file must be a regular executable file.

\item {\it viewimage}\verb" "

Ensure the {\it viewimage} field is valid.  If the filename starts with a '/',
the file must be a regular executable file.

\end{description}

{\bf Roll-over Daemon Checks}

The following checks are performed for \cmd{rollerd} values:

\begin{description}

\item {\it roll\_logfile}\verb" "

Ensure that the log file for the \cmd{rollerd} is valid.  If the file
exists, it must be a regular file.

\item {\it roll\_loglevel}\verb" "

Ensure that the logging level for the \cmd{rollerd} is reasonable.  The
log level must be one of the following text or numeric values:

\begin{verbatim}
    tmi        1       (Overly verbose informational messages.)
    info       3       (Informational messages.)
    curphase   5       (Current state of zone.)
    err        7       (Error messages.)
    fatal      9       (Fatal errors.)
\end{verbatim}

Specifying a particular log level will causes messages of a higher numeric
value to also be displayed.

\item {\it roll\_sleeptime}\verb" "

Ensure that the \cmd{rollerd}'s sleep-time is reasonable.
\cmd{rollerd}'s sleep-time must be at least one minute.

\end{description}

{\bf Miscellaneous Checks}

The following miscellaneous checks are performed:

\begin{description}

\item {\it archivedir}\verb" "

Ensure that the {\it archivedir} directory is actually a directory.
This check is only performed if the {\it savekeys} flag is set on.

\item {\it entropy\_msg}\verb" "

Ensure that the {\it entropy\_msg} flag is either 0 or 1.

\item {\it savekeys}\verb" "

Ensure that the {\it savekeys} flag is either 0 or 1.
If this flag is set to 1, then the {\it archivedir} field will also be checked.

\item {\it usegui}\verb" "

Ensure that the {\it usegui} flag is either 0 or 1.

\end{description}

{\bf OPTIONS}

\begin{description}

\item {\it -expert}\verb" "

This option will bypass the following checks:

\begin{itemize}

\item KSK has a longer lifespan than the configuration file's default minimum
lifespan

\item KSK has a shorter lifespan than the configuration file's default maximum
lifespan

\item ZSKs have a longer lifespan than the configuration file's default minimum
lifespan

\item ZSKs have a shorter lifespan than the configuration file's default maximum
lifespan

\end{itemize}

\item {\it -quiet}\verb" "

No output will be given.
The number of errors will be used as the exit code.

\item {\it -summary}\verb" "

A final summary of success or failure will be printed.
The number of errors will be used as the exit code.

\item {\it -verbose}\verb" "

Success or failure status of each check will be given.
A {\bf +} or {\bf -} prefix will be given for each valid and invalid entry.
The number of errors will be used as the exit code.

\item {\it -help}\verb" "

Display a usage message.

\end{description}

{\bf SEE ALSO}

\cmd{dtdefs(8)},
\cmd{dtinitconf(8)},
\cmd{rollerd(8)},
\cmd{zonesigner(8)}

\perlmod{Net::DNS::SEC::Tools::conf.pm(3)},
\perlmod{Net::DNS::SEC::Tools::defaults.pm(3)}

\path{dnssec-tools.conf(5)}

\clearpage

\subsubsection{\bf timetrans}

{\bf NAME}

\cmd{timetrans} - Converts time into time.

{\bf SYNOPSIS}

\begin{verbatim}
    timetrans [units-options] [-count]
\end{verbatim}

{\bf DESCRIPTION}

{\it timetrans} converts time from one type of unit to another.  If any of the
units options are specified, then {\it timetrans} will convert those time units
into the number of seconds to which they add up.  If given the count option,
{\it timetrans} will convert that number of seconds into the appropriate number
of weeks, days, hours, minutes, and seconds.  The converted result is printed
out.  Units options cannot be specified in the same execution as the count
option, and vice versa.

{\it timetrans} is intended for use with DNSSEC-Tools, for calculating
a zone's expiration time.

{\bf OPTIONS}

{\bf Units Options}

The converted value of each unit is totaled and a single result printed.
These options may be shortened to their first letter.

\begin{description}

\item {\it -seconds seconds}\verb" "

Count of seconds to convert to seconds.

\item {\it -minutes minutes}\verb" "

Count of minutes to convert to seconds.

\item {\it -hours hours}\verb" "

Count of hours to convert to seconds.

\item {\it -days days}\verb" "

Count of days to convert to seconds.

\item {\it -weeks weeks}\verb" "

Count of weeks to convert to seconds.

\end{description}

{\bf Count Option}

The specified seconds count is converted to the appropriate number of weeks,
days, hours, minutes, and seconds.  This option may be shortened to its first
letter.

\begin{description}

\item {\it -count seconds}\verb" "

Count of seconds to convert to the appropriate set of units.

\end{description}

{\bf EXAMPLES}

Example 1:  Converting 5 days into seconds

\begin{verbatim}
    $(42)> timetrans -days 5
    432000
\end{verbatim}

Example 2:  Converting 2 weeks into seconds

\begin{verbatim}
    $(43)> timetrans -w 2
    1209600
\end{verbatim}

Example 3:  Converting 8 days and 8 hours into seconds

\begin{verbatim}
    $(44)> timetrans -d 8 -hours 8
    720000
\end{verbatim}

Example 4:  Converting 1 week, 1 day, and 8 hours into seconds

\begin{verbatim}
    $(46)> timetrans -w 1 -days 1 -h 8
    720000
\end{verbatim}

Example 5:  Converting 14 weeks, 4 days, 21 hours, 8 minutes, and 8 seconds into seconds

\begin{verbatim}
    $(47)> timetrans -w 14 -d 4 -h 21 -m 8 -s 8
    8888888
\end{verbatim}

Example 6:  Converting 720000 seconds into time units

\begin{verbatim}
    $(48)> timetrans -c 720000
    1 week, 1 day, 8 hours
\end{verbatim}

Example 7:  Converting 1814421 seconds into time units

\begin{verbatim}
    $(49)> timetrans -c 1814421
    3 weeks, 21 seconds
\end{verbatim}

Example 8:  Converting 8888888 seconds into time units

\begin{verbatim}
    $(50)> timetrans -c 8888888
    14 weeks, 4 days, 21 hours, 8 minutes, 8 seconds
\end{verbatim}


{\bf SEE ALSO}

\cmd{zonesigner(8)}

\perlmod{Net::DNS::SEC::Tools::timetrans.pm(3)}


\clearpage
\subsection{\bf DNS Zone File Commands}
\label{ssect-cmds-zone}

The DNS zone file commands provide analytical and visualization tools for DNS
zone files.  These commands are:

\begin{table}[ht]
\begin{center}
\begin{tabular}{ll}
\cmd{dnspktflow} & analyzes and draw DNS flow diagrams			\\
\cmd{donuts}	 & analyzes DNS zone files for errors			\\
\cmd{donutsd}	 & daemon to periodically run \cmd{donuts}		\\
\cmd{getdnskeys} & manage lists of DNSKEYs from DNS zones		\\
\cmd{mapper}	 & creates maps of DNS zone data			\\
\cmd{TrustMan}	 & manage keys used as trust anchors			\\
\cmd{tachk}	 & verifies trust anchors in a \path{named.conf} file	\\
\cmd{validate}	 & query the Domain Name System 			\\
\end{tabular} 
\end{center}
\end{table}

\clearpage

\subsubsection{dnspktflow}

{\bf NAME}

\cmd{dnspktflow} - Analyze and draw DNS flow diagrams from a \path{tcpdump} file

{\bf SYNOPSIS}

\begin{verbatim}
    dnspktflow -o output.png file.tcpdump

    dnspktflow -o output.png -x -a -t -q file.tcpdump
\end{verbatim}

{\bf DESCRIPTION}

The \cmd{dnspktflow} application takes a \cmd{tcpdump} network traffic dump
file, passes it through the \cmd{tshark} application and then displays the
resulting DNS packet flows in a ``flow-diagram'' image.  \cmd{dnspktflow}
can output a single image or a series of images which can then be
shown in sequence as an animation.

\cmd{dnspktflow} was written as a debugging utility to help trace DNS
queries and responses, especially as they apply to DNSSEC-enabled lookups.

{\bf REQUIREMENTS}

This application requires the following Perl modules and software
components to work:

\begin{verbatim}
     graphviz                  http://www.graphviz.org/
     GraphViz                  Perl module
     tshark                    http://www.wireshark.org/
\end{verbatim}

The following is required for outputting screen presentations:

\begin{verbatim}
    MagicPoint                 http://member.wide.ad.jp/wg/mgp/
\end{verbatim}

If the following modules are installed, a GUI interface will be enabled for
communication with \cmd{dnspktflow}:

\begin{verbatim}
     QWizard                   Perl module
     Getopt::GUI::Long         Perl module
\end{verbatim}

{\bf OPTIONS}

\cmd{dnspktflow} takes a wide variety of command-line options.  These options
are described below in the following functional groups:  input packet
selection, output file options, output visualization options, graphical
options, and debugging.

{\bf Input Packet Selection}

These options determine the packets that will be selected by \cmd{dnspktflow}.
Short versions of the options are given in parentheses.

\begin{description}

\item {\bf --ignore-hosts=STRING (-i)}\verb" "

A regular expression of host names to ignore in the query/response fields.

\item {\bf --only-hosts=STRING (-r)}\verb" "

A regular expression of host names to analyze in the query/response fields.

\item {\bf --show-frame-num (-f)}\verb" "

Display the packet frame numbers.

\item {\bf --begin-frame=INTEGER (-b)}\verb" "

Begin at packet frame NUMBER.

\end{description}

{\bf Output File Options}

These options determine the type and location of \cmd{dnspktflow}'s output.

\begin{description}

\item {\bf --output-file=STRING (-o)}\verb" "

Output file name (default: out%03d.png as PNG format.)

\item {\bf --fig}\verb" "

Output format should be fig.

\item {\bf --tshark-out=STRING (-O)}\verb" "

Save \cmd{tshark} output to this file.

\item {\bf --multiple-outputs (-m)}\verb" "

One picture per request (use %03d in the filename.)

\item {\bf --magic-point=STRING (-M)}\verb" "

Saves a MagicPoint presentation for the output.

\end{description}

{\bf Output Visualization Options:}

These options determine specifics of \cmd{dnspktflow}'s output.

\begin{description}

\item {\bf --last-line-labels-only (-L)}\verb" "

Only show data on the last line drawn.

\item {\bf --most-lines=INTEGER (-z)}\verb" "

Only show at most INTEGER connections.

\item {\bf --input-is-tshark-out (-T)}\verb" "

The input file is already processed by \cmd{tshark}.

\end{description}

{\bf Graphical Options:}

These options determine fields included in \cmd{dnspktflow}'s output.

\begin{description}

\item {\bf --show-type (-t)}\verb" "

Shows message type in result image.

\item {\bf --show-queries (-q)}\verb" "

Shows query questions in result image.

\item {\bf --show-answers (-a)}\verb" "

Shows query answers in result image.

\item {\bf --show-authoritative (-A)}\verb" "

Shows authoritative information in result image.

\item {\bf --show-additional (-x)}\verb" "

Shows additional information in result image.

\item {\bf --show-label-lines (-l)}\verb" "

Shows lines attaching labels to lines.

\item {\bf --fontsize=INTEGER}\verb" "

Font Size

\end{description}

{\bf Debugging:}

These options may assist in debugging \cmd{dnspktflow}.

\begin{description}

\item {\bf --dump-pkts (-d)}\verb" "

Dump data collected from the packets.

\item {\bf --help (-h)}\verb" "

Show help for command line options.

\end{description}

{\bf SEE ALSO}

Getopt::GUI::Long(3)
Net::DNS(3)
QWizard.pm(3)


\clearpage

\subsubsection{\bf donuts}


{\bf NAME}

\cmd{donuts} - analyze DNS zone files for errors and warnings

{\bf SYNOPSIS}

\begin{verbatim}
    donuts -h -H -v -l LEVEL -r RULEFILES -i IGNORELIST -C -c configfile
        	 ZONEFILE DOMAINNAME...}
\end{verbatim}

{\bf DESCRIPTION}

DoNutS is a DNS Lint application that examines DNS zone files looking for
particular problems.  This is especially important for zones making use of
DNSSEC security records, since many subtle problems can occur.

If the \perlmod{Text::Wrap} Perl module is installed, \cmd{donuts} will give
better output formatting.

{\bf OPTIONS}

\begin{description}

\item {\it -h}\verb" "

Displays a help message.

\item {\it -v}\verb" "

Turns on more verbose output.

\item {\it -q}\verb" "

Turns on more quiet output.

\item {\it -l LEVEL}\verb" "

Sets the level of errors to be displayed.  The default is level 5.
The maximum value is level 9, which displays many debugging results.
You probably want to run no higher than level 8.

\item {\it -r RULEFILES}\verb" "

A comma-separated list of rule files to load.  The strings will be
passed to \func{glob()} so * wildcards can be used to specify multiple files.

\item {\it -i IGNORELIST}\verb" "

A comma-separated list of regex patterns which are checked against
rule names to determine if some should be ignored.  Run with {\it -v}
to figure out rule names if you're not sure which rule is generating
errors you don't wish to see.

\item {\it -L}\verb" "

Include rules that require live queries of data.  Generally, these
rules concentrate on pulling remote DNS data to test;
for example, parent/child zone relationships.

\item {\it -c CONFIGFILE}\verb" "

Parse a configuration file to change constraints specified by rules.
This defaults to \path{\$HOME/.donuts.conf}.

\item {\it -C}\verb" "

Don't read user configuration files at all, such as those specified by
the {\it -c} option or the \path{\$HOME/.donuts.conf} file.

\item {\it -t INTERFACE}\verb" "

Specifies that \cmd{tcpdump} should be started on {\it INTERFACE} (e.g.,
``eth0'') just before \cmd{donuts} begins its run of rules for each domain
and will stop it just after it has processed the rules.  This is
useful when you wish to capture the traffic generated by the {\it live}
feature, described above.

\item {\it -T FILTER}\verb" "

When \cmd{tcpdump} is run, this {\it FILTER} is passed to it for purposes of
filtering traffic.  By default, this is set to {\it port 53 || ip$[$6:2$]$ \&
0x1fff != 0}, which limits the traffic to traffic destined to port 53
(DNS) or fragmented packets.

\item {\it -o FILE}\verb" "

Saves the \cmd{tcpdump} captured packets to {\it FILE}.  The following
special fields can be used to help generate unique file names:

\begin{description}

\item {\it \%d}\verb" "

This is replaced with the current domain name being analyzed (e.g.,
``example.com''.)

\item {\it \%t}\verb" "

This is replaced with the current epoch time (i.e., the number of
seconds since Jan 1, 1970).

\end{description}

This field defaults to {\it \%d.\%t.pcap}.

\item {\it -H}\verb" "

Displays the personal configuration file rules and tokens that are acceptable
in a configuration file.  The output will consist of a rule name, a token, and
a description of its meaning.

Your configuration file (e.g., \path{\$HOME/.donuts.conf}) may have lines in
it that look like this:

\begin{verbatim}
    # change the default minimum number of legal NS records from 2 to 1
    name: DNS_MULTIPLE_NS
    minnsrecords: 1

    # change the level of the following rule from 8 to 5
    name: DNS_REASONABLE_TTLS
    level: 5
\end{verbatim}

This allows you to override certain aspects of how rules are executed.

\item {\it -R}\verb" "

Displays a list of all known rules along with their description (if available).

\item {\it -F LIST}\verb" "

\item {\it --features=LIST}\verb" "

The {\it --features} option specifies additional rule features that should
be executed.  Some rules are turned off by default because they are
more intensive or require a live network connection, for instance.
Use the {\it --features} flag to turn them on.  The LIST argument should be
a comma-separated list.  Example usage:

\begin{verbatim}
    --features live,data_check
\end{verbatim}

Features available in the default rule set:

\begin{description}

\item {\it live}\verb" "

The {\it live} feature allows rules that need to perform live DNS queries
to run.  Most of these {\it live} rules query parent and children of the
current zone, when appropriate, to see that the parent/child
relationships have been built properly.  For example, if you have a
DS record which authenticates the key used in a child zone the {\it live}
feature will let a rule run which checks to see if the child is
actually publishing the DNSKEY that corresponds to the test zone's DS
record.

\end{description}

\item {\it --show-gui}\verb" "

alpha code

Displays a browsable GUI screen showing the results of the \cmd{donuts} tests.

The \perlmod{QWizard} and \perlmod{Gtk2} Perl modules must be installed
for this to work.

\item {\it --live}\verb" "

Obsolete command line option.  Please use {\it --features live} instead.

\end{description}

{\bf SEE ALSO}

For writing rules that can be loaded by \cmd{donuts}:
\perlmod{Net::DNS::SEC::Tools::Donuts::Rule}

General DNS and DNSSEC usage: \perlmod{Net::DNS}, \perlmod{Net::DNS::SEC}

\perlmod{Gtk2.pm(3)}, \perlmod{QWizard.pm(3)}

\clearpage

\subsection{\it donutsd}

{\bf NAME}

\begin{verbatim}  DoNutSD - Run the donuts syntax checker periodically and email the results\end{verbatim}

{\bf SYNOPSIS}

\begin{verbatim}
  donutsd [-z FREQ] [-t TMPDIR] [-f FROM] [-s SMTPSERVER] [-a DONUTSARGS]
          [-x] [-v] [-i zonelistfile] [ZONEFILE ZONENAME ZONECONTACT]
\end{verbatim}

{\bf DESCRIPTION}

{\it donutsd} runs {\it donuts} on a set of zone files every so often (the
frequency is specified by the {\it -z} flag which defaults to 24 hours) and
watches for changes in the results.  These changes may be due to the
time-sensitive nature of DNSSEC-related records (e.g., RRSIG validity
periods) or because parent/child relationships have changed.  If any
changes have occurred in the output since the last run of {\it donuts} on a
particular zone file, the results are emailed to the specified zone
administrator's email address.

{\bf OPTIONS}

\begin{description}

\item [-v] Turns on more verbose output.

\item [-o] Run once and quit, as opposed to sleeping or re-running forever.

\item [-a ARGUMENTS]\verb" "

Passes arguments to command line arguments of {\it donuts} runs.

\item [-z TIME]\verb" "

Sleeps TIME seconds between calls to {\it donuts}.

\item [-e ADDRESS]\verb" "

Mail ADDRESS with a summary of the results from all the files.
These are the last few lines of the {\it donuts} output for each zone that
details the number of errors found.

\item [-s SMTPSERVER]\verb" "

When sending mail, send it to the SMTPSERVER specified.  The default
is {\it localhost}.

\item [-f FROMADDR]\verb" "

When sending mail, use FROMADDR for the From: address.

\item [-x] Send the {\it diff} output in the email message as well as
the {\it donuts} output.

\item [-t TMPDIR]\verb" "

Store temporary files in TMPDIR.

\item [-i INPUTZONES]\verb" "

See the next section for details.

\end{description}

{\bf ZONE ARGUMENTS}

The rest of the arguments to {\it donutsd} should be triplets of the
following information:

\begin{description}

\item [ZONEFILE] The zone file to examine

\item [ZONENAME] The zonename that file is supposed to be defining

\item [ZONECONTACT] An email address of the zone administrator (or a
comma-separated list of addresses).  The results will be sent to this
email address.

\end{description}

Additionally, instead of listing all the zones you wish to monitor on
the command line, you can instead use the {\it -i} flag which specifies a
file to be read listing the TRIPLES instead.  Each line in this file
should contain one triple with white-space separating the arguments.

Example:

\begin{verbatim}
   db.zonefile1.com   zone1.com   admin@zone1.com
   db.zonefile2.com   zone2.com   admin@zone2.com,admin2@zone2.com
\end{verbatim}

For even more control you can specify an XML file (which must end in
{\bf .xml} and you must have the {\bf XML::Smart} Perl module installed) that
describes the same information but also allows for per-zone
customization of the {\it donuts} arguments:

\begin{verbatim}
 <donutsd>
   <zones>
    <zone>
      <file>db.example.com</file>
      <name>example.com</name>
      <contact>admin@example.com</contact>
      <!-- this is not a signed zone therefore we'll
           add these args so we don't display DNSSEC errors -->
      <donutsargs>-i DNSSEC</donutsargs>
    </zone>
   </zones>
 </donutsd>
\end{verbatim}

The {\it donutsd} tree can also contain a {\it configs} section where command
line flags can be specified as well:

\begin{verbatim}
 <donutsd>
  <configs>
   <config><flag>a</flag><value>--live --level 8</value></config>
   <config><flag>e</flag><value>wes@example.com</value></config>
  </configs>
  <zones>
   ...
  </zones>
 </donutsd>
\end{verbatim}

Real command line flags will be used in preference to those specified
in the {\bf .xml} file, however.

{\bf EXAMPLE}

\begin{verbatim}
  donutsd -a "--live --level 8" -f root@somewhere.com \
     db.example.com example.com admin@example.com
\end{verbatim}

{\bf SEE ALSO}

{\bf donuts(8)}

\url{http://dnssec-tools.sourceforge.net}


\clearpage

\subsubsection{getdnskeys}

{\bf NAME}

\cmd{getdnskeys} - Manage lists of DNSKEYs from DNS zones

{\bf SYNOPSIS}

\begin{verbatim}
    getdnskeys [-i file] [-o file] [-k] [-T] [-t] [-v] [zones]
\end{verbatim}

{\bf DESCRIPTION}

\cmd{getdnskeys} manages lists of DNSKEYs from DNS zones.  It may be used
to retrieve and compare DNSKEYs.  The output from \cmd{getdnskeys} may be
included (directly or indirectly) in a \path{named.conf} file.

{\bf OPTIONS}

\begin{description}

\item {\bf -h}\verb" "

Gives a help message.

\item {\bf -i path}\verb" "

Reads {\it path} as a \path{named.conf} with which to compare key lists.

\item {\bf -k}\verb" "

Only looks for Key Signing Keys (KSKs); all other keys are ignored.

\item {\bf -o file}\verb" "

Writes the results to {\it file}.

\item {\bf -T}\verb" "

Checks the current trusted key list from \path{named.conf}.

\item {\bf -t}\verb" "

Encloses output in needed \path{named.conf} syntax markers.

\item {\bf -v}\verb" "

Turns on verbose mode for additional output.

\end{description}

{\bf EXAMPLES}

This \cmd{getdnskeys} will retrieve the KSK for example.com:

\begin{verbatim}
    getdnskeys -o /etc/named.trustkeys.conf -k -v -t example.com
\end{verbatim}

This \cmd{getdnskeys} will check saved keys against a live set of keys:

\begin{verbatim}
    getdnskeys -i /etc/named.trustkeys.conf -T -k -v -t
\end{verbatim}

This \cmd{getdnskeys} will automatically update a set of saved keys:

\begin{verbatim}
    getdnskeys -i /etc/named.trustkeys.conf -k -t -T -v
               -o /etc/named.trustkeys.conf
\end{verbatim}

{\bf SECURITY ISSUES}

Currently this does not validate new keys placed in the file in any
way, nor does it validate change over keys which have been added.

It also does not handle revocation of keys.

It should prompt you before adding a new key so that you can always
run the auto-update feature.


\clearpage

\subsubsection{mapper}

{\bf NAME}

\cmd{mapper} - Create graphical maps of DNS zone data

{\bf SYNOPSIS}

\begin{verbatim}
    mapper [options] zonefile1 ... zonefileN
\end{verbatim}

{\bf DESCRIPTION}

This application creates a graphical map of one or more zone files.  The
output gives a graphical representation of a DNS zone or zones.  The output
is written in the PNG format.  The result can be useful for getting a more
intuitive view of a zone or set of zones.  It is extremely useful for
visualizing DNSSEC deployment within a given zone as well as to help discover
problem spots.

{\bf OPTIONS}

\begin{description}

\item -h\verb" "

Prints a help summary.

\item -o OUTFILE.png\verb" "

Saves the results to a given filename.  If this option is not given, the map
will be saved to \path{map.png}.

\item -r\verb" "

Lists resource records assigned to each node within the map.

\item -t TYPE,TYPE...\verb" "

Adds the data portion of a resource record to the displayed node
information.  Data types passed will be automatically converted to
upper-case for ease of use.

Example usage: {\it -t A} will add IPv4 addresses to
all displayed nodes that have A records.

\item -L\verb" "

Adds a legend to the map.

\item -l (neato|dot|twopi|circo|fdp)\verb" "

Selects a layout format.  The default is {\it neato}, which is circular in
pattern.  See the documentation on the \cmd{GraphViz} package and the
\perlmod{GraphViz} Perl module for further details.

\item -a\verb" "

Allows overlapping of nodes.  This makes much tighter maps with the
downside being that they are somewhat cluttered.  Maps of extremely
large zones will be difficult to decipher if this option is not used.

\item -e WEIGHT\verb" "

Assigns an edge weight to edges.  In theory, $>$1 means shorter and $<$1 means
longer, although, it may not have any effect as implemented.
This should work better in the future.

\item -f INTEGER\verb" "

Uses the INTEGER value for the font size to print node names with.
The default value is 10.

\item -w WARNTIME\verb" "

Specifies how far in advance expiration warnings are enabled for signed 
resource records.  The default is 7 days.  The warning time is measured
in seconds.

\item -i REGEX\verb" "

Ignores record types matching a {\it REGEX} regular expression.

\item -s TYPE,TYPE...\verb" "

Specifies a list of record types that will not be analyzed or displayed
in the map.  By default, this is set to NSEC and CNAME in order to reduce
clutter.  Setting it to ``'' will display these results again.

\item -T TYPE,TYPE...\verb" "

Restrict record types that will be processed to those of type {\it TYPE}.
This is the converse of the {\it -s} option.  It is not meaningful to use both
{\it -s} and {\it -t} in the same invocation.  They will both work at once,
however, so if {\it -T} specifies a type which {\it -s} excludes, it will not
be shown.

\item -g\verb" "

Attempts to cluster nodes around the domain name.  For ``dot'' layouts,
this actually means drawing a box around the cluster.  For the other
types, it makes very little difference, if any.

\item -q\verb" "

Prevents output of warnings or errors about records that have DNSSEC
signatures that are near or beyond their signature lifetimes.

\end{description}

{\bf EXAMPLE INVOCATIONS}

\begin{description}

\item {\it mapper -s cname,nsec -i dhcp -L zonefile zone.com}\verb" "

Writes to the default file (\path{map.png}) of a {\it zone.com} zone
stored in \path{zonefile}.  It excludes any hosts with a name containing
\cmd{dhcp} and ignores any record of type {\it CNAME} or {\it NSEC}.  A legend
is included in the output.

\item {\it mapper -s txt,hinfo,cname,nsec,a,aaaa,mx,rrsig -L zonefile zone.com zonefile2 sub.zone.com ...}\verb" "

Removes a lot of records from the display in order to primarily display
a map of a zone hierarchy.

\item {\it mapper -l dot -s txt,hinfo,cname,nsec,a,aaaa,mx,rrsig -L zonefile zone.com zonefile2 sub.zone.com ...}\verb" "

As the previous example, but this command draws a more vertical tree-style
graph of the zone.  This works well for fairly deep but narrow hierarchies.
Tree-style diagrams rarely look as nice for full zones.

\end{description}

{\bf SEE ALSO}

Net::DNS(3)


\clearpage

\subsubsection{\bf TrustMan}

{\bf NAME}

\cmd{TrustMan} - manage keys used as trust anchors

{\bf SYNOPSIS}

\begin{verbatim}
    TrustMan [options]
\end{verbatim}

{\bf DESCRIPTION}

\cmd{TrustMan} runs by default as a daemon to verify if keys stored locally in
configuration files like \path{named.conf} still match the same keys as fetched
from the zone where they are defined.  If mismatches are detected, the daemon
notifies via email the contact person defined in the DNSSEC-Tools
configuration file or on the command line.

\cmd{TrustMan} can also be run in the foreground ({\it -f}) to run this check
a single time.

\cmd{TrustMan} can also be used to set up configuration data in the file
\path{dnssec-tools.conf} for later use by the daemon, making fewer command
line arguments necessary.  Configuration data are stored in
\path{dnssec-tools.conf}.  The current version requires the
\path{dnssec-tools.conf} file to be edited by hand to add values for the
contact person's email address ({\it tacontact}) and the SMTP server ({\it
tasmtpserver}).  Also, the location of \path{named.conf} and
\path{dnsval.conf} must also be added to that file, if necessary.

{\bf OPTIONS}

\begin{description}

\item -f\verb" "

Run in the foreground.

\item -c\verb" "

Create a configure file for \cmd{TrustMan} from the command line options given.

\item -o\verb" "

Output file for configuration.

\item -k \verb" "

A \path{dnsval.conf} file to read.

\item -n \verb" "

A \path{named.conf} file to read.

\item -d\verb" "

The domain to check (supersedes configuration file.)

\item -t\verb" "

The number of seconds to sleep between checks.  Default is 3600 (one hour.)

\item -m\verb" "

Mail address for the contact person to whom reports should be sent.

\item -p\verb" "

Log messages to {\it stdout}.

\item -L\verb" "

Log messages to {\bf syslog}.

\item -s\verb" "

SMTP server \cmd{TrustMan} should use to send reports.

\item -N\verb" "

Send report when there are no errors.

\item -v\verb" "

Verbose.

\end{description}

{\bf SEE ALSO}

\path{dnssec-tools.conf(5)},
\path{dnsval.conf(5)},
\path{named.conf(5)}


\clearpage

\subsubsection{\bf tachk}

{\bf NAME}

\cmd{tachk} - Check the validity of the trust anchors in a \path{named.conf}
file.

{\bf SYNOPSIS}

\begin{verbatim}
    tachk [options] named.conf
\end{verbatim}

{\bf DESCRIPTION}

\cmd{tachk} checks the validity of the trust anchors in the specified
{\bf named.conf} file.  The output given depends on the options selected.

Note:  \cmd{tachk} may be removed in future releases.

{\bf OPTIONS}

{\bf tachk} takes two types of options:  record-attribute options
and output-style options.  These option sets are detailed below.

{\bf Record-Attribute Options}

\begin{description}

\item {\it -valid}\verb" "

This option displays the valid trust anchors in a {\bf named.conf} file.

\item {\it -invalid}\verb" "

This option displays the invalid trust anchors in a {\bf named.conf} file.

\end{description}

{\bf Output-Format Options}

These options define how the trust anchor information will be displayed.
Without any of these options, the zone name and key tag will be displayed
for each trust anchor.

\begin{description}

\item {\it -count}\verb" "

The count of matching records will be displayed, but the matching records
will not be.

\item {\it -long}\verb" "

The long form of output will be given:  the zone name and key tag will be
displayed for each trust anchor.

\item {\it -terse}\verb" "

This option displays only the name of the zones selected by other options.

\item {\it -help}\verb" "

Display a usage message.

\end{description}

\clearpage

\subsubsection{\bf validate}

{\bf NAME}

\cmd{validate} - Query the Domain Name System and display results of the
DNSSEC validation process

{\bf SYNOPSIS}

\begin{verbatim}
    validate

    validate [options] DOMAIN_NAME
\end{verbatim}

{\bf DESCRIPTION}

\cmd{validate} is a diagnostic tool built on top of the DNSSEC validator.  If
given a domain name argument (as {\it DOMAIN\_NAME}), \cmd{validate} queries
the DNS for that domain name.  It outputs the series of responses that were
received from the DNS and the DNSSEC validation results for each domain name.
An examination of the queries and validation results can help an administrator
uncover errors in DNSSEC configuration of DNS zones.

If no options are specified and no {\it DOMAIN\_NAME} argument is given,
\cmd{validate} will perform a series of pre-defined test queries against the
{\it test.dnssec-tools.org} zone.  This serves as a test-suite for the
validator.  If any options are specified (e.g., configuration file locations),
{\it -s} or {\it --selftest} must be specified to run the test-suite.

{\bf OPTIONS}

\begin{description}

\item {\it -c CLASS, --class=CLASS}\verb" "

This option can be used to specify the DNS class of the Resource Record
queried.  If this option is not given, the default class {\bf IN} is used.

\item {\it -h, --help}\verb" "

Display the help and exit.

\item {\it -m, --merge}\verb" "

When this option is given, \cmd{validate} will merge different RRsets in the
response into a single answer.  If this option is not given, each RRset is
output as a separate response.  This option makes sense only when used with
the {\it -p} option.

\item {\it -p, --print}\verb" "

Print the answers and validation results.  By default, \cmd{validate} just
outputs a series of responses and their validation results on {\it stderr}.
When the {\it -p} option is used, \cmd{validate} will also output the final
result on {\it stdout}.

\item {\it -t TYPE, --type=TYPE}\verb" "

This option can be used to specify the DNS type of the Resource Record
queried.  If this option is not given, \cmd{validate} will query for the
{\bf A} record for the given {\it DOMAIN\_NAME}.

\item {\it -v FILE, --dnsval-conf=FILE}\verb" "

This option can be used to specify the location of the \path{dnsval.conf}
configuration file.

\item {\it -r FILE, --resolv-conf=FILE}\verb" "

This option can be used to specify the location of the \path{resolv.conf}
configuration file containing the name servers to use for lookups.

\item {\it -i FILE, --root-hints=FILE}\verb" "

This option can be used to specify the location of the \path{root.hints}
configuration file, containing the root name servers. This is only used when
no name server is found, and \cmd{validate} must do recursive lookups itself.

\item {\it -s, --selftest}\verb" "

This option can be used to specify that the application should perform its
test-suite against the {\it dnssec-tools.org} test domain. If the name servers
configured in the system \path{resolv.conf} do not support DNSSEC, use the
{\it -r} and {\it -i} options to enable \cmd{validate} to use its own internal
recursive resolver.

\item {\it -T testcase number>}\verb" "

This option can be used to run a specific test from the test-suite.

\item {\it -o, --output=debug-level:dest-type[:dest-options]}\verb" "

\var{debug-level} is 1-7, corresponding to syslog levels ALERT-DEBUG. \\
\var{dest-type} is one of {\it file}, {\it net}, {\it syslog},.
{\it stderr}, {\it stdout}. \\
\var{dest-options} depends on \var{dest-type}:
\begin{verbatim}
    file:<file-name>              (opened in append mode)
    net[:<host-name>:<host-port>] (127.0.0.1:1053)
    syslog[:facility]             (0-23 (default 1 USER))
\end{verbatim}

\end{description}

{\bf PRE-REQUISITES}

\lib{libval(3)}

{\bf SEE ALSO}

\cmd{drawvalmap(1)}

\lib{libval(3)}, \lib{val\_query(3)}



\clearpage
\subsection{\bf Zone-Signing Commands}
\label{ssect-cmds-sign}

The zone-signing commands provide tools to assist in the signing DNS zone
files and keeping records about those signed zones.  These commands are:

\begin{table}[ht]
\begin{center}
\begin{tabular}{ll}
\cmd{zonesigner}	& generates encryption keys and signs a DNS zone \\
\cmd{genkrf}		& creates a new {\it keyrec} file		 \\
\cmd{krfcheck}		& checks a {\it keyrec} file for errors		 \\
\cmd{lskrf}		& lists the contents of a {\it keyrec} file	 \\
\cmd{expchk}		& checks a {\it keyrec} file for expired zones	 \\
\cmd{fixkrf}		& fixes a {\it keyrec} file for missing keys	 \\
\cmd{cleankrf}		& cleans a {\it keyrec} file of orphaned keys	 \\
\cmd{signset-editor}	& GUI editor for signing sets in a {\it keyrec} file \\
\end{tabular} 
\end{center}
\end{table}

\clearpage

\subsubsection{zonesigner}

{\bf NAME}

\cmd{zonesigner} - Generates encryption keys and signs a DNS zone

{\bf SYNOPSIS}

\begin{verbatim}
  zonesigner [options] <zone-file> <signed-zone-file>

  # get started immediately examples:

  # first run on a zone for example.com:
  zonesigner -genkeys -endtime +2678400 example.com

  # future runs before expiration time (reuses the same keys):
  zonesigner -endtime +2678400 example.com
\end{verbatim}

{\bf DESCRIPTION}

This script combines into a single command many actions that are required to
sign a DNS zone.  It generates the required KSK and ZSK keys, adds the key
data to a zone record file, signs the zone file, and runs checks to ensure
that everything worked properly.  It also keeps records about the keys and
how the zone was signed in order to facilitate re-signing of the zone in the
future.

The \cmd{zonesigner}-specific zone-signing records are kept in \struct{keyrec}
files.  Using \struct{keyrec} files, defined and maintained by DNSSEC-Tools,
\cmd{zonesigner} can automatically gather many of the options used to
previously sign and generate a zone and its keys.  This allows the zone to be
maintained using the same key lengths and expiration times, for example,
without an administrator needing to manually track these fields.

{\bf QUICK START}

The following are examples that will allow a quick start on using
\cmd{zonesigner}:

\begin{description}

\item first run on example.com\verb" "

The following command will generate keys and sign the zone file for
example.com, giving an expiration date 31 days in the future.  The
zone file is named \path{example.com} and the signed zone file will be
named \path{example.com.signed}.

\begin{verbatim}
    zonesigner -genkeys -endtime +2678400 example.com
\end{verbatim}

\item subsequent runs on example.com\verb" "

The following command will re-sign example.com's zone file, but will not
generate new keys.  The files and all key-generation and zone-signing
arguments will remain the same.

\begin{verbatim}
    zonesigner example.com
\end{verbatim}

\end{description}

{\bf USING ZONESIGNER}

\cmd{zonesigner} is used in this way:

\begin{verbatim}
    zonesigner [options] <zone-file> <signed-zone-file>
\end{verbatim}

The {\it zone-file} argument is required.

{\it zone-file} is the name of the zone file from which a signed zone file
will be created.  If the {\it -zone} option is not given, then {\it zone-file}
will be used as the name of the zone that will be signed.  Generated keys are
given this name as their base.

The zone file is modified to have {\bf include} commands, which will include
the KSK and ZSK keys.  These lines are placed at the end of the file and
should not be modified by the user.  If the zone file already includes any key
files, those inclusions will be deleted.  These lines are distinguished by
starting with ``\$INCLUDE'' and end with \path{.key}.  Only the actual include
lines are deleted; any related comment lines are left untouched.

An intermediate file is used in signing the zone.  {\it zone-file} is copied
to the intermediate file and is modified in preparation of signing the zone
file.  Several \$INCLUDE lines will be added at the end of the file and the
SOA serial number will be incremented.

{\it signed-zone} is the name of the signed zone file.  If it is not given on
the command line, the default signed zone filename is the {\it zone-file}
appended with \path{.signed}.  Thus, executing \xqt{zonesigner example.com}
will result in the signed zone being stored in \path{example.com.signed}.

Unless the {\it -genkeys}, {\it -genksk}, {\it -genzsk}, or {\it -newpubksk}
options are specified, the last keys generated for a particular zone will be
used in subsequent \cmd{zonesigner} executions.

{\bf KEYREC FILES}

\struct{keyrec} files retain information about previous key-generation and
zone-signing operations.  If a \struct{keyrec} file is not specified (by way
of the {\it -krfile} option), then a default \struct{keyrec} file is used.  If
this default is not specified in the system's DNSSEC-Tools configuration file,
the filename will be the zone name appended with \path{.krf}.  If the {\it
-nokrfile} option is given, then no \struct{keyrec} file will be consulted or
saved.

\struct{keyrec} files contain three types of entries:  zone \struct{keyrec}s,
set \struct{keyrec}s, and key \struct{keyrec}s.  Zone \struct{keyrec}s contain
information specifically about the zone, such as the number of ZSKs used to
sign the zone, the end-time for the zone, and the key signing set names (names
of set \struct{keyrec}s.) Set \struct{keyrec}s contain lists of key
\struct{keyrec} names used for a specific purpose, such as the current ZSK
keys or the published ZSK keys.  Key \struct{keyrec}s contain information
about the generated keys themselves, such as encryption algorithm, key length,
and key lifetime.

Each \struct{keyrec} contains a set of ``key/value'' entries, one per line.
Example 4 below contains the contents of a sample \struct{keyrec} file.

{\bf ENTROPY}

On some systems, the implementation of the pseudo-random number generator
requires keyboard activity.  This keyboard activity is used to fill a buffer
in the system's random number generator.  If \cmd{zonesigner} appears hung,
you may have to add entropy to the random number generator by randomly
striking keys until the program completes.  Display of this message is
controlled by the {\bf entropy\_msg} configuration file parameter.

{\bf DETERMINING OPTION VALUES}

\cmd{zonesigner} checks four places in order to determine option values.  
In descending order of precedence, these places are:

\begin{description}
\item command line options
\item keyrec file
\item DNSSEC-Tools configuration file
\item zonesigner defaults
\end{description}

Each is checked until a value is found.  That value is then used for that
\cmd{zonesigner} execution and the value is stored in the \struct{keyrec} file.

{\bf Example}

For example, the KSK length has the following values:

\begin{table}[ht]
\begin{tabular}{clr}
& -ksklength command line option  & 8192 \\
& keyrec file                     & 1024 \\
& DNSSEC-Tools configuration file & 2048 \\
& zonesigner defaults             & 512 \\
\end{tabular}
\end{table}

If all are present, then the KSK length will be 8192.

If the {\it -ksklength} command line option wasn't given, the KSK length
will be 1024.

If the KSK length wasn't given in the configuration file, it will be 8192.

If the KSK length wasn't in the \struct{keyrec} file or the configuration
file, the KSK length will be 8192.

If the {\it -ksklength} command line option wasn't given and the KSK length
wasn't in the configuration file, it'll be 1024.

If the command line option wasn't given, the KSK length wasn't in the
\struct{keyrec} file, and it wasn't in the configuration file, then the KSK
length will be 512.

{\bf OPTIONS}

Three types of options may be given, based on the command for which they are
intended.  These commands are  \cmd{dnssec-keygen}, \cmd{dnssec-signzone}, and
\cmd{zonesigner}.

{\bf \cmd{zonesigner}-specific Options}

\begin{description}

\item {\bf -nokrfile}\verb" "

No \struct{keyrec} file will be consulted or created.

\item {\bf -krfile}\verb" "

\struct{keyrec} file to use in processing options.  See the man page for the
DNSSEC-Tools \perlmod{tooloptions.pm} module for more details about this file.

\item {\bf -genkeys}\verb" "

Generate new KSKs and ZSKs for the zone.

\item {\bf -genksk}\verb" "

Generate new Current KSKs for the zone.  Any existing Current KSKs will be
marked as obsolete.  If this option is not given, the last KSKs generated for
this zone will be used.

\item {\bf -genzsk}\verb" "

Generate new ZSKs for the zone.  By default, the last ZSKs generated for this
zone will be used.

\item {\bf -newpubksk}\verb" "

Generate new Published KSKs for the zone.  Any existing Published KSKs will
be marked as obsolete.

\item {\bf -useboth}\verb" "

Use the existing Current {\bf and} Published ZSKs to sign the zone.

\item {\bf -usezskpub}\verb" "

Use the existing Published ZSKs to sign the zone.

\item {\bf -archivedir}\verb" "

The key archive directory.  If a key archive directory hasn't been specified
(on the command line or in the DNSSEC-Tools configuration file) and the {\it
-nosave} option was {\bf not} given, an error message will be displayed and
\cmd{zonesigner} will exit.

When the files are saved into the archive directory, the existing file names
are prepended with a timestamp.  The timestamp indicates when the files are
archived.

This directory {\bf may not} be the root directory.

\item {\bf -nosave}\verb" "

Do not save obsolete keys to the key archive directory.  The default behavior
is to save obsolete keys.

\item {\bf -kskcount}\verb" "

The number of KSK keys to generate and with which to sign the zone.  The
default is to use a single KSK key.

\item {\bf -ksklife}\verb" "

The time between KSK rollovers.  This is measured in seconds.

\item {\bf -ksignset}\verb" "

The name of the KSK signing set to use.  If the signing set does not exist,
then this must be used in conjunction with either {\it -genkeys} or {\it
-genksk}.  The name may contain alphanumerics, underscores, hyphens, periods,
and commas.

The default signing set name is ``signing-set-{\it N}'', where {\it N} is a
number.  If {\it -signset} is not specified, then \cmd{zonesigner} will use
the default and increment the number for subsequent signing sets.

\item {\bf -zsklife}\verb" "

The time between ZSK rollovers.  This is measured in seconds.

\item {\bf -zskcount}\verb" "

The number of ZSK keys to generate and with which to sign the zone.  The
default is to use a single ZSK key.

\item {\bf -signset}\verb" "

The name of the ZSK signing set to use as the Current ZSK signing set.  The
zone is signed and the given signing set becomes the zone's new Current ZSK
signing set.  If the signing set does not exist, then this must be used in
conjunction with either {\it -genkeys} or {\it -genzsk}.

The name may contain alphanumerics, underscores, hyphens, periods, and commas.
The default signing set name is ``signing-set-{\it N}'', where {\it N} is a
number.  If {\it -signset} is not specified, then \cmd{zonesigner} will use
the default and increment the number for subsequent signing sets.

\item {\bf -rollksk}\verb" "

Force a rollover of the KSK keys.  The Current KSK keys are marked as Obsolete
and the Published KSK keys are marked as Current.  The zone is then signed
with the new set of Current KSK keys.  If the zone's \struct{keyrec} does not
list a Current or Published KSK, an error message is printed and
\cmd{zonesigner} stops execution.

The zone's \struct{keyrec} file is updated to show the new key state.

The \struct{keyrec}s of the KSK keys are adjusted as follows:

\begin{enumerate}
\item The Current KSK keys are marked as Obsolete.
\item The Published KSK keys are marked as Current.
\item The obsolete KSK keys are moved to the archive directory.
\end{enumerate}

{\bf Warning}:  The timing of key-rolling is critical.  Great care must be
taken when using this option.  \cmd{rollerd} automates the KSK rollover
process and may be used to safely take care of this aspect of DNSSEC
management.

{\bf Warning}:  Using the {\it -rollksk} option should only be used if you
know what you're doing.

{\bf Warning}:  This is a {\it temporary} method of KSK rollover.  It {\it
may} be changed in the future.

\item {\bf -rollzsk}\verb" "

Force a rollover of the ZSK keys using the Pre-Publish Key Rollover method.
The rollover process adjusts the keys used to sign the specified zone,
generates new keys, signs the zone with the appropriate keys, and updates the
\struct{keyrec} file.  The Pre-Publish Key Rollover process is described in the
DNSSEC Operational Practices document.

Three sets of ZSK keys are used in the rollover process:  Current, Published,
and New.  Current ZSKs are those which are used to sign the zone.  Published
ZSKs are available in the zone data, and therefore in cached zone data, but
are not yet used to sign the zone.  New ZSKs are not available in zone data
nor yet used to sign the zone, but are waiting in the wings for future use.

The \struct{keyrec}s of the ZSK keys are adjusted as follows:

\begin{enumerate}
\item The Current ZSK keys are marked as obsolete.
\item The Published ZSK keys are marked as Current.
\item The New ZSK keys, if they exist, are marked as Published.
\item Another set of ZSK keys are generated, which will be
        marked as the New ZSK keys.
\item The Published ZSK keys' zsklife field is copied to the
        new ZSK keys' keyrecs.
\item The obsolete ZSK keys are moved to the archive directory.
\end{enumerate}

The quick summary of proper ZSK rolling (which \cmd{rollerd} does for you if
you use it):

\begin{enumerate}
\item wait 2 * max(TTL in zone)
\item run zonesigner using -usezskpub
\item wait 2 * max(TTL in zone)
\item run zonesigner using -rollzsk
\item wait 2 * max(TTL in zone)
\end{enumerate}

{\bf Warning}:  The timing of key-rolling is critical.  Great care must be taken
when using this option.  {\bf rollerd} automates the rollover process and may be
used to safely take care of this aspect of DNSSEC management.  Using the
{\it -rollzsk} option should only be used if you know what you're doing.

\item {\bf -intermediate}\verb" "

Filename to use for the temporary zone file.  The zone file will be copied to
this file and then the key names appended.

\item {\bf -zone}\verb" "

Name of the zone that will be signed.  This zone name may be given with this
option or as the first non-option command line argument.

\item {\bf -help}\verb" "

Display a usage message.

\item {\bf -Version}\verb" "

Display the version information for \cmd{zonesigner} and the DNSSEC-Tools
package.

\item {\bf -verbose}\verb" "

Verbose output will be given.  As more instances of {\it -verbose} are given
on the command line, additional levels of verbosity are achieved.

\begin{table}[ht]
\begin{center}
\begin{tabular}{|c|l|}
\hline
{\bf Verbosity Level} & {\bf Output} \\
\hline
1 & operations being performed \\
  & (e.g., generating key files, signing zone)  \\
2 & details on operations and some operation results \\
  & (e.g., new key names, zone serial number) \\
3 & operations' parameters and additional details \\
  & (e.g., key lengths, encryption algorithm, \\
  & executed commands) \\
\hline
\end{tabular}
\end{center}
\caption{\cmd{zonesigner} Verbosity Levels}
\end{table}

Higher levels of verbosity are cumulative.  Specifying two instances of
{\it -verbose} will get the output from the first and second levels of output.

\item {\bf -showkeycmd}\verb" "

Display the actual key-generation command (with options and arguments) that is
executed.  This is a small subset of verbose level 3 output.

\item {\bf -showsigncmd}\verb" "

Display the actual zone-signing command (with options and arguments) that is
executed.  This is a small subset of verbose level 3 output.

\end{description}

{\bf \cmd{dnssec-keygen}-specific Options}

\begin{description}

\item {\bf -algorithm}\verb" "

Cryptographic algorithm used to generate the zone's keys.  The default value
is RSASHA1.  The option value is passed to \cmd{dnssec-keygen} as the the {\it
-a} flag.  Consult \cmd{dnssec-keygen}'s manual page to determine legal values.

\item {\bf -ksklength}\verb" "

Bit length of the zone's KSK key.
The default is 1024.

\item {\bf -random}\verb" "

Source of randomness used to generate the zone's keys.  This is assumed to be
a file, for example \path{/dev/urandom}.

\item {\bf -zsklength}\verb" "

Bit length of the zone's ZSK key.
The default is 512.

\item {\bf -kgopts}\verb" "

Additional options for \cmd{dnssec-keygen} may be specified using this option.
The additional options are passed as a single string value as an argument to
the {\it -kgopts} option.

\end{description}

{\bf \cmd{dnssec-signzone}-specific Options}

\begin{description}

\item {\bf -endtime}\verb" "

Time that the zone expires, measured in seconds.  See the man page for
\cmd{dnssec-signzone} for the valid format of this field.
The default value is 2592000 seconds (30 days.)

\item {\bf -gends}\verb" "

Force \cmd{dnssec-signzone} to generate DS records for the zone.  This option
is translated into {\it -g} when passed to \cmd{dnssec-signzone}.

\item {\bf -ksdir}\verb" "

Specify a directory for storing keysets.  This is passed to
\cmd{dnssec-signzone} as the {\it -d} option.

\item {\bf -szopts}\verb" "

Additional options for \cmd{dnssec-signzone} may be specified using this
option.  The additional options are passed as a single string value as an
argument to the {\it -szopts} option.

\end{description}

{\bf Other Options}

\begin{description}

\item {\bf -zcopts}\verb" "

Additional options for \cmd{named-checkzone} may be specified using this option.
The additional options are passed as a single string value as an argument to
the {\it -zcopts} option.

\end{description}

{\bf Examples}

Example 1.

In the first example, an existing \struct{keyrec} file is used to assist in
signing the example.com domain.  Zone data are stored in \path{example.com},
and the keyrec is in \path{example.krf}.  The final signed zone file will be
\path{db.example.com.signed}.  Using this execution:

\begin{verbatim}
    # zonesigner -krfile example.krf example.com db.example.com.signed
\end{verbatim}

the following files are created:

\eject

\begin{table}[ht]
\begin{tabular}{cl}
 & \path{Kexample.com.+005+45842.private} \\
 & \path{Kexample.com.+005+45842.key} \\
 & \path{Kexample.com.+005+50186.private} \\
 & \path{Kexample.com.+005+50186.key} \\
 & \path{Kexample.com.+005+59143.private} \\
 & \path{Kexample.com.+005+59143.key} \\
 & \path{dsset-example.com.} \\
 & \path{keyset-example.com.} \\
 & \path{db.example.com.signed} \\
\end{tabular}
\end{table}

The first six files are the KSK and ZSK keys required for the zone.  The next
two files are created by the zone-signing process.  The last file is the 
final signed zone file.

Example 2.

In the second example, an existing \struct{keyrec} file is used to assist in
signing the example.com domain.  Zone data are stored in \path{example.com},
and the keyrec is in \path{example.krf}.  The generated keys, an intermediate
zone file, and final signed zone file will use \path{example.com} as a base.
Using this execution:

\begin{verbatim}
    # zonesigner -krfile example.krf -intermediate example.zs example.com
\end{verbatim}

the following files are created:

\begin{table}[ht]
\begin{tabular}{cl}
 & \path{Kdb.example.com.+005+12354.key} \\
 & \path{Kdb.example.com.+005+12354.private} \\
 & \path{Kdb.example.com.+005+82197.key} \\
 & \path{Kdb.example.com.+005+82197.private} \\
 & \path{Kdb.example.com.+005+55888.key} \\
 & \path{Kdb.example.com.+005+55888.private} \\
 & \path{dsset-db.example.com.} \\
 & \path{keyset-db.example.com.} \\
 & \path{example.zs} \\
 & \path{example.com.signed} \\
\end{tabular}
\end{table}

The first six files are the KSK and ZSK keys required for the zone.  The next
two files are created by the zone-signing process.  The second last file is
an intermediate file that will be signed.  The last file is file is the final
signed zone.

Example 3.

In the third example, no \struct{keyrec} file is specified for the signing of
the example.com domain.  In addition to files created as shown in previous
examples, a new \struct{keyrec} file is created.  The new \struct{keyrec} file
uses the domain name as its base.  Using this execution:

\begin{verbatim}
    # zonesigner example.com db.example.com
\end{verbatim}

The \struct{keyrec} file is created as \path{example.com.krf}.

The signed zone file is created in \path{db.example.com}.

Example 4.

This example shows a \struct{keyrec} file generated by \cmd{zonesigner}.

The command executed is:

\begin{verbatim}
    # zonesigner example.com db.example.com
\end{verbatim}

The generated \struct{keyrec} file contains six \struct{keyrec}s:  a zone
\struct{keyrec}, two set \struct{keyrec}s, one KSK \struct{keyrec}, and two
ZSK \struct{keyrec}s.

\begin{verbatim}
    zone        "example.com"
        zonefile        "example.com"
        signedzone      "db.example.com"
        endtime         "+2592000"
        kskcur          "signing-set-24"
        kskdirectory    "."
        zskcur          "signing-set-42"
        zskpub          "signing-set-43"
        zskdirectory    "."
        keyrec_type     "zone"
        keyrec_signsecs "1115166642"
        keyrec_signdate "Wed May  4 00:30:42 2005"

    set                "signing-set-24"
        zonename        "example.com"
        keys            "Kexample.com.+005+24082"
        keyrec_setsecs  "1110000042"
        keyrec_setdate  "Sat Mar  5 05:20:42 2005"

    set                "signing-set-42"
        zonename        "example.com"
        keys            "Kexample.com.+005+53135"
        keyrec_setsecs  "1115166640"
        keyrec_setdate  "Wed May  4 00:30:40 2005"

    set                "signing-set-43"
        zonename        "example.com"
        keys            "Kexample.com.+005+13531"
        keyrec_setsecs  "1115166641"
        keyrec_setdate  "Wed May  4 00:30:41 2005"

    key                "Kexample.com.+005+24082"
        zonename        "example.com"
        keyrec_type     "kskcur"
        algorithm       "rsasha1"
        random          "/dev/urandom"
        keypath         "./Kexample.com.+005+24082.key"
        ksklength       "1024"
        ksklife         "15768000"
        keyrec_gensecs  "1110000042"
        keyrec_gendate  "Sat Mar  5 05:20:42 2005"

    key                "Kexample.com.+005+53135"
        zonename        "example.com"
        keyrec_type     "zskcur"
        algorithm       "rsasha1"
        random          "/dev/urandom"
        keypath         "./Kexample.com.+005+53135.key"
        zsklength       "512"
        zsklife         "604800"
        keyrec_gensecs  "1115166638"
        keyrec_gendate  "Wed May  4 00:30:38 2005"

    key                "Kexample.com.+005+13531"
        zonename        "example.com"
        keyrec_type     "zskpub"
        algorithm       "rsasha1"
        random          "/dev/urandom"
        keypath         "./Kexample.com.+005+13531.key"
        zsklength       "512"
        zsklife         "604800"
        keyrec_gensecs  "1115166638"
        keyrec_gendate  "Wed May  4 00:30:38 2005"
\end{verbatim}

{\bf NOTES}

\begin{itemize}

\item One Zone in a \struct{keyrec} File\verb" "

There is a bug in the signing-set code that necessitates only storing one
zone in a \struct{keyrec} file.

\item SOA Serial Numbers\verb" "

Serial numbers in SOA records are merely incremented in this version.
Future plans are to allow for more flexible serial number manipulation.

\end{itemize}

{\bf SEE ALSO}

dnssec-keygen(8),
dnssec-signzone(8)

Net::DNS::SEC::Tools::conf.pm(3),
Net::DNS::SEC::Tools::defaults.pm(3),\\
Net::DNS::SEC::Tools::keyrec.pm(3),
Net::DNS::SEC::Tools::tooloptions.pm(3)


\clearpage

\subsubsection{\bf genkrf}

{\bf NAME}

\cmd{genkrf} - Generate a {\it keyrec} file from Key Signing Key (KSK)
and/or Zone Signing Key (ZSK) files.

{\bf SYNOPSIS}

\begin{verbatim}
    genkrf [options] <zone-file> [<signed-zone-file>]
\end{verbatim}

{\bf DESCRIPTION}

{\it genkrf} generates a {\it keyrec} file from KSK and/or ZSK files.  It
generates new KSK and ZSK keys if needed.

The name of the {\it keyrec} file to be generated is given by the
{\it -krfile} option.  If this option is not specified, \path{zone-name.krf}
is used as the name of the {\it keyrec} file.  If the {\it keyrec} file
already exists, it will be overwritten with new {\it keyrec} definitions.

The {\it zone-file} argument is required.  It specifies the name of the
zone file from which the signed zone file was created.  The optional
{\it signed-zone-file} argument specifies the name of the signed zone file.
If it is not given, then it defaults to \path{zone-file.signed}.

{\bf OPTIONS}

\cmd{genkrf} has a number of options that assist in creation of the {\it
keyrec} file.  These options will be set to the first value found from this
search path:

\begin{verbatim}
    command line options
    DNSSEC-Tools configuration file
    DNSSEC-Tools defaults
\end{verbatim}

See \perlmod{tooloptions.pm(3)} for more details.
Exceptions to this are given in the option descriptions below.

The \cmd{genkrf} options are described below.

{\bf General \cmd{genkrf} Options}

\begin{description}

\item {\it -zone zone-name}\verb" "

This option specifies the name of the zone.  If it is not given then
{\it zone-file} will be used as the name of the zone.

\item {\it -krfile keyrec-file}\verb" "

This option specifies the name of the {\it keyrec} file to be generated.
If it is not given, then \path{zone-name.krf} will be used.

\item {\it -algorithm algorithm}\verb" "

This option specifies the algorithm used to generate encryption keys.

\item {\it -endtime endtime}\verb" "

This option specifies the time that the signature on the zone expires,
measured in seconds.

\item {\it -random random-device}\verb" "

Source of randomness used to generate the zone's keys. See the man
page for \cmd{dnssec-signzone} for the valid format of this field.

\item {\it -verbose}\verb" "

Display additional messages during processing.  If this option is given at
least once, then a message will be displayed indicating the successful
generation of the {\it keyrec} file.  If it is given twice, then the values
of all options will also be displayed.

\item {\it -help}\verb" "

Display a usage message.

\end{description}

{\bf KSK-related Options}

\begin{description}

\item {\it -ksk KSK-name}\verb" "

This option specifies the KSK's key file being used to sign the zone.  If this
option is not given, a new KSK will be created.

\item {\it -kskdir KSK-directory}\verb" "

This option specifies the absolute or relative path of the directory
where the KSK resides.  If this option is not given, it defaults to
the current directory ``.''.

\item {\it -ksklength KSK-length}\verb" "

This option specifies the length of the KSK encryption key.

\item {\it -ksklife KSK-lifespan}\verb" "

This option specifies the lifespan of the KSK encryption key.  This lifespan
is {\bf not} inherent to the key itself.  It is {\bf only} used to determine
when the KSK must be rolled over.

\end{description}

{\bf ZSK-related Options}

\begin{description}

\item {\it -zskcur ZSK-name}\verb" "

This option specifies the current ZSK being used to sign the zone.
If this option is not given, a new ZSK will be created.

\item {\it -zskpub ZSK-name}\verb" "

This option specifies the published ZSK for the zone.  If this option
is not given, a new ZSK will be created.

\item {\it -zskcount ZSK-count}\verb" "

This option specifies the number of current and published ZSK keys that will
be generated.  If this option is not given, the default given in the
DNSSEC-Tools configuration file will be used.

\item {\it -zskdir ZSK-directory}\verb" "

This option specifies the absolute or relative path of the directory
where the ZSKs reside.  If this option is not given, it defaults to
the current directory ``.''.

\item {\it -zsklength ZSK-length}\verb" "

This option specifies the length of the ZSK encryption key.

\item {\it -zsklife ZSK-lifespan}\verb" "

This option specifies the lifespan of the ZSK encryption key.  This lifespan
is {\bf not} inherent to the key itself.  It is {\bf only} used to determine
when the ZSK must be rolled over.

\end{description}

{\bf SEE ALSO}

\cmd{dnssec-keygen(8)},
\cmd{dnssec-signzone(8)},
\cmd{zonesigner(8)}

\perlmod{Net::DNS::SEC::Tools::conf.pm(3)},
\perlmod{Net::DNS::SEC::Tools::defaults.pm(3)}, \\
\perlmod{Net::DNS::SEC::Tools::keyrec.pm(3)}

\path{conf(5)},
\path{keyrec(5)}

\clearpage

\subsubsection{\bf krfcheck}

{\bf NAME}

\cmd{krfcheck} - Check a DNSSEC-Tools {\it keyrec} file for problems and
inconsistencies.

{\bf SYNOPSIS}

\begin{verbatim}
    krfcheck [options] keyrec-file
\end{verbatim}

{\bf DESCRIPTION}

\cmd{krfcheck} checks a {\it keyrec} file for problems, potential problems,
and inconsistencies.

Recognized problems include:

\begin{description}

\item {\it no zones defined}\verb" "

The {\it keyrec} file does not contain any zone {\it keyrec}s.

\item {\it no sets defined}\verb" "

The {\it keyrec} file does not contain any set {\it keyrec}s.

\item {\it no keys defined}\verb" "

The {\it keyrec} file does not contain any key {\it keyrec}s.

\item {\it unknown zone {\it keyrec}s}\verb" "

A set {\it keyrec} or a key {\it keyrec} references a non-existent zone
{\it keyrec}.

\item {\it missing key from zone {\it keyrec}}\verb" "

A zone {\it keyrec} does not have both a KSK key and a ZSK key.

\item {\it missing key from set {\it keyrec}}\verb" "

A key listed in a set {\it keyrec} does not have a key {\it keyrec}.

\item {\it expired zone {\it keyrec}s}\verb" "

A zone has expired.

\item {\it mislabeled key}\verb" "

A key is labeled as a KSK (or ZSK) and its owner zone has it labeled as the
opposite.

\item {\it invalid zone data values}\verb" "

A zone's {\it keyrec} data are checked to ensure that they are valid.  The
following conditions are checked:  existence of the zone file, existence of
the KSK file, existence of the KSK and ZSK directories, the end-time is
greater than one day, and the seconds-count and date string match.

\item {\it invalid key data values}\verb" "

A key's {\it keyrec} data are checked to ensure that they are valid.  The
following conditions are checked:  valid encryption algorithm, key length
falls within algorithm's size range, random generator file exists, and the
seconds-count and date string match.

\end{description}

Recognized potential problems include:

\begin{description}

\item {\it imminent zone expiration}\verb" "

A zone will expire within one week. 

\item {\it odd zone-signing date}\verb" "

A zone's recorded signing date is later than the current system clock.

\item {\it orphaned keys}\verb" "

A key {\it keyrec} is unreferenced by any set {\it keyrec}.

\item {\it missing key directories}\verb" "

A zone {\it keyrec}'s key directories ({\it kskdirectory} or {\it zskdirectory}) does
not exist.

\end{description}

Recognized inconsistencies include:

\begin{description}

\item {\it key-specific fields in a zone {\it keyrec}}\verb" "

A zone {\it keyrec} contains key-specific entries.  To allow for site-specific
extensibility, \cmd{krfcheck} does not check for undefined {\it keyrec} fields.

\item {\it zone-specific fields in a key {\it keyrec}}\verb" "

A key {\it keyrec} contains zone-specific entries.  To allow for site-specific
extensibility, \cmd{krfcheck} does not check for undefined {\it keyrec} fields.

\item {\it mismatched zone timestamp}\verb" "

A zone's seconds-count timestamp does not match its textual timestamp.

\item {\it mismatched set timestamp}\verb" "

A set's seconds-count timestamp does not match its textual timestamp.

\item {\it mismatched key timestamp}\verb" "

A key's seconds-count timestamp does not match its textual timestamp.

\end{description}

{\bf OPTIONS}

\begin{description}

\item {\it -zone}\verb" "

Only perform checks of zone {\it keyrec}s.  This option may not be combined
with the {\bf -set} or {\bf -key} options.

\item {\it -set}\verb" "

Only perform checks of set {\it keyrec}s.  This option may not be combined
with the {\bf -zone} or {\bf -key} options.

\item {\it -key}\verb" "

Only perform checks of key {\it keyrec}s.  This option may not be combined
with the {\bf -set} or {\bf -zone} options.

\item {\it -count}\verb" "

Display a final count of errors.

\item {\it -quiet}\verb" "

Do not display messages.  This option supersedes the setting of the {\it -v}
option.

\item {\it -verbose}\verb" "

Display many messages.  This option is subordinate to the {\it -q} option.

\item {\it -Version}\verb" "

Display the \cmd{krfcheck} version number and exit.

\item {\it -help}\verb" "

Display a usage message.

\end{description}

{\bf SEE ALSO}

\cmd{cleankrf(8)},
\cmd{fixkrf(8)},
\cmd{lskrf(1)},
\cmd{zonesigner(8)}

\perlmod{Net::DNS::SEC::Tools::keyrec.pm(3)}

\path{keyrec(5)}

\clearpage

\subsection{{\it lskrf}}


{\bf NAME}

lskrf - List the {\it keyrec}s in a DNSSEC-Tools {\it keyrec} file.

{\bf SYNOPSIS}

\begin{verbatim}  lskrf [options] <keyrec-files>\end{verbatim}

{\bf DESCRIPTION}

This script lists the contents of the specified {\it keyrec} files.  All {\it
keyrec} files are loaded before the output is displayed.  If any {\it keyrec}s
have duplicated names, whether within one file or across multiple files, the
later {\it keyrec} will be the one whose data are displayed.  The output given
depends on the options selected.

{\bf OPTIONS}

{\bf lskrf} has three types of options it can take:  record-selection options,
record-attribute options, and output-style options.  These option sets are
detailed below.

{\bf Record-Selection Options}

These options select the types of {\it keyrec} that will be displayed.

\begin{description}

\item [-all]\verb" "

This option displays all the records in a {\it keyrec} file.

\item [-zones]\verb" "

This option displays the zones in a {\it keyrec} file.

\item [-keys]\verb" "

This option displays the keys in a {\it keyrec} file.

\item [-ksk]\verb" "

This option displays the KSK keys in a {\it keyrec} file.

\item [-zsk]\verb" "

This option displays the ZSK keys in a {\it keyrec} file.  It does not include
obsolete ZSK keys; the {\it -obs} option must be specified to display obsolete
keys.

\item [-cur]\verb" "

This option displays the current ZSK keys in a {\it keyrec} file.

\item [-new]\verb" "

This option displays the new ZSK keys in a {\it keyrec} file.

\item [-pub]\verb" "

This option displays the published ZSK keys in a {\it keyrec} file.

\item [-obs]\verb" "

This option displays the obsolete ZSK keys in a {\it keyrec} file.  This option
must be give if obsolete ZSK keys are to be displayed.

\end{description}

{\bf Record-Attribute Options}

These options select subsets of the {\it keyrec}s chosen by the
record-selection options. 

\begin{description}

\item [-valid]\verb" "

This option displays the valid zones in a {\it keyrec} file.
It implies the {\it -zones} option.

\item [-expired]\verb" "

This option displays the expired zones in a {\it keyrec} file.
It implies the {\it -zones} option.

\item [-ref]\verb" "

This option displays the referenced key {\it keyrec}s in a {\it keyrec} file.
If no record-selection options were specified, then the {\it -keys} option will
be set.

\item [-unref]\verb" "

This option displays the unreferenced key {\it keyrec}s in a {\it keyrec} file.
If no record-selection options were specified, then the {\it -keys} option will
be set.

\end{description}

{\bf Output-Format Options}

These options define how the {\it keyrec} information will be displayed.

Without any of these options, the zone name, zone file, zone-signing date,
and a label will be displayed for zones.  For types, the key name, the key's
zone, the key's generation date, and a label will be displayed if these
options aren't given.

\begin{description}

\item [-count]\verb" "

The count of matching records will be displayed, but the matching records
will not be.

\item [-nodate]\verb" "

The key's generation date will not be printed if this flag is given.

\item [-long]\verb" "

The long form of output will be given.  For zones, the zone name, the zone
file, the zone's signing date, the zone's expiration date, and a label will
be displayed.  For keys, the key name, the key's zone, the key's encryption
algorithm, the key's length, the key's generation date, and a label are
given.

\item [-terse]\verb" "

This options displays only the name of the zones or keys selected by other
options.

\item [-help]\verb" "

Display a usage message.

\end{description}

{\bf SEE ALSO}

\perlmod{Net::DNS::SEC::Tools::keyrec.pm(3)}


\clearpage

\subsubsection{expchk}

{\bf NAME}

\cmd{expchk} - Check a \struct{keyrec} file for expired zones

{\bf SYNOPSIS}

\begin{verbatim}
  expchk -all -expired -valid -warn numdays -zone zonename
                -count -help keyrec_files
\end{verbatim}

{\bf DESCRIPTION}

\cmd{expchk} checks a set of \struct{keyrec} files to determine if the zone
\struct{keyrec}s are valid or expired.  The type of zones displayed depends on
the options chosen; if no options are given the expired zones will be listed.

{\bf OPTIONS}

\begin{description}

\item {\bf -all}\verb" "

Display expiration information on all zones, expired or valid, in the
specified \struct{keyrec} files.

\item {\bf -expired}\verb" "

Display expiration information on the expired zones in the specified
\struct{keyrec} files.  This is the default action.

\item {\bf -valid}\verb" "

Display expiration information on the valid zones in the specified
\struct{keyrec} files.

\item {\bf -warn numdays}\verb" "

A warning will be given for each valid zone that will expire in {\it numdays}
days.  This option has no effect on expired zones.

\item {\bf -zone zonename}\verb" "

Display expiration information on the zone specified in {\it zonename}.

\item {\bf -count}\verb" "

Only the count of matching zones (valid or expired) will be given.  If both
types of zones are selected, then the count will be the number of zones in the
specified \struct{keyrec} files.

\item {\bf -help}\verb" "

Display a usage message.

\end{description}

{\bf SEE ALSO}

zonesigner(8)

Net::DNS::SEC::Tools::keyrec.pm(3)


\clearpage

\subsubsection{fixkrf}

{\bf NAME}

\cmd{fixkrf} - Fixes DNSSEC-Tools \struct{keyrec} files whose encryption key
files have been moved.

{\bf SYNOPSIS}

\begin{verbatim}
  fixkrf [options] <keyrec-file> <dir 1> ... <dir N>
\end{verbatim}

{\bf DESCRIPTION}

\cmd{fixkrf} checks a specified \struct{keyrec} file to ensure that the
referenced encryption key files exist where listed.  If a key is not where
the \struct{keyrec} specifies it should be, then \cmd{fixkrf} will search the
given directories for those keys and adjust the \struct{keyrec} to match
reality.  If a key of a particular filename is found in multiple places, a
warning will be printed and the \struct{keyrec} file will not be changed for
that key.

{\bf OPTIONS}

\begin{description}

\item {\bf -list}\verb" "

Display output about missing keys, but don't fix the \struct{keyrec} file.

\item {\bf -verbose}\verb" "

Display output about found keys as well as missing keys.

\item {\bf -help}\verb" "

Display a usage message.

\end{description}

{\bf SEE ALSO}

cleankrf(8),
genkrf(8),
lskrf(8),
zonesigner(8)

Net::DNS::SEC::Tools::keyrec.pm(3)

file-keyrec.pm(5)


\clearpage

\subsection{{\it clean-keyrec}}


{\bf NAME}

keyrec-clean - Clean a DNSSEC-Tools {\it keyrec} file of orphaned keys.

{\bf SYNOPSIS}

\begin{verbatim}
  keyrec-clean [options] <keyrec-files>
\end{verbatim}

{\bf DESCRIPTION}

This script cleans the orphaned {\it keyrec}s from a set of DNSSEC-Tools
{\it keyrec} files.  Orphaned keys are those keys which are not referenced
by a zone.  A warning is given if a key is found that is referenced by a
different zone {\it keyrec} than the one referenced by the key's {\it keyrec}.

{\it keyrec-clean}'s exit code is the count of orphaned keys found.

{\bf OPTIONS}

\begin{description}

\item [-count]\verb" "

Display a final count of orphaned keys.  This option allows the count to be
displayed even if the {\it -quiet} option is given.

\item [-list]\verb" "

The key {\it keyrec}s are checked for orphans, but are not removed from the
{\it keyrec} file.  The names of the orphaned {\it keyrec}s are displayed.

\item [-rm]\verb" "

Delete each the key files, both {\it .key} and {\it .private}, from orphaned and
expired {\it keyrec}s.

\item [-quiet]\verb" "

Display no output.

\item [-verbose]\verb" "

Display output about referenced keys as well as unreferenced keys.

\item [-help]\verb" "

Display a usage message.

\end{description}

{\bf SEE ALSO}

\perlmod{Net::DNS::SEC::Tools::keyrec.pm(3)}


\clearpage

\subsubsection{signset-editor}

{\bf NAME}

\cmd{signset-editor} - DNSSEC-Tools Signing Set GUI Editor

{\bf SYNOPSIS}

\begin{verbatim}
  signset-editor <keyrec-file>
\end{verbatim}

{\bf DESCRIPTION}

\cmd{signset-editor} provides the capability for easy management of signing
sets in a GUI.  A signing set contains zero or more names of key
\struct{keyrec}s.  These sets are used by other DNSSEC-Tools utilities for
signing zones.  The signing sets found in the given \struct{keyrec} file are
displayed in a new window.  New signing sets may be created and existing
signing sets may be modified or deleted from \cmd{signset-editor}.

\cmd{signset-editor} has two display modes.  The Signing Set Display shows the
names of all the set \struct{keyrec}s in the given \struct{keyrec} file.  The
Keyrec Display shows the names of all the key \struct{keyrec}s in the given
\struct{keyrec} file.  \cmd{signset-editor} starts in Signing Set Display
mode, but the mode can be toggled back and forth as needed.

An additional toggle controls the display of additional data.  If the Extended
Data toggle is turned on, then the Signing Set Display shows the names of the
key \struct{keyrec}s in each signing set and the Keyrec Display shows the
names of each signing set each key \struct{keyrec} is in.  If the Extended
Data toggle is turned off, then the Signing Set Display only shows the names
of the set \struct{keyrec}s and the Keyrec Display only shows the names key
\struct{keyrec}s.

\cmd{signset-editor} has a small number of commands.  These commands are all
available through the menus, and most have a keyboard accelerator.  The
commands are described in the next section.

Management of signing sets may be handled using a normal text editor.
However, \cmd{signset-editor} provides a nice GUI that {\bf only} manipulates
signing sets without the potential visual clutter of the rest of the
\struct{keyrec} entries.

{\bf UNDOING MODIFICATIONS}

\cmd{signset-editor} has the ability to reverse modifications it has made to a
\struct{keyrec} file.  This historical restoration will only work for
modifications made during a particular execution of \cmd{signset-editor};
modifications made during a previous execution may not be undone.

When undoing modifications, \cmd{signset-editor} does not necessarily restore
name-ordering within a \struct{keyrec}'s {\bf signing\_set} field.  However,
the signing-set data are maintained.  This means that an ``undone''
\struct{keyrec} file may not be exactly the same, byte-for-byte, as the
original file, but the proper meaning of the data is kept.

After a ``Save'' operation, the data required for reversing modifications are
deleted.  This is not the case for the ``Save As'' operation.

{\bf COMMANDS}

\cmd{signset-editor} provides the following commands, organized by menus:

\begin{itemize}

\item {\bf Open} (File menu)\verb" "

Open a new \struct{keyrec} file.  If the specified file does not exist, the
user will be prompted for the action to take.  If the user chooses the
``Continue'' action, then \cmd{signset-editor} will continue editing the current
\struct{keyrec} file.  If the ``Quit'' action is selected, then
\cmd{signset-editor} will exit.

\item {\bf Save} (File menu)\verb" "

Save the current \struct{keyrec} file.  The data for the ``Undo Changes''
command are purged, so this file will appear to be unmodified.

Nothing will happen if no changes have been made.

\item {\bf Save As} (File menu)\verb" "

Save the current \struct{keyrec} file to a name selected by the user.

\item {\bf Quit} (File menu)\verb" "

Exit \cmd{signset-editor}.

\item {\bf Undo Changes} (Edit menu)\verb" "

Reverse modifications made to the signing sets and keyrecs.  This is {\bf only}
for the in-memory version of the \struct{keyrec} file.

\item {\bf New Signing Set} (Commands menu)\verb" "

Create a new signing set.   The user is given the option of adding key
\struct{keyrec}s to the new set.

This command is available from both viewing modes.

\item {\bf Delete Signing Set/Key} (Commands menu)\verb" "

Delete the selected signing set or key.

This command is available from both viewing modes.  If used from the Signing
Set Display mode, then all the keys in the selected signing set will be
removed from that set.  If used from the Keyrec Display mode, then the
selected key will no longer be part of any signing set.

\item {\bf Modify Signing Set/Key} (Commands menu)\verb" "

Modify the selected signing set or key.

This command is available from both viewing modes.  If used from the Signing
Set Display mode, then the selected signing set may be modified by adding keys
to that set or deleting them from that set.  If used from the Keyrec Display
mode, then the selected key may be added to or deleted from any of the defined
signing sets.

\item {\bf View Signing Sets} (Display menu)\verb" "

The main window will display the \struct{keyrec} file's signing sets.  If
Extended Data are to be displayed, then each key \struct{keyrec} in the
signing set will also be shown.  If Extended data are not to be displayed,
then only the signing set names will be shown.

This command is a toggle that switches between View Signing Sets mode and View
Keyrecs mode.

\item {\bf View Keyrecs} (Display menu)\verb" "

The main window will display the names of the \struct{keyrec} file's key
\struct{keyrec}s.  If Extended Data are to be displayed, then the name of each
signing set of the \struct{keyrec} will also be shown.  If Extended data are
not to be shown, then only the \struct{keyrec} names will be displayed.

This command is a toggle that switches between View Keyrecs mode and View
Signing Sets mode.

\item {\bf Display Extended Data} (Display menu)\verb" "

Additional data will be shown in the main window.  For Signing Sets Display
mode, the names of the signing set and their constituent key \struct{keyrec}s
will be displayed.  For Keyrec Display mode, the names of the key
\struct{keyrec}s and the Signing Sets it is in will be displayed.

This command is a toggle that switches between Extended Data display and No
Extended Data display.

\item {\bf Do Not Display Extended Data} (Display menu)\verb" "

No additional data will be shown in the main window.  For Signing Sets Display
mode, only the names of the Signing Sets will be displayed.  For Keyrec
Display mode, only the names of the \struct{keyrec}s will be displayed.

This command is a toggle that switches between No Extended Data display and
Extended Data display.

\item {\bf Help} (Help menu)\verb" "

Display a help window.

\end{itemize}

\eject

{\bf KEYBOARD ACCELERATORS}

Below are the keyboard accelerators for the \cmd{signset-editor} commands:

\begin{table}[ht]
\begin{center}
\begin{tabular}{|c|l|}
\hline
{\bf Accelerator} & {\bf Function} \\
\hline
Ctrl-D & Delete Signing Set \\
Ctrl-E & Display Extended Data / Do Not Display Extended Data \\
Ctrl-H & Help \\
Ctrl-M & Modify Signing Set \\
Ctrl-N & New Signing Set \\
Ctrl-O & Open \\
Ctrl-Q & Quit \\
Ctrl-S & Save \\
Ctrl-U & Undo Changes \\
Ctrl-V & View Signing Sets / View Keyrecs \\
Ctrl-W & Close Window (New Signing Set, Modify Signing Set, Help) \\
\hline
\end{tabular}
\end{center}
\caption{Keyboard Accelerators for \cmd{signset-editor}}
\end{table}

These accelerators are all lowercase letters.

{\bf REQUIREMENTS}

\cmd{signset-editor} is implemented in Perl/Tk, so both Perl and Perl/Tk must
be installed on your system.

{\bf SEE ALSO}

cleankrf(8),
fixkrf(8),
genkrf(8),
krfcheck(8),
lskrf(1),
zonesigner(8)

Net::DNS::SEC::Tools::keyrec(3)

file-keyrec(5)



\clearpage
\subsection{\bf Zone-Rollover Commands}
\label{ssect-cmds-roll}

The zone-rollover commands provide tools to assist in DNSSEC zone
rollover and keeping records about those zones' rollover status.  The
commands in this group with together with the Zone-Signing Commands
(see Section~\ref{ssect-cmds-sign}.)  These commands are:

\begin{table}[ht]
\begin{center}
\begin{tabular}{ll}
\cmd{rollerd}	& the DNSSEC-Tools key rollover daemon			      \\
\cmd{rollinit}	& create a new {\it rollrec} file			      \\
\cmd{rollchk}	& verify the validity of the contents of a {\it rollrec} file \\
\cmd{lsroll}	& list the contents of a {\it rollrec} file		      \\
\cmd{rollctl}	& communicate with the \cmd{rollerd} rollover daemon	      \\
\cmd{rolllog}	& write a message to the rollover log file		      \\
\cmd{blinkenlights} & GUI tool for monitoring and controlling \cmd{rollerd}   \\
\end{tabular} 
\end{center}
\end{table}

\clearpage

\subsubsection{\bf rollerd}

{\bf NAME}

\cmd{rollerd} - DNSSEC-Tools daemon to manage DNSSEC key rollover.

{\bf SYNOPSIS}

\begin{verbatim}
    rollerd [-options] -rrfile <rollrec_file>
\end{verbatim}

{\bf DESCRIPTION}

The \cmd{rollerd} daemon manages key rollover for zones.  The Pre-Publish
Method of key rollover is used for ZSK key rollovers.  (Currently,
\cmd{rollerd} only handles ZSK rollover.) This method has four phases that are
entered when it is time to perform the ZSK rollover:

\begin{enumerate}
\item wait for old zone data to expire from caches
\item sign the zone with the KSK and Published ZSK
\item wait for old zone data to expire from caches
\item adjust keys in keyrec and sign the zone with new Current ZSK
\end{enumerate}

\cmd{rollerd} uses the \cmd{zonesigner} command during rollover phases 2 and 4.
\cmd{zonesigner} will generate keys as required and sign the zone during these
two phases.

The Pre-Publish Method of key rollover is defined in the Step-by-Step DNS
Security Operator Guidance Document.  See that document for more detailed
information.

The zones to be managed by \cmd{rollerd} are defined in a {\it rollrec} file.
Each zone's entry contains data needed by \cmd{rollerd} and some data useful
to a user.  Below is a sample {\it rollrec} entry:

\begin{verbatim}
        roll "example.com"
                zonefile        "example.com.signed"
                keyrec          "example.com.krf"
                curphase        "3"
                maxttl          "2400"
                phasestart      "Thu May  4 19:19:21 2006"
\end{verbatim}

The first line gives the {\it rollrec} entry's name.  The ``roll'' keyword
indicates that \cmd{rollerd} should include the zone in its roll queue.  Using
``skip'' in place of ``roll'' allows a zone to be stored in the {\it rollrec}
file, but it will not be included in rollover processing.  The first three
fields tell \cmd{rollerd} where to find example.com's signed zone file and
{\it keyrec} file and the zone's current rollover phase.  The last two are
for reference by the user.  The {\it maxttl} field is derived from the signed
zone file.

If either of the {\it zonefile} or {\it keyrec} files do not exist, then a
``roll'' {\it rollrec} will be changed into a ``skip'' {\it rollrec}.  That
record will not be processed.

The \cmd{rollctl} command is used to control the behavior of \cmd{rollerd}.
A number of commands are available, such as starting or stopping rollover for
a selected zone or all zones, turning on or off a GUI rollover display, and
halting \cmd{rollerd} execution.  The communications path between
\cmd{rollerd} and \cmd{rollctl} is operating system-dependent.  On Unix-like
systems, it is a Unix pipe that should {\bf only} be writable by root.

{\bf A Note About Files and Filenames}

There are a number of files and filenames used by \cmd{rollerd} and
\cmd{zonesigner}.  The user must be aware of the files used by these programs,
where the files are located, and where the programs are executed.

By default, \cmd{rollerd} will change directory to the DNSSEC-Tools directory,
though this may be changed by the {\it -directory} option.  Any programs
started by \cmd{rollerd}, most importantly \cmd{zonesigner}, will run in this
same directory.  If files and directories referenced by these programs are
named with relative paths, those paths must be relative to this directory.

The {\it rollrec} entry name is used as a key to the {\it rollrec} file and to
the zone's {\it keyrec} file.  This entry does not have to be the name of the
entry's domain, but it is a very good idea to make it so.  Whatever is used
for this entry name, the same name {\bf must} be used for the zone
{\it keyrec} in that zone's {\it keyrec} file.

It is probably easiest to store {\it rollrec} files, {\it keyrec} files, zone
files, and key files in a single directory.

{\bf INITIALIZATION AND USAGE}

The following steps must be taken to initialize and use \cmd{rollerd}.  This
assumes that zone files have been created, and that BIND and DNSSEC-Tools
have been installed.

\begin{enumerate}

\item sign zones\verb" "

The zones to be managed by \cmd{rollerd} must be signed.  Use \cmd{zonesigner}
to create the signed zone files and the {\it keyrec} files needed by
\cmd{rollerd}.  The {\it rollrec} file created in the next step {\bf must} use
the {\it keyrec} file names and the signed zone file names created here.

\item create {\it rollrec} file\verb" "

Before \cmd{rollerd} may be used, a {\it rollrec} file must first be created.
While this file may be built by hand, the \cmd{rollinit} command was
written specifically to build the file.

\item select operational parameters\verb" "

A number of \cmd{rollerd}'s operational parameters are taken from the
DNSSEC-Tools configuration file.  However, these may be overridden
by command-line options.  See the {\bf OPTIONS} section below for more details.
If non-standard parameters are desired to always be used, the appropriate
fields in the DNSSEC-Tools configuration file may be modified to use these
values.

\item install the rollover configuration\verb" "

The complete rollover configuration -- \cmd{rollerd}, {\it rollrec} file,
DNSSEC-Tools configuration file values, zone files -- should be installed.  
The appropriate places for these locations are both installation-dependent
and operating system-dependent.

\item test the rollover configuration\verb" "

The complete rollover configuration should be tested.  

Edit the zone files so that their zones have short TTL values.  A one-minute
TTL should be sufficient.  Test rollovers of this speed should {\bf only} be
done in a test environment without the real signed zone.

Run the following command:

\begin{verbatim}
    rollerd -rrfile test.rollrec -logfile - -loglevel info -sleep 60
\end{verbatim}

This command assumes the test {\it rollrec} file is \path{test.rollrec}.  It
writes a fair amount of log messages to the terminal, and checks its queue
every 60 seconds.  Follow the messages to ensure that the appropriate actions,
as required by the Pre-Publish Method, are taking place.

\item set \cmd{rollerd} to start at boot\verb" "

Once the configuration is found to work, \cmd{rollerd} should be set to start
at system boot.  The actual operations required for this step are operating
system-dependent.

\item reboot and verify\verb" "

The system should be rebooted and the \cmd{rollerd} logfile checked to ensure
that \cmd{rollerd} is operating properly.

\end{enumerate}

{\bf OPTIONS}

The following options are recognized:

\begin{description}

\item {\it -rrfile rollrec\_file}\verb" "

Name of the {\it rollrec} file to be processed.  This is the only required
``option''.

\item {\it -directory dir}\verb" "

Sets the \cmd{rollerd} execution directory.  This must be a valid directory.

\item {\it -logfile log\_file}\verb" "

Sets the \cmd{rollerd} log file to {\it log\_file}.  This must be a valid
logging file, meaning that if {\it logfile} already exists, it must be a
regular file.  The only exceptions to this are if {\it logfile} is
\path{/dev/stdout}, \path{/dev/tty}, and \path{-}.
Of these three, using a {\it logfile} of \path{-} is preferable since Perl
will properly convert the \path{-} to the process' standard output.


\item {\it -loglevel level}\verb" "

Sets \cmd{rollerd}'s logging level to {\it level}.  \perlmod{rollmgr.pm(3)}
contains a list of valid logging levels.

\item {\it -sleep sleeptime}\verb" "

Sets \cmd{rollerd}'s sleep time to {\it sleeptime}.  The sleep time is the
amount of time \cmd{rollerd} waits between processing its {\it rollrec}-based
queue.

\item {\it -parameters}\verb" "

Prints a set of \cmd{rollerd} parameters and then exits.

\item {\it -display}\verb" "

Starts the \cmd{blinkenlights} graphical display program to show the status of
zones managed by \cmd{rollerd}.

\item -{\it help}\verb" "

Display a usage message.

\item {\it -verbose}\verb" "

Verbose output will be given.

\end{description}

{\bf ASSUMPTIONS}

\cmd{rollerd} uses the \cmd{rndc} command to communicate with the BIND
\cmd{named} daemon.  Therefore, it assumes that appropriate measure have been
taken so that this communication is possible.

{\bf KNOWN PROBLEMS}

The following problems (or potential problems) are known:

\begin{description}

\item - Only deals with ZSK rollover.

\item - Any process that can write to the rollover socket can send commands to
\cmd{rollerd}.  This is probably not a Good Thing.

\item - {\it rollrec} files do not have zone state updated when the zones start
or stop rollover as a result of \cmd{rollctl} commands.

\item - No testing with zone files and key files not in the process' directory.

\end{description}

{\bf POSSIBLE ENHANCEMENTS}

The following potential enhancements may be made:

\begin{description}

\item - It'd be good to base \cmd{rollerd}'s sleep time on when the next
operation must take place, rather than a simple seconds count.

\item - It'd be nice to allow each {\it rollrec} entry to specify its own
logging level.

\end{description}

{\bf SEE ALSO}

\cmd{blinkenlights(8)},
\cmd{named(8)},
\cmd{rndc(8)},
\cmd{rollchk(8)},
\cmd{rollctl(8)},	\\
\cmd{rollinit(8)},
\cmd{zonesigner(8)}

\perlmod{Net::DNS::SEC::Tools::conf.pm(3)},
\perlmod{Net::DNS::SEC::Tools::defaults.pm(3)}, \\
\perlmod{Net::DNS::SEC::Tools::keyrec.pm(3)},
\perlmod{Net::DNS::SEC::Tools::rollmgr.pm(3)}, \\
\perlmod{Net::DNS::SEC::Tools::rollrec.pm(3)}

\path{rollrec(5)}

\clearpage

\subsubsection{rollinit}

{\bf NAME}

\cmd{rollinit} - Create new \struct{rollrec} records for a DNSSEC-Tools
\struct{rollrec} file

{\bf SYNOPSIS}

\begin{verbatim}
  rollinit [options] <zonename1> ... <zonenameN>
\end{verbatim}

{\bf DESCRIPTION}

\cmd{rollinit} creates new \struct{rollrec} entries for a \struct{rollrec}
file.  This \struct{rollrec} file will be used by \cmd{rollerd} to manage key
rollover for the named domains.

A \struct{rollrec} entry has this format:

\begin{verbatim}
    roll "example.com"
        zonefile        "example.com.signed"
        keyrec          "example.com.krf"
        kskphase        "0"
        zskphase        "0"
        administrator   "bob@bobhost.example.com"
        directory       "/var/dns/zones/example.com"
        loglevel        "phase"
        ksk_rolldate    " "
        ksk_rollsecs    "0"
        zsk_rolldate    " "
        zsk_rollsecs    "0"
        maxttl          "604800"
        display         "1"
        phasestart      "Mon Jan 9 16:00:00 2006"
\end{verbatim}

The {\it zonefile} and \struct{keyrec} fields are set according to command-line
options and arguments.  The manner of generating the \struct{rollrec}'s actual
values is a little complex and is described in the ZONEFILE And KEYREC FIELDS
section below.

The {\it administrator} field is set to ``bob\@bobhost.example.com'' to indicate
that the email messages to the zone's administrator should be sent to
``bob\@bobhost.example.com''.

The {\it directory} field is set to ``/var/dns/zones/example.com''
to indicate that the files for this zone should be found in
\path{/var/dns/zones/example.com}.  This includes the zone file,
the signed zone file, and the \struct{keyrec} file.

The {\it loglevel} field is set to ``phase'' to indicate that \cmd{rollerd}
should only log phase-level (and greater) log messages for this zone.

The {\it kskphase} field is set to 0 to indicate that the zone is in normal
operation (non-rollover) for KSK keys.  The {\it zskphase} field is set to 0
to indicate that the zone is in normal operation (non-rollover) for ZSK keys.

The {\it ksk\_rolldate} and {\it ksk\_rollsecs} fields are set to indicate
that the zone has not yet undergone KSK rollover.

The {\it zsk\_rolldate} and {\it zsk\_rollsecs} fields are set to indicate
that the zone has not yet undergone ZSK rollover.

The {\it display} field is set to indicate that \cmd{blinkenlights} should
display the record.  The {\it maxttl} and {\it phasestart} fields are set to
dummy values.

The keywords {\bf roll} and {\bf skip} indicate whether \cmd{rollerd} should
process or ignore a particular \struct{rollrec} entry.  {\bf roll} records are
created by default; {\bf skip} entries are created if the {\it -skip} option
is specified.

The newly generated \struct{rollrec} entries are written to standard output,
unless the {\it -out} option is specified.

{\bf ZONEFILE and KEYREC FIELDS}

The {\it zonefile} and \struct{keyrec} fields may be given by using the {\it
-zone} and {\it -keyrec} options, or default values may be used.

The default values use the \struct{rollrec}'s zone name, taken from the
command line, as a base.  \path{.signed} is appended to the domain name
for the zone file; \path{.krf} is appended to the domain name for the
\struct{keyrec} file.

If {\it -zone} or {\it -keyrec} are specified, then the options values are
used in one of two ways:

\begin{itemize}

\item A single domain name is given on the command line.\verb" "

The option values for {\it -zone} and/or {\it -keyrec} are used for the actual
\struct{rollrec} fields.

\item Multiple domain names are given on the command line.\verb" "

The option values for {\it -zone} and/or {\it -keyrec} are used as templates
for the actual \struct{rollrec} fields.  The option values must contain the
string {\bf =}.  This string is replaced by the domain whose \struct{rollrec}
is being created.

\end{itemize}

See the EXAMPLES section for examples of how options are used by \cmd{rollinit}.

{\bf OPTIONS}

\cmd{rollinit} may be given the following options:

\begin{description}

\item {\bf -zone zonefile}\verb" "

This specifies the value of the {\it zonefile} field.
See the ZONEFILE And KEYREC FIELDS and EXAMPLES sections for more details.

\item {\bf -keyrec keyrec-file}\verb" "

This specifies the value of the \struct{keyrec} field.
See the ZONEFILE And KEYREC FIELDS and EXAMPLES sections for more details.

\item {\bf -admin}\verb" "

This specifies the value of the {\it administrator} field.  If it is not given,
an {\it administrator} field will not be included for the record.

\item {\bf -directory}\verb" "

This specifies the value of the {\it directory} field.  If it is not given,
a {\it directory} field will not be included for the record.

\item {\bf -loglevel}\verb" "

This specifies the value of the {\it loglevel} field.  If it is not given, a
{\it loglevel} field will not be included for the record.

\item {\bf -skip}\verb" "

By default, {\bf roll} records are generated.  If this option is given, then
{\bf skip} records will be generated instead.

\item {\bf -out output-file}\verb" "

The new \struct{rollrec} entries will be appended to {\it output-file}.
The file will be created if it does not exist.

If this option is not given, the new \struct{rollrec} entries will be written
to standard output.

\item {\bf -help}\verb" "

Display a usage message.

\item {\bf -Version}\verb" "

Display version information for \cmd{rollinit} and DNSSEC-Tools.

\end{description}

{\bf EXAMPLES}

The following options should make clear how \cmd{rollinit} deals with options
and the new \struct{rollrec}s.  Example 1 will show the complete new
\struct{rollrec} record.  For the sake of brevity, the remaining examples will
only show the newly created {\it zonefile} and \struct{keyrec} records.

{\bf Example 1.  One domain, no options}

This example shows the \struct{rollrec} generated by giving \cmd{rollinit} a
single domain, without any options.

\begin{verbatim}
    $ rollinit example.com
        roll    "example.com"
            zonefile        "example.com.signed"
            keyrec          "example.com.krf"
            kskphase        "0"
            zskphase        "0"
            ksk_rolldate    " "
            ksk_rollsecs    "0"
            zsk_rolldate    " "
            zsk_rollsecs    "0"
            maxttl          "0"
            display         "1"
            phasestart      "new"
\end{verbatim}

{\bf Example 2.  One domain, -zone option}

This example shows the \struct{rollrec} generated by giving \cmd{rollinit} a
single domain, with the {\it -zone} option.

\begin{verbatim}
    $ rollinit -zone signed-example example.com
        roll    "example.com"
            zonefile        "signed-example"
            keyrec          "example.com.krf"
\end{verbatim}

{\bf Example 3.  One domain, -keyrec option}

This example shows the \struct{rollrec} generated by giving \cmd{rollinit} a
single domain, with the {\it -keyrec} option.

\begin{verbatim}
    $ rollinit -keyrec x-rrf example.com
        roll    "example.com"
            zonefile        "example.com.signed"
            keyrec          "x-rrf"
\end{verbatim}

{\bf Example 4.  One domain, -zone and -keyrec options}

This example shows the \struct{rollrec} generated by giving \cmd{rollinit} a
single domain, with the {\it -zone} and {\it -keyrec} options.

\begin{verbatim}
    $ rollinit -zone signed-example -keyrec example.rrf example.com
        roll    "example.com"
            zonefile        "signed-example"
            keyrec          "example.rrf"
\end{verbatim}

{\bf Example 5.  One domain, -skip option}

This example shows the \struct{rollrec} generated by giving \cmd{rollinit} a
single domain, with the {\it -zone} and {\it -keyrec} options.

\begin{verbatim}
    $ rollinit -skip example.com
        skip    "example.com"
            zonefile        "example.com.signed"
            keyrec          "example.com.krf"
\end{verbatim}

{\bf Example 6.  Multiple domains, no options}

This example shows the \struct{rollrec}s generated by giving \cmd{rollinit}
several domains, without any options.

\begin{verbatim}
    $ rollinit example1.com example2.com
        roll    "example1.com"
                zonefile        "example1.com.signed"
                keyrec          "example1.com.krf"

        roll    "example2.com"
                zonefile        "example2.com.signed"
                keyrec          "example2.com.krf"
\end{verbatim}

{\bf Example 7.  Multiple domains, -zone option}

This example shows the \struct{rollrec}s generated by giving \cmd{rollinit}
several domains, with the {\it -zone} option.

\begin{verbatim}
    $ rollinit -zone =-signed example1.com example2.com
        roll    "example1.com"
                zonefile        "example1.com-signed"
                keyrec          "example1.com.krf"

        roll    "example2.com"
                zonefile        "example2.com-signed"
                keyrec          "example2.com.krf"
\end{verbatim}

{\bf Example 8.  Multiple domains, -keyrec option}

This example shows the \struct{rollrec}s generated by giving \cmd{rollinit}
several domains, with the {\it -keyrec} option.

\begin{verbatim}
    $ rollinit -keyrec zone-=-keyrec example1.com example2.com
        roll    "example1.com"
                zonefile        "example1.com.signed"
                keyrec          "zone-example1.com-keyrec"

        roll    "example2.com"
                zonefile        "example2.com.signed"
                keyrec          "zone-example2.com-keyrec"
\end{verbatim}

{\bf Example 9.  Multiple domains, -zone and -keyrec options}

This example shows the \struct{rollrec}s generated by giving \cmd{rollinit}
several domains, with the {\it -zone} and {\it -keyrec} options.

\begin{verbatim}
    $ rollinit -zone Z-= -keyrec =K example1.com example2.com
        roll    "example1.com"
                zonefile        "Z-example1.com"
                keyrec          "example1.comK"

        roll    "example2.com"
                zonefile        "Z-example2.com"
                keyrec          "example2.comK"
\end{verbatim}

{\bf Example 10.  Single domain, -zone and -keyrec options with template}

This example shows the \struct{rollrec} generated by giving \cmd{rollinit} a
single domain, with the {\it -zone} and {\it -keyrec} options.  The options
use the multi-domain {\bf =} template.

\begin{verbatim}

    $ rollinit -zone Z-= -keyrec =.K example.com
        roll    "example.com"
                zonefile        "Z-="
                keyrec          "=.K"

\end{verbatim}

This is probably not what is wanted, since it results in the {\it zonefile}
and \struct{keyrec} field values containing the {\bf =}.

{\bf Example 11.  Multiple domains, -zone and -keyrec options without template}

This example shows the \struct{rollrec}s generated by giving \cmd{rollinit}
several domains, with the {\it -zone} and {\it -keyrec} options.  The options
do not use the multi-domain {\bf =} template.

\begin{verbatim}
    $ rollinit -zone ex.zone -keyrec ex.krf example1.com example2.com
        roll    "example1.com"
                zonefile        "ex.zone"
                keyrec          "ex.krf"

        roll    "example2.com"
                zonefile        "ex.zone"
                keyrec          "ex.krf"
\end{verbatim}

This may not be what is wanted, since it results in the same {\it zonefile}
and \struct{keyrec} fields values for each \struct{rollrec}.

{\bf SEE ALSO}

lsroll(1)

rollerd(8),
rollchk(8),
zonesigner(8)

Net::DNS::SEC::Tools::keyrec.pm(3),
Net::DNS::SEC::Tools::rollrec.pm(3)

Net::DNS::SEC::Tools::file-keyrec.pm(5),
Net::DNS::SEC::Tools::file-rollrec.pm(5)


\clearpage

\subsubsection{\bf rollchk}

{\bf NAME}

\cmd{rollchk} - Check a DNSSEC-Tools {\it rollrec} file for problems and
inconsistencies.

{\bf SYNOPSIS}

\begin{verbatim}
    rollchk [-roll|-skip] [-count] [-quiet] [-verbose] [-help] rollrec-file
\end{verbatim}

{\bf DESCRIPTION}

This script checks the {\it rollrec} file specified by {\it rollrec-file} for
problems and inconsistencies.

Recognized problems include:

\begin{description}

\item {\it non-existent rollrec file}\verb" "

The specified {\it rollrec} file does not exist.

\item {\it no zones defined}\verb" "

No zones are defined in the specified {\it rollrec} file.

\item {\it invalid rollover phase}\verb" "

A zone has an invalid rollover phase.  These phases may be 0, 1, 2, 3, or 4;
any other value is invalid.

\item {\it invalid display flag}\verb" "

A zone has an invalid display flag.  This flag may be 0 or 1;
any other value is invalid.

\item {\it non-positive maxttl}\verb" "

The maximum TTL value must be greater than zero.

\item {\it zone file checks}\verb" "

Several checks are made for a zone's zone file.  The zone file must exist, it
must be a regular file, and it must not be of zero length.

\item {\it keyrec file checks}\verb" "

Several checks are made for a zone's {\it keyrec} file.  The {\it keyrec} file
must exist, it must be a regular file, and it must not be of zero length.

\end{description}

{\bf OPTIONS}

\begin{description}

\item {\it -roll}\verb" "

Only display {\it rollrec}s that are active (``roll'') records.
This option is mutually exclusive of the {\it -skip} option.

\item {\it -skip}\verb" "

Only display {\it rollrec}s that are inactive (``skip'') records.
This option is mutually exclusive of the {\it -roll} option.

\item {\it -count}\verb" "

Display a final count of errors.

\item {\it -quiet}\verb" "

Do not display messages.  This option supersedes the setting of the {\it -v}
option.

\item {\it -verbose}\verb" "

Display many messages.  This option is subordinate to the {\it -q} option.

\item {\it -help}\verb" "

Display a usage message.

\end{description}


{\bf SEE ALSO}

\cmd{lsroll(8)},
\cmd{rollerd(8)},
\cmd{rollinit(8)}

\perlmod{Net::DNS::SEC::Tools::rollrec.pm(3)}

\path{rollrec(5)}

\clearpage

\subsubsection{\bf lsroll}

{\bf NAME}

\cmd{lsroll} - List the {\it rollrec}s in a DNSSEC-Tools {\it rollrec} file.

{\bf SYNOPSIS}

\begin{verbatim}
    lsroll [options] <rollrec-files>
\end{verbatim}

{\bf DESCRIPTION}

This script lists the contents of the specified {\it rollrec} files.  All
{\it rollrec} files are loaded before the output is displayed.  If any
{\it rollrec}s have duplicated names, whether within one file or across
multiple files, the later {\it rollrec} will be the one whose data are
displayed.

Each record's name is always included in the output.  Additional output
depends on the options selected.

{\bf OPTIONS}

There are three types of options recognized by \cmd{lsroll}:  record-selection
options, attribute-selection options, and output-format options.  Each type
is described in the sections below.

{\bf Record-selection Options}

These options select the records that will be displayed by \cmd{lsroll}.

\begin{description}

\item {\it -all}\verb" "

List all records in the {\it rollrec} file.

\item {\it -roll}\verb" "

List all ``roll'' records in the {\it rollrec} file.

\item {\it -skip}\verb" "

List all ``skip'' records in the {\it rollrec} file.

\end{description}

{\bf Attribute-selection Options}

These options select the attributes of the records that will be displayed
by \cmd{lsroll}.

\begin{description}

\item {\it -type}\verb" "

Include each {\it rollrec} record's type in the output.  The type will be
either ``roll'' or ``skip''.  The type is given parenthetically.

\item {\it -zone}\verb" "

The record's zonefile is included in the output.  This field is part
of the default output.

\item {\it -keyrec}\verb" "

The record's {\it keyrec} file is included in the output.  This field is part
of the default output.

\item {\it -phase}\verb" "

The record's rollover phase is included in the output.  This field is part of
the default output.

\item {\it -ttl}\verb" "

The record's TTL value is included in the output.

\item {\it -display}\verb" "

The record's display flag, used by \cmd{blinkenlights}, is included in the
output.

\item {\it -phstart}\verb" "

The record's rollover phase is included in the output.

\end{description}

{\bf Output-format Options}

These options select the type of output that will be given by \cmd{lsroll}.

\begin{description}

\item {\it -count}\verb" "

Only a count of matching keyrecs in the {\it rollrec} file is given.

\item {\it -terse}\verb" "

Terse output is given.  Only the record name and any other fields specifically
selected are included in the output.

\item -help\verb" "

Display a usage message.

\end{description}

{\bf SEE ALSO}

\cmd{blinkenlights(8)},
\cmd{rollchk(8)},
\cmd{rollinit(8)},
\cmd{rollerd(8)}

\perlmod{Net::DNS::SEC::Tools::rollrec.pm(3)}

\path{rollrec(5)}

\clearpage

\subsubsection{\bf rollctl}

{\bf NAME}

\cmd{rollctl} - Send commands to the DNSSEC-Tools rollover daemon.

{\bf SYNOPSIS}

\begin{verbatim}
    rollctl [options]
\end{verbatim}

{\bf DESCRIPTION}

The \cmd{rollctl} command sends commands to the DNSSEC-Tools rollover daemon,
\cmd{rollerd}.  Multiple options may be specified on a single command line and
they will be executed in {\it alphabetical} order.  The exception to this
ordering is that the {\it -shutdown} command will always be executed last.

In most cases, \cmd{rollerd} will send a response to \cmd{rollctl}.
\cmd{rollctl} will print a success or failure message, as appropriate.

{\bf OPTIONS}

The following options are handled by \cmd{rollctl}.

\begin{description}

\item {\it -halt}\verb" "

Cleanly halts \cmd{rollerd} execution.

\item {\it -logfile logfile}\verb" "

Sets the \cmd{rollerd} log file to {\it logfile}.  This must be a valid
logging file, meaning that if {\it logfile} already exists, it must be a
regular file.  The only exceptions to this are if {\it logfile} is
\path{/dev/stdout} or \path{/dev/tty}.

\item {\it -loglevel loglevel}\verb" "

Sets the \cmd{rollerd} logging level to {\it loglevel}.  This must be one of
the valid logging levels defined in \perlmod{rollmgr.pm(3)}.

\item {\it -rollall}\verb" "

Initiates rollover for all the zones defined in the current \cmd{rollrec}
file.

\item {\it -rollrec rollrec\_file}\verb" "

Sets the \cmd{rollrec} file to be processed by \cmd{rollerd} to
{\it rollrec\_file}.

\item {\it -rollzone zone}\verb" "

Initiates rollover for the zone named by {\it zone}.

\item {\it -runqueue}\verb" "

Wakes up \cmd{rollerd} and has it run its queue of {\it rollrec} entries.

\item {\it -shutdown}\verb" "

Synonym for {\it -halt}.

\item {\it -skipall}\verb" "

Stops rollover for all zones in the current {\it rollrec} file.

\item {\it -skipzone zone}\verb" "

Stops rollover for the zone named by {\it zone}.

\item {\it -sleeptime sleeptime}\verb" "

Sets \cmd{rollerd}'s sleep time to {\it sleeptime}.  {\it sleeptime} must be
an integer at least as large as the {\bf \$MIN\_SLEEP} value in \cmd{rollerd}.

\item {\it -status}\verb" "

Retrieves and prints several of \cmd{rollerd}'s operational parameters.
The parameters are also written to the log file.

\item {\it -zonestatus}\verb" "

Retrieves and prints the status of the zones managed by \cmd{rollerd}.
Status is also written to the log file.

For each zone in the {\it rollrec} file, the zone name, the record type
(``skip'' or ``roll''), and the current rollover phase are given.

\item {\it -quiet}\verb" "

Prevents output from being given.  Both error and non-error output is stopped.

\item {\it -help}\verb" "

Displays a usage message.

\end{description}

{\bf FUTURE}

The following modifications may be made in the future:

\begin{description}

\item {\it command execution order}\verb" "

The commands will be executed in the order given on the command line rather
than in alphabetical order.

\end{description}

{\bf SEE ALSO}

\cmd{rollerd(8)}

\perlmod{Net::DNS::SEC::Tools::rollmgr.pm(3)},
\perlmod{Net::DNS::SEC::Tools::rollrec.pm(3)}

\clearpage

\subsubsection{rolllog}

{\bf NAME}

\cmd{rolllog} - DNSSEC-Tools utility to write messages to the DNSSEC rollover
log file

{\bf SYNOPSIS}

\begin{verbatim}
  rolllog -loglevel <level> <log_message>
\end{verbatim}

{\bf DESCRIPTION}

The \cmd{rolllog} program writes log messages to the DNSSEC rollover log
file.  \cmd{rolllog} does not actually write the messages itself; rather,
it sends them to the \cmd{rollerd} rollover daemon to write the messages.
\cmd{rollerd} keeps track of a logging level, and only messages of that
level or higher are written to the log file.

{\bf OPTIONS}

The following options are recognized:

\begin{description}

\item {\bf -loglevel level}\verb" "

Logging level of this message.  The valid levels are defined in
\perlmod{rollmgr.pm}(3).   This option is required.

\item {\bf -help}\verb" "

Display a usage message.

\item {\bf -Version}\verb" "

Display a version message.

\end{description}

{\bf SEE ALSO}

rollctl(8),
rollerd(8)

Net::DNS::SEC::Tools::rollmgr.pm(3)


\clearpage

\subsubsection{blinkenlights}

{\bf NAME}

\cmd{blinkenlights} - DNSSEC-Tools rollerd GUI

{\bf SYNOPSIS}

\begin{verbatim}

  blinkenlights <rollrec-file>

\end{verbatim}

{\bf DESCRIPTION}

\cmd{blinkenlights} is a GUI tool for use with monitoring and controlling the
DNSSEC-Tools \cmd{rollerd} program.  It displays information on the current
state of the zones \cmd{rollerd} is managing.  The user may control some
aspects of \cmd{rollerd}'s execution using \cmd{blinkenlights} menu commands.

\cmd{blinkenlights} creates a window in which to display information about
each zone \cmd{rollerd} is managing.  (These zones are those in
\cmd{rollerd}'s current \struct{rollrec} file.)  As a zone's rollover status
changes, \cmd{blinkenlights} will update its display for that zone.  Skipped
zones, zones listed in the \struct{rollrec} file but which are not in rollover
or normal operation, are displayed but have very little useful information to
display.

The user may also select a set of zones to hide from the display.  These
zones, if in the rolling state, will continue to roll; however, their zone
information will not be displayed.  Display state for each zone will persist
across \cmd{blinkenlights} executions.

Menu commands are available for controlling \cmd{rollerd}.  The commands which
operate on a single zone may be executed by keyboard shortcuts.  The zone may
be selected either by clicking in its ``zone stripe'' or by choosing from a
dialog box.  Display and execution options for \cmd{blinkenlights} are also
available through menu commands.  More information about the menu commands is
available in the MENU COMMANDS section.

\cmd{blinkenlights} is only intended to be started by \cmd{rollerd}, not
directly by a user.  There are two ways to have \cmd{rollerd} start
\cmd{blinkenlights}.  First, \cmd{rollctl} may be given the {\it -display}
option.  Second, the {\it -display} option may be given on \cmd{rollerd}'s
command line.

{\bf SCREEN LAYOUT}

The \cmd{blinkenlights} window is laid out as a series of ``stripes''.  The
top stripe contains status information about \cmd{rollerd}, the second stripe
contains column headers, and the bulk of the window consists of zone stripes.
The list below provides more detail on the contents of each stripe.

See the WINDOW COLORS section for a discussion of the colors used for the
zone stripes.

\begin{itemize}

\item \cmd{rollerd} information stripe\verb" "

The information stripe contains four pieces of information:  \cmd{rollerd}'s
current \struct{rollrec} file, the count of rolling zones, the count of
skipped zones, and the amount of time \cmd{rollerd} waits between processing
its queue.  Coincidentally, that last datum is also the amount of time between
\cmd{blinkenlights} screen updates.

\item column headers stripe\verb" "

This stripe contains the column headers for the columns of each zone stripe.

\item zone stripes\verb" "

Each zone managed by \cmd{rollerd} (i.e., every zone in the current
\struct{rollrec} file) will have a zone stripe which describes that zone's
current state.  The stripe is divided into four sections:  the zone name,
the current rollover state, and the zone's DNSSEC keys.

The zone name section just contains the name of the zone.

The rollover state section contains the rollover phase number, a text
explanation of the phase, and the amount of time remaining in that rollover
phase.  The phase explanation is ``normal operation'' when the zone isn't
currently in rollover.

The DNSSEC key section contains two subsections, one for the zone's ZSK keys
and another for the zone's KSK keys.  Each subsection contains the names of
the signing sets active for the zone.  The ZSK subsection lists the Current,
Published, and New ZSK keys; the KSK subsection lists the Current and
Published.

See the WINDOW COLORS section for a discussion of the colors used for the
zone stripes.

\end{itemize}

{\bf WINDOW COLORS}

The default \cmd{blinkenlights} configuration uses window coloring to provide
visual cues and to aid in easily distinguishing zone information.  The default
window coloring behavior gives each zone stripe has its own color and the
rollover state section of each zone stripe is shaded to show the zone's phase.
Window coloring can be turned off (and on) with configuration options and menu
commands.

{\bf Color Usage}

The two window coloring behaviors are discussed more fully below:

\begin{itemize}

\item zone stripe colors\verb" "

Each rolling zone's stripe is given one of three colors:  blue, red, or green.
The color is assigned on a top-down basis and the colors wrap if there are
more than three zones.  So, the first zone is always blue, the second zone
red, the third zone green, the fourth zone blue, etc.

The colors do not stay with a particular zone.  If a rolling zone becomes a
skipped zone, the zone stripes will be reassigned new colors to account for
that skipped zone.

Skipped zones are not colored with these three colors.  Stripes for skipped
zones are colored either grey or a color set in the configuration file.  If
you choose to use a non-standard color for skipped zones your should ensure
that it is {\bf not} one of the colors used for rolling zones' stripes.
Modifying the {\it skipcolor} configuration field allows the skipped-zone color
to be changing.

The {\it colors} configuration field can be used to turn on or off the use of
colors for zone stripes.  If stripe coloring is turned off, then every stripe
will be displayed using the {\it skipcolor} color.

\item rollover-state shading\verb" "

The only portion of a zone stripe that changes color is the status column; the
color of the rest of the zone stripe stays constant.  Before a zone enters
rollover, the status column is the same color as the rest of the stripe.  When
the zone enters rollover, the status column's color is changed to a very light
shade of the stripe's normal color.  As the rollover phases progress towards
rollover completion, the status column's shade darkens.  Once rollover
completes, the status column returns again to the same shade as the rest of
that stripe.

The {\it shading} configuration field can be used to turn on or off the use of
shading in the rollover-state column.  If shading is turned off, then the zone
stripe will be a solid color.

See the CONFIGURATION FILE section for information on setting the
configuration fields.

\end{itemize}

{\bf Colors Used}

The color names are taken from the X11 \path{rgb.txt} file (X11 1.1.3 -
XFree86 4.4.0 for MacOS X.)  If these aren't available in your \path{rgb.txt}
file, similar names should be selected.  The actual red/green/blue values used
are given below to assist in finding suitable replacements.  These values were
taken from the \path{rgb.txt} file.

Blue Shades:

\begin{table}[ht]
\begin{center}
\begin{tabular}{|l|r|r|r|}
\hline
{\bf Color Name} & {\bf Red Value} & {\bf Green Value} & {\bf Blue Value} \\
\hline
blue            &   0 &   0 & 255 \\
lightblue2      & 178 & 223 & 238 \\
darkslategray1  & 151 & 255 & 255 \\
skyblue1        & 135 & 206 & 255 \\
steelblue1      &  99 & 184 & 255 \\
turquoise1      &   0 & 245 & 255 \\
cornflower blue & 100 & 149 & 237 \\
dodger blue     &  30 & 144 & 255 \\
\hline
\end{tabular}
\end{center}
\caption{Blue Shades}
\end{table}

\eject

Red Shades:

\begin{table}[ht]
\begin{center}
\begin{tabular}{|l|r|r|r|}
\hline
{\bf Color Name} & {\bf Red Value} & {\bf Green Value} & {\bf Blue Value} \\
\hline
red          &    255 &   0 &   0 \\
pink         &    255 & 192 & 203 \\
lightsalmon1 &    255 & 160 & 122 \\
tomato       &    255 &  99 &  71 \\
indianred    &    205 &  92 &  92 \\
violetred1   &    255 &  62 & 150 \\
orangered1   &    255 &  69 &   0 \\
firebrick1   &    255 &  48 &  48 \\
\hline
\end{tabular}
\end{center}
\caption{Red Shades}
\end{table}

Green Shades:

\begin{table}[ht]
\begin{center}
\begin{tabular}{|l|r|r|r|}
\hline
{\bf Color Name} & {\bf Red Value} & {\bf Green Value} & {\bf Blue Value} \\
\hline
green           &   0 & 255 &   0 \\
darkseagreen1   & 193 & 255 & 193 \\
darkolivegreen1 & 202 & 255 & 112 \\
lightgreen      & 144 & 238 & 144 \\
seagreen1       &  84 & 255 & 159 \\
spring green    &   0 & 255 & 127 \\
greenyellow     & 173 & 255 &  47 \\
lawngreen       & 124 & 252 &   0 \\
\hline
\end{tabular}
\caption{Green Shades}
\end{center}
\end{table}

{\bf MENU COMMANDS}

A number of menu commands are available to control the behavior of
\cmd{blinkenlights} and to send commands to \cmd{rollerd}.  These
commands are discusses in this section.

{\bf File Menu}

The commands in this menu are basic GUI commands.

\begin{itemize}

\item Quit\verb" "

\cmd{blinkenlights} will stop execution.

\end{itemize}

{\bf Options Menu}

The commands in this menu control the appearance and behavior of
\cmd{blinkenlights}.

\begin{itemize}

\item Row Colors (toggle)\verb" "

This menu item is a toggle to turn on or off the coloring of zone stripes.
If row coloring is turned off, zone stripes will all be the same color.
If row coloring is turned on, zone stripes will be displayed in varying
colors.  See the WINDOW COLORS section for a discussion of row coloring.

\item Status Column Shading (toggle)\verb" "

This menu item is a toggle to turn on or off the shading of the zone status
column.  If shading is turned off, the zone stripes will present a solid,
unchanging band of color for each zone.  If shading is turned on, the color
of the zone status column will change according to the zone's rollover state.

\item Skipped Zones Display (toggle)\verb" "

This menu item is a toggle to turn on or off the display of skipped zones.  If
display is turned off, zone stripes for skipped zones will not be displayed.
If display is turned on, zone stripes for all zones will be displayed.

\item Modification Commands (toggle)\verb" "

In some situations, it may be desirable to turn off \cmd{blinkenlights}'
ability to send commands to \cmd{rollerd}.  This menu item is a toggle to turn
on or off this ability.  If the commands are turned off, then the ``Zone
Control'' menu and keyboard shortcuts are disabled.  If the commands are
turned on, then the ``Zone Control'' menu and keyboard shortcuts are enabled.

\item Font Size\verb" "

This menu item allows selection of font size of text displayed in the main
window.

Normally, changing the font size causes the window to grow and shrink as
required.  However, on Mac OS X there seems to be a problem when the size
selected increases the window size to be greater than will fit on the screen.
If the font size is subsequently reduced, the window size does not shrink in
response.

\end{itemize}

{\bf General Control Menu}

The commands in this menu are GUI interfaces for the \cmd{rollctl} commands
related to {\it general} zone management.

\begin{itemize}

\item Roll Selected Zone\verb" "

The selected zone will be moved to the rollover state.  This only has an
effect on skipped zones.  A zone may be selected by clicking on its zone
stripe.  If this command is selected without a zone having been selected,
a dialog box is displayed from which a currently skipped zone may be chosen.

\item Roll All Zones\verb" "

All zones will be moved to the rollover state.  This has no effect on
currently rolling zones.

\item Run the Queue\verb" "

\cmd{rollerd} is awoken and runs through its queue of zones.  The operation
required for each zone is then performed.

\item Skip Selected Zone\verb" "

The selected zone will be moved to the skipped state.  This only has an effect
on rolling zones.  A zone may be selected by clicking on its zone stripe.  If
this command is selected without a zone having been selected, a dialog box is
displayed from which a currently rolling zone may be chosen.

\item Skip All Zones\verb" "

All zones will be moved to the skipped state.  This has no effect on
currently skipped zones.

\item Halt Rollerd\verb" "

\cmd{rollerd}'s execution is halted.  As a result, \cmd{blinkenlights}'
execution will also be halted.

\end{itemize}

{\bf KSK Control Menu}

The commands in this menu are GUI interfaces for the \cmd{rollctl} commands
related to KSK-specific zone management.

\begin{itemize}

\item DS Published Selected Zone\verb" "

This command is used to indicate that the selected zone's parent has published
a new DS record for the zone.  It moves the zone from phase 6 to phase 7 of
KSK rollover.

\item DS Published All Zones\verb" "

This command is used to indicate that all the zones in KSK rollover phase 6
have new DS records published by their parents.  It moves all these zones from
phase 6 to phase 7 of KSK rollover.

\end{itemize}

{\bf Zone Display Menu}

The commands in this menu are GUI interfaces parts of the zone display.  There
are commands for displaying and hiding both zone stripes and key columns.  The
commands allow all, some, or none of the zone stripes and key columns to be
displayed.  Undisplayed rolling zones will continue to roll, but they will do
so without the \cmd{blinkenlights} window indicating this.

\begin{itemize}

\item Zone Selection\verb" "

A dialog box is created that holds a list of the zones currently managed by
\cmd{rollerd}.  The user may select which zones should be displayed by clicking
on the zone's checkbox.  Zones with a selected checkbox will be displayed;
zones without a selected checkbox will not be displayed.

\item Display All Zones\verb" "

All zones will be displayed in the \cmd{blinkenlights} window.

\item Hide All Zones\verb" "

No zones will be displayed in the \cmd{blinkenlights} window.

\item KSK Sets (toggle)\verb" "

This menu item is a toggle to turn on or off the display of KSK signing set
names.  If display is turned off, the columns holding the KSK signing set
names and labels will be removed from the display and the display window will
shrink.  If display is turned on, the columns holding the KSK signing set
names and labels will be restored to the display and the display window will
be expanded.

When displayed, KSK signing sets will always be the right-most columns.

\item ZSK Sets (toggle)\verb" "

This menu item is a toggle to turn on or off the display of ZSK signing set
names.  If display is turned off, the columns holding the ZSK signing set
names and labels will be removed from the display and the display window will
shrink.  If display is turned on, the columns holding the ZSK signing set
names and labels will be restored to the display and the display window will
be expanded.

When displayed, ZSK signing sets will always be immediately to the right of
the zone status column.

\item Hide All Keysets\verb" "

Turns off display of the KSK and ZSK signing set names.

\item Show All Keysets\verb" "

Turns on display of the KSK and ZSK signing set names.

\end{itemize}

{\bf Help Menu}

The commands in this menu provide assistance to the user.

\begin{itemize}

\item Help\verb" "

Display a window containing help information.

\end{itemize}

{\bf CONFIGURATION FILE}

Several aspects of \cmd{blinkenlights}' behavior may be controlled from
configuration files.  Configuration value may be specified in the DNSSEC-Tools
configuration file or in a more specific \path{rc.blinkenlights}.  The
system-wide \cmd{blinkenlights} configuration file is in the DNSSEC-Tools
configuration directory and is named \path{blinkenlights.conf}.  Multiple
\path{rc.blinkenlights} files may exist on a system, but only the one in the
directory in which \cmd{blinkenlights} is executed is used.

The following are the available configuration values:

\begin{table}[ht]
\begin{center}
\begin{tabular}{|l|c|l|}
\hline
{\bf Configuration Value} & {\bf Meaning} \\
colors    & Turn on/off use of colors on zone stripes \\
fontsize  & The size of the font in the output window \\
modify    & Turn on/off execution of rollerd modification commands \\
shading   & Turn on/off shading of the status columns \\
showskip  & Turn on/off display of skipped zones \\
skipcolor & The background color used for skipped zones \\
\hline
\end{tabular}
\end{center}
\caption{\cmd{blinkenlights} Configuration File Entries}
\end{table}

The \path{rc.blinkenlights} file is {\bf only} searched for in the directory
in which \cmd{blinkenlights} is executed.  The potential problems inherent in
this may cause these \cmd{blinkenlights}-specific configuration files to be
removed in the future.

This file is in the ``field value'' format, where {\it field} specifies the
output aspect and {\it value} defines the value for that field.  The following
are the recognized fields:

Empty lines and comments are ignored.  Comment lines are lines that start with
an octothorpe (`\#').

Spaces are not allowed in the configuration values.

Choose your skipcolors carefully.  The only foreground color used is black, so
your background colors must work well with black.

{\bf REQUIREMENTS}

\cmd{blinkenlights} is implemented in Perl/Tk, so both Perl and Perl/Tk must be
installed on your system.

{\bf WARNINGS}

\cmd{blinkenlights} has several potential problems that must be taken into
account.

\begin{description}

\item development environment\verb" "

\cmd{blinkenlights} was developed and tested on a single-user system running
X11.  While it works fine in this environment, it has not been run on a system
with many users or in a situation where the system console hasn't been in use
by the \cmd{blinkenlights} user.

\item long-term performance issues\verb" "

In early tests, the longer \cmd{blinkenlights} runs, the slower the updates
become.  This is {\it probably} a result of the Tk implementation or the way
Tk interfaces with X11.  This is pure supposition, though.

This performance impact is affected by a number of things, such as the number
of zones managed by \cmd{rollerd} and the length of \cmd{rollerd}'s sleep
interval.  Large numbers of zones or very short sleep intervals will increase
the possibility of \cmd{blinkenlights}' performance degrading.

This appears to have been resolved by periodically performing a complete
rebuild of the screen.  \cmd{blinkenlights} keeps track of the number of
screen updates it makes and rebuilds the screen when this count exceeds a
threshold.  The threshold is built into \cmd{blinkenlights} and stored in the
\var{\$PAINTMAX} variable.  This threshold may be adjusted if there are too
many screen rebuilds or if \cmd{blinkenlights}' performance slows too much.
Raising the number will reduce the screen rebuilds; lowering the number will
(may) increase performance.

\end{description}

{\bf SEE ALSO}

rollctl(8),
rollerd(8),
zonesigner(8)

Net::DNS::SEC::Tools::timetrans(3)

Net::DNS::SEC::Tools::keyrec(5),
Net::DNS::SEC::Tools::rollrec(5),




%%%%%%%%%%%%%%%%%%%%%%%%%%%%%%%%%%%%%%%%%%%%%%%%%%%%%%%%%%%%%%%%%%%%%%%%%%%%%%

\clearpage

\markboth{DNSSEC-Tools Software User Manual (vers. 4) -- Manual Pages}{DNSSEC-Tools Software User Manual (vers. 4) -- Manual Pages}
\section{DNSSEC Library Routines}
\markboth{DNSSEC-Tools Software User Manual (vers. 4) -- Manual Pages}{DNSSEC-Tools Software User Manual (vers. 4) -- Manual Pages}
\label{sect-libraries}


Several libraries have been developed to provide DNSSEC-validated resolution,
translation, and querying services.  These DNSSEC-Tools libraries are:

\begin{description}

\item{\lib{libsres}} - Secure Resolver Library

\lib{libsres} provides the resolver component of the ``validating resolver''.
It is capable of recursively obtaining answers for an application (validator)
from a DNSSEC-aware name server.  Resolver policy will eventually be used to
control the flags (CD, RD etc) that are sent in the query to the name servers,
as well as other parameters, such as the name server to which the query is
to be sent.

This library provides very basic functionality for name resolution.  The data
structures and interfaces exported to applications are still in a state of
flux and are expected to change. Many corner cases are still not supported.

\item{\lib{libval}} - DNSSEC validation library

\lib{libval} provides DNSSEC resource-record validation functionality.  It
relies on the resolver component to fetch answers from a DNSSEC-aware name
server.

As of now there is no functionality to traverse the chain-of-trust while
performing record validation and also no support for validation policies.
As such, the interfaces defined herein are in a state of flux and are
expected to change.

\end{description}

This section contains man pages describing these libraries.

\clearpage

\subsection{{\bf query\_send()} Secure Resolver Library Routines}


{\bf NAME}

query\_send, response\_rcv, get, free\_name\_server, free\_name\_servers -
send queries and receive responses from a DNS name server.

print\_response - Display answers returned from the name server

{\bf SYNOPSIS}

\begin{verbatim}
  #include <resolver.h>

  int query_send( const char    *name,
            const u_int16_t     type,
            const u_int16_t     class,
            struct name_server  *nslist,
            int                 *trans_id);

  int response_recv(int         *trans_id,
            struct name_server  **respondent,
            u_int8_t            **response,
            u_int32_t           *response_length);

  int get(const char          *name_n,
          const u_int16_t     type_h,
          const u_int16_t     class_h,
          struct name_server  *nslist,
          struct name_server  **respondent,
          u_int8_t            **response,
          u_int32_t           *response_length);

  void free_name_server(struct name_server **ns);

  void free_name_servers(struct name_server **ns);

  void print_response(u_int8_t *response, int response_length);
\end{verbatim}

{\bf DESCRIPTION}

The {\bf query\_send()} function can be used to send a query comprised of the
$<${\it name, class, type}$>$ tuple to the name servers specified in {\it
nslist}.  {\it trans\_id} provides a handle to this transaction within the
{\it libsres} library.

The {\bf response\_recv()} function returns the answers, if available, from the
name server that responds within the transaction identified by {\it trans\_id}.
The response is available in {\it response} and the responding name server is
returned in {\it respondent}.  The length of the response in bytes is returned
in {\it response\_length}.

The {\bf get()} function provides a wrapper around the {\bf query\_send()} and
{\bf response\_recv()} functions.  It blocks until a response is received from
some name server or the request times out.  The {\it libsres} library does
not automatically send out recursive queries; referral requests are also
treated as valid responses.

The memory pointed to by {\it *respondent} is allocated by the {\it libsres}
library and this must be freed by the invoker using {\bf free\_name\_server()}.
An entire list of name servers can be freed using {\bf free\_name\_servers()}.

{\bf print\_response()} provides a convenient way to display answers returned
in {\it response} by the name server.

{\it struct name\_server} is defined in {\bf resolver.h} as follows.

\begin{verbatim}
  struct name_server
  {
        u_int8_t *ns_name_n;
        void *ns_tsig_key;
        u_int32_t ns_security_options;
        u_int32_t ns_status;
        struct name_server *ns_next;
        int ns_number_of_addresses;
        struct sockaddr ns_address[1];
  };
\end{verbatim}


\begin{description}

\item [{\it ns\_name\_n}]\verb" "

The name of the zone for which this name server is authoritative.  This field
provides a convenient way for the invoker to index a list of name servers
while sending queries to different name servers, especially during a referral.
It is not used directly by the resolver and can be set to an empty string.

\item [{\it ns\_tsig\_key}]\verb" "

The {\it tsig} key that should be used to protect messages sent to this name
server.  This field is currently unused.

\item [{\it ns\_security\_options}]\verb" "

The security options for the zone.  This can be set to either ZONE\_USE\_NOTHING
or ZONE\_USE\_TSIG.

\item [{\it ns\_status}]\verb" "

The status of the zone.  This field is used internally by the invoker to
maintain properties of the zone.  Currently defined values for this field are
SR\_ZI\_STATUS\_UNSET, SR\_ZI\_STATUS\_PERMANENT and SR\_ZI\_STATUS\_LEARNED.

\item [{\it ns\_next}]\verb" "

The address of the next name server in the list.

\item [{\it ns\_number\_of\_addresses}]\verb" "

The number of elements in the array {\it ns\_addresses}.  This field is
currently unused.

\item [{\it ns\_addresses}]\verb" "

The IP address of the name server.  Currently, only IPv4 addresses can be
stored.

\end{description}

{\bf OTHER SYMBOLS EXPORTED}

The {\it libsres} library also exports the following BIND symbols:
\begin{packed}
\item {\it \_\_ns\_name\_ntop}
\item {\it \_\_ns\_name\_pton}
\item {\it \_\_ns\_name\_unpack}
\item {\it \_\_p\_class}
\item {\it \_\_p\_section}
\item {\it \_\_p\_type}
\end{packed}

Documentation for these symbols can be found in the BIND sources and
documentation manuals.

{\bf RETURN VALUES}

\begin{description}

\item [SR\_UNSET]\verb" "

No error.

\item [SR\_CALL\_ERROR]\verb" "

An invalid parameter was passed to {\bf get()}, {\bf query\_send()}, or
{\bf response\_recv()}.

\item [SR\_MEMORY\_ERROR]\verb" "

Memory allocation failed.

\item [SR\_MKQUERY\_INTERNAL\_ERROR]\verb" "

An internal error was encountered while trying to construct a
query message.

\item [SR\_TSIG\_INTERNAL\_ERROR]\verb" "

An internal error was encountered while trying to construct a
signed TSIG message.

\item [SR\_SEND\_INTERNAL\_ERROR]\verb" "

An internal error was encountered while trying to send the
message to the name server(s).

\item [SR\_NO\_ANSWER\_YET]\verb" "

No answer currently available; the query is still active.

\item [SR\_NO\_ANSWER]\verb" "

No answers were received from any name server.

\item [SR\_RCV\_INTERNAL\_ERROR]\verb" "

An internal error was encountered while trying to receive
responses from a name server.

\item [SR\_WRONG\_ANSWER]\verb" "

The header bits did not correctly identify the message as a response.

\item [SR\_HEADER\_BADSIZE]\verb" "

The length and count of records in the header were incorrect.

\item [SR\_TSIG\_ERROR]\verb" "

TSIG validation on the response message failed.

\item [SR\_NXDOMAIN]\verb" "

The queried name did not exist.

\item [SR\_FORMERR]\verb" "

The name server was not able to parse the query message.

\item [SR\_SERVFAIL]\verb" "

The name server was not reachable.

\item [SR\_NOTIMPL]\verb" "

A particular functionality is not yet implemented.

\item [SR\_REFUSED]\verb" "

The name server refused to answer this query.

\item [SR\_GENERIC\_FAILURE]\verb" "

Other failure returned by the name server and reflected in the
returned message RCODE.

\item [SR\_EDNS\_VERSION\_ERROR]\verb" "

Wrong EDNS version used.  Not implemented.

\item [SR\_UNSUPP\_EDNS0\_LABEL]\verb" "

Unsupported EDNS version used.  Not implemented.

\item [SR\_SUSPICIOUS\_BIT]\verb" "

A bit in the header was set to an unexpected value.  Not implemented.

\item [SR\_NAME\_EXPANSION\_FAILURE]\verb" "

Could not expand name from wire format.  Not used.

\end{description}

{\bf CURRENT STATUS}

There is currently no support for IPv6.

There is limited support for specifying resolver policy; members of the
{\it struct name\_server} are still subject to change.

The library is not thread-safe.

{\bf SEE ALSO}

{\bf libval(3)}

\url{http://dnssec-tools.sourceforge.net}


\clearpage

\subsection{\bf libval Library}

{\bf NAME}

\func{val\_resolve\_and\_check()}, \func{val\_free\_result\_chain()} - query
and validate answers from a DNS name server

\func{val\_istrusted()} - check if status value corresponds to that of a
trustworthy answer

\func{val\_create\_context()}, \func{val\_free\_context()},
\func{val\_switch\_policy\_scope()} - manage validator context

\func{dnsval\_conf\_get()}, \func{resolver\_config\_get()},
\func{root\_hints\_get()} - get the current location for the validator
configuration files

\func{dnsval\_conf\_set()}, \func{resolver\_config\_set()},
\func{root\_hints\_set()} - set the current location for the validator
configuration files

\func{p\_ac\_status()}, \func{p\_val\_status()} - display validator status
information

{\bf SYNOPSIS}

\begin{verbatim}
    #include <validator.h>

    int val_resolve_and_check(val_context_t               *context,
                              u_char                      *domain_name_n,
                              const u_int16_t             type,
                              const u_int16_t             class,
                              const u_int8_t              flags,
                              struct val_result_chain     **results);

    void val_free_result_chain(struct val_result *results);

    int val_istrusted(val_status_t val_status);

    int val_create_context(const char *label, val_context_t **newcontext);

    void val_free_context(val_context_t *context);

    char *resolver_config_get(void);

    int resolver_config_set(const char *name);

    char *root_hints_get(void);

    int root_hints_set(const char *name);

    char *dnsval_conf_get(void);

    int dnsval_conf_set(const char *name);

    char *p_ac_status(val_astatus_t valerrno);

    char *p_val_status(val_status_t valerrno);

\end{verbatim}

{\bf DESCRIPTION}

The \func{val\_resolve\_and\_check()} function queries a set of name servers
for the \var{$<$domain\_name\_n, type, class$>$} tuple and to verifies and
validates the response.  Verification involves checking the RRSIGs, and
validation is verification up the chain-of-trust to a trust anchor.  The
\var{domain\_name\_n} parameter is the queried name in DNS wire format.  The
conversion from host format to DNS wire format can be done using the
\func{ns\_name\_pton()} function exported by the \lib{libsres(3)} library.

Answers returned by \func{val\_resolve\_and\_check()} are made available in
the \var{*results} array.  Each answer is a distinct RRset; multiple RRs
within the RRset are treated as the same answer.  Multiple answers are
possible when \var{type} is \var{ns\_t\_any}.

Individual elements in \var{*results} point to the authentication chain linked
list.  The authentication chain elements contain the actual RRsets returned by
the name server in response to the query.

Most applications only require the status value within \var{*results} since
this provides a single error code for representing the authenticity of
returned data.  Other more intrusive applications, such as a DNSSEC
troubleshooting utility, may look at individual authentication chain elements
to identify what particular component in the chain-of-trust led to a
validation failure.  \func{val\_istrusted()} is a helper function that easily
identifies if a given validator status value corresponds to one of the
authenticated and/or trusted data codes.  Validator status values returned in
the \var{val\_result\_chain} and \var{val\_authentication\_chain} linked lists
can be can be converted into ASCII format using the \func{p\_val\_status()}
and \func{p\_ac\_status()} functions.

The \lib{libval} library internally allocates memory for \var{*results} and
this must be freed by the invoking application using the
\func{free\_result\_chain()} interface.

The first parameter to \func{val\_resolve\_n\_check()} is the validator
context.  Applications can create a new validator context using the
\func{val\_create\_context()} function.  This function parses the resolver and
validator configuration files and creates the handle \var{newcontext} to this
parsed information.  Information stored as part of validator context includes
the validation policy and resolver policy.  Validator and resolver policy are
read by default from the \path{/etc/dnsval.conf} and \path{/etc/resolv.conf}
files.  ``Root hints'' that allows the library to bootstrap its lookup process
when functioning as a full resolver is read from \path{/etc/root.hints}.  The
locations of each of these files may be changed using the
\func{dnsval\_conf\_set}, \func{resolver\_config\_set} and
\func{root\_hints\_set} interfaces.  The corresponding ``get'' interfaces,
(namely, \func{dnsval\_conf\_get}, \func{resolver\_config\_get} and
\func{root\_hints\_get}) can be used to return the current location from where
these configuration files are read.

Applications can use local policy to influence the validation outcome.
Examples of local policy elements include trust anchors for different zones
and untrusted algorithms for cryptographic keys and hashes.  Local policy may
vary for different applications and operating scenarios.

Local policy for the validator is stored in the configuration file,
\path{/etc/dnsval.conf}.  Policies are identified by simple text strings
called labels, which must be unique within the configuration system.  For
example, ``browser'' could be used as the label that defines the validator
policy for all web-browsers in a system.  A label value of ``:'' identifies
the default policy, the policy that is used when a NULL context is specified
as the \var{ctx} parameter for interfaces such as
\func{val\_resolve\_and\_check()}, \func{val\_getaddrinfo()}, and
\func{val\_gethostbyname()}.  The default policy is unique within the
configuration system.

{\bf DATA STRUCTURES}

\begin{description}

\item \struct{struct val\_result\_chain}\verb" "

\begin{verbatim}
  struct val_result_chain}
  {
      val_status_t                     val_rc_status;
      struct val_authentication_chain *val_rc_answer;
      int                              val_rc_proof_count;
      struct val_authentication_chain *val_rc_proofs[MAX_PROOFS];
      struct val_result_chain         *val_rc_next;
  };
\end{verbatim}

\begin{description}

\item \var{val\_rc\_answer}\verb" "

Authentication chain for a given RRset.

\item \var{val\_rc\_next}\verb" "

Pointer to the next RRset in the set of answers returned for a query.

\item \var{val\_rc\_proofs}\verb" "

Pointer to any proofs that were returned for the query.

\item \var{val\_rc\_proof\_count}\verb" "

Number of proof elements stored in \var{val\_rc\_proofs}.

\item \var{val\_rc\_status}\verb" "

Validation status for a given RRset.  This can be one of the following:

\begin{itemize}

\item \const{VAL\_SUCCESS}\verb" "

Answer received and validated successfully.

\item \const{VAL\_LOCAL\_ANSWER}\verb" "

Answer was available from a local file.

\item \const{VAL\_BARE\_RRSIG}\verb" "

No DNSSEC validation possible, query was for an RRSIG.

\item \const{VAL\_NONEXISTENT\_NAME}\verb" "

No name was present and a valid proof of non-existence confirming the missing
name (NSEC or NSEC3 span) was returned.  The proof was verified and the
authentication chains for each component in the proof led to a trust anchor.

\item \const{VAL\_NONEXISTENT\_TYPE}\verb" "

No type exists for the name and a valid proof of non-existence confirming the
missing name (NSEC or NSEC3 span) was returned.  The proof was verified and
the authentication chains for each component in the proof led to a trust
anchor.

\item \const{VAL\_ERROR}\verb" "

Did not have sufficient or relevant data to complete validation, or
encountered a DNS error.

\item \const{VAL\_DNS\_ERROR\_BASE} $+$ \const{SR\_}error\verb" "

This value contains a resolver error from \lib{libsres}.  The \lib{libsres}
error is \\
added to \const{VAL\_DNS\_ERROR\_BASE}, so this value will lie between \\
\const{VAL\_DNS\_ERROR\_BASE} and \const{VAL\_DNS\_ERROR\_LAST}.

\item \const{VAL\_INDETERMINATE}\verb" "

Lacking information to give a more conclusive answer.

\item \const{VAL\_BOGUS}\verb" "

Validation failure condition.

\item \const{VAL\_NOTRUST}\verb" "

All available components in the authentication chain verified properly, but
there was no trust anchor available.

\item \const{VAL\_PROVABLY\_UNSECURE}\verb" "

The record or some ancestor of the record in the authentication chain towards
the trust anchor was known to be provably unsecure.

\end{itemize}

Error values in \var{val\_status\_t} returned by the validator can be displayed
in a more user-friendly format using \func{p\_val\_status()}.

\end{description}

\item \struct{struct val\_authentication\_chain}\verb" "

\begin{verbatim}
  struct val_authentication_chain
  {
      val_astatus_t                    val_ac_status;
      struct val_rrset                *val_ac_rrset;
      struct val_authentication_chain *val_ac_trust;
  };
\end{verbatim}

\begin{description}

\item \var{val\_ac\_status}\verb" "

Validation state of the authentication chain element.  This field will contain
the error or success code for DNSSEC validation over the current authentication
chain element upon completion of \func{val\_resolve\_n\_check()}.  This field
may contain the following values:

\begin{itemize}

\item \const{VAL\_AC\_UNSET}\verb" "

The status was not set.

\item \const{VAL\_AC\_DATA\_MISSING}\verb" "

No data were returned for a query and the DNS did not indicate an error.

\item \const{VAL\_AC\_RRSIG\_MISSING}\verb" "

RRSIG data could not be retrieved for a resource record.

\item \const{VAL\_AC\_DNSKEY\_MISSING}\verb" "

The DNSKEY for an RRSIG covering a resource record could not be retrieved.

\item \const{VAL\_AC\_DS\_MISSING}\verb" "

The DS record covering a DNSKEY record was not available.

\item \const{VAL\_AC\_UNTRUSTED\_ZONE}\verb" "

Local policy defined a given zone as untrusted, with no further validation
being deemed necessary.

\item \const{VAL\_AC\_UNKNOWN\_DNSKEY\_PROTOCOL}\verb" "

The DNSKEY protocol number was unrecognized.

\item \const{VAL\_AC\_NOT\_VERIFIED}\verb" "

All RRSIGs covering the RRset could not be verified.

\item \const{VAL\_AC\_VERIFIED}\verb" "

A status of \const{VAL\_AC\_RRSIG\_VERIFIED} was found for at least one RRSIG
covering a resource record.

\item \const{VAL\_AC\_LOCAL\_ANSWER}\verb" "

The answer was obtained locally (e.g., from \path{/etc/hosts}) and validation
was not performed on the results.

\item \const{VAL\_AC\_TRUST\_KEY}\verb" "

A given DNSKEY or a DS record was locally defined to be a trust anchor.

\item \const{VAL\_AC\_TRUST\_ZONE}\verb" "

Local policy defined a given zone as trusted, with no further validation being
deemed necessary.

\item \const{VAL\_AC\_PROVABLY\_UNSECURE}\verb" "

The authentication chain from a trust anchor to a given zone could not be
constructed due to the provable absence of a DS record for this zone in the
parent.

\item \const{VAL\_AC\_BARE\_RRSIG}\verb" "

The response was for a query of type RRSIG.  RRSIGs contain the cryptographic
signatures for other DNS data and cannot themselves be validated.

\item \const{VAL\_AC\_NO\_TRUST\_ANCHOR}\verb" "

There was no trust anchor configured for a given authentication chain.

\item \const{VAL\_DNS\_ERROR\_BASE} $+$ \const{SR\_}error\verb" "

This value contains a resolver error from \lib{libsres}.  The \lib{libsres}
error is added \\
to \const{VAL\_DNS\_ERROR\_BASE}, so this value will lie between \\
\const{VAL\_DNS\_ERROR\_BASE} and \const{VAL\_DNS\_ERROR\_LAST}.
These values include the following:

\begin{itemize}

\item \const{SR\_CONFLICTING\_ANSWERS} - Multiple conflicting answers received
for a query.

\item \const{SR\_REFERRAL\_ERROR} - Some error encountered while following
referrals.

\item \const{SR\_MISSING\_GLUE} - Glue was missing.

\end{itemize}

\end{itemize}

\item \var{val\_ac\_rrset}\verb" "

Pointer to an RRset of type \var{struct val\_rrset} obtained from the DNS
response.

\item \var{val\_ac\_trust}\verb" "

Pointer to an authentication chain element that either contains a DNSKEY RRset
that can be used to verify RRSIGs over the current record, or contains a DS
RRset that can be used to build the chain-of-trust towards a trust anchor.

\end{description}

\item \struct{struct val\_rrset}\verb" "

\begin{verbatim}
    struct val_rrset
    {
        u_int8_t      *val_msg_header; 
        u_int16_t      val_msg_headerlen;
        u_int8_t      *val_rrset_name_n; 
        u_int16_t      val_rrset_class_h;
        u_int16_t      val_rrset_type_h;
        u_int32_t      val_rrset_ttl_h;
        u_int8_t       val_rrset_section;
        struct rr_rec *val_rrset_data;
        struct rr_rec *val_rrset_sig;
    };
\end{verbatim}

\begin{description}

\item \var{val\_msg\_header}\verb" "

Header of the DNS response in which the RRset was received.

\item \var{val\_msg\_headerlen}\verb" "

Length of the header information in \var{val\_msg\_header}.

\item \var{val\_rrset\_name\_n}\verb" "

Owner name of the RRset represented in on-the-wire format.

\item \var{val\_rrset\_class\_h}\verb" "

Class of the RRset.

\item \var{val\_val\_rrset\_type\_h}\verb" "

Type of the RRset.

\item \var{val\_rrset\_ttl\_h}\verb" "

TTL of the RRset.

\item \var{val\_rrset\_section}\verb" "

Section in which the RRset was received.  This may be one of the following:
\begin{itemize}
\item \const{VAL\_FROM\_ANSWER}
\item \const{VAL\_FROM\_AUTHORITY}
\item \const{VAL\_FROM\_ADDITIONAL}
\end{itemize}

\item \var{val\_rrset\_data}\verb" "

Response RDATA.

\item \var{val\_rrset\_sig}\verb" "

Any associated RRSIGs for the RDATA returned in \var{val\_rrset\_data}.

\end{description}

\item \struct{struct rr\_rec}\verb" "

\begin{verbatim}
    struct rr_rec
    {
        u_int16_t        rr_rdata_length_h;
        u_int8_t        *rr_rdata;
        val_astatus_t    rr_status;
        struct rr_rec   *rr_next;
    };
\end{verbatim}

\begin{description}

\item \var{rr\_rdata\_length\_h}\verb" "

Length of data stored in {\it rr\_rdata}.

\item \var{rr\_rdata}\verb" "

RDATA bytes.

\item \var{rr\_status}\verb" "

For each signature \var{rr\_rec} member within the authentication chain
\var{val\_ac\_rrset}, the validation status stored in the variable
\var{rr\_status} can return one of the following values:

\begin{itemize}

\item \const{VAL\_AC\_RRSIG\_VERIFIED}\verb" "

The RRSIG verified successfully.

\item \const{VAL\_AC\_WCARD\_VERIFIED}\verb" "

A given RRSIG covering a resource record shows that the record was wildcard
expanded.

\item \const{VAL\_AC\_RRSIG\_VERIFY\_FAILED}\verb" "

A given RRSIG covering an RRset was bogus.

\item \const{VAL\_AC\_DNSKEY\_NOMATCH}\verb" "

An RRSIG was created by a DNSKEY that did not exist in the apex keyset.

\item \const{VAL\_AC\_RRSIG\_ALGORITHM\_MISMATCH}\verb" "

The keytag referenced in the RRSIG matched a DNSKEY but the algorithms were
different.

\item \const{VAL\_AC\_WRONG\_LABEL\_COUNT}\verb" "

The number of labels on the signature was greater than the count given in
the RRSIG RDATA.

\item \const{VAL\_AC\_BAD\_DELEGATION}\verb" "

An RRSIG was created with a key that did not exist in the parent DS record
set.

\item \const{VAL\_AC\_RRSIG\_NOTYETACTIVE}\verb" "

The RRSIG's inception time is in the future.

\item \const{VAL\_AC\_RRSIG\_EXPIRED}\verb" "

The RRSIG had expired.

\item \const{VAL\_AC\_INVALID\_RRSIG}\verb" "

The RRSIG could not be parsed.

\item \const{VAL\_AC\_ALGORITHM\_NOT\_SUPPORTED}\verb" "

The RRSIG algorithm was not supported.

\item \const{VAL\_AC\_UNKNOWN\_ALGORITHM}\verb" "

The RRSIG algorithm was unknown.

\item \const{VAL\_AC\_ALGORITHM\_REFUSED}\verb" "

The RRSIG algorithm was not allowed as per local policy.

\end{itemize}

For each \var{rr\_rec} member of type DNSKEY (or DS, where relevant) within the
authentication chain \var{val\_ac\_rrset}, the validation status is stored in
the variable \var{rr\_status} and can return one of the following values:

\begin{itemize}

\item \const{VAL\_AC\_SIGNING\_KEY}\verb" "

This DNSKEY was used to create an RRSIG for the resource record set.

\item \const{VAL\_AC\_VERIFIED\_LINK}\verb" "

This DNSKEY provided the link in the authentication chain from the trust
anchor to the signed record.

\item \const{VAL\_AC\_UNKNOWN\_ALGORITHM\_LINK}\verb" "

This DNSKEY provided the link in the authentication chain from the trust
anchor to the signed record, but the DNSKEY algorithm was unknown.

\item \const{VAL\_AC\_INVALID\_KEY}\verb" "

The key used to verify the RRSIG was not a valid DNSKEY.

\item \const{VAL\_AC\_KEY\_TOO\_LARGE}\verb" "

Local policy defined the DNSKEY size as being too large.

\item \const{VAL\_AC\_KEY\_TOO\_SMALL}\verb" "

Local policy defined the DNSKEY size as being too small.

\item \const{VAL\_AC\_KEY\_NOT\_AUTHORIZED}\verb" "

Local policy defined the DNSKEY to be unauthorized for validation.

\item \const{VAL\_AC\_ALGORITHM\_NOT\_SUPPORTED}\verb" "

The DNSKEY or DS algorithm was not supported.

\item \const{VAL\_AC\_UNKNOWN\_ALGORITHM}\verb" "

The DNSKEY or DS algorithm was unknown.

\item \const{VAL\_AC\_ALGORITHM\_REFUSED}\verb" "

The DNSKEY or DS algorithm was not allowed as per local policy.

\end{itemize}

\item \var{rr\_next}\verb" "

Points to the next resource record in the RRset.

\end{description}

\end{description}

{\bf RETURN VALUES}

Return values for \func{val\_resolve\_n\_check()} and
\func{val\_create\_context()} are given below.

\begin{description}

\item \func{val\_resolve\_n\_check()}

\begin{description}

\item \const{VAL\_NO\_ERROR} - No error was encountered.

\item \const{VAL\_GENERIC\_ERROR} - Generic error encountered.

\item \const{VAL\_NOT\_IMPLEMENTED} - Functionality not yet implemented.

\item \const{VAL\_BAD\_ARGUMENT} - Bad arguments passed as parameters.

\item \const{VAL\_INTERNAL\_ERROR} - Encountered some internal error.

\item \const{VAL\_NO\_PERMISSION} - No permission to perform operation.
Currently not implemented.

\item \const{VAL\_RESOURCE\_UNAVAILABLE} - Some resource (crypto possibly) was
unavailable.  Currently not implemented.

\end{description}

\item \func{val\_create\_context()}

\begin{description}

\item \const{VAL\_NO\_ERROR} - No error was encountered.

\item \const{VAL\_RESOURCE\_UNAVAILABLE} - Could not allocate memory.

\item \const{VAL\_CONF\_PARSE\_ERROR} - Error in parsing some configuration
file.

\item \const{VAL\_CONF\_NOT\_FOUND} - A configuration file was not available.

\end{description}

\end{description}

{\bf FILES}

The validator library reads configuration information from two files,
\path{/etc/resolv.conf} and \path{/etc/dnsval.conf}.

Only the ``nameserver'' option is supported in the \path{resolv.conf} file.
This option is used to specify the IP address of the name server to which
queries must be sent by default.  For example,

\begin{verbatim}
    nameserver 10.0.0.1
\end{verbatim}

See \path{dnsval.conf(5)} for a description of the validator configuration file.

{\bf CURRENT STATUS}

There is currently no support for IPv6.

The caching functionality is very basic and no timeout logic currently exists.

There are a number of feature enhancements that still remain to be done.

{\bf SEE ALSO}

{\bf libsres(3)}

{\bf dnsval.conf(5)}

\clearpage

\subsection{{\bf val\_getaddrinfo()} DNSSEC-validated address translation}


{\bf NAME}

val\_getaddrinfo, val\_x\_getaddrinfo, val\_get\_addrinfo\_dnssec\_status,
val\_dupaddrinfo, val\_freeaddrinfo - get DNSSEC-validated network address
and service translation

{\bf SYNOPSIS}

\begin{verbatim}
  #include <validator.h>

  int val_getaddrinfo(const char *nodename, const char *servname,
                    const struct addrinfo *hints,
                    struct addrinfo **res);

  int val_x_getaddrinfo(const struct val_context *ctx,
                    const char *nodename, const char *servname,
                    const struct addrinfo *hints,
                    struct addrinfo **res);

  int val_get_addrinfo_dnssec_status(const struct addrinfo *ainfo);

  struct addrinfo* val_dupaddrinfo(const struct addrinfo *ainfo);

  void val_freeaddrinfo(struct addrinfo *ainfo);
\end{verbatim}

{\bf DESCRIPTION}

{\bf val\_getaddrinfo()} is a DNSSEC-aware version of {\bf getaddrinfo(3)}.  It
performs DNSSEC validation of DNS queries.  It returns a network address value
of type {\it addrinfo}.  (See {\bf getaddrinfo(3)} for more information on
{\it addrinfo}.)

{\bf val\_x\_getaddrinfo()} performs the same function as {\bf
val\_getaddrinfo()}, but is optimized for multiple calls.  The two routines
take the same parameters, but {\bf val\_x\_getaddrinfo()} takes an additional
parameter {\it ctx}, of type {\it val\_context}, which passes the validation
context for use in call optimization.  The {\it ctx} parameter also gives the
caller more control over the resolver and validator policies.  If a {\bf NULL}
value is passed for the {\it ctx} parameter, the default validation context
is used.  (See {\bf get\_context(3)} for information on creating a validation
context.) {\bf val\_getaddrinfo()} is equivalent to calling {\bf
val\_x\_getaddrinfo()} with a {\bf NULL} {\it ctx} parameter.

{\bf val\_dupaddrinfo()} duplicates the {\it addrinfo} structure and its
auxiliary data.  It performs a deep copy; i.e., the internal strings, arrays,
and other structures are also copied.

{\bf val\_freeaddrinfo()} frees a {\it addrinfo} structure, such as those
returned by the {\bf val\_getaddrinfo()}, {\bf val\_x\_getaddrinfo()} and {\bf
val\_dupaddrinfo()} functions.

{\bf val\_get\_addrinfo\_dnssec\_status()} extracts the DNSSEC validation
status from the returned {\it addrinfo} structure.  This function must be
called only for the values returned from {\bf val\_getaddrinfo()}, {\bf
val\_x\_getaddrinfo()}, and {\bf val\_dupaddrinfo()} functions.

{\bf RETURN VALUES}

The {\bf val\_getaddrinfo()} and {\bf val\_x\_getaddrinfo()} functions return
a value of type {\it addrinfo} on success, and {\bf NULL} on error.  The
memory for the returned value is dynamically allocated by these functions.
Hence, the caller must only call the {\bf val\_freeaddrinfo()} function on
the returned value in order to avoid memory leaks.

The {\bf val\_get\_addrinfo\_dnssec\_status()} function returns the result
of the DNSSEC validation.  The possible values for the DNSSEC status are given
in {\bf val\_errors.h}.

The {\bf val\_dupaddrinfo()} function returns a copy of the specified {\it
addrinfo} structure.  The returned value must be freed using {\bf
val\_freeaddrinfo()} to avoid memory leaks.

{\bf EXAMPLE}

\begin{verbatim}
 #include <stdio.h>
 #include <validator.h>

 int main(int argc, char *argv[])
 {
          int dnssec_status = ERROR;
          struct addrinfo *ainfo = NULL;

          if (argc < 2) {
                  printf("Usage: %s <hostname>\n", argv[0]);
                  exit(1);
          }
 
          ainfo = val_getaddrinfo(argv[1]);

          if (ainfo) {
                  dnssec_status = val_get_addrinfo_dnssec_status(h);

                  printf("DNSSEC Status = %d [%s]\n", dnssec_status,
                         p_val_error(dnssec_status));
                  val_freeaddrinfo(h);
          }

          return 0;
  }
\end{verbatim}

{\bf SEE ALSO}

{\bf gethostbyname}(3)

{\bf get\_context(3)}, {\bf val\_duphostent(3)}, {\bf val\_freehostent(3)},\\
{\bf val\_gethostbyname(3)}, {\bf val\_query(3)},\\
{\bf val\_x\_gethostbyname(3)}, {\bf val\_x\_query(3)}

{\it p\_val\_error}

\url{http://dnssec-tools.sourceforge.net}


\clearpage

\subsection{\bf val\_gethostbyname()}

{\bf NAME}

\func{val\_gethostbyname()}, \func{val\_gethostbyname2()},
\func{val\_gethostbyname\_r()}, \func{val\_gethostbyname2\_r()} -
get DNSSEC-validated network host entry

{\bf SYNOPSIS}

\begin{verbatim}
    #include <validator.h>

    extern int h_errno;
    struct hostent *val_gethostbyname(const val_context_t *ctx,
                                      const char *name,
                                      val_status_t *val_status);

    struct hostent *val_gethostbyname2(const val_context_t *ctx,
                                       const char *name,
                                       int af,
                                       val_status_t *val_status);

    int val_gethostbyname_r(const val_context_t *ctx,
                            const char *name,
                            struct hostent *ret,
                            char *buf,
                            size_t buflen,
                            struct hostent **result,
                            int *h_errnop,
                            val_status_t *val_status);

    int val_gethostbyname2_r(const val_context_t *ctx,
                             const char *name,
                             int af,
                             struct hostent *ret,
                             char *buf,
                             size_t buflen,
                             struct hostent **result,
                             int *h_errnop,
                             val_status_t *val_status);
\end{verbatim}

{\bf DESCRIPTION}

\func{val\_gethostbyname()}, \func{val\_gethostbyname2()},
\func{val\_gethostbyname\_r()} and \func{val\_gethostbyname2\_r()}
perform DNSSEC validation of DNS queries.  They return a network host entry
value of type \struct{struct hostent} and are DNSSEC-aware versions of the
\func{gethostbyname(3)}, \func{gethostbyname2(3)}, \func{gethostbyname\_r()}
and \func{gethostbyname2\_r()} functions, respectively.  (See
\func{gethostbyname(3)} for more information on type \struct{struct hostent}).

\func{val\_gethostbyname()} and \func{val\_gethostbyname\_r()}
support only IPv4 addresses.  IPv4 and IPv6 addresses are supported by
\func{val\_gethostbyname2()} and \func{val\_gethostbyname2\_r()}.

The \func{val\_gethostbyname\_r()} and \func{val\_gethostbyname2\_r()} are
reentrant versions and can be safely used in multi-threaded applications.

The \var{ctx} parameter specifies the validation context, which can be set to
NULL for default values.  The caller can use \var{ctx} to control the resolver
and validator policies.  Using a non-NULL validator context over multiple
calls can provide some optimization.  (See \lib{libval(3)} for
information on creating a validation context.)

\func{val\_gethostbyname()} and \func{val\_gethostbyname2()} set the global
\var{h\_errno} variable to return the resolver error code.  The reentrant
versions \func{val\_gethostbyname\_r()} and \func{val\_gethostbyname2\_r()}
use the \var{h\_errnop} parameter to return this value.  This ensures thread
safety, by avoiding the global \var{h\_errno} variable.  \var{h\_errnop} must
not be NULL.  (See the man page for \func{gethostbyname(3)} for possible
values of \var{h\_errno}.)

The \var{name}, \var{af}, \var{ret}, \var{buf}, \var{buflen}, and \var{result}
parameters have the same meaning and syntax as the corresponding parameters
for the original \func{gethostbyname*()} functions.  See the manual page for
\func{gethostbyname(3)} for more details about these parameters.

The \var{val\_status} parameter is used to return the validator error code.
\func{val\_istrusted()} determines whether this validation status represents a
trusted value and \func{p\_val\_status()} displays this value to the user in a
useful manner (See {\bf libval(3)} more for information).  \var{val\_status}
must not be NULL.

{\bf RETURN VALUE}

The \func{val\_gethostbyname()} and \func{val\_gethostbyname2()} functions
return a pointer to a \struct{hostent} structure when they can resolve the
given host name (with or without DNSSEC validation), and NULL on error.  The
memory for the returned value may be statically allocated by these two
functions.  Hence, the caller must not free the memory for the returned value.

The \func{val\_gethostbyname\_r()} and \func{val\_gethostbyname2\_r()}
functions return 0 when they can resolve the given host name (with or
without DNSSEC validation), and a non-zero error-code on failure.

{\bf EXAMPLE}

\begin{verbatim}
    #include <stdio.h>
    #include <stdlib.h>
    #include <validator.h>

    int main(int argc, char *argv[])
    {
             int val_status;
             struct hostent *h = NULL;

             if (argc < 2) {
                     printf("Usage: %s <hostname>\n", argv[0]);
                     exit(1);
             }
    
             h = val_gethostbyname(NULL, argv[1], &val_status);
             printf("h_errno = %d [%s]\n", h_errno,
                    hstrerror(h_errno));
             if (h) {
                     printf("Validation Status = %d [%s]\n", val_status,
                            p_val_status(val_status));
             }

             return 0;
    }
\end{verbatim}

{\bf NOTES}

These functions do not currently read the order of lookup from
\path{/etc/hosts.conf}.  This functionality will be provided in
future versions.  At present, the default order is set to consult
the \path{/etc/hosts} file first and then query DNS.

The current versions of these functions do not support NIS lookups.

{\bf SEE ALSO}

\func{gethostbyname(3)}, \func{gethostbyname2(3)}, \func{gethostbyname\_r(3)},
\func{gethostbyname2\_r(3)} \\
\func{get\_context(3)}, \func{val\_getaddrinfo(3)}, \func{val\_freeaddrinfo(3)},
\func{val\_query(3)} 

\lib{libval(3)}

\clearpage

\subsection{\bf val\_query()}

{\bf NAME}

\func{val\_query()} - DNSSEC-validated resolution of DNS queries

{\bf SYNOPSIS}

\begin{verbatim}
    #include <validator.h>

    int val_query(const val_context_t *ctx,
                  const char *dname,
                  const u_int16_t class,
                  const u_int16_t type,
                  const u_int8_t flags,
                  struct val_response **resp);

    int val_free_response(struct val_response *resp);

    int val_res_query(const val_context_t *ctx,
                      const char *dname,
                      int class,
                      int type,
                      u_char *answer,
                      int anslen,
                      val_status_t *val_status);
\end{verbatim}

{\bf DESCRIPTION}

The \func{val\_query()} and \func{val\_res\_query()} functions perform DNSSEC
validation of DNS queries.  They are DNSSEC-aware substitutes for
\func{res\_query(3)}.

The \var{ctx} parameter is the validator context and can be set to NULL for
default settings.  (More information about this field can be found in
\lib{libval(3)}.)

The \var{dname} parameter specifies the domain name, \var{class} specifies the
DNS class and \var{type} specifies the DNS type.

The \func{val\_query()} function returns results in the \var{resp} linked-list
which encapsulates the results into the following structure:

\begin{verbatim}
    struct val_response
    {
        unsigned char       *vr_response;
        int                  vr_length;
        val_status_t         vr_val_status;
        struct val_response *vr_next;
    };
\end{verbatim}

The \var{vr\_response} and \var{vr\_length} fields are functionally similar to
the \var{answer} and \var{anslen} parameters in \func{res\_query(3)}.  Memory
for the \var{resp} linked-list is internally allocated and must be released
after a successful invocation of the function using the
\func{val\_free\_response()} function.  Each element in the \var{resp} linked
list will contain an answer corresponding to a single RRSet in the DNS reply.

If DNSSEC validation succeeds for a given RRSet, a value of
\const{VAL\_SUCCESS} is returned in the \var{vr\_val\_status} field of the
\var{val\_response} structure for that RRSet.  Other values are returned in
case of errors.  (See \path{val\_errors.h} for a listing of possible error
codes.)

\func{p\_val\_status()} returns a brief string description of the error
code.  \func{val\_istrusted()} determines if the error code indicates that the
response can be trusted.  (See \lib{libval(3)} for further information.)

The \var{flags} parameter controls the scope of validation and name
resolution, and the output format.  At present only the
\const{VAL\_QUERY\_MERGE\_RRSETS} flag is defined.  This flag is provided for
legacy applications that already use \func{res\_query(3)} and want to
transition to \func{val\_query()} with minimal change.  When this flag is
specified, all RRsets in the answer are merged into a single response and
returned in the first element of the \var{resp} array.  The response field of
this element will have a format similar to the answer returned by
\func{res\_query(3)}.  The validation status will be \const{VAL\_SUCCESS} only
if all the individual RRsets have been successfully validated.  Otherwise, the
validation status will be one of the other error codes.  If a value other than
\const{VAL\_SUCCESS} is returned and if multiple RRsets are present in the
answer, it is not possible to know which RRset resulted in the error status,
if this flag is used.

\func{val\_res\_query()} is provided as a closer substitute for
\func{res\_query(3)}.  It calls \func{val\_query()} internally with the
\const{VAL\_QUERY\_MERGE\_RRSETS} flag and returns the answers in the field
answer with length of \var{anslen}.  The validation status is returned in the
field \var{val\_status}.

{\bf RETURN VALUES}

The \func{val\_query()} function returns 0 on success.  Errors returned
by \func{resolve\_n\_check()} may be returned, as it is called internally
by \func{val\_query()}.

The \func{val\_res\_query()} function returns the number of bytes received on
success and -1 on failure.

{\bf EXAMPLES}

\begin{verbatim}
    #include <stdio.h>
    #include <stdlib.h>
    #include <strings.h>
    #include <arpa/nameser.h>
    #include <validator.h>

    #define BUFLEN 8096
    #define RESPCOUNT 3

    int main(int argc, char *argv[])
    {
             int retval;
                 int i;
             int class = ns_c_in;
             int type = ns_t_a;
             struct val_response *resp, *iter;

             if (argc < 2) {
                     printf("Usage: %s <domain-name>\n", argv[0]);
                     exit(1);
             }

             retval = val_query(NULL, argv[1], class, type, 0, &resp);

             if (retval == 0) {
                     for (iter=resp; iter; iter=iter->vr_next) {
                             printf("Validation Status = %d [%s]\n",
                                        iter->vr_val_status,
                                        p_val_status(iter->vr_val_status));
                     }
             }

             free_val_response(resp);

             return 0;
    }
\end{verbatim}

{\bf SEE ALSO}

\func{res\_query(3)}

\func{get\_context(3)}, \func{val\_getaddrinfo(3)},
\func{val\_gethostbyname(3)}

\lib{libval(3)}



%%%%%%%%%%%%%%%%%%%%%%%%%%%%%%%%%%%%%%%%%%%%%%%%%%%%%%%%%%%%%%%%%%%%%%%%%%%%%%

\clearpage

\markboth{DNSSEC-Tools Software User Manual (vers. 4) -- Manual Pages}{DNSSEC-Tools Software User Manual (vers. 4) -- Manual Pages}
\section{Supporting Modules}
\markboth{DNSSEC-Tools Software User Manual (vers. 4) -- Manual Pages}{DNSSEC-Tools Software User Manual (vers. 4) -- Manual Pages}
\label{sect-modules}


A number of Perl modules have been developed for DNSSEC-Tools to assist in
maintaining DNSSEC-secured domains.  These routines manipulate DNSSEC-Tools
files, provide GUI interfaces, and manipulate common command options.

These DNSSEC-Tools Perl modules are:

\begin{table}[ht]
\begin{center}
\begin{tabular}{ll}
\perlmod{BootStrap.pm}	  & optionally load Perl modules		\\
\perlmod{QWPrimitives.pm} & \perlmod{QWizard} primitives		\\
\perlmod{conf.pm}	  & DNSSEC-Tools configuration file routines	\\
\perlmod{defaults.pm}	  & DNSSEC-Tools defaults routines		\\
\perlmod{dnssectools.pm}  & general routines for DNSSEC-Tools		\\
\perlmod{keyrec.pm}	  & {\it keyrec} file manipulation routines	\\
\perlmod{rolllog.pm}      & DNSSEC-Tools rollover logging routines	\\
\perlmod{rollmgr.pm}	  & \cmd{rollerd} interfaces			\\
\perlmod{rollrec.pm}	  & {\it rollrec} file manipulation routines	\\
\perlmod{timetrans.pm}	  & time/text conversion routines		\\
\perlmod{tooloptions.pm}  & DNSSEC-Tools common option routines		\\
\end{tabular} 
\end{center}
\end{table}

This section contains man pages describing these modules.

\clearpage

\subsubsection{BootStrap.pm}

{\bf NAME}

\perlmod{Net::DNS::SEC::Tools::BootStrap} - Optional loading of Perl modules

{\bf SYNOPSIS}

\begin{verbatim}
  use Net::DNS::SEC::Tools::BootStrap;

  dnssec_tools_load_mods(PerlModule => 'Additional help/error text');
\end{verbatim}

{\bf DESCRIPTION}

The DNSSEC-Tools package requires a number of Perl modules that are only
needed by some of the tools.  This module helps determine at run-time, rather
than at installation time, if the right tools are available on the system.  If
any module fails to load, \func{dnssec\_tools\_load\_mods()} will display an
error message and calls \func{exit()}.  The error message describes how to
install a module via CPAN.

The arguments to \func{dnssec\_tools\_load\_mods()} are given in pairs.  Each
pair is a module to try to load (and import) and a supplemental error message.
If the module fails to load, the supplemental error message will be displayed
along with the installation-via-CPAN message.  If the error message consists
of the string ``noerror'', then no error message will be displayed before the
function exits.

{\bf CAVEATS}

The module will try to import any exported subroutines from the module into
the \var{main} namespace.  This means that the \perlmod{BootStrap.pm} module
is likely to not be useful for importing symbols into other modules.
Work-arounds for this are:

\begin{description}

\item - import the symbols by hand\verb" "

\begin{verbatim}

  dnssec_tools_load_mods(PerlModule => 'Additional help/error text');

  import PerlModule qw(func1 func2);

  func1(arg1, arg2);

\end{verbatim}

\item - call the fully qualified function name\verb" "

\begin{verbatim}

  dnssec_tools_load_mods(PerlModule => 'Additional help/error text');

  PerlModule::func1(arg1, arg2);

\end{verbatim}

\end{description}


\clearpage

\subsection{\perlmod{QWPrimitives.pm}}

{\bf NAME}

Net::DNS::SEC::Tools::QWPrimitives - QWizard primitives for DNSSEC-Tools

{\bf SYNOPSIS}

\begin{verbatim}
  use Net::DNS::SEC::Tools::QWPrimitives;
  use Getopt::Long::GUI;

  GetOptions(...,['GUI:otherprimaries',dnssec_tools_get_qwprimitives()]);
\end{verbatim}

{\bf DESCRIPTION}

\perlmod{QWizard} is a dynamic GUI-construction kit.  It displays a series of
questions, and then retrieves and acts upon the answers.  This module
provides access to \perlmod{QWizard} for DNSSEC-Tools software.

{\bf SEE ALSO}

\perlmod{Net::DNS}

\perlmod{QWizard}

\url{http://www.dnssec-tools.org}


\clearpage

\subsection{\perlmod{conf.pm}}


{\bf NAME}

Net::DNS::SEC::Tools::conf - DNSSEC tools configuration file routines.

{\bf SYNOPSIS}

\begin{verbatim}
  use Net::DNS::SEC::Tools::conf;

  %dtconf = parseconfig();
  %dtconf = parseconfig("localzone.keyrec");
\end{verbatim}

{\bf DESCRIPTION}

The DNSSEC tools have a configuration file for commonly used values.
These values are the defaults for a variety of things, such as
encryption algorithm and encryption key length.

{\bf /usr/local/etc/dnssec/dnssec-tools.conf} is the path for the DNSSEC tools
configuration file.  The {\bf Net::DNS::SEC::Tools::conf} module provides
methods for accessing the configuration data in this file.

The DNSSEC tools configuration file consists of a set of configuration
value entries, with only one entry per line.  Each entry has the
``keyword value'' format.  During parsing, the line is broken into
tokens, with tokens being separated by spaces and tabs.  The first
token in a line is taken to be the keyword.  All other tokens in that
line are concatenated into a single string, with a space separating
each token.  The untokenized string is added to a hash table, with the
keyword as the value's key.

Comments may be included by prefacing them with the `\#' or `;'
comment characters.  These comments can encompass an entire line or may
follow a configuration entry.  If a comment shares a line with an entry,
value tokenization stops just prior to the comment character.

An example configuration file follows:

\begin{verbatim}
    # Sample configuration entries.

    algorithm       rsasha1     # Encryption algorithm.
    ksk_length      1024        ; KSK key length.
\end{verbatim}

{\bf CONFIGURATION INTERFACES}

\begin{description}

\item [{\bf parseconfig()}]\verb" "

This routine reads and parses the system's DNSSEC tools configuration file.
The parsed contents are put into a hash table, which is returned to the caller.

\item [{\bf parseconfig(conffile)}]\verb" "

This routine reads and parses a caller-specified DNSSEC tools configuration
file.  The parsed contents are put into a hash table, which is returned to
the caller.  The routine quietly returns if the configuration file does not
exist. 

\end{description}

{\bf SEE ALSO}

{\it zonesigner(1)}

{\bf Net::DNS::SEC::Tools::keyrec(3)}

{\bf dnssec-tools.conf(5)}


\clearpage

\subsection{\bf defaults.pm Module}

{\bf NAME}

\perlmod{Net::DNS::SEC::Tools::defaults.pm} - DNSSEC-Tools default values.

{\bf SYNOPSIS}

\begin{verbatim}
    use Net::DNS::SEC::Tools::defaults;

    $defalg = dnssec_tools_defaults("algorithm");

    $cz_path = dnssec_tools_defaults("bind_checkzone");

    $ksklife = dnssec_tools_defaults("ksklife");

    @default_names = dnssec_tools_defnames();
\end{verbatim}

{\bf DESCRIPTION}

This module maintains a set of default values used by DNSSEC-Tools
programs.  This allows these defaults to be centralized in a single
place and prevents them from being spread around multiple programs.

{\bf INTERFACES}

\begin{description}

\item \func{dnssec\_tools\_defaults(default)}\verb" "

This interface returns the value of a DNSSEC-Tools default.  The interface
is passed {\it default}, which is the name of a default to look up.  The value
of this default is returned to the caller.

\item \func{dnssec\_tools\_defnames()}\verb" "

This interface returns the names of all the DNSSEC-Tools defaults.
No default values are returned, but the default names returned by
\func{dnssec\_tools\_defnames()} may then be passed to
\func{dnssec\_tools\_defaults()}.

\end{description}

{\bf DEFAULT FIELDS}

The following are the defaults defined for DNSSEC-Tools.

\begin{description}

\item {\it algorithm}\verb" "

This default holds the default encryption algorithm.

\item {\it bind\_checkzone}\verb" "

This default holds the path to the \cmd{named-checkzone} BIND program.

\item {\it bind\_keygen}\verb" "

This default holds the path to the \cmd{dnssec-keygen} BIND program.

\item {\it bind\_signzone}\verb" "

This default holds the path to the \cmd{dnssec-signzone} BIND program.

\item {\it enddate}\verb" "

This default holds the default zone life, in seconds.

\item {\it entropy\_msg}\verb" "

This default indicates whether or not \cmd{zonesigner} should display an
entropy message.

\item {\it ksklength}\verb" "

This default holds the default length of a KSK key.

\item {\it ksklife}\verb" "

This default holds the default lifespan of a KSK key.  This is only used
for determining when to rollover the KSK key.  Keys otherwise have no
concept of a lifespan.  This is measured in seconds.

\item {\it random}\verb" "

This default holds the default random number generator device.

\item {\it savekeys}\verb" "

This default indicates whether or not keys should be deleted when they are no
longer in use.

\item {\it usegui}\verb" "

This default indicates whether or not the DNSSEC-Tools GUI should be used for
option entry.

\item {\it viewimage}\verb" "

This default holds the default image viewer.

\item {\it zskcount}\verb" "

This default holds the default number of ZSK keys to generate for a zone.

\item {\it zsklength}\verb" "

This default holds the default length of the ZSK key.

\item {\it zsklife}\verb" "

This default holds the default lifespan of the ZSK key.  This is only used
for determining when to rollover the ZSK key.  Keys otherwise have no
concept of a lifespan.  This is measured in seconds.

\end{description}

{\bf SEE ALSO}

\path{conf(5)}

\clearpage

\subsubsection{dnssectools.pm}

{\bf NAME}

\perlmod{Net::DNS::SEC::Tools::dnssectools} - General routines for the
DNSSEC-Tools package.

{\bf SYNOPSIS}

\begin{verbatim}
  use Net::DNS::SEC::Tools::dnssectools;

  dt_adminmail($subject,$msgbody,$recipient);

  $zspath = dt_cmdpath('zonesigner');

  $ftype = dt_findtype($path);
\end{verbatim}

{\bf DESCRIPTION}

The \perlmod{dnssectools} module provides a general set of methods for use
with DNSSEC-Tools utilities.

{\bf INTERFACES}

The interfaces to the \perlmod{dnssectools} module are given below.

\begin{description}

\item \func{dt\_adminmail(subject,msgbody,recipient)}\verb" "

This routine emails a message to the administrative user listed in the
DNSSEC-Tools configuration file.

\func{dt\_adminmail()} requires two parameters, both scalars.
The \var{subject} parameter is the subject for the mail message.
The \var{msgbody} parameter is the body of the mail message.

A third parameter, \var{recipient}, may be given to specify the message's
recipient.  If this is not given, then the recipient will be taken from
the \var{admin-email} record of the DNSSEC-Tools configuration file.

Return values:

\begin{itemize}

\item 1 - the message was created and sent.
\item 0 - an invalid recipient was specified. 

\end{itemize}

\item \func{dt\_cmdpath(command)}\verb" "

This routine returns the path to a specified DNSSEC-Tools command.
\var{command} should be the name only, without any leading directories.
The command name is checked to ensure that it is a valid DNSEC-Tools command,

Return values:

\begin{itemize}

\item The absolute path to the command is returned if the command is valid.
\item Null is returned if the command is not valid.

\end{itemize}

\item \func{dt\_filetype(path)}\verb" "

This routine returns the type of the file named in \var{path}.  The
\struct{rollrec} and \struct{keyrec} records contained therein are
counted and a type determination is made.

Return values:

\begin{description}

\item {\it keyrec}\verb" "

At least one \struct{keyrec} record was found and no \struct{rollrec} records
were found.

\item {\it rollrec}\verb" "

At least one \struct{rollrec} record was found and no \struct{keyrec} records
were found.

\item {\it mixed}\verb" "

At least one \struct{rollrec} record and at least one \struct{keyrec} record
were found.  This is most likely an erroneous file.

\item {\it unknown}\verb" "

No \struct{rollrec} records nor \struct{keyrec} records were found.

\item {\it nofile}\verb" "

The specified file does not exist.

\end{description}

\end{description}

{\bf SEE ALSO}

Mail::Send.pm(3),
Net::DNS::SEC::Tools::conf.pm(3)



\clearpage

\subsection{\perlmod{keyrec.pm}}

{\bf NAME}

Net::DNS::SEC::Tools::keyrec - DNSSEC-Tools {\it keyrec} file operations

{\bf SYNOPSIS}

\begin{verbatim}
  use Net::DNS::SEC::Tools::keyrec;

  keyrec_read("localzone.keyrec");

  @krnames = keyrec_names();

  $krec = keyrec_fullrec("example.com");
  %keyhash = %$krec;
  $zname = $keyhash{"algorithm"};

  $val = keyrec_recval("example.com","zonefile");

  keyrec_add("zone","example.com",\%zone_krfields);
  keyrec_add("key","Kexample.com.+005+12345",\%keydata);

  keyrec_del("example.com");
  keyrec_del("Kexample.com.+005+12345");

  keyrec_setval("zone","example.com","zonefile","db.example.com");

  @keyfields = keyrec_keyfields();
  @zonefields = keyrec_zonefields();

  keyrec_write();
  keyrec_close();
  keyrec_discard();
\end{verbatim}

{\bf DESCRIPTION}

The \perlmod{Net::DNS::SEC::Tools::keyrec} module manipulates the contents of
a DNSSEC-Tools {\it keyrec} file.  {\it keyrec} files contain data about
zones signed by and keys generated by the DNSSEC-Tools programs.  Module
interfaces exist for looking up {\it keyrec} records, creating new
records, and modifying existing records.

A {\it keyrec} file is organized in sets of {\it keyrec} records.  Each
{\it keyrec} must be either of {\it key} type or {\it zone} type.  Key
{\it keyrec}s describe how encryption keys were generated, zone {\it keyrec}s
describe how zones were signed.  A {\it keyrec} consists of a set of
keyword/value entries.  The following is an example of a key {\it keyrec}:

\begin{verbatim}
    key     "Kexample.com.+005+30485"
          zonename        "example.com"
          keyrec_type     "ksk"
          algorithm       "rsasha1"
          random          "/dev/urandom"
          zsklength       "512"
          keyrec_gensecs  "1101183727"
          keyrec_gendate  "Tue Nov 23 04:22:07 2004"
\end{verbatim}

The first step in using this module {\bf must} be to read the {\it keyrec}
file.  The {\bf keyrec\_read()} interface reads the file and parses it
into an internal format.  The file's records are copied into a hash
table (for easy reference by the \perlmod{Net::DNS::SEC::Tools::keyrec}
routines) and in an array (for preserving formatting and comments.)

After the file has been read, the contents are referenced using
{\bf keyrec\_fullrec()} and {\bf keyrec\_recval()}.  The contents are modified
using {\bf keyrec\_add()} and {\bf keyrec\_setval()}.  {\it keyrec}s may be
deleted with the {\bf keyrec\_del()} interface.

If the {\it keyrec} file has been modified, it must be explicitly written or
the changes are not saved.  {\bf keyrec\_write()} saves the new contents to
disk.  {\bf keyrec\_close()} saves the file and close the Perl file handle to
the {\it keyrec} file.  If a {\it keyrec} file is no longer wanted to be open,
yet the contents should not be saved, {\bf keyrec\_discard()} gets rid of the
data, and closes the file handle {\bf without} saving any modified data.

{\bf KEYREC INTERFACES}

The interfaces to the \perlmod{Net::DNS::SEC::Tools::keyrec} module are given
below.

{\bf keyrec\_add(keyrec\_type,keyrec\_name,fields)}

This routine adds a new {\it keyrec} to the {\it keyrec} file and the internal
representation of the file contents.  The {\it keyrec} is added to both the
{\it \%keyrecs} hash table and the {\it $@$keyreclines} array.

{\it keyrec\_type} specifies the type of the {\it keyrec} -- ``key'' or
``zone''.  {\it keyrec\_name} is the name of the {\it keyrec}.  {\it fields}
is a reference to a hash table that contains the name/value {\it keyrec}
fields.  The keys of the hash table are always converted to lowercase, but
the entry values are left as given.

The {\it ksklength} entry is only added if {\it keyrec\_type} is ``ksk''.

The {\it zsklength} entry is only added if {\it keyrec\_type} is ``zsk''.

Timestamp fields are added at the end of the {\it keyrec}.  For key
{\it keyrec}s, the {\it keyrec\_gensecs} and {\it keyrec\_gendate} timestamp
fields are added.  For zone {\it keyrec}s, the {\it keyrec\_signsecs} and
{\it keyrec\_signdate} timestamp fields are added.

If a specified field isn't defined for the {\it keyrec} type, the entry isn't
added.  This prevents zone {\it keyrec} data from getting mingled with key
{\it keyrec} data.

A blank line is added after the final line of the new {\it keyrec}.  After
adding all new {\it keyrec} entries, the {\it keyrec} file is written but is
not closed.

Return values are:

\begin{verbatim}
    0     success
    -1    invalid I<krtype>
\end{verbatim}

{\bf keyrec\_del(keyrec\_name)}

This routine deletes a {\it keyrec} from the {\it keyrec} file and the internal
representation of the file contents.  The {\it keyrec} is deleted from both
the {\it \%keyrecs} hash table and the {\it $@$keyreclines} array.

Only the {\it keyrec} itself is deleted from the file.  Any associated comments
and blank lines surrounding it are left intact.

Return values are:

\begin{verbatim}
    0     successful I<keyrec> deletion
    -1    invalid I<krtype> (empty string or unknown name)
\end{verbatim}

{\bf keyrec\_close()}

This interface saves the internal version of the {\it keyrec} file (opened
with {\bf keyrec\_read()}) and closes the file handle. 

{\bf keyrec\_discard()}

This routine removes a {\it keyrec} file from use by a program.  The internally
stored data are deleted and the {\it keyrec} file handle is closed.  However,
modified data are not saved prior to closing the file handle.  Thus, modified
and new data will be lost.

{\bf keyrec\_fullrec(keyrec\_name)}

{\bf keyrec\_fullrec()} returns a reference to the {\it keyrec} specified in
{\it keyrec\_name}.

{\bf keyrec\_keyfields()}

This routine returns a list of the recognized fields for a key {\it keyrec}.

{\bf keyrec\_names()}

This routine returns a list of the {\it keyrec} names from the file.

{\bf keyrec\_read(keyrec\_file)}

This interface reads the specified {\it keyrec} file and parses it into a
{\it keyrec} hash table and a file contents array.  {\bf keyrec\_read()}
{\bf must} be called prior to any of the other
\perlmod{Net::\-DNS::\-SEC::\-Tools::\-keyrec} calls.  If another {\it keyrec}
is already open, then it is saved and closed prior to opening the new {\it
keyrec}.

Upon success, {\bf keyrec\_read()} returns the number of {\it keyrec}s read
from the file.

Failure return values:

\begin{verbatim}
    -1    specified I<keyrec> file doesn't exit
    -2    unable to open I<keyrec> file
    -3    duplicate I<keyrec> names in file
\end{verbatim}

{\bf keyrec\_recval(keyrec\_name,keyrec\_field)}

This routine returns the value of a specified field in a given {\it keyrec}.
{\it keyrec\_name} is the name of the particular {\it keyrec} to consult.
{\it keyrec\_field} is the field name within that {\it keyrec}.

For example, the current {\it keyrec} file contains the following {\it keyrec}:

\begin{verbatim}
    zone        "example.com"
        zonefile        "db.example.com"
\end{verbatim}

The call:

\begin{verbatim}
    keyrec_recval("example.com","zonefile")
\end{verbatim}

will return the value ``db.example.com''.

{\bf keyrec\_setval(keyrec\_type,keyrec\_name,field,value)}

Set the value of a {\it name/field} pair in a specified {\it keyrec}.  The
file is {\bf not} written after updating the value.  The value is saved in
both {\it \%keyrecs} and in {\it $@$keyreclines}, and the file-modified flag
is set.

{\it keyrec\_type} specifies the type of the {\it keyrec}.  This is only used
if a new {\it keyrec} is being created by this call.
{\it keyrec\_name} is the name of the {\it keyrec} that will be modified.
{\it field} is the {\it keyrec} field which will be modified.
{\it value} is the new value for the field.

Return values are:

\begin{verbatim}
    0  if the creation succeeded
    -1 invalid type was given
\end{verbatim}

{\bf keyrec\_write()}

This interface saves the internal version of the {\it keyrec} file (opened with
{\bf keyrec\_read()}).  It does not close the file handle.  As an efficiency
measure, an internal modification flag is checked prior to writing the file.
If the program has not modified the contents of the {\it keyrec} file, it is not
rewritten.

{\bf keyrec\_zonefields()}

This routine returns a list of the recognized fields for a zone {\it keyrec}.

{\bf KEYREC INTERNAL INTERFACES}

The interfaces described in this section are intended for internal use by the
\perlmod{Net::\-DNS::\-SEC::\-Tools::\-keyrec} module.  However, there are
situations where external entities may have need of them.  Use with caution,
as misuse may result in damaged or lost {\it keyrec} files.

{\bf keyrec\_init()}

This routine initializes the internal {\it keyrec} data.  Pending changes
will be lost.  An open {\it keyrec} file handle will remain open, though the
data are no longer held internally.  A new {\it keyrec} file must be read in
order to use the \perlmod{Net::\-DNS::\-SEC::\-Tools::\-keyrec} interfaces
again.

{\bf keyrec\_newkeyrec(kr\_name,kr\_type)}

This interface creates a new {\it keyrec}.  The {\it keyrec\_name} and {\it
keyrec\_hash} fields in the {\it keyrec} are set to the values of the {\it
kr\_name} and {\it kr\_type} parameters.  {\it kr\_type} must be either
``key'' or ``zone''.

Return values are:

\begin{verbatim}
    0      if the creation succeeded
    -1     if an invalid I<keyrec> type was given
\end{verbatim}

{\bf KEYREC DEBUGGING INTERFACES}

The following interfaces display information about the currently parsed
{\it keyrec} file.  They are intended to be used for debugging and testing,
but may be useful at other times.

{\bf keyrec\_dump\_hash()}

This routine prints the {\it keyrec} file as it is stored internally in a
hash table.  The {\it keyrec}s are printed in alphabetical order, with the
fields alphabetized for each {\it keyrec}.  New {\it keyrec}s and {\it keyrec}
fields are alphabetized along with current {\it keyrec}s and fields.  Comments
from the {\it keyrec} file are not included with the hash table.

{\bf keyrec\_dump\_array()}

This routine prints the {\it keyrec} file as it is stored internally in
an array.  The {\it keyrec}s are printed in the order given in the file,
with the fields ordered in the same manner.  New {\it keyrec}s are
appended to the end of the array.  {\it keyrec} fields added to existing
{\it keyrec}s are added at the beginning of the {\it keyrec} entry.
Comments and vertical whitespace are preserved as given in the
{\it keyrec} file.

{\bf SEE ALSO}

\perlmod{Net::DNS::SEC::Tools::keyrec(5)}


\clearpage

\subsubsection{rolllog.pm}

{\bf NAME}

\perlmod{Net::DNS::SEC::Tools::rolllog} - DNSSEC-Tools rollover logging
interfaces.

{\bf SYNOPSIS}

\begin{verbatim}
  use Net::DNS::SEC::Tools::rolllog;

  @levels = rolllog_levels();

  $curlevel = rolllog_level();
  $oldlevel = rolllog_level("info");
  $oldlevel = rolllog_level(LOG_ERR,1);

  $curlogfile = rolllog_file();
  $oldlogfile = rolllog_file("-");
  $oldlogfile = rolllog_file("/var/log/roll.log",1);

  $loglevelstr = rolllog_str(8)
  $loglevelstr = rolllog_str("info")

  $ret = rolllog_num("info");

  rolllog_log(LOG_INFO,"example.com","zone is valid");
\end{verbatim}

{\bf DESCRIPTION}

The \perlmod{Net::DNS::SEC::Tools::rolllog} module provides logging interfaces
for the rollover programs.  The logging interfaces allow log messages to be
recorded.  \cmd{rollerd} must be running, as it is responsible for updating
the log file.

Each log message is assigned a particular logging level.  The valid logging
levels are:

\begin{table}[h]
\begin{center}
\begin{tabular}{|l|c|l|}
\hline
{\bf Textual Level} & {\bf Numeric Level} & {\bf Meaning} \\
\hline
{\bf tmi}    & 1 & The highest level -- all log messages are saved.	\\
{\bf expire} & 3 & A verbose countdown of zone expiration is given.	\\
{\bf info}   & 4 & Many informational messages are recorded.		\\
{\bf phase}  & 6 & Each zone's current rollover phase is given.		\\
{\bf err}    & 8 & Errors are recorded.					\\
{\bf fatal}  & 9 & Fatal errors are saved.				\\
\hline
\end{tabular}
\end{center}
\caption{Logging Levels}
\end{table}

The logging levels include all numerically higher levels.  For example, if
the logging level is set to {\bf phase}, then {\bf err} and {\bf fatal}
messages will also be recorded.

{\bf LOGGING INTERFACES}

\begin{description}

\item \func{rolllog\_levels()}\verb" "

This routine returns an array holding the text forms of the user-settable
logging levels.  The levels are returned in order, from most verbose to least.

\item \func{rolllog\_level(newlevel,useflag)}\verb" "

This routine sets and retrieves the logging level for \cmd{rollerd}.
The \var{newlevel} argument specifies the new logging level to be set.
\var{newlevel} may be given in either text or numeric form.

The \var{useflag} argument is a boolean that indicates whether or not to give
a descriptive message and exit if an invalid logging level is given.  If
\var{useflag} is true, the message is given and the process exits; if false,
-1 is returned.

If given with no arguments, the current logging level is returned.  In fact,
the current level is always returned unless an error is found.  -1 is returned
on error.

\item \func{rolllog\_file(newfile,useflag)}\verb" "

This routine sets and retrieves the log file for \cmd{rollerd}.  The
\var{newfile} argument specifies the new log file to be set.  If \var{newfile}
exists, it must be a regular file.

The \var{useflag} argument is a boolean that indicates whether or not to give
a descriptive message if an invalid logging level is given.  If \var{useflag}
is true, the message is given and the process exits; if false, no message is
given.  For any error condition, an empty string is returned.

\item \func{rolllog\_num(loglevel)}\verb" "

This routine translates a text log level (given in \var{loglevel}) into the
associated numeric log level.  The numeric log level is returned to the
caller.

If \var{loglevel} is an invalid log level, -1 is returned.

\item \func{rolllog\_str(loglevel)}\verb" "

This routine translates a log level (given in \var{loglevel}) into the
associated text log level.  The text log level is returned to the caller.

If \var{loglevel} is a text string, it is checked to ensure it is a valid log
level.  Case is irrelevant when checking \var{loglevel}.

If \var{loglevel} is numeric, it is must be in the valid range of log levels.
\var{undef} is returned if \var{loglevel} is invalid.

\item \func{rolllog\_log(level,group,message)}\verb" "

The \func{rolllog\_log()} interface writes a message to the log file.  Log
messages have this format:

\begin{verbatim}
    timestamp: group: message
\end{verbatim}

The \var{level} argument is the message's logging level.  It will only be
written to the log file if the current log level is numerically equal to or
less than \var{level}.

\var{group} allows messages to be associated together.  It is currently used
by \cmd{rollerd} to group messages by the zone to which the message applies.

The \var{message} argument is the log message itself.  Trailing newlines are
removed.

\end{description}

{\bf SEE ALSO}

rollctl(1)

rollerd(8)

Net::DNS::SEC::Tools::rollmgr.pm(3)


\clearpage

\subsubsection{rollmgr.pm}

{\bf NAME}

\perlmod{Net::DNS::SEC::Tools::rollmgr} - Communicate with the DNSSEC-Tools
rollover manager.

{\bf SYNOPSIS}

\begin{verbatim}
  use Net::DNS::SEC::Tools::rollmgr;

  $dir = rollmgr_dir();

  $idfile = rollmgr_idfile();

  $id = rollmgr_getid();

  rollmgr_dropid();

  rollmgr_rmid();

  rollmgr_cmdint();

  rollmgr_halt();

  rollmgr_channel(1);
  ($cmd,$data) = rollmgr_getcmd();
  $ret = rollmgr_verifycmd($cmd);

  rollmgr_sendcmd(CHANNEL_CLOSE,ROLLCMD_ROLLZONE,"example.com");

  rollmgr_sendcmd(CHANNEL_WAIT,ROLLCMD_ROLLZONE,"example.com");
  ($retcode, $respmsg) = rollmgr_getresp();
\end{verbatim}

{\bf DESCRIPTION}

The \perlmod{Net::DNS::SEC::Tools::rollmgr} module provides standard,
platform-independent methods for a program to communicate with DNSSEC-Tools'
\cmd{rollerd} rollover manager.  There are two interface classes described
here:  general interfaces and communications interfaces.

{\bf GENERAL INTERFACES}

The interfaces to the \perlmod{Net::DNS::SEC::Tools::rollmgr} module are given
below.

\begin{description}

\item \cmd{rollmgr\_dir()}\verb" "

This routine returns \cmd{rollerd}'s directory.

\item \cmd{rollmgr\_idfile()}\verb" "

This routine returns \cmd{rollerd}'s id file.

\item \cmd{rollmgr\_getid()}\verb" "

This routine returns \cmd{rollerd}'s process id.  If a non-zero value is
passed as an argument, the id file will be left open and accessible through
the PIDFILE file handle.  See the WARNINGS section below.

Return Values:

\begin{description}
\item On success, the first portion of the file contents (up to 80 characters)
is returned.
\item -1 is returned if the id file does not exist.
\end{description}

\item \cmd{rollmgr\_dropid()}\verb" "

This interface ensures that another instance of \cmd{rollerd} is not
running and then creates a id file for future reference.

Return Values:

\begin{description}
\item 1 - the id file was successfully created for this process
\item 0 - another process is already acting as \cmd{rollerd}
\end{description}

\item \cmd{rollmgr\_rmid()}\verb" "

This interface deletes \cmd{rollerd}'s id file.

Return Values:

\begin{description}
\item  1 - the id file was successfully deleted
\item  0 - no id file exists
\item -1 - the calling process is not \cmd{rollerd}
\item -2 - unable to delete the id file
\end{description}

\item \cmd{rollmgr\_cmdint()}\verb" "

This routine informs \cmd{rollerd} that a command has been sent via
\cmd{rollmgr\_sendcmd()}.

Return Values:

\begin{description}
\item -1 - an invalid process id was found for \cmd{rollerd}
\item Anything else indicates the number of processes that were signaled.\\
(This should only ever be 1.)
\end{description}

\item \cmd{rollmgr\_halt()}\verb" "

This routine informs \cmd{rollerd} to shut down.

In the current implementation, the return code from the \func{kill()} command
is returned.

\begin{description}
\item -1 - an invalid process id was found for \cmd{rollerd}
\item Anything else indicates the number of processes that were signaled.\\
(This should only ever be 1.)
\end{description}

\end{description}

{\bf ROLLERD COMMUNICATIONS INTERFACES}

\begin{description}

\item \cmd{rollmgr\_channel(serverflag)}\verb" "

This interface sets up a persistent channel for communications with
\cmd{rollerd}.  If \var{serverflag} is true, then the server's side of the
channel is created.  If \var{serverflag} is false, then the client's side of
the channel is created.

Currently, the connection may only be made to the localhost.  This may be
changed to allow remote connections, if this is found to be needed.

\item \cmd{rollmgr\_getcmd()}\verb" "

\cmd{rollmgr\_getcmd()} retrieves a command sent over \cmd{rollerd}'s
communications channel by a client program.  The command and the command's
data are sent in each message.

The command and the command's data are returned to the caller.

\item \cmd{rollmgr\_sendcmd(closeflag,cmd,data)}\verb" "

\cmd{rollmgr\_sendcmd()} sends a command to \cmd{rollerd}.  The command must
be one of the commands from the table below.  This interface creates a
communications channel to \cmd{rollerd} and sends the message.  The channel is
not closed, in case the caller wants to receive a response from \cmd{rollerd}.

The available commands and their required data are:

\begin{table}[h]
\begin{center}
\begin{tabular}{|l|c|l|}
\hline
{\bf Command} & {\bf Data} & {\bf Purpose} \\
\hline
\const{ROLLCMD\_DISPLAY}   & 1/0           & start/stop \cmd{rollerd}'s graphical display \\
\const{ROLLCMD\_DSPUB}     & zone-name     & a DS record has been published \\
\const{ROLLCMD\_DSPUBALL}  & none          & DS records published for all zones \\
                           &               & in KSK rollover phase 6 \\
\const{ROLLCMD\_ROLLALL}   & none          & force all zones to start ZSK rollover \\
\const{ROLLCMD\_ROLLKSK}   & zone-name     & force a zone to start KSK rollover \\
\const{ROLLCMD\_ROLLREC}   & rollrec-name  & change \cmd{rollerd}'s \struct{rollrec} file \\
\const{ROLLCMD\_ROLLZONE}  & zone-name     & force a zone to start ZSK rollover \\
\const{ROLLCMD\_RUNQUEUE}  & none          & \cmd{rollerd} runs through its queue \\
\const{ROLLCMD\_SHUTDOWN}  & none          & stop \cmd{rollerd} \\
\const{ROLLCMD\_SLEEPTIME} & seconds-count & set \cmd{rollerd}'s sleep time \\
\const{ROLLCMD\_STATUS}    & none          & get \cmd{rollerd}'s status \\
\hline
\end{tabular}
\end{center}
\caption{\cmd{rollerd} Commands}
\end{table}

The data aren't checked for validity by \cmd{rollmgr\_sendcmd()}; validity
checking is a responsibility of \cmd{rollerd}.

If the caller does not need a response from \cmd{rollerd}, then
\var{closeflag} should be set to \const{CHANNEL\_CLOSE}; if a response is
required then \var{closeflag} should be \const{CHANNEL\_WAIT}.  These values
are boolean values, and the constants aren't required.

Return Values:

\begin{description}
\item 1 is returned on success.
\item 0 is returned if an invalid command is given.
\end{description}

\item \cmd{rollmgr\_getresp()}\verb" "

After executing a client command sent via \cmd{rollmgr\_sendcmd()},
\cmd{rollerd} will send a response to the client.  \cmd{rollmgr\_getresp()}
allows the client to retrieve the response.

A return code and a response string are returned, in that order.  Both are
specific to the command sent.

\item \cmd{rollmgr\_verifycmd(cmd)}\verb" "

\cmd{rollmgr\_verifycmd()} verifies that \var{cmd} is a valid command for
\cmd{rollerd}.

Return Values:

\begin{description}
\item 1 is returned for a valid command.
\item 0 is returned for an invalid command.
\end{description}

\end{description}

{\bf WARNINGS}

1.  \cmd{rollmgr\_getid()} attempts to exclusively lock the id file.
Set a timer if this matters to you.

2.  \cmd{rollmgr\_getid()} has a nice little race condition.  We should lock
the file prior to opening it, but we can't do so without it being open.

{\bf SEE ALSO}

rollctl(1)

Net::DNS::SEC::Tools::keyrec.pm(3),
Net::DNS::SEC::Tools::rolllog.pm(3), \\
Net::DNS::SEC::Tools::rollrec.pm(3)

rollerd(8)


\clearpage

\subsection{\bf rollrec.pm Module}

{\bf NAME}

\perlmod{}Net::DNS::SEC::Tools::rollrec.pm - Manipulate a DNSSEC-Tools rollrec file.

{\bf SYNOPSIS}

\begin{verbatim}
    use Net::DNS::SEC::Tools::rollrec;

    rollrec_lock();
    rollrec_read("localhost.rollrec");

    @rrnames = rollrec_names();

    $rrec = rollrec_fullrec("example.com");
    %rrhash = %$rrec;
    $zname = $rrhash{"maxttl"};

    $val = rollrec_recval("example.com","zonefile");

    rollrec_add("roll","example.com",\%rollfields);
    rollrec_add("skip","example.com",\%rollfields);

    rollrec_del("example.com");

    rollrec_type("example.com","skip");
    rollrec_type("example.com","roll");

    rollrec_setval("example.com","zonefile","db.example.com");

    rollrec_settime("example.com");

    @rollrecfields = rollrec_fields();

    $default_file = rollrec_default();

    rollrec_write();
    rollrec_close();
    rollrec_discard();

    rollrec_unlock();
\end{verbatim}

{\bf DESCRIPTION}

\perlmod{\bf Net::DNS::SEC::Tools::rollrec} module manipulates the contents of
a DNSSEC-Tools {\it rollrec} file.  {\it rollrec} files describe the status of
a zone rollover process, as performed by the DNSSEC-Tools programs.  Module
interfaces exist for looking up {\it rollrec} records, creating new records,
and modifying existing records.

A {\it rollrec} file is organized in sets of {\it rollrec} records.  {\it
rollrec}s describe the state of a rollover operation.  A {\it rollrec}
consists of a set of keyword/value entries.  The following is an example
of a {\it rollrec}:

\begin{verbatim}
    roll "example.com"
        zonefile              "/usr/etc/dnssec/zones/db.example.com"
        keyrec                "/usr/etc/dnssec/keyrec/example.keyrec"
        curphase              "2"
        maxttl                "86400"
        phasestart            "Wed Mar 09 21:49:22 2005"
        display               "0"
        rollrec_rollsecs      "1115923362"
        rollrec_rolldate      "Tue Mar 09 19:12:54 2005"
\end{verbatim}

The first step in using this module must be to read the {\it rollrec} file.
The \func{rollrec\_read()} interface reads the file and parses it into an
internal format.  The file's records are copied into a hash table (for easy
reference by the \perlmod{Net::DNS::SEC::Tools::rollrec} routines) and in an
array (for preserving formatting and comments.)

After the file has been read, the contents are referenced using
\func{rollrec\_fullrec()} and \func{rollrec\_recval()}.  The
\func{rollrec\_add()}, \func{rollrec\_setval()}, and \func{rollrec\_settime()}
interfaces
are used to modify the contents of a {\it rollrec} record.

If the {\it rollrec} file has been modified, it must be explicitly written or
the changes will not saved.  \func{rollrec\_write()} saves the new contents to
disk.  \func{rollrec\_close()} saves the file and close the Perl file handle to
the {\it rollrec} file.  If a {\it rollrec} file is no longer wanted to be
open, yet the contents should not be saved, \func{rollrec\_discard()} gets rid
of the data closes and the file handle {\bf without} saving any modified data.

{\bf ROLLREC LOCKING}

This module includes interfaces for synchronizing access to the {\it rollrec}
files.  This synchronization is very simple and relies upon locking and
unlocking a single lock file for all {\it rollrec} files.

{\it rollrec} locking is not required before using this module, but it is
recommended.  The expected use of these facilities follows:

\begin{verbatim}
    rollrec_lock() || die "unable to lock rollrec file\n";
    rollrec_read();
    ... perform other rollrec operations ...
    rollrec_close();
    rollrec_unlock();
\end{verbatim}

Synchronization is performed in this manner due to the way the module's
functionality is implemented, as well as providing flexibility to users
of the module.  It also provides a clear delineation in callers' code as
to where and when {\it rollrec} locking is performed.

This synchronization method has the disadvantage of having a single lockfile
as a bottleneck to all {\it rollrec} file access.  However, it reduces
complexity in the locking interfaces and cuts back on the potential number of
required lockfiles.

Using a single synchronization file may not be practical in large
installations.  If that is found to be the case, then this will be reworked.

{\bf ROLLREC INTERFACES}

The interfaces to the \perlmod{Net::DNS::SEC::Tools::rollrec} module are given
below.

\begin{description}

\item \func{rollrec\_add(rollrec\_type,rollrec\_name,fields)}\verb" "

This routine adds a new {\it rollrec} to the {\it rollrec} file and the
internal representation of the file contents.  The {\it rollrec} is added to
both the {\it \%rollrecs} hash table and the {\it \@rollreclines} array.
Entries are only added if they are defined for {\it rollrec}s.

{\it rollrec\_type} is the type of the {\it rollrec}.  This must be either
``roll'' or ``skip''.  {\it rollrec\_name} is the name of the {\it rollrec}.
{\it fields} is a reference to a hash table that contains the name/value {\it
rollrec} fields.  The keys of the hash table are always converted to
lowercase, but the entry values are left as given.

Timestamp fields are added at the end of the {\it rollrec}.  These fields have
the key values {\it rollrec\_gensecs} and {\it rollrec\_gendate}.

A blank line is added after the final line of the new {\it rollrec}.
The {\it rollrec} file is not written after \func{rollrec\_add()}, though
it is marked as having been modified.

\item \func{rollrec\_del(rollrec\_name)}\verb" "

This routine deletes a {\it rollrec} from the {\it rollrec} file and the
internal representation of the file contents.  The {\it rollrec} is deleted
from both the {\it \%rollrecs} hash table and the {\it \@rollreclines} array.

Only the {\it rollrec} itself is deleted from the file.  Any associated
comments and blank lines surrounding it are left intact.
The {\it rollrec} file is not written after \func{rollrec\_del()}, though
it is marked as having been modified.

Return values are:

\begin{table}[ht]
\begin{center}
\begin{tabular}{cl}
0 & successful rollrec deletion \\
-1 & unknown name \\
\end{tabular} 
\end{center}
\end{table}

\item \func{rollrec\_close()}\verb" "

This interface saves the internal version of the {\it rollrec} file (opened
with \func{rollrec\_read()}) and closes the file handle.

\item \func{rollrec\_discard()}\verb" "

This routine removes a {\it rollrec} file from use by a program.  The internally
stored data are deleted and the {\it rollrec} file handle is closed.  However,
modified data are not saved prior to closing the file handle.  Thus, modified
and new data will be lost.

\item \func{rollrec\_fullrec(rollrec\_name)}\verb" "

\func{rollrec\_fullrec()} returns a reference to the {\it rollrec} specified in
{\it rollrec\_name}.

\item \func{rollrec\_lock()}\verb" "

\func{rollrec\_lock()} locks the {\it rollrec} lockfile.  An exclusive lock is
requested, so the execution will suspend until the lock is available.  If the
{\it rollrec} synchronization file does not exist, it will be created.  If the
process can't create the synchronization file, an error will be returned.
Success or failure is returned.

\item \func{rollrec\_names()}\verb" "

This routine returns a list of the {\it rollrec} names from the file.

\item \func{rollrec\_read(rollrec\_file)}\verb" "

This interface reads the specified {\it rollrec} file and parses it into a
{\it rollrec} hash table and a file contents array.  \func{rollrec\_read()}
{\bf must} be called prior to any of the other
\perlmod{Net::DNS::SEC::Tools::rollrec} calls.  If another {\it rollrec} is
already open, then it is saved and closed prior to opening the new
{\it rollrec}.

Upon success, \func{rollrec\_read()} returns the number of {\it rollrec}s read
from the file.

Failure return values:

\begin{table}[ht]
\begin{center}
\begin{tabular}{cl}
-1 & specified rollrec file doesn't exit	\\
-2 & unable to open rollrec file		\\
-3 & duplicate rollrec names in file		\\
\end{tabular} 
\end{center}
\end{table}

\item \func{rollrec\_recval(rollrec\_name,rollrec\_field)}\verb" "

This routine returns the value of a specified field in a given {\it rollrec}.
{\it rollrec\_name} is the name of the particular {\it rollrec} to consult.
{\it rollrec\_field} is the field name within that {\it rollrec}.

For example, the current {\it rollrec} file contains the following {\it
rollrec}.

\begin{verbatim}
    roll        "example.com"
        zonefile        "db.example.com"
\end{verbatim}

The call:

\begin{verbatim}
    rollrec_recval("example.com","zonefile")
\end{verbatim}

will return the value ``db.example.com''.

\item \func{rollrec\_rectype(rollrec\_name,rectype)}\verb" "

Set the type of the specified {\it rollrec} record.  The file is {\bf not}
written after updating the value, but the internal file-modified flag is set.
The value is saved in both {\it \%rollrecs} and in {\it \@rollreclines}.

{\it rollrec\_name} is the name of the {\it rollrec} that will be modified.
{\it rectype} is the new type of the {\it rollrec}, which must be either
``roll'' or ``skip''.

Return values:

\begin{table}[ht]
\begin{center}
\begin{tabular}{cl}
0 & failure (invalid record type or rollrec not found)	\\
1 & success						\\
\end{tabular} 
\end{center}
\end{table}

\item \func{rollrec\_setval(rollrec\_name,field,value)}\verb" "

Set the value of a name/field pair in a specified {\it rollrec}.  The file is
{\bf not} written after updating the value, but the internal file-modified
flag is set.  The value is saved in both {\it \%rollrecs} and in {\it
\@rollreclines}.

{\it rollrec\_name} is the name of the {\it rollrec} that will be modified.  If
the named {\it rollrec} does not exist, it will be created as a ``roll''-type
{\it rollrec}.  {\it field} is the {\it rollrec} field which will be modified.
{\it value} is the new value for the field.

\item \func{rollrec\_settime(rollrec\_name)}\verb" "

Set the timestamp in the {\it rollrec} specified by {\it rollrec\_name}.
The file is {\bf not} written after updating the value.

\item \func{rollrec\_unlock()}\verb" "

\func{rollrec\_unlock()} unlocks the {\it rollrec} synchronization file.

\item \func{rollrec\_write()}\verb" "

This interface saves the internal version of the {\it rollrec} file (opened
with \func{rollrec\_read()}).  It does not close the file handle.  As an
efficiency measure, an internal modification flag is checked prior to writing
the file.  If the program has not modified the contents of the {\it rollrec}
file, it is not rewritten.

\end{description}

{\bf ROLLREC INTERNAL INTERFACES}

The \perlmod{Net::DNS::SEC::Tools::rollrec} module has a number of interfaces,
described in this section, that are intended for internal use only.  However,
there are situations where external entities may have need of them.  Use with
caution, as misuse may result in damaged or lost {\it rollrec} files.

\begin{description}

\item \func{rollrec\_init()}\verb" "

This routine initializes the internal {\it rollrec} data.  Pending changes
will be lost.  An open {\it rollrec} file handle will remain open, though the
data are no longer held internally.  A new {\it rollrec} file must be read in
order to use the \perlmod{Net::DNS::SEC::Tools::rollrec} interfaces again.

\item \func{rollrec\_newrec(type,name)}\verb" "

This interface creates a new {\it rollrec}.  The {\it rollrec\_name} field in
the {\it rollrec} is set to the values of the {\it name} parameter.  The {\it
type} parameter must be either ``roll'' or ``skip''.

\item \func{rollrec\_default()}\verb" "

This routine returns the name of the default {\it rollrec} file.

\end{description}

{\bf ROLLREC DEBUGGING INTERFACES}

The following interfaces display information about the currently parsed {\it
rollrec} file.  They are intended to be used for debugging and testing, but
may be useful at other times.

\begin{description}

\item \func{rollrec\_dump\_hash()}\verb" "

This routine prints the {\it rollrec} file as it is stored internally in a
hash table.  The {\it rollrec}s are printed in alphabetical order, with the
fields alphabetized for each {\it rollrec}.  New {\it rollrec}s and {\it
rollrec} fields are alphabetized along with current {\it rollrec}s and fields.
Comments from the {\it rollrec} file are not included with the hash table.

\item \func{rollrec\_dump\_array()}\verb" "

This routine prints the {\it rollrec} file as it is stored internally in an
array.  The {\it rollrec}s are printed in the order given in the file, with
the fields ordered in the same manner.  New {\it rollrec}s are appended to the
end of the array.  {\it rollrec} fields added to existing {\it rollrec}s are
added at the beginning of the {\it rollrec} entry.  Comments and vertical
whitespace are preserved as given in the {\it rollrec} file.

\end{description}

{\bf SEE ALSO}

\cmd{lsroll(1)},
\cmd{rollchk(8)},
\cmd{rollinit(8)}

\perlmod{Net::DNS::SEC::Tools::keyrec.pm(3)}

\perlmod{Net::DNS::SEC::Tools::keyrec(5)}

\clearpage

\subsubsection{timetrans.pm}

{\bf NAME}

\perlmod{Net::DNS::SEC::Tools::timetrans} - Convert an integer seconds
count into text units.

{\bf SYNOPSIS}

\begin{verbatim}
  use Net::DNS::SEC::Tools::timetrans;

  $timestring = timetrans(86488);

  $timestring = fuzzytimetrans(86488);
\end{verbatim}

{\bf DESCRIPTION}

The \func{timetrans}() interface in \perlmod{Net::DNS::SEC::Tools::timetrans}
converts an integer seconds count into the equivalent number of weeks, days,
hours, and minutes.  The time converted is a relative time, {\bf not} an
absolute time.  The returned time is given in terms of weeks, days, hours,
minutes, and seconds, as required to express the seconds count appropriately.

The \func{fuzzytimetrans}() interface converts an integer seconds count into
the equivalent number of weeks {\bf or} days {\bf or} hours {\bf or} minutes.
The unit chosen is that which is most natural for the seconds count.  One
decimal place of precision is included in the result.

{\bf INTERFACES}

The interfaces to the \perlmod{Net::DNS::SEC::Tools::timetrans} module are
given below.

\begin{description}

\item \func{timetrans()}

This routine converts an integer seconds count into the equivalent number of
weeks, days, hours, and minutes.  This converted seconds count is returned
as a text string.  The seconds count must be greater than zero or an error
will be returned.

Return Values:

\begin{description}

\item If a valid seconds count was given, the count converted into the
appropriate text string will be returned.

\item An empty string is returned if no seconds count was given or if
the seconds count is less than one.

\end{description}

\item \func{fuzzytimetrans()}

This routine converts an integer seconds count into the equivalent number of
weeks, days, hours, or minutes.  This converted seconds count is returned
as a text string.  The seconds count must be greater than zero or an error
will be returned.

Return Values:

\begin{description}

\item If a valid seconds count was given, the count converted into the
appropriate text string will be returned.

\item An empty string is returned if no seconds count was given or if
the seconds count is less than one.

\end{description}

\end{description}

{\bf EXAMPLES}

{\it timetrans(400)} returns 6 minutes, 40 seconds

{\it timetrans(420)} returns 7 minutes

{\it timetrans(888)} returns 14 minutes, 48 seconds

{\it timetrans(86400)} returns 1 day

{\it timetrans(86488)} returns 1 day, 28 seconds

{\it timetrans(715000)} returns 1 week, 1 day, 6 hours, 36 minutes, 40 second

{\it timetrans(720000)} returns 1 week, 1 day, 8 hours

{\it fuzzytimetrans(400)} returns 6.7 minutes

{\it fuzzytimetrans(420)} returns 7.0 minutes

{\it fuzzytimetrans(888)} returns 14.8 minutes

{\it fuzzytimetrans(86400)} returns 1.0 day

{\it fuzzytimetrans(86488)} returns 1.0 day

{\it fuzzytimetrans(715000)} returns 1.2 weeks

{\it fuzzytimetrans(720000)} returns 1.2 weeks

{\bf SEE ALSO}

timetrans(1)


\clearpage

\subsection{\perlmod{tooloptions.pm}}


{\bf NAME}

Net::DNS::SEC::Tools::tooloptions - DNSSEC-Tools option routines.

{\bf SYNOPSIS}

\begin{verbatim}
  use Net::DNS::SEC::Tools::tooloptions;

  $keyrec_file = "example.keyrec";
  $keyrec_name = "Kexample.com.+005+10988";
  @specopts = ("propagate+", "waittime=i");

  $optsref = tooloptions($keyrec_file,$keyrec_name);
  %options = %$optsref;

  $optsref = tooloptions($keyrec_file,$keyrec_name,@specopts);
  %options = %$optsref;

  $optsref = tooloptions("",@specopts);
  %options = %$optsref;

  ($krfile,$krname,$optsref) = opts_krfile($keyrec_file,"");
  %options = %$optsref;

  ($krfile,$krname,$optsref) = opts_krfile("",$keyrec_name,@specopts);
  %options = %$optsref;

  ($krfile,$krname,$optsref) = opts_krfile("","");
  %options = %$optsref;

  $key_ref = opts_keykr();
  %key_kr  = %$key_ref;

  $optsref = opts_keykr($keyrec_file,$keyrec_name);
  %options = %$optsref;

  $zoneref = opts_zonekr();
  %zone_kr = %$zoneref;

  $zoneref = opts_zonekr($keyrec_file,$keyrec_name);
  %zone_kr = %$zoneref;

  opts_setcsopts(@specopts);

  opts_createkrf();

  opts_suspend();

  opts_restore();

  opts_drop();

  opts_reset();
\end{verbatim}


{\bf DESCRIPTION}

DNSSEC-Tools supports a set of options common to all the tools in the suite.
These options may have defaults set in the {\bf dnssec-tools.conf}
configuration file, in a {\it keyrec} file, from command-line options, or
from any combination of the three.  In order to enforce a common sequence of
option interpretation, all DNSSEC-Tools should use the {\bf tooloptions()}
routine to initialize its options.

The {\it keyrec\_file} argument specifies a {\it keyrec} file that will be
consulted.  The {\it keyrec} named by the {\it keyrec\_name} argument will
be loaded.  If no {\it keyrec} file should be used, then {\it keyrec\_file}
should be an empty string and the {\it keyrec\_name} parameter not included.
The {\it @specopts} array contains command-specific arguments; the arguments
must be in the format prescribed by the {\bf Getopt::Long} Perl module.

{\bf tooloptions()} combines data from these three option sources into a hash
table.  The hash table is returned to the caller, which will then use the
options as needed.

The command-line options are saved between calls, so a command may call {\bf
tooloptions()} multiple times and still have the command-line options included
in the final hash table.  This is useful for examining multiple {\it keyrec}s
in a single command.  Inclusion of command-line options may be suspended and
restored using the {\bf opts\_suspend()} and {\bf opts\_restore()} calls.
Options may be discarded entirely by calling {\bf opts\_drop()}; once dropped,
command-line options may never be restored.  Suspension, restoration, and
dropping of command-line options are only effective after the initial {\bf
tooloptions()} call.

The options sources are combined in this manner:

\begin{description}

\item [1.  {\bf dnssec-tools.conf}]\verb" "

The system-wide configuration file is read and these option values are used
as the defaults.  These options are put into a hash table, with the option
names as the hash key.

\item [2. {\it keyrec} File]\verb" "

If a {\it keyrec} file was specified, then the {\it keyrec} named by {\it
keyrec\_name} will be retrieved.  The {\it keyrec}'s fields are added to the
hash table.  Any field whose keyword matches an existing hash key will
override the existing value.

\item [3. Command-line Options]\verb" "

The command-line options, specified in {\it @specopts}, are parsed using {\bf
Getoptions()} from the {\bf Getopt::Long} Perl module.  These options are
folded into the hash table; again possibly overriding existing hash values.
The options given in {\it @specopts} must be in the format required by {\bf
Getoptions()}.

\end{description}

A reference to the hash table created in these three steps is returned to the
caller.


{\bf EXAMPLE}

{\bf dnssec-tools.conf} has these entries:

\begin{verbatim}
    ksklength      1024
    zsklength      512
\end{verbatim}

{\bf example.keyrec} has this entry:

\begin{verbatim}
    key         "Kexample.com.+005+10988"
        zsklength        "1024"
\end{verbatim}

{\it zonesigner} is executed with this command line:

\begin{verbatim}
    zonesigner -ksklength 512 -zsklength 4096 -wait 600 ...  example.com
\end{verbatim}

{\bf tooloptions("example.keyrec","Kexample.com.+005+10988",("wait=i"))}
will read each option source in turn, ending up with:
\begin{verbatim}
    I<ksklength>           512
    I<zsklength>          4096
    I<wait>                600
\end{verbatim}


{\bf TOOL OPTION ARGUMENTS}

Many of the DNSSEC-Tools option interfaces take the same set of arguments:
{\it \$keyrec\_file}, {\it \$keyrec\_name}, and {\it @csopts}.  These arguments
are used similarly by most of the interfaces; differences are noted in the
interface descriptions in the next section.

\begin{description}

\item [{\it \$keyrec\_file}] Name of the {\it keyrec} file to be searched.

\item [{\it \$keyrec\_name}] Name of the {\it keyrec} that is being sought

\item [{\it @csopts}] Command-specific options.

\end{description}

The {\it keyrec} named in {\it \$keyrec\_name} is selected from the {\it
keyrec} file given in {\it \$keyrec\_file}.  If either {\it \$keyrec\_file}
or {\it \$keyrec\_name} are given as empty strings, their values will be taken
from the {\it -krfile} and {\it -keyrec} command line options.

A set of command-specific options may be specified in {\it @csopts}.  These
options are in the format required by the {\bf Getopt::Long} Perl module.  If
{\it @csopts} is left off the call, then no command-specific options will be
included in the final option hash.  The {\it @csopts} array may be passed
directly to several interfaces or it may be saved in a call to {\it
opts\_setcsopts()}.


{\bf TOOL OPTION INTERFACES}

\begin{description}

\item [{\bf tooloptions(\$keyrec\_file,\$keyrec\_name,@csopts)}]\verb" "

This {\bf tooloptions()} call builds an option hash from the system
configuration file, a {\it keyrec}, and a set of command-specific options.
A reference to this option hash is returned to the caller.

If {\it \$keyrec\_file} is given as an empty string, then no {\it keyrec}
file will be consulted.  In this case, it is assumed that {\it \$keyrec\_name}
will be left out altogether.

If a non-existent {\it \$keyrec\_file} is given and {\bf opts\_createkrf()}
has been called, then the named {\it keyrec} file will be created.  {\it
opts\_createkrf()} must be called for each {\it keyrec} file that must be
created, as the {\bf tooloptions} {\it keyrec}-creation state is reset after
{\bf tooloptions()} has completed.

\item [{\bf opts\_krfile(\$keyrec\_file,\$keyrec\_name,@csopts)}]\verb" "

The {\bf opts\_krfile()} routine looks up the {\it keyrec} file and {\it
keyrec} name and uses those fields to help build an options hash.  References
to the {\it keyrec} file name, {\it keyrec} name, and the option hash table
are returned to the caller.

The {\it \$keyrec\_file} and {\it \$keyrec\_name} arguments are required
parameters.  They may be given as empty strings, but they {\bf must} be given.

If the {\it \$keyrec\_file} file and {\it \$keyrec\_name} name are both
specified by the caller, then this routine will have the same effect as
directly calling {\bf tooloptions()}.


\item [{\bf opts\_getkeys(\$keyrec\_file,\$keyrec\_name,@csopts)}]\verb" "

This routine returns references to the KSK and ZSK {\it keyrec}s associated
with a specified {\it keyrec} entry.  This gives an easy way to get a zone's
{\it keyrec} entries in a single step.

This routine acts as a front-end to the {\bf opts\_krfile()} routine.
Arguments to {\bf opts\_getkeys()} conform to those of {\bf opts\_krfile()}.

If {\bf opts\_getkeys()} isn't passed any arguments, it will act as if both
{\it \$keyrec\_file} and {\it \$keyrec\_name} were given as empty strings.  In
this case, their values will be taken from the {\it -krfile} and {\it -keyrec}
command line options.


\item [{\bf opts\_keykr(\$keyrec\_file,\$keyrec\_name,@csopts)}]\verb" "

This routine returns a reference to the key {\it keyrec} named by
{\it \$keyrec\_name}.  It ensures that the named {\it keyrec} is a
key {\it keyrec}; if it isn't, {\it undef} is returned.

This routine acts as a front-end to the {\bf opts\_krfile()} routine.
{\bf opts\_keykr()}'s arguments conform to those of {\bf opts\_krfile()}.

If {\bf opts\_keykr()} isn't passed any arguments, it will act as if both
{\it \$keyrec\_file} and {\it \$keyrec\_name} were given as empty strings.
In this case, their values will be taken from the {\it -krfile} and {\it
-keyrec} command line options.


\item [{\bf opts\_zonekr(\$keyrec\_file,\$keyrec\_name,@csopts)}]\verb" "

This routine returns a reference to the zone {\it keyrec} named by
{\it \$keyrec\_name}.  The {\it keyrec} fields from the zone's KSK and ZSK
are folded in as well, but the key's {\it keyrec\_} fields are excluded.
This call ensures that the named {\it keyrec} is a zone {\it keyrec};
if it isn't, {\it undef} is returned.

This routine acts as a front-end to the {\bf opts\_krfile()} routine.
{\bf opts\_zonekr()}'s arguments conform to those of {\bf opts\_krfile()}.

If {\bf opts\_zonekr()} isn't passed any arguments, it will act as if both
{\it \$keyrec\_file} and {\it \$keyrec\_name} were given as empty strings.
In this case, their values will be taken from the {\it -krfile} and {\it
-keyrec} command line options.

\item [{\bf opts\_setcsopts(@csopts)}]\verb" "

This routine saves a copy of the command-specific options given in {\it
@csopts}.  This collection of options is added to the {\it @csopts} array
that may be passed to {\bf tooloptions()}.

\item [{\bf opts\_createkrf()}]\verb" "

Force creation of an empty {\it keyrec} file if the specified file does not
exist.  This may happen on calls to {\bf tooloptions()}, {\bf opts\_getkeys()},
{\bf opts\_krfile()}, and {\bf opts\_zonekr()}.

\item [{\bf opts\_suspend()}]\verb" "

Suspend inclusion of the command-line options in building the final hash
table of responses.

\item [{\bf opts\_restore()}]\verb" "

Restore inclusion of the command-line options in building the final hash
table of responses.

\item [{\bf opts\_drop()}]\verb" "

Discard the command-line options.  They will no longer be available for
inclusion in building the final hash table of responses for this execution
of the command.

\item [{\bf opts\_reset()}]\verb" "

Reset an internal flag so that the command-line arguments may be
re-examined.  This is usually only useful if the arguments have been
modified by the calling program itself.

\end{description}

{\bf SEE ALSO}

{\bf zonesigner(8)}

{\bf Getopt::Long(3)}

{\bf Net::DNS::SEC::Tools::conf(3)}, {\bf Net::DNS::SEC::Tools::keyrec(3)},

{\bf Net::DNS::SEC::Tools::keyrec(5)}




%%%%%%%%%%%%%%%%%%%%%%%%%%%%%%%%%%%%%%%%%%%%%%%%%%%%%%%%%%%%%%%%%%%%%%%%%%%%%%

\clearpage

\markboth{DNSSEC-Tools Software User Manual (vers. 4) -- Manual Pages}{DNSSEC-Tools Software User Manual (vers. 4) -- Manual Pages}
\section{Data Files}
\markboth{DNSSEC-Tools Software User Manual (vers. 4) -- Manual Pages}{DNSSEC-Tools Software User Manual (vers. 4) -- Manual Pages}
\label{sect-files}


Several data files are used by the DNSSEC-Tools components.

These DNSSEC-Tools files are:

\begin{description}

\item{\path{dnssec-tools.conf}} - Configuration file for DNSSEC-Tools programs

\item{\path{dnsval.conf}} - Configuration file for ...

\item{\path{keyrec}} - Key and zone configuration files for DNSSEC-Tools
programs

\item{\path{rollrec}} - Rollover configuration files for DNSSEC-Tools programs

\item{\cmd{blinkenlights} rules} - Rule definition files for
\cmd{blinkenlights}.

\item{\cmd{donuts} rules} - Rule definition files for \cmd{donuts}.

\end{description}

This section contains man pages describing these commands.

\clearpage

\subsection{\bf dnssec-tools.conf}

{\bf NAME}

\path{dnssec-tools.conf} - Configuration file for the DNSSEC-Tools programs.

{\bf DESCRIPTION}

This file contains configuration information for the DNSSEC-Tools programs.
These configuration data are used if nothing else has been specified for a
particular program.  The \path{conf.pm} module is used to parse this
configuration file.

A line in the configuration file contains either a comment or a configuration
entry.  Comment lines start with either a `\#' character or a `;' character.
Comment lines and blank lines are ignored by the DNSSEC-Tools programs.

Configuration entries are in a {\it keyword/value} format.  The keyword is a
character string that contains no whitespace.  The value is a tokenized list
of the remaining character groups, with each token separated by a single space.

True/false flags must be given a {\bf 1} (true) or {\bf 0} (false) value.

{\bf Configuration Records}

The following records are recognized by the DNSSEC-Tools programs.
Not every DNSSEC-Tools program requires each of these records.

\begin{description}

\item [algorithm]\verb" "

The default encryption algorithm to be passed to \cmd{dnssec-keygen}.

\item [archivedir]\verb" "

The pathname to the archived-key directory.

\item [checkzone]\verb" "

The path to the \cmd{named-checkzone} command.

\item [default\_keyrec]\verb" "

The default {\it keyrec} filename to be used by the \perlmod{keyrec.pm} module.

\item [endtime]\verb" "

The zone default expiration time to be passed to \cmd{dnssec-signzone}.

\item [entropy\_msg]\verb" "

A true/false flag indicating if the \cmd{zonesigner} command should display
a message about entropy generation.  This is primarily dependent on the
implementation of a system's random number generation.

\item [keygen]\verb" "

The path to the \cmd{dnssec-keygen} command.

\item [ksklength]\verb" "

The default KSK key length to be passed to \cmd{dnssec-keygen}.

\item [ksklife]\verb" "

The default length of time between KSK roll-overs.  This is measured in
seconds.

This value is {\bf only} used for key roll-over.  Keys do not have a life-time
in any other sense.

\item [lifespan-max]\verb" "

The maximum length of time a key should be in use before it is rolled over.
This is measured in seconds.

\item [lifespan-min]\verb" "

The minimum length of time a key should be in use before it is rolled over.
This is measured in seconds.

\item [random]\verb" "

The random device generator to be passed to \cmd{dnssec-keygen}.

\item [savekeys]\verb" "

A true/false flag indicating if old keys should be moved to the
archive directory.

\item [signzone]\verb" "

The path to the \cmd{dnssec-signzone} command.

\item [usegui]\verb" "

Flag to allow/disallow usage of the GUI for specifying command options.

\item [zonesigner]\verb" "

The path to the \cmd{zonesigner} command.

\item [zskcount]\verb" "

The default number of ZSK keys that will be generated for each zone.

\item [zsklength]\verb" "

The default ZSK key length to be passed to \cmd{dnssec-keygen}.

\item [zsklife]\verb" "

The default length of time between ZSK roll-overs.  This is measured in
seconds.

This value is {\bf only} used for key roll-over.  Keys do not have a life-time
in any other sense.

\end{description}

{\bf Sample Times}

Several configuration fields measure various times.  This section is a
convenient reference for several common times, as measured in seconds.

\begin{verbatim}
    3600       - hour
    86400      - day
    604800     - week
    2592000    - 30-day month
    15768000   - half-year
    31536000   - year
\end{verbatim}

{\bf Example File}

The following is an example \path{dnssec-tools.conf} configuration file.

\begin{verbatim}
    #
    # Paths to required programs.  These may need adjusting for
    # individual hosts.
    #
    checkzone       /usr/local/sbin/named-checkzone
    keygen          /usr/local/sbin/dnssec-keygen
    rndc            /usr/local/sbin/rndc
    signzone        /usr/local/sbin/dnssec-signzone
    viewimage       /usr/X11R6/bin/xview

    rollrec-chk     /usr/bin/rollrec-check
    zonesigner      /usr/bin/zonesigner

    #
    # Settings for dnssec-keygen.
    #
    algorithm        rsasha1
    ksklength        2048
    zsklength        1024
    random        /dev/urandom
    
    #
    # Settings for dnssec-signzone.
    #
    endtime         +2592000      # RRSIGs good for 30 days.
    
    # Life-times for keys.  These defaults indicate how long a key has
    # between roll-overs.  The values are measured in seconds.
    # 
    ksklife         15768000      # Half-year.
    zsklife         604800        # One week.
    lifespan-max    94608000      # Two years.
    lifespan-min    3600          # One hour.

    #
    # Settings that will be noticed by zonesigner.
    #
    archivedir          /usr/local/etc/dnssec/KEY-SAFE
    default_keyrec      default.krf
    entropy_msg         0
    savekeys            1
    zskcount            1

    #
    # Settings for rollover-manager.
    #
    roll_logfile    /usr/local/etc/dnssec/rollerd
    roll_loglevel   info
    roll_sleeptime  60

    #
    # GUI-usage flag.
    #
    usegui                0
\end{verbatim}

{\bf SEE ALSO}

\cmd{dtinitconf(8)},
\cmd{dtconfchk(8)},
\cmd{rollerd(8)},
\cmd{zonesigner(8)}

\perlmod{Net::DNS::SEC::Tools::conf.pm(3)},
\perlmod{Net::DNS::SEC::Tools::keyrec.pm(3)}

\clearpage

\subsection{\bf dnsval.conf}

{\bf NAME}

\path{/etc/dnsval.conf} - Configuration policy for the DNSSEC validator
library \lib{libval(3)}.

{\bf DESCRIPTION}

The validator library reads configuration information from three files,
\path{/etc/resolv.conf}, \path{/etc/root.hints}, and \path{/etc/dnsval.conf}.

\begin{description}

\item [/etc/resolv.conf]\verb" "

Only the {\it nameserver} option is supported in the \path{resolv.conf} file.
This option is used to specify the IP address of the name server to which
queries must be sent by default.  For example,

\begin{verbatim}
    nameserver 10.0.0.1
\end{verbatim}

The \path{/etc/resolv.conf} file may be empty in which case the validator in
\lib{libval(3)} tries to recursively answer the query using information
present in \path{/etc/root.hints}.

\item [/etc/root.hints]\verb" "

The \path{/etc/root.hints} file contains bootstrapping information for the
resolver while it attempts to recursively answer queries.  The contents of
this file may be generated by the following command:

\begin{verbatim}
    dig @e.root-servers.net . ns > root.hints
\end{verbatim}

\item [/etc/dnsval.conf]\verb" "

The \path{/etc/dnsval.conf} file contains a sequence of the following
"policy-fragments":

\begin{verbatim}
    <label> <KEYWORD> <additional-data>; 
\end{verbatim}

{\it label} identifies the policy fragment 
and {\it KEYWORD} is the specific policy component that is 
configured.  The format of additional-data depends on the 
keyword specified.

If multiple policy fragments are defined for the same label and keyword
combination then the last definition in the file is used.

Currently two different keywords are specified:

\begin{description}

\item [trust-anchor]\verb" "

Specifies the trust anchors for a sequence of zones.  The additional
data portion for this keyword is a sequence of the zone name and a 
quoted string containing the RDATA portion for the trust anchor's 
DNSKEY.

\item [zone-security-expectation]\verb" "

Specifies the local security expectation for a zone.  The additional
data portion for this keyword is a sequence of the zone name and 
its trust status - {\it ignore}, {\it validate}, or {\it untrusted}.

\end{description}
\end{description}

{\bf EXAMPLE}

The \path{/etc/dnsval.conf} configuration file might appear as follows:

\begin{verbatim}
    mozilla trust-anchor]
        dnssec-tools.org.
            "257 3 5 AQO8XS4y9r77X9SHBmrx-
            MoJf1Pf9AT9Mr/L5BBGtO9/e9f/zl4FFgM2l
            B6M2XEm6mp6mit4tzpB/sAEQw1McYz6bJdKkTiqtuWTCfDmgQhI6/Ha0
            EfGPNSqnY 99FmbSeWNIRaa4fgSCVFhvbrYq1nXkNVyQPeEVHkoDNCAlr
            qOA3lw=="]
        netsec.tislabs.com.
            "257 3 5 AQO8XS4y9r77X9SHBmrx-
            MoJf1Pf9AT9Mr/L5BBGtO9/e9f/zl4FFgM2l
            B6M2XEm6mp6mit4tzpB/sAEQw1McYz6bJdKkTiqtuWTCfDmgQhI6/Ha0
            EfGPNSqnY 99FmbSeWNIRaa4fgSCVFhvbrYq1nXkNVyQPeEVHkoDNCAlr
            qOA3lw==" ;]

    : zone-security-expectation
        org ignore ]
        net ignore]
        dnssec-tools.org validate]
        com ignore;]
\end{verbatim}

{\bf FILES}

\path{/etc/dnsval.conf(5)}

\path{/etc/resolv.conf(5)}

\path{/etc/root.hints(5)}

{\bf SEE ALSO}

\lib{libval(3)}

\clearpage

\subsection{{\it keyrec} Files}

{\bf NAME}

{\bf keyrec} - Zone and key data used by DNSSEC-Tools programs.

{\bf DESCRIPTION}

{\it keyrec} files contain data about zones signed by and keys generated by
the DNSSEC-Tools.  A {\it keyrec} file is organized in sets of {\it keyrec}
records.  Each {\it keyrec} must be either of {\it key} type or {\it zone}
type.  Key {\it keyrec}s describe how encryption keys were generated; zone
{\it keyrec}s describe how zones were signed.  A {\it keyrec} consists of a
set of keyword/value entries.

The DNSSEC-Tools {\bf keyrec} module manipulates the contents of a {\it
keyrec} file.  Module interfaces exist for looking up {\it keyrec} records,
creating new records, and modifying existing records.

The following is an example of a key {\it keyrec}:

\begin{verbatim}
    key        "Kexample.com.+005+39936"
            zonename        "example.com"
            keyrec_type     "ksk"
            algorithm       "rsasha1"
            random          "/dev/urandom"
            keypath         "./Kexample.com.+005+39936.key"
            ksklength       "1024"
            keyrec_gensecs  "1123771354"
            keyrec_gendate  "Thu Aug 11 14:42:34 2005"
\end{verbatim}

The following is an example of a zone {\it keyrec}:

\begin{verbatim}
    zone        "example.com"
            zsknew          "Kexample.com.+005+60521"
            zskpubpath      "./Kexample.com.+005+23057.key"
            zskcurpath      "./Kexample.com.+005+41702.key"
            kskpath         "./Kexample.com.+005+39936.key"
            zskpub          "Kexample.com.+005+23057"
            zskcur          "Kexample.com.+005+41702"
            kskpath         ""
            zskdirectory    "."
            signedfile      "db.example.com.signed"
            kskpath         ""
            kskkey          "Kexample.com.+005+39936"
            kskdirectory    "."
            endtime         "+604800"
            zonefile        "db.example.com"
            keyrec_type     "zone"
            keyrec_signsecs "1123771721"
            keyrec_signdate "Thu Aug 11 14:48:41 2005"
\end{verbatim}

{\bf SEE ALSO}

{\bf Net::DNS::SEC::Tools::keyrec(3)}


\clearpage

\subsubsection{Rollrec Files}

{\bf NAME}

\struct{rollrec} - Rollover-related zone data used by DNSSEC-Tools programs.

{\bf DESCRIPTION}

\struct{rollrec} files contain data used by the DNSSEC-Tools to manage key
rollover.  A \struct{rollrec} file is organized in sets of \struct{rollrec}
records.  Each \struct{rollrec} record describes the rollover state of a
single zone and must be either of {\it roll} type or {\it skip} type.  Zone
\struct{rollrec}s record information about currently rolling zones.  Skip
\struct{rollrec}s record information about zones that are not being rolled.
A \struct{rollrec} consists of a set of keyword/value entries.

The DNSSEC-Tools \perlmod{rollrec.pm} module manipulates the contents of
a \struct{rollrec} file.  Module interfaces exist for looking up
\struct{rollrec} records, creating new records, and modifying existing
records.

Comment lines and blank lines are ignored by the DNSSEC-Tools programs.
Comment lines start with either a `\#' character or a `;' character.

A \struct{rollrec}'s name may consist of alphabetic characters, numbers, and
several special characters.  The special characters are the minus sign, the
plus sign, the underscore, the comma, the period, the colon, the
forward-slash, the space, and the tab.

The values in a \struct{rollrec}'s entries may consist of alphabetic
characters, numbers, and several special characters.  The special characters
are the minus sign, the plus sign, the underscore, the comma, the period, the
colon, the forward-slash, the space, and the tab.

{\bf FIELDS}

The fields in a \struct{rollrec} record are:

\begin{description}

\item {\it administrator}\verb" "

This is the email address for the zone's administrative user.  If it is not
set, the default from the DNSSEC-Tools configuration file will be used.

\item {\it directory}\verb" "

This field contains the name of the directory in which {\bf rollerd} will
execute for the \struct{rollrec}'s zone.  If it is not specified, then the
normal {\bf rollerd} execution directory will be used.

\item {\it display}\verb" "

This boolean field indicates whether or not the zone should be displayed by
the {\bf blinkenlights} program.

\item {\it keyrec}\verb" "

The zone's \struct{keyrec} file.

\item {\it kskphase}\verb" "

The zone's current KSK rollover phase.  A value of zero indicates that the
zone is not in rollover, but is in normal operation.  A numeric value of 1-7
indicates that the zone is in that phase of KSK rollover.

\item {\it ksk\_rolldate}\verb" "

The time at which the zone's last KSK rollover completed.  This is only used
to provide a human-readable format of the timestamp.  It is derived from the
{\it ksk\_rollsecs} field.

\item {\it ksk\_rollsecs}\verb" "

The time at which the zone's last KSK rollover completed.  This value is used
to derive the {\it ksk\_rolldate} field.

\item {\it loglevel}\verb" "

The {\bf rollerd} logging level for this zone.

\item {\it maxttl}\verb" "

The maximum time-to-live for the zone.  This is measured in seconds.

\item {\it phasestart}\verb" "

The time-stamp of the beginning of the zone's current phase.

\item {\it zonefile}\verb" "

The zone's zone file.

\item {\it zskphase}\verb" "

The zone's current ZSK rollover phase.  A value of zero indicates that the zone
is not in rollover, but is in normal operation.  A value of 1, 2, 3, 4
indicates that the zone is in that phase of ZSK rollover.

\item {\it zsk\_rolldate}\verb" "

The time at which the zone's last ZSK rollover completed.  This is only used
to provide a human-readable format of the timestamp.  It is derived from the
{\it ksk\_rollsecs} field.

\item {\it zsk\_rollsecs}\verb" "

The time at which the zone's last ZSK rollover completed.  This value is used
to derive the {\it ksk\_rolldate} field.

\end{description}

{\bf EXAMPLES}

The following is an example of a roll \struct{rollrec} record:

\begin{verbatim}

    roll "example.com"
            zonefile        "example.signed"
            keyrec          "example.krf"
            kskphase        "1"
            zskphase        "0"
            administrator   "bob@bobbox.example.com"
            loglevel        "info"
            maxttl          "60"
            display         "1"
            ksk_rollsecs    "1172614842"
            ksk_rolldate    "Tue Feb 27 22:20:42 2007"
            zsk_rollsecs    "1172615087"
            zsk_rolldate    "Tue Feb 27 22:24:47 2007"
            phasestart      "Mon Feb 20 12:34:56 2007"

\end{verbatim}

The following is an example of a skip \struct{rollrec} record:

\begin{verbatim}

    skip "test.com"
            zonefile        "test.com.signed"
            keyrec          "test.com.krf"
            kskphase        "0"
            zskphase        "2"
            administrator   "tess@test.com"
            loglevel        "info"
            maxttl          "60"
            display         "1"
            ksk_rollsecs    "1172614800"
            ksk_rolldate    "Tue Feb 27 22:20:00 2007"
            zsk_rollsecs    "1172615070"
            zsk_rolldate    "Tue Feb 27 22:24:30 2007"
            phasestart      "Mon Feb 20 12:34:56 2007"

\end{verbatim}

{\bf SEE ALSO}

lsroll(1),

blinkenlights(8),
rollerd(8),
zonesigner(8)

Net::DNS::SEC::Tools::keyrec(3),
Net::DNS::SEC::Tools::rollrec(3)

keyrec(5)


\clearpage

\subsubsection{blinkenlights.conf}

{\bf NAME}

\path{blinkenlights.conf} - Configuration file for the DNSSEC-Tools
\cmd{blinkenlights} program

{\bf DESCRIPTION}

This file contains configuration information for the DNSSEC-Tools
\cmd{blinkenlights} program.  These configuration data are used as default
values The \perlmod{conf.pm} module is used to parse this configuration file.

A line in this file contains either a comment or a configuration
entry.  Comment lines start with either a `\#' character or a `;' character.
Comment lines and blank lines are ignored by the DNSSEC-Tools programs.

Configuration entries are in a {\it keyword/value} format.  The keyword is a
character string that contains no whitespace.  The value is a tokenized list
of the remaining character groups, with each token separated by a single space.

True/false flags must be given a true or false value.
True values are:  1, ``yes'', ``on''.
False values are:  0, ``no'', ``off''.

{\bf Configuration Records}

The following records are recognized by \cmd{blinkenlights}.

\begin{description}

\item colors\verb" "

Toggle indicating whether or not to use different background colors for
\cmd{blinkenlights} zone stripes.
If on, different colors will be used.
If off, the {\it skipcolor} value will be used.

\item fontsize\verb" "

The font size used to display information in the \cmd{blinkenlights} window.
If this is not specified, the default font size is 18.

\item modify\verb" "

Toggle indicating whether or not to allow access to \cmd{blinkenlights}'
zone-modification commands.  These commands are the GUI's front-end to
some of \cmd{rollerd}'s commands.
If on, the commands are enabled.
If off, the commands are disabled.

\item shading\verb" "

Toggle indicating whether or not to use color shading in \cmd{blinkenlights}'
status column.
If on, shading is enabled.
If off, shading is disabled.

\item showskip\verb" "

Toggle indicating whether or not to display skipped zones in
\cmd{blinkenlights}' window.
If on, skipped zones are displayed.
If off, skipped zones are not displayed.

\item skipcolor\verb" "

The background color to use in displaying skipped zones.
If this is not specified, the default color is grey.

\end{description}

{\bf Example File}

The following is an example \cmd{blinkenlights.conf} configuration file.

\begin{verbatim}
    #
    # DNSSEC-Tools configuration file for blinkenlights
    #
    #   Recognized values:
    #         colors          use different colors for stripes (toggle)
    #         fontsize        size of demo output font
    #         modify          allow modification commands (toggle)
    #         shading         shade the status columns (toggle)
    #         showskip        show skipped zones (toggle)
    #         skipcolor       color to use for skip records

    fontsize        24
    modify          no

    colors          on
    skipcolor       orange
    showskip        1
    shading         yes
\end{verbatim}

{\bf SEE ALSO}

blinkenlights(8),
rollerd(8)

Net::DNS::SEC::Tools::conf.pm(3)


\clearpage

\subsection{\cmd{donuts} Rule Files}

{\bf NAME}

\begin{verbatim}
  Donuts Rules Files - Define donuts DNS record-checking rules
\end{verbatim}

{\bf DESCRIPTION}

This class wraps around a rule definition which is used by the {\it donuts}
DNS zone file checker.  It stores the data that implements a given rule.

Rules are defined in {\it donuts} rule configuration files using the
following syntax.  See the {\it donuts} manual page for details on where to
place those files and how to get them loaded.

{\bf RULE FILE FORMAT}

Each rule file can contain multiple rules.  Each rule is composed of a
number of parts.  Minimally, it must contain a {\bf name} and a {\bf test}
portion.  Everything else is optional and/or has defaults associated
with it.  The rule file format follows this example:

\begin{verbatim}
  name: rulename
  class: Warning
  test:
    my ($record) = @_;
    return "problem found"
      if ($record{xxx} != yyy);
\end{verbatim}

Further details about each section can be found below.  Besides the
tokens below, other rule-specific data can be stored in also tokens
and each rule is a hash of the above tokens as keys and their
associated data.  However, there are a few exceptions where special
tokens imply special meanings.  These special tokens include {\it test}
and {\it init}.  See below for details.

Each rule definition within a file should be separated using a blank line.

Lines beginning with the `\#' character will be discarded as a comment.

\begin{description}

\item [{\it name}]\verb" "

The name of the rule.  This is mandatory, as the user may need to be
able to refer to names in the future for use with the {\it -i} flag,
specifying behavior in configuration files, and for other uses.

By convention, all names should be specified using capital letters and
`\_' characters between the words.  The leftmost word should give an
indication of a global category of test, such as ``DNSSEC''.  The
better-named the rules, the more power the user will have for
selecting certain types of rules via {\it donuts -i} and other flags.

Example:

\begin{verbatim}  name: DNSSEC_TEST_SOME_SECURE_FEATURE\end{verbatim}

\item [{\it level}]\verb" "

The rule's execution level, as recognized by {\it donuts}.  Only those
rules at or above {\it donuts}' current execution level will be run by
{\it donuts}.  The execution level is specified by the {\it -l} option to
{\it donuts}; if not given, then the default execution level is 5.

The default {\it level} of every rule is 5.

Generally, more serious problems should receive lower numbers and
less serious problems should be placed at a higher number.  The
maximum value is 9, which is reserved for debugging rules only.
8 is the maximum rule level that user-defined rules should use.

Example:

\begin{verbatim}
  name: DNSSEC_TEST_SOME_SECURE_FEATURE
  level: 2
\end{verbatim}

\item [{\it class}]\verb" "

The {\it class} code indicates the type of problem associated with the
rule.  It defaults to ``{\it Error}'', and the only other value that should
be used is ``{\it Warning}''.

This value is displayed to the user.  Technically, any value could be
specified, but using anything other than {\it Error}/{\it Warning} convention
could break portability in future versions.

Example:
\begin{verbatim}
  name: DNSSEC_TEST_SOME_SECURE_FEATURE
  class: Warning
\end{verbatim}

\item [{\it ruletype}]\verb" "

Rules fall into one of two types (currently): {\it record} or {\it name}.
{\it record} rules have their test evaluated for each record being in
a zone file.  {\it name} rules, on the other hand, get called once per
name stored in the database.  See the {\it test} description below for
further details on the arguments passed to each rule type.

The default value for this clause is {\it record}.

Example:

\begin{verbatim}
  name: DNSSEC_TEST_SOME_SECURE_FEATURE
  ruletype: record
\end{verbatim}

\item [{\it type}]\verb" "

Rules that test a particular type of record should specify the
{\it type} field with the type of record it wants to test.  The rule
will only be executed for records of that type.  This will result
in less error checking for the user in the {\it test} section.

For example, if a rule is testing a particular aspect of an MX record,
it should specify MX in this field.

Example:

\begin{verbatim}
  name: DNSSEC_TEST_SOME_SECURE_FEATURE
  type: MX
\end{verbatim}

\item [{\it init}]\verb" "

A block of code to be executed immediately. This is useful for
boot-strap code to be performed only at start-up, rather than
at every rule-test invocation.  For example, ``use MODULE;''
type statements should be used in {\it init} sections.

{\it init} sections contain special formatting such as the following.
The code lines {\bf MUST} begin with whitespace.

Example:

\begin{verbatim}
  init:
    use My::Module;
    $value = calculate();
\end{verbatim}

\item [{\it test}]\verb" "

A block of code that defining the test for each record or name.
The test statement follows the same multi-line code specification
described in the {\it init} clause above.  Specifically, the first line
follows the line with the {\it test:} token and each line of code {\bf MUST}
begin with whitespace.

The end result must be a subroutine reference which will be called by
the {\it donuts} program.  When the code is evaluated, if it does not
begin with ``sub \{'' then a ``sub \{'' prefix and ``\}'' suffix will be
automatically added to the code to turn the code-snippet into a
subroutine.

If the test fails, it should return an error string which will be displayed
for the user.  The text will be line-wrapped before display (and thus should
be unformatted text.)  If the test is testing for multiple problems, a
reference to an array of error strings may be returned.  A reference to an
empty array being returned also indicates no error.

There are two types of tests (currently), and the code snippet is
called depending on the {\it ruletype} clause above.

\begin{description}

\item [{\it record} tests]\verb" "

These code snippets are expected to test a single {\bf Net::DNS::RR} record.

It is called with two arguments:

\begin{itemize}
\item the record which is to be tested
\item the rule definition itself.
\end{itemize}

\item [{\it name} tests]\verb" "

These code snippets are expected to test all the records
associated with a given name record.

It is called with three arguments:

\begin{enumerate}
\item a hash reference to all the record types associated
     with that name (e.g., `A', `MX', ...) and each value of
     the hash will contain an array of all the records for
     that type (i.e., for names containing multiple `A'
     records then more than one entry in the array reference
     will exist).

\item The rule definition

\item The record name being checked (the name associated with
     the data from 1) above).
\end{enumerate}

\end{description}

Examples:

\begin{verbatim}
  # local rule to mandate that each record must have a
  # TTL > 60 seconds
  name: DNS_TTL_AT_LEAST_60
  level: 8
  type: record
  test:
    return "TTL too small" if ($_[0]->ttl < 60);

  # local policy to mandate that anything with an A record
  # must have a HINFO record too
  name: DNS_MX_MUST_HAVE_A
  level: 8
  type: name
  test:
    return "A records must have a HINFO record too"
      if (exists($_[0]{'A'}) && !exists($_[0]{'HINFO'}));
\end{verbatim}

\end{description}

{\bf SEE ALSO}

{\bf donuts(8)}

{\bf Net::DNS}, {\bf Net::DNS::RR}

\url{http://dnssec-tools.sourceforge.net}




%%%%%%%%%%%%%%%%%%%%%%%%%%%%%%%%%%%%%%%%%%%%%%%%%%%%%%%%%%%%%%%%%%%%%%%%%%%%%%

\end{document}
