\clearpage

\subsubsection{conf.pm}

{\bf NAME}

\perlmod{Net::DNS::SEC::Tools::conf} - DNSSEC-Tools configuration routines.

{\bf SYNOPSIS}

\begin{verbatim}
  use Net::DNS::SEC::Tools::conf;

  %dtconf = parseconfig();

  %dtconf = parseconfig("localzone.keyrec");

  cmdcheck(\%options_hashref);

  $prefixdir = getprefixdir();

  $confdir = getconfdir();

  $conffile = getconffile();

  $statedir = getlocalstatedir();

  erraction(ERR_MSG);
  err("unable to open keyrec file",1);
\end{verbatim}

{\bf DESCRIPTION}

The routines in this module perform configuration operations.  Some routines
access the DNSSEC-Tools configuration file, while others validate the
execution environment.

The DNSSEC tools have a configuration file for commonly used values.  These
values are the defaults for a variety of things, such as encryption algorithm
and encryption key length.  The \perlmod{Net::DNS::SEC::Tools::conf} module
provides methods for accessing the configuration data in this file.

\path{dnssec-tools.conf} is the filename for the DNSSEC tools configuration
file.  The full path depends on how DNSSEC-Tools was configured; see the
DIRECTORIES section for the complete path.  The paths required by
\perlmod{conf.pm} are set at DNSSEC-Tools configuration time.

The DNSSEC tools configuration file consists of a set of configuration value
entries, with only one entry per line.  Each entry has the ``keyword value''
format.  During parsing, the line is broken into tokens, with tokens being
separated by spaces and tabs.  The first token in a line is taken to be the
keyword.  All other tokens in that line are concatenated into a single string,
with a space separating each token.  The untokenized string is added to a hash
table, with the keyword as the value's key.

Comments may be included by prefacing them with the `\#' or `;' comment
characters.  These comments can encompass an entire line or may follow a
configuration entry.  If a comment shares a line with an entry, value
tokenization stops just prior to the comment character.

An example configuration file follows:

\begin{verbatim}
    # Sample configuration entries.

    algorithm       rsasha1     # Encryption algorithm.
    ksk_length      1024        ; KSK key length.
\end{verbatim}

Another aspect of DNSSEC-Tools configuration is the error action used by the
DNSSEC-Tools Perl modules.  The action dictates whether an error condition
will only give an error return, print an error message to STDERR, or print an
error message and exit.  The \func{erraction()} and \func{err()} interfaces
are used for these operations.

{\bf INTERFACES}

\begin{description}

\item \func{parseconfig()}\verb" "

This routine reads and parses the system's DNSSEC tools configuration file.
The parsed contents are put into a hash table, which is returned to the caller.

\item \func{parseconfig(conffile)}\verb" "

This routine reads and parses a caller-specified DNSSEC tools configuration
file.  The parsed contents are put into a hash table, which is returned to
the caller.  The routine quietly returns if the configuration file does not
exist. 

\item \func{cmdcheck(\%options\_hashref)}\verb" "

This routine ensures that the needed commands are available and executable.
If any of the commands either don't exist or aren't executable, then an error
message will be given and the process will exit.  If all is well, everything
will proceed quietly onwards.

The commands keys currently checked are {\it zonecheck}, {\it keygen}, and
{\it zonesign}.  The pathnames for these commands are found in the given
options hash referenced by \var{\%options\_hashref}.  If the hash doesn't
contain an entry for one of those commands, it is not checked.

\item \func{getconfdir()}\verb" "

This routine returns the name of the DNSSEC-Tools configuration directory.

\item \func{getconffile()}\verb" "

This routine returns the name of the DNSSEC-Tools configuration file.

\item \func{getprefixdir()}\verb" "

This routine returns the name of the DNSSEC-Tools prefix directory.

\item \func{getlocalstatedir()}\verb" "

This routine returns the name of the local state directory.

\item \func{erraction(error\_action)}\verb" "

This interface sets the error action for DNSSEC-Tools Perl modules.
The valid actions are:

\begin{description}
\item \const{ERR\_SILENT} - Do not print an error message, do not exit.
\item \const{ERR\_MSG} - Print an error message, do not exit.
\item \const{ERR\_EXIT} - Print an error message, exit.
\end{description}

\const{ERR\_SILENT} is the default action.

The previously set error action is returned.

\item \func{err(``error message'',exit\_code}\verb" "

The \func{err()} interface is used by the DNSSEC-Tools Perl modules to report
an error and exit, depending on the error action.

The first argument is an error message to print -- if the error action allows
error messages to be printed.

The second argument is an exit code -- if the error action requires that the
process exit.

\end{description}

{\bf DIRECTORIES}

The default directories for this installation are:

\begin{description}
\item \const{prefix} -  \path{/usr/local}
\item \const{sysconf} -  \path{/usr/local/etc}
\item \const{DNSSEC-Tools configuration file} - \path{/usr/local/etc/dnssec-tools}
\end{description}

{\bf SEE ALSO}

dnssec-tools.conf(5)

