\clearpage

\subsubsection{libval}

{\bf NAME}

\func{val\_create\_context()}, \func{val\_create\_context\_with\_conf()},
\func{val\_free\_context()} - manage validator contex

\func{val\_resolve\_and\_check()}, \func{val\_free\_result\_chain()} -
query and validate answers from a DNS name server

\func{val\_istrusted()} - check if status value corresponds to that of a
trustworthy answer

\func{val\_isvalidated()} - check if status value represents an answer tha
cryptographically chains down from a configured trust anchor

\func{val\_does\_not\_exist()} - check if status value represents
one of the non-existence types

\func{p\_val\_status()}, \func{p\_ac\_status()}, \func{p\_val\_error()} -
display validation status, authentication chain status and error information

\func{dnsval\_conf\_get()}, \func{resolv\_conf\_get()},
\func{root\_hints\_get()} - get the default location for the validator
configuration files

\func{dnsval\_conf\_set()}, \func{resolv\_conf\_set()},
\func{root\_hints\_set()} - set the default location for the validator
configuration files

{\bf SYNOPSIS}

\begin{verbatim}
    #include <validator.h>

    int val_create_context(const char *label, val_context_t **newcontext);

    int val_create_context_with_conf(char *label,
                                     char *dnsval_conf,
                                     char *resolv_conf,
                                     char *root_conf,
                                     val_context_t ** newcontext);

    int val_resolve_and_check(val_context_t *context,
                              u_char *domain_name_n,
                              const u_int16_t class,
                              const u_int16_t type,
                              const u_int8_t  flags,
                              struct val_result_chain  **results);

    char *p_val_status(val_status_t valerrno);

    char *p_ac_status(val_astatus_t auth_chain_status);

    char *p_val_error(int err);

    int val_istrusted(val_status_t val_status);

    int val_isvalidated(val_status_t val_status);

    int val_does_not_exist(val_status_t status);

    void val_free_result_chain(struct val_result_chain *results);

    void val_free_context(val_context_t *context);

    char *resolv_conf_get(void);

    int resolv_conf_set(const char *name);

    char *root_hints_get(void);

    int root_hints_set(const char *name);

    char *dnsval_conf_get(void);

    int dnsval_conf_set(const char *name);

\end{verbatim}

{\bf DESCRIPTION}

The \func{val\_resolve\_and\_check()} function queries a set of name servers
for the \var{$<$domain\_name\_n, type, class}$>$ tuple and to verifies and
validates the response.  Verification involves checking the RRSIGs, and
validation is verification of an authentication chain from a configured trus
anchor.  The \var{domain\_name\_n} parameter is the queried name in DNS wire
format.  The conversion from host format to DNS wire format can be done using
the \func{ns\_name\_pton()} function exported by the \lib{libsres(3)} library.

The \var{flags} parameter can be used to control the results of validation.
Two values, which may be ORed together, are currently defined for this field.
\const{VAL\_QUERY\_DONT\_VALIDATE} causes the validator to disable
validation for this query.  \const{VAL\_QUERY\_NO\_DLV} causes the
validator to disable DLV processing for this query.  The second flag is only
available if the \lib{libval(3)} library has been compiled with DLV support.

The first parameter to \func{val\_resolve\_and\_check()} is the validator
context.  Applications can create a new validator context using the
\func{val\_create\_context()} function.  This function parses the resolver and
validator configuration files and creates the handle \var{newcontext} to this
parsed information.  Information stored as part of validator context includes
the validation policy and resolver policy.

Validator and resolver policies are read from the \path{/etc/dnsval.conf} and
\path{/etc/resolv.conf} files by default.  \path{/etc/root.hints} provides
bootstrapping information for the validator when it functions as a full
resolver (see \path{dnsval.conf(3)}).  The default locations of each of these
files may be changed using the interfaces \func{dnsval\_conf\_set()},
\func{resolv\_conf\_set()}, and \func{root\_hints\_set()}, respectively.  The
corresponding "get" interfaces (namely \func{dnsval\_conf\_get()},
\func{resolv\_conf\_get()}, and \func{root\_hints\_get()}) can be used to
return the current location from where these configuration files are read.
The configuration file locations may also be specified on a per-context basis
using the \func{val\_create\_context\_with\_conf()} function.

Answers returned by \func{val\_resolve\_and\_check()} are made available in
the \var{*results} linked list.  Each answer corresponds to a distinct RRset;
multiple RRs within the RRset are part of the same answer.  Multiple answers
are possible when \var{type} is \var{ns\_t\_any} or \var{ns\_t\_rrsig}.

Individual elements in \var{*results} point to
\var{val\_authentication\_chain} linked lists.  The authentication chain
elements in \var{val\_authentication\_chain} contain the actual RRsets
returned by  the name server in response to the query.

Validation result values returned in \var{val\_result\_chain} and
authentication chain status values returned in each element of the
\var{val\_authentication\_chain} linked list can be can be converted into
ASCII format using the \func{p\_val\_status()} and \func{p\_ac\_status()}
functions respectively.

While some applications such as DNSSEC troubleshooting utilities and packet
inspection tools may look at individual authentication chain elements to
identify the actual reasons for validation error, most applications will only
be interested in a single error code for determining the authenticity of data.

\func{val\_isvalidated()} identifies if a given validation result status value
corresponds to an answer that was cryptographically verified and validated
using a locally configured trust anchor.

\func{val\_istrusted()} identifies if a given validator status value is
trusted.  An answer may be locally trusted without being validated.

\func{val\_does\_not\_exist()} identifies if a given validator status value
corresponds to one of the non-existence types.

The \lib{libval} library internally allocates memory for \var{*results} and
this must be freed by the invoking application using the
\func{free\_result\_chain()} interface.

{\bf DATA STRUCTURES}

\begin{description}

\item {\it struct val\_result\_chain}\verb" "

\begin{verbatim}
    struct val_result_chain
    {
        val_status_t                     val_rc_status;
        struct val_authentication_chain *val_rc_answer;
        int                              val_rc_proof_count;
        struct val_authentication_chain *val_rc_proofs[MAX_PROOFS];
        struct val_result_chain         *val_rc_next;
    };
\end{verbatim}

\begin{description}

\item {\it val\_rc\_answer} \verb" "

Authentication chain for a given RRset.

\item {\it val\_rc\_next} \verb" "

Pointer to the next RRset in the set of answers returned for a query.

\item {\it val\_rc\_proofs} \verb" "

Pointer to authentication chains for any proof of non-existence that were
returned for the query.

\item {\it val\_rc\_proof\_count} \verb" "

Number of proof elements stored in {\it val\_rc\_proofs}. The number canno
exceed \const{MAX\_PROOFS}.

\item {\it val\_rc\_status} \verb" "

Validation status for a given RRset.  This can be one of the following:

\begin{description}
\item \const{VAL\_SUCCESS}\verb" "

Answer received and validated successfully.

\item \const{VAL\_LOCAL\_ANSWER}\verb" "

Answer was available from a local file.

\item \const{VAL\_BARE\_RRSIG}\verb" "

No DNSSEC validation possible, query was for
an RRSIG.

\item \const{VAL\_NONEXISTENT\_NAME}\verb" "

No name was present and a valid proof of non-
existence confirming the missing name (NSEC
or NSEC3 span) was returned. The components of
the proof were also individually validated.

\item \const{VAL\_NONEXISTENT\_TYPE}\verb" "

No type exists for the name and a valid proof
of non-existence confirming the missing name was
returned.  The components of the proof were also
individually validated.

\item \const{VAL\_NONEXISTENT\_NAME\_NOCHAIN}\verb" "

No name was present and a valid proof of non-
existence confirming the missing name was returned.
The components of the proof were also identified
to be trustworthy, but they were not individually
validated.

\item \const{VAL\_NONEXISTENT\_TYPE\_NOCHAIN}\verb" "

No type exists for the name and a valid proof
of non-existence confirming the missing name (NSEC
or NSEC3 span) was returned.  The components of the
proof were also identified to be trustworthy, but
they were not individually validated.

\item \const{VAL\_DNS\_CONFLICTING\_ANSWERS}\verb" "

Multiple conflicting answers received for a query.

\item \const{VAL\_DNS\_QUERY\_ERROR}\verb" "

Some error was encountered while sending the query.

\item \const{VAL\_DNS\_RESPONSE\_ERROR}\verb" "

No response returned or response with an error
rcode value returned.

\item \const{VAL\_DNS\_WRONG\_ANSWER}\verb" "

Wrong answer received for a query.

\item \const{VAL\_DNS\_REFERRAL\_ERROR}\verb" "

Some error encountered while following referrals.

\item \const{VAL\_DNS\_MISSING\_GLUE}\verb" "

Glue was missing.

\item \const{VAL\_BOGUS\_PROVABLE}\verb" "

Response could not be validated due to signature
verification failures or the inability to verify
proofs for exactly one component in the
authentication chain below the trust anchor.

\item \const{VAL\_BOGUS}\verb" "

Response could not be validated due to signature
verification failures or the inability to verify
proofs for an indeterminate number of components
in the authentication chain.

\item \const{VAL\_NOTRUST}\verb" "

All available components in the authentication
chain verified properly, but there was no trust
anchor available.

\item \const{VAL\_IGNORE\_VALIDATION}\verb" "

Local policy was configured to ignore validation
for the zone from which this data was received.

\item \const{VAL\_TRUSTED\_ZONE}\verb" "

Local policy was configured to trust any data
received from the given zone.

\item \const{VAL\_UNTRUSTED\_ZONE}\verb" "

Local policy was configured to reject any data
received from the given zone.

\item \const{VAL\_PROVABLY\_UNSECURE}\verb" "

The record or some ancestor of the record in the
authentication chain towards the trust anchor was
known to be provably unsecure.

\item \const{VAL\_BAD\_PROVABLY\_UNSECURE}\verb" "

The record or some ancestor of the record in the
authentication chain towards the trust anchor
was known to be provably unsecure. But the
provably unsecure condition was configured as
untrustworthy.

\end{description}

Status values in \var{val\_status\_t} returned by the validator can be
displayed in ASCII format using \func{p\_val\_status()}.

\end{description}

\end{description}

\begin{description}

\item {\it struct val\_authentication\_chain}\verb" "

\begin{verbatim}
    struct val_authentication_chain
    {
        val_astatus_t                    val_ac_status;
        struct val_rrset                *val_ac_rrset;
        struct val_authentication_chain *val_ac_trust;
    };
\end{verbatim}

\begin{description}

\item {\it val\_ac\_status} \verb" "

Validation state of the authentication chain element.  This field will
contain the status code for the given component in the authentication chain.
This field may contain one of the following values:

\begin{description}
\item \const{VAL\_AC\_UNSET}\verb" "

The status was not set.

\item \const{VAL\_AC\_DATA\_MISSING}\verb" "

No data were returned for a query and the DNS did not
indicate an error.

\item \const{VAL\_AC\_RRSIG\_MISSING}\verb" "

RRSIG data could not be retrieved for a resource record.

\item \const{VAL\_AC\_DNSKEY\_MISSING}\verb" "

The DNSKEY for an RRSIG covering a resource record
could not be retrieved.

\item \const{VAL\_AC\_DS\_MISSING}\verb" "

The DS record covering a DNSKEY record was not available.

\item \const{VAL\_AC\_NOT\_VERIFIED}\verb" "

All RRSIGs covering the RRset could not be verified.

\item \const{VAL\_AC\_VERIFIED}\verb" "

At least one RRSIG covering a resource record had a status of\\
\const{VAL\_AC\_RRSIG\_VERIFIED}.

\item \const{VAL\_AC\_TRUST\_KEY}\verb" "

A given DNSKEY or a DS record was locally defined to be
a trust anchor.

\item \const{VAL\_AC\_IGNORE\_VALIDATION}\verb" "

Validation for the given resource record was ignored,
either because of some local policy directive or because
of some protocol-specific behavior.

\item \const{VAL\_AC\_TRUSTED\_ZONE}\verb" "

Local policy defined a given zone as trusted, with no
further validation being deemed necessary.

\item \const{VAL\_AC\_UNTRUSTED\_ZONE}\verb" "

Local policy defined a given zone as untrusted, with no
further validation being deemed necessary.

\item \const{VAL\_AC\_PROVABLY\_UNSECURE}\verb" "

The authentication chain from a trust anchor to a given
zone could no be constructed due to the provable absence
of a DS record for this zone in the parent.

\item \const{VAL\_AC\_BARE\_RRSIG}\verb" "

The response was for a query of type RRSIG.  RRSIGs
contain the cryptographic signatures for other DNS
data and cannot themselves be validated.

\item \const{VAL\_AC\_NO\_TRUST\_ANCHOR}\verb" "

There was no trust anchor configured for a given
authentication chain.

\item \const{VAL\_AC\_DNS\_CONFLICTING\_ANSWERS}\verb" "

Multiple conflicting answers received for a query.

\item \const{VAL\_AC\_DNS\_QUERY\_ERROR}\verb" "

Some error was encountered while sending the query.

\item \const{VAL\_AC\_DNS\_RESPONSE\_ERROR}\verb" "

No response returned or response with an error rcode
value returned.

\item \const{VAL\_AC\_DNS\_WRONG\_ANSWER}\verb" "

Wrong answer received for a query.

\item \const{VAL\_AC\_DNS\_REFERRAL\_ERROR}\verb" "

Some error encountered while following referrals.

\item \const{VAL\_AC\_DNS\_MISSING\_GLUE}\verb" "

Glue was missing.

\end{description}

\item {\it val\_ac\_rrset} \verb" "

Pointer to an RRset of type \struct{val\_rrset} obtained from the DNS response.

\item {\it val\_ac\_trust} \verb" "

Pointer to an authentication chain element that either contains a DNSKEY RRset
that can be used to verify RRSIGs over the current record, or contains a DS
RRset that can be used to build the chain-of-trust towards a trust anchor.
\end{description}

\end{description}

\begin{description}

\item {\it struct val\_rrset}\verb" "

\begin{verbatim}
    struct val_rrset
    {
        u_int8_t      *val_msg_header;
        u_int16_t      val_msg_headerlen;
        u_int8_t      *val_rrset_name_n;
        u_int16_t      val_rrset_class_h;
        u_int16_t      val_rrset_type_h;
        u_int32_t      val_rrset_ttl_h;
        u_int32_t      val_rrset_ttl_x;
        u_int8_t       val_rrset_section;
        struct rr_rec *val_rrset_data;
        struct rr_rec *val_rrset_sig;
    };
\end{verbatim}

\begin{description}

\item \var{val\_msg\_header} \verb" "

Header of the DNS response in which the RRset was received.

\item \var{val\_msg\_headerlen} \verb" "

Length of the header information in \struct{val\_msg\_header}.

\item \var{val\_rrset\_name\_n} \verb" "

Owner name of the RRset represented in on-the-wire format.

\item \var{val\_rrset\_class\_h} \verb" "

Class of the RRset.

\item \var{val\_val\_rrset\_type\_h} \verb" "

Type of the RRset.

\item \var{val\_rrset\_ttl\_h} \verb" "

TTL of the RRset.

\item \var{val\_rrset\_ttl\_x} \verb" "

The time when the TTL for this RRset is set to expire.

\item \var{val\_rrset\_section} \verb" "

Section in which the RRset was received.  This value may be\\
\const{VAL\_FROM\_ANSWER}, \const{VAL\_FROM\_AUTHORITY},\\
or \const{VAL\_FROM\_ADDITIONAL}.

\item \var{val\_rrset\_data} \verb" "

Response RDATA.

\item \var{val\_rrset\_sig} \verb" "

Any associated RRSIGs for the RDATA returned in \var{val\_rrset\_data}.

\end{description}
\end{description}

\begin{description}

\item {\it struct rr\_rec}\verb" "

\begin{verbatim}
    struct rr_rec
    {
         u_int16_t        rr_rdata_length_h;
         u_int8_t        *rr_rdata;
         val_astatus_t    rr_status;
         struct rr_rec   *rr_next;
    };
\end{verbatim}

\begin{description}

\item \var{rr\_rdata\_length\_h} \verb" "

Length of data stored in \var{rr\_rdata}.

\item \var{rr\_rdata} \verb" "

RDATA bytes.

\item \var{rr\_status} \verb" "

For each signature \var{rr\_rec} member within the authentication chain
\var{val\_ac\_rrset}, the validation status stored in the variable
\var{rr\_status} can return one of the following values:

\begin{description}
\item \const{VAL\_AC\_RRSIG\_VERIFIED}\verb" "

The RRSIG verified successfully.

\item \const{VAL\_AC\_WCARD\_VERIFIED}\verb" "

A given RRSIG covering a resource record shows that
the record was wildcard expanded.

\item \const{VAL\_AC\_RRSIG\_VERIFIED\_SKEW}\verb" "

The RRSIG verified successfully after clock skew was
taken into account.

\item \const{VAL\_AC\_WCARD\_VERIFIED\_SKEW}\verb" "

A given RRSIG covering a resource record shows that
the record was wildcard expanded, but it was verified
only after clock skew was taken into account.

\item \const{VAL\_AC\_RRSIG\_VERIFY\_FAILED}\verb" "

A given RRSIG covering an RRset was bogus.

\item \const{VAL\_AC\_DNSKEY\_NOMATCH}\verb" "

An RRSIG was created by a DNSKEY that did not exist
in the apex keyset.

\item \const{VAL\_AC\_RRSIG\_ALGORITHM\_MISMATCH}\verb" "

The keytag referenced in the RRSIG matched a DNSKEY
but the algorithms were different.

\item \const{VAL\_AC\_WRONG\_LABEL\_COUNT}\verb" "

The number of labels on the signature was greater
than the count given in the RRSIG RDATA.

\item \const{VAL\_AC\_BAD\_DELEGATION}\verb" "

An RRSIG was created with a key that did not exist
in the parent DS record set.

\item \const{VAL\_AC\_RRSIG\_NOTYETACTIVE}\verb" "

The RRSIG's inception time is in the future.

\item \const{VAL\_AC\_RRSIG\_EXPIRED}\verb" "

The RRSIG had expired.

\item \const{VAL\_AC\_INVALID\_RRSIG}\verb" "

The RRSIG could not be parsed.

\item \const{VAL\_AC\_ALGORITHM\_NOT\_SUPPORTED}\verb" "

The RRSIG algorithm was not supported.

\end{description}

For each \var{rr\_rec} member of type DNSKEY (or DS, where relevant) within
the authentication chain \var{val\_ac\_rrset}, the validation status is stored
in the variable \var{rr\_status} and can return one of the following values:

\begin{description}
\item \const{VAL\_AC\_SIGNING\_KEY}\verb" "

This DNSKEY was used to create an RRSIG for the resource record set.

\item \const{VAL\_AC\_VERIFIED\_LINK}\verb" "

This DNSKEY provided the link in the authentication chain from the trust
anchor to the signed record.

\item \const{VAL\_AC\_UNKNOWN\_DNSKEY\_PROTOCOL}\verb" "

The DNSKEY protocol number was unrecognized.

\item \const{VAL\_AC\_UNKNOWN\_ALGORITHM\_LINK}\verb" "

This DNSKEY provided the link in the authentication chain from the trust
anchor to the signed record, but the DNSKEY algorithm was unknown.

\item \const{VAL\_AC\_INVALID\_KEY}\verb" "

The key used to verify the RRSIG was not a valid DNSKEY.

\item \const{VAL\_AC\_ALGORITHM\_NOT\_SUPPORTED}\verb" "

The DNSKEY or DS algorithm was not supported.

\end{description}

\end{description}

\item \var{rr\_next} \verb" "

Points to the next resource record in the RRset.

\end{description}

{\bf RETURN VALUES}

Return values for various functions are given below. These values can be
displayed in ASCII format using the \func{p\_val\_error()} function.

\begin{description}

\item \func{val\_resolve\_and\_check()}\verb" "

\begin{description}

\item \const{VAL\_NO\_ERROR}\verb" "

No error was encountered.

\item \const{VAL\_GENERIC\_ERROR}\verb" "

Generic error encountered.

\item \const{VAL\_NOT\_IMPLEMENTED}\verb" "

Functionality not yet implemented.

\item \const{VAL\_BAD\_ARGUMENT}\verb" "

Bad arguments passed as parameters.

\item \const{VAL\_INTERNAL\_ERROR}\verb" "

Encountered some internal error.

\item \const{VAL\_NO\_PERMISSION}\verb" "

No permission to perform operation.  Currently not implemented.

\item \const{VAL\_RESOURCE\_UNAVAILABLE}\verb" "

Some resource (crypto possibly) was unavailable.  Currently not implemented.

\end{description}

\item \func{val\_create\_context()} and
\func{val\_create\_context\_with\_conf()}\verb" "

\begin{description}

\item \const{VAL\_NO\_ERROR}\verb" "

No error was encountered.

\item \const{VAL\_RESOURCE\_UNAVAILABLE}\verb" "

Could not allocate memory.

\item \const{VAL\_CONF\_PARSE\_ERROR}\verb" "

Error in parsing some configuration file.

\item \const{VAL\_CONF\_NOT\_FOUND}\verb" "

A configuration file was not available.

\item \const{VAL\_NO\_POLICY}\verb" "

The policy identifier being referenced was invalid.

\end{description}

\end{description}

{\bf FILES}

The validator library reads configuration information from two files,
\path{/etc/resolv.conf} and \path{/etc/dnsval.conf}.

See \path{dnsval.conf(5)} for a description of syntax for these files.

{\bf SEE ALSO}

libsres(3)

dnsval.conf(5)

