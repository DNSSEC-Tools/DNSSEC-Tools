\clearpage

\subsubsection{validate}

{\bf NAME}

\cmd{validate} - Query the Domain Name System and display results of
the DNSSEC validation process

{\bf SYNOPSIS}

\begin{verbatim}
    validate

    validate [options] DOMAIN_NAME
\end{verbatim}

{\bf DESCRIPTION}

\cmd{validate} is a diagnostic tool which uses the DNSSEC validator.  It
takes {\it DOMAIN\_NAME} as an argument and queries the DNS for that domain
name.  It outputs the series of responses that were received from the DNS and
the DNSSEC validation results for each domain name.  An examination of the
queries and validation results can help an administrator uncover errors in
DNSSEC configuration of DNS zones.

If no options are specified and no {\it DOMAIN\_NAME} argument is given,
\cmd{validate} will perform a series of pre-defined test queries against the
{\it test.dnssec-tools.org} zone.  This serves as a test-suite for the
validator.  If any options are specified (e.g., configuration file locations),
{\it -s} or {\it --selftest} must be specified to run the test-suite.

{\bf OPTIONS}

\begin{description}

\item -c {\it CLASS}, --class={\it CLASS}\verb" "

This option can be used to specify the DNS class of the Resource
Record queried.  If this option is not given, the default class
{\bf IN} is used.

\item -h, --help\verb" "

Display the help and exit.

\item -m, --merge\verb" "

When this option is given, \cmd{validate} will merge different RRsets
in the response into a single answer.  If this option is not given,
each RRset is output as a separate response.  This option makes
sense only when used with the {\it -p} option.

\item -p, --print\verb" "

Print the answers and validation results.  By default, \cmd{validate}
just outputs a series of responses and their validation results on
{\it stderr}.  When the {\it -p} option is used, \cmd{validate} will also
output the final result on {\it stdout}.

\item -t {\it TYPE}, --type={\it TYPE}\verb" "

This option can be used to specify the DNS type of the Resource Record
queried.  If this option is not given, \cmd{validate} will query for the
{\bf A} record for the given {\it DOMAIN\_NAME}.

\item -v {\it FILE}, --dnsval-conf={\it FILE}\verb" "

This option can be used to specify the location of the \path{dnsval.conf}
configuration file.

\item -r {\it FILE}, --resolv-conf={\it FILE}\verb" "

This option can be used to specify the location of the \path{resolv.conf}
configuration file containing the name servers to use for lookups.

\item -i {\it FILE}, --root-hints={\it FILE}\verb" "

This option can be used to specify the location of the root.hints
configuration file, containing the root name servers.  This is only
used when no name server is found, and \cmd{validate} must do recursive
lookups itself.

\item -S {\it suite}[:{\it suite}], --test-suite={\it suite}[:{\it suite}]\verb" "

This option specifies the test suite (or range of test suites) to use 
for the internal tests.

\item -s, --selftest\verb" "

This option can be used to specify that the application should perform its
test-suite against the {\it dnssec-tools.org} test domain.  If the name
servers configured in the system \path{resolv.conf} do not support DNSSEC, use
the {\it -r} and {\it -i} options to enable \cmd{validate} to use its own
internal recursive resolver.

\item -T {\it number}[:{\it number}], --testcase={\it number}[:{\it number}]\verb" "

This option can be used to run a specific test (or range of tests) 
from the test suite.

\item -F {\it file}, --testcase-conf={\it file} \verb" "

This option is used to specify the file containing the test cases.

\item -l {\it label}, --label={\it label} \verb" "

This option can be used to specify the policy from within the {\it dnsval.conf} 
file to use during validation. 

\item -w {\it seconds}, --wait={\it seconds} \verb" "

This option can be used to run the queries specified by other flags in a loop,
with the specified interval between successive queries.

\item -o, --output=$<$debug-level$>$:$<$dest-type$>$[:$<$dest-options$>$]\verb" "

$<$debug-level$>$ is 1-7, corresponding to syslog levels ALERT-DEBUG \\
$<$dest-type$>$ is one of file, net, syslog, stderr, stdout \\
$<$dest-options$>$ depends on $<$dest-type$>$ \\
\begin{verbatim}
    file:<file-name>   (opened in append mode)
    net[:<host-name>:<host-port>] (127.0.0.1:1053
    syslog[:facility] (0-23 (default 1 USER))
\end{verbatim}

\end{description}

{\bf PRE-REQUISITES}

\lib{libval}

{\bf SEE ALSO}

libval(3),
syslog(3)

