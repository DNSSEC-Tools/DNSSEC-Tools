\clearpage

\subsubsection{rollrec-editor}

{\bf NAME}

\cmd{rollrec-editor} - DNSSEC-Tools Rollrec GUI Editor

{\bf SYNOPSIS}

\begin{verbatim}
  rollrec-editor [options] <rollrec-file>
\end{verbatim}

{\bf DESCRIPTION}

\cmd{rollrec-editor} provides the capability for easy GUI-based management of
\struct{rollrec} files.  A \struct{rollrec} file contains one or more
\struct{rollrec} records.  These records are used by the DNSSEC-Tools rollover
utilities ({\bf rollerd}, etc.) to describe zones' rollover state.  Each
zone's \struct{rollrec} record contains such information as the zone file, the
rollover phase, and logging level.  \struct{rollrec} files are text files and
may be edited by any text editor.  \cmd{rollrec-editor} allows editing of only
those records a user should change and performs error checking on the data.

When \cmd{rollrec-editor} starts, a window is created that has ``buttons'' for
each \struct{rollrec} record in the given \struct{rollrec} file.  (In this
documentation, this window is called the Button Window.)  Clicking on the
buttons selects (or deselects) that zone.  After one or more zones are
selected, one of several commands may be executed.  Commands allow
modification and deletion of existing \struct{rollrec} records, creation of
new \struct{rollrec} records, merging of \struct{rollrec} files, and
verification of file validity.

\cmd{rollrec-editor}'s commands are available through the menus and most have
a keyboard accelerator.  The commands are described below in the COMMANDS
section.

When a \struct{rollrec} record is selected for modification, a new window is
created to hold the editable fields of the record.  The fields may be modified
in place.  When editing is complete, the record is ``saved''.  This does not
save the modified \struct{rollrec} into its on-disk file; the file must be
saved explicitly from the Button Window.

As stated above, verification checks are performed when saving an edited
\struct{rollrec} record.  One set of checks ensures that files and directories
associated with a zone actually exist.  This check may be turned off at
command start-up with the {\it -ignore-warns} command line option.  It may be
modified during execution with the ``Ignore Edit Warnings'' menu command.

{\bf Button Window Layout}

The Button Window contains a button for each \struct{rollrec} record in the
selected file.  The buttons are arranged in a table that with at least three
columns.  The number of columns may be set at command start-up with the {\it
-columns} command line option.  It may be modified during execution with the
``Columns in Button Window'' menu command.

When setting the number of columns, \cmd{rollrec-editor} minimizes the space
required to display the selected file's \struct{rollrec} buttons.  This will
sometimes cause the number of columns to differ from that requested.

For example, assume a \struct{rollrec} file has 12 \struct{rollrec} records.
The following table shows how many rows and columns are given for each of the
given column selections.

\begin{table}[ht]
\begin{center}
\begin{tabular}{|c|c|c|}
\hline
{\bf Column Count} & {\bf Rows} & {\bf Columns} \\
\hline
column count & rows & columns \\
1            & 12   & 1 \\
2            & 6    & 2 \\
3            & 4    & 3 \\
4            & 3    & 4 \\
5            & 3    & 4 \\
6            & 2    & 6 \\
7            & 2    & 6 \\
8            & 2    & 6 \\
9            & 2    & 6 \\
10           & 2    & 6 \\
11           & 2    & 6 \\
12           & 1    & 12 \\
\hline
\end{tabular}
\end{center}
\caption{Example Button Layouts}
\end{table}

The actual rows and columns for a requested column count will vary in order
to allow a ``best-fit''.

{\bf UNDOING MODIFICATIONS}

{\bf The ``undo'' capability is not currently implemented.}

\cmd{rollrec-editor} has the ability to reverse modifications it has made to a
\struct{rollrec} file.  This historical restoration will only work for
modifications made during a particular execution of \cmd{rollrec-editor};
modifications made during a previous execution may not be undone.

After a ``Save'' operation, the data required for reversing modifications are
deleted.  This is not the case for the ``Save As'' operation.

{\bf OPTIONS}

\cmd{rollrec-editor} supports the following options.

\begin{itemize}

\item {\bf -columns columns}\verb" "

This option allows the user to specify the number of columns to be used in
the Button Window.

\item {\bf -ignore-warns}\verb" "

This option causes \cmd{rollrec-editor} to ignore edit warnings when performing
validation checks on the contents of \struct{rollrec} records.

\item {\bf -no-filter}\verb" "

This option turns off name filtering when \cmd{rollrec-editor} presents a
file-selection dialog for choosing a new \struct{rollrec} file.  If this option
is not given, then the file-selection dialog will only list regular files
with a suffix of \path{.rrf}.

\item -{\bf V}\verb" "

This option displays the program and DNSSEC-Tools version.

\end{itemize}

{\bf COMMANDS}

\cmd{rollrec-editor} provides the following commands, organized by menus:
File, Edit, Commands, Rollrecs, and Options.

{\bf File Menu}

The File Menu contains commands for manipulating \struct{rollrec} files and
stopping execution.

\begin{itemize}

\item {\bf Open}\verb" "

Open a new \struct{rollrec} file.  If the specified file does not exist, the
user will be prompted for the action to take.  If the user chooses the
``Continue'' action, then \cmd{rollrec-editor} will continue editing the
current \struct{rollrec} file.  If the ``Quit'' action is selected, then
\cmd{rollrec-editor} will exit.

\item {\bf Save}\verb" "

Save the current \struct{rollrec} file.  The data for the ``Undo Changes''
command are purged, so this file will appear to be unmodified.

Nothing will happen if no changes have been made.

\item {\bf Save As}\verb" "

Save the current \struct{rollrec} file to a name selected by the user.

\item {\bf Quit}\verb" "

Exit \cmd{rollrec-editor}.

\end{itemize}

{\bf Edit Menu}

The Edit Menu contains commands for general editing operations.

\begin{itemize}

\item {\bf Undo Changes}\verb" "

Reverse modifications made to the \struct{rollrec} records.  This is {\bf only}
for the in-memory version of the \struct{rollrec} file.

{\bf Not currently implemented.}

\end{itemize}

{\bf Commands Menu}

The Commands Menu contains commands for modifying \struct{rollrec} records.

\begin{itemize}

\item {\bf New Rollrec}\verb" "

Create a new \struct{rollrec} record.   The user is given a new window in
which to edit the user-modifiable \struct{rollrec} fields.  A button for the
new \struct{rollrec} record will be inserted into the Button Window.

After editing is completed, the ``Save'' button will add the new
\struct{rollrec} record to the in-memory \struct{rollrec} file.  The file
must be saved in order to have the new \struct{rollrec} added to the file.

Potentially erroneous conditions will be reported to the user at save time.
If the {\it ignore-warnings} flag has been turned on, then these warnings will
not be reported.  Errors (e.g., invalid log conditions) will always be
reported.

\item {\bf Delete Selected Rollrecs}\verb" "

Delete the selected \struct{rollrec} records.  The buttons for each selected
record will be removed from the Button Window.

\item {\bf Edit Selected Rollrecs}\verb" "

Modify the selected \struct{rollrec} records.   For every record selected for
modification, the user is given a new window in which to edit the
user-modifiable \struct{rollrec} fields.  When the edit window goes away
(after a ``Save'' or ``Cancel''), the \struct{rollrec} record's button is
deselected.

After editing is completed, the ``Save'' button will add the new
\struct{rollrec} record to the in-memory \struct{rollrec} file.  The file
must be saved in order to have the new \struct{rollrec} added to the file.

Potentially erroneous conditions will be reported to the user at save time.  
If the {\it ignore-warnings} flag has been turned on, then these warnings will
not be reported.  Errors (e.g., invalid log conditions) will always be
reported.

\item {\bf Merge Rollrec Files}\verb" "

Merge a \struct{rollrec} file with the currently open \struct{rollrec} file.
After a successful merge, the \struct{rollrec} records in the second file will
be added to the {\it end} of the currently open \struct{rollrec} file.

If there are any \struct{rollrec} name collisions in the files, then the user
will be asked whether to continue with the merge or cancel it.  If the merge
continues, then the conflicting \struct{rollrec} records from the ``new'' file
will be discarded in favor of the currently open \struct{rollrec} file.

\item {\bf Verify Rollrec File}\verb" "

Verify the validity of the \struct{rollrec} file.  The user-editable fields in
the open \struct{rollrec} file are checked for validity.  An edit window is
opened for each \struct{rollrec} record that registers an error or warning.

If the {\it ignore-warnings} flag has been turned on, then potentially
erroneous conditions will not be reported.  Errors (e.g., invalid log
conditions) will always be reported.

\item {\bf Summarize Problems}\verb" "

Summarize the warning and error counts of the \struct{rollrec} file.  Each
\struct{rollrec} record in the open \struct{rollrec} file is checked for
validity.  If any warnings or errors are found, a window is displayed that
lists the name of each \struct{rollrec} record and its warning and error
counts.

If the {\it ignore-warnings} flag has been turned on, then potentially
erroneous conditions will not be reported.  Errors (e.g., invalid log
conditions) will always be reported.

\end{itemize}

{\bf View Menu}

The View Menu contains miscellaneous commands for viewing edit windows.

\begin{itemize}

\item {\bf Select All Rollrecs/Unselect All Rollrecs}\verb" "

Select or unselect all \struct{rollrec} buttons.  All the buttons in the Button
Window will be selected or unselected.  If {\bf any} of the buttons are not
selected, this command will cause all the buttons to become selected.  If
{\it all} of the buttons are selected, this command will cause all the buttons
to be deselected.

This command is a toggle that switches between Select All Rollrecs and
Unselect All Rollrecs.

\item {\bf Reveal Rollrec Edit Windows}\verb" "

Raise all \struct{rollrec} edit windows.  This command brings all
\struct{rollrec} edit windows to the front so that any that are hidden
behind other windows will become visible.

\item {\bf Close Rollrec Edit Windows}\verb" "

Close all \struct{rollrec} edit windows.  This command closes and deselects all
\struct{rollrec} edit windows.

\end{itemize}

{\bf Options Menu}

The Options Menu allows the user to set several options that control
\cmd{rollrec-editor} behavior.

\begin{itemize}

\item {\bf Ignore Edit Warnings/Don't Ignore Edit Warnings}\verb" "

Certain operations perform validation checks on the contents of
\struct{rollrec} records.  Warnings and errors may be reported by these
checks.  This option controls whether or not warnings are flagged during
validation.  If they are flagged, then the operation will not continue until
the warning condition is fixed or the operation canceled.  If the warnings are
ignored, then the operation will complete without the condition being fixed.

This command is a toggle that switches between Ignore Edit Warnings mode and
Don't Ignore Edit Warnings mode.

\item {\bf Filter Name Selection/Don't Filter Name Selection}\verb" "

When opening \struct{rollrec} files for editing or merging, a file-selection
dialog box is displayed to allow the user to select a \struct{rollrec} file.
This option controls whether a filename filter is used by the dialog box.  If
Filter Names Selection mode is used, then only regular files with a suffix of
\path{.rrf} will be displayed by the dialog box.  If Don't Filter Name
Selection mode is used, then all regular files will be displayed by the dialog
box.

This command is a toggle that switches between Filter Name Selection display
and Don't Filter Name Selection display.

\item {\bf Columns in Button Window}\verb" "

Set the number of columns used in the Button Window.  See the Button Window
Layout section for more information on columns in the Button Window.

\end{itemize}

{\bf Help Menu}

The Help Menu contains commands for getting assistance.

\begin{itemize}

\item {\bf Help}\verb" "

Display a help window.

\end{itemize}

{\bf KEYBOARD ACCELERATORS}

Below are the keyboard accelerators for the \cmd{rollrec-editor} commands:

\begin{table}[ht]
\begin{center}
\begin{tabular}{|c|l|}
\hline
{\bf Accelerator} & {\bf Function} \\
\hline
Ctrl-A & Select All Rollrecs \\
Ctrl-D & Delete Selected Rollrecs \\
Ctrl-E & Edit Selected Rollrecs \\
Ctrl-H & Help \\
Ctrl-K & Close Rollrec Edit Windows \\
Ctrl-M & Merge Rollrec Files \\
Ctrl-N & New Rollrec \\
Ctrl-O & Open \\
Ctrl-Q & Quit \\
Ctrl-R & Reveal Rollrec Edit Windows \\
Ctrl-S & Save \\
Ctrl-U & Undo Changes \\
       & (not currently implemented) \\
Ctrl-V & Verify Rollrec File \\
\hline
\end{tabular}
\end{center}
\caption{Keyboard Accelerators for \cmd{rollrec-editor}}
\end{table}

These accelerators are all lowercase letters.

{\bf REQUIREMENTS}

\cmd{rollrec-editor} is implemented in Perl/Tk, so both Perl and Perl/Tk must be
installed on your system.

{\bf KNOWN ISSUES}

The following are known issues.  These will be resolved in the fullness of time.

\begin{description}

\item There is an issue with the column count and adding new \struct{rollrec}s.
It doesn't always work properly, thus occasionally leaving some
\struct{rollrec} buttons undisplayed.

\item Segmentation violation on exit.  Should not affect anything as file has
already been saved.  Occurs on MacOS X 10.4.9 with Perl 5.8.6; does not occur
on Linux FC5 with Perl 5.8.8.

\item There is no undo command.

\end{description}

{\bf SEE ALSO}

lsroll(1)

rollchk(8),
rollerd(8)
rollinit(8),
rollset(8),
zonesigner(8)

Net::DNS::SEC::Tools::rollrec(3)

file-rollrec(5)

