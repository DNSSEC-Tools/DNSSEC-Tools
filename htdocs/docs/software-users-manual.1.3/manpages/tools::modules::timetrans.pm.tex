\clearpage

\subsubsection{timetrans.pm}

{\bf NAME}

\perlmod{Net::DNS::SEC::Tools::timetrans} - Convert an integer seconds
count into text units.

{\bf SYNOPSIS}

\begin{verbatim}
  use Net::DNS::SEC::Tools::timetrans;

  $timestring = timetrans(86488);

  $timestring = fuzzytimetrans(86488);
\end{verbatim}

{\bf DESCRIPTION}

The \func{timetrans}() interface in \perlmod{Net::DNS::SEC::Tools::timetrans}
converts an integer seconds count into the equivalent number of weeks, days,
hours, and minutes.  The time converted is a relative time, {\bf not} an
absolute time.  The returned time is given in terms of weeks, days, hours,
minutes, and seconds, as required to express the seconds count appropriately.

The \func{fuzzytimetrans}() interface converts an integer seconds count into
the equivalent number of weeks {\bf or} days {\bf or} hours {\bf or} minutes.
The unit chosen is that which is most natural for the seconds count.  One
decimal place of precision is included in the result.

{\bf INTERFACES}

The interfaces to the \perlmod{Net::DNS::SEC::Tools::timetrans} module are
given below.

\begin{description}

\item \func{timetrans()}

This routine converts an integer seconds count into the equivalent number of
weeks, days, hours, and minutes.  This converted seconds count is returned
as a text string.  The seconds count must be greater than zero or an error
will be returned.

Return Values:

\begin{description}

\item If a valid seconds count was given, the count converted into the
appropriate text string will be returned.

\item An empty string is returned if no seconds count was given or if
the seconds count is less than one.

\end{description}

\item \func{fuzzytimetrans()}

This routine converts an integer seconds count into the equivalent number of
weeks, days, hours, or minutes.  This converted seconds count is returned
as a text string.  The seconds count must be greater than zero or an error
will be returned.

Return Values:

\begin{description}

\item If a valid seconds count was given, the count converted into the
appropriate text string will be returned.

\item An empty string is returned if no seconds count was given or if
the seconds count is less than one.

\end{description}

\end{description}

{\bf EXAMPLES}

{\it timetrans(400)} returns 6 minutes, 40 seconds

{\it timetrans(420)} returns 7 minutes

{\it timetrans(888)} returns 14 minutes, 48 seconds

{\it timetrans(86400)} returns 1 day

{\it timetrans(86488)} returns 1 day, 28 seconds

{\it timetrans(715000)} returns 1 week, 1 day, 6 hours, 36 minutes, 40 second

{\it timetrans(720000)} returns 1 week, 1 day, 8 hours

{\it fuzzytimetrans(400)} returns 6.7 minutes

{\it fuzzytimetrans(420)} returns 7.0 minutes

{\it fuzzytimetrans(888)} returns 14.8 minutes

{\it fuzzytimetrans(86400)} returns 1.0 day

{\it fuzzytimetrans(86488)} returns 1.0 day

{\it fuzzytimetrans(715000)} returns 1.2 weeks

{\it fuzzytimetrans(720000)} returns 1.2 weeks

{\bf SEE ALSO}

timetrans(1)

