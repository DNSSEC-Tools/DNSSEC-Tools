\clearpage

\subsection{{\bf query\_send()} Secure Resolver Library Routines}


{\bf NAME}

query\_send, response\_rcv, get, free\_name\_server, free\_name\_servers -
send queries and receive responses from a DNS name server.

print\_response - Display answers returned from the name server

{\bf SYNOPSIS}

\begin{verbatim}
  #include <resolver.h>

  int query_send( const char    *name,
            const u_int16_t     type,
            const u_int16_t     class,
            struct name_server  *nslist,
            int                 *trans_id);

  int response_recv(int         *trans_id,
            struct name_server  **respondent,
            u_int8_t            **response,
            u_int32_t           *response_length);

  int get(const char          *name_n,
          const u_int16_t     type_h,
          const u_int16_t     class_h,
          struct name_server  *nslist,
          struct name_server  **respondent,
          u_int8_t            **response,
          u_int32_t           *response_length);

  void free_name_server(struct name_server **ns);

  void free_name_servers(struct name_server **ns);

  void print_response(u_int8_t *response, int response_length);
\end{verbatim}

{\bf DESCRIPTION}

The {\bf query\_send()} function can be used to send a query comprised of the
$<${\it name, class, type}$>$ tuple to the name servers specified in {\it
nslist}.  {\it trans\_id} provides a handle to this transaction within the
{\it libsres} library.

The {\bf response\_recv()} function returns the answers, if available, from the
name server that responds within the transaction identified by {\it trans\_id}.
The response is available in {\it response} and the responding name server is
returned in {\it respondent}.  The length of the response in bytes is returned
in {\it response\_length}.

The {\bf get()} function provides a wrapper around the {\bf query\_send()} and
{\bf response\_recv()} functions.  It blocks until a response is received from
some name server or the request times out.  The {\it libsres} library does
not automatically send out recursive queries; referral requests are also
treated as valid responses.

The memory pointed to by {\it *respondent} is allocated by the {\it libsres}
library and this must be freed by the invoker using {\bf free\_name\_server()}.
An entire list of name servers can be freed using {\bf free\_name\_servers()}.

{\bf print\_response()} provides a convenient way to display answers returned
in {\it response} by the name server.

{\it struct name\_server} is defined in {\bf resolver.h} as follows.

\begin{verbatim}
  struct name_server
  {
        u_int8_t *ns_name_n;
        void *ns_tsig_key;
        u_int32_t ns_security_options;
        u_int32_t ns_status;
        struct name_server *ns_next;
        int ns_number_of_addresses;
        struct sockaddr ns_address[1];
  };
\end{verbatim}


\begin{description}

\item [{\it ns\_name\_n}]\verb" "

The name of the zone for which this name server is authoritative.  This field
provides a convenient way for the invoker to index a list of name servers
while sending queries to different name servers, especially during a referral.
It is not used directly by the resolver and can be set to an empty string.

\item [{\it ns\_tsig\_key}]\verb" "

The {\it tsig} key that should be used to protect messages sent to this name
server.  This field is currently unused.

\item [{\it ns\_security\_options}]\verb" "

The security options for the zone.  This can be set to either ZONE\_USE\_NOTHING
or ZONE\_USE\_TSIG.

\item [{\it ns\_status}]\verb" "

The status of the zone.  This field is used internally by the invoker to
maintain properties of the zone.  Currently defined values for this field are
SR\_ZI\_STATUS\_UNSET, SR\_ZI\_STATUS\_PERMANENT and SR\_ZI\_STATUS\_LEARNED.

\item [{\it ns\_next}]\verb" "

The address of the next name server in the list.

\item [{\it ns\_number\_of\_addresses}]\verb" "

The number of elements in the array {\it ns\_addresses}.  This field is
currently unused.

\item [{\it ns\_addresses}]\verb" "

The IP address of the name server.  Currently, only IPv4 addresses can be
stored.

\end{description}

{\bf OTHER SYMBOLS EXPORTED}

The {\it libsres} library also exports the following BIND symbols:
\begin{packed}
\item {\it \_\_ns\_name\_ntop}
\item {\it \_\_ns\_name\_pton}
\item {\it \_\_ns\_name\_unpack}
\item {\it \_\_p\_class}
\item {\it \_\_p\_section}
\item {\it \_\_p\_type}
\end{packed}

Documentation for these symbols can be found in the BIND sources and
documentation manuals.

{\bf RETURN VALUES}

\begin{description}

\item [SR\_UNSET]\verb" "

No error.

\item [SR\_CALL\_ERROR]\verb" "

An invalid parameter was passed to {\bf get()}, {\bf query\_send()}, or
{\bf response\_recv()}.

\item [SR\_MEMORY\_ERROR]\verb" "

Memory allocation failed.

\item [SR\_MKQUERY\_INTERNAL\_ERROR]\verb" "

An internal error was encountered while trying to construct a
query message.

\item [SR\_TSIG\_INTERNAL\_ERROR]\verb" "

An internal error was encountered while trying to construct a
signed TSIG message.

\item [SR\_SEND\_INTERNAL\_ERROR]\verb" "

An internal error was encountered while trying to send the
message to the name server(s).

\item [SR\_NO\_ANSWER\_YET]\verb" "

No answer currently available; the query is still active.

\item [SR\_NO\_ANSWER]\verb" "

No answers were received from any name server.

\item [SR\_RCV\_INTERNAL\_ERROR]\verb" "

An internal error was encountered while trying to receive
responses from a name server.

\item [SR\_WRONG\_ANSWER]\verb" "

The header bits did not correctly identify the message as a response.

\item [SR\_HEADER\_BADSIZE]\verb" "

The length and count of records in the header were incorrect.

\item [SR\_TSIG\_ERROR]\verb" "

TSIG validation on the response message failed.

\item [SR\_NXDOMAIN]\verb" "

The queried name did not exist.

\item [SR\_FORMERR]\verb" "

The name server was not able to parse the query message.

\item [SR\_SERVFAIL]\verb" "

The name server was not reachable.

\item [SR\_NOTIMPL]\verb" "

A particular functionality is not yet implemented.

\item [SR\_REFUSED]\verb" "

The name server refused to answer this query.

\item [SR\_GENERIC\_FAILURE]\verb" "

Other failure returned by the name server and reflected in the
returned message RCODE.

\item [SR\_EDNS\_VERSION\_ERROR]\verb" "

Wrong EDNS version used.  Not implemented.

\item [SR\_UNSUPP\_EDNS0\_LABEL]\verb" "

Unsupported EDNS version used.  Not implemented.

\item [SR\_SUSPICIOUS\_BIT]\verb" "

A bit in the header was set to an unexpected value.  Not implemented.

\item [SR\_NAME\_EXPANSION\_FAILURE]\verb" "

Could not expand name from wire format.  Not used.

\end{description}

{\bf CURRENT STATUS}

There is currently no support for IPv6.

There is limited support for specifying resolver policy; members of the
{\it struct name\_server} are still subject to change.

The library is not thread-safe.

{\bf SEE ALSO}

{\bf libval(3)}

\url{http://dnssec-tools.sourceforge.net}

