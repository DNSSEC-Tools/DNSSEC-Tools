\clearpage

\subsection{\it donuts}

{\bf NAME}

DoNutS - analyze DNS zone files for errors and warnings

{\bf SYNOPSIS}

\begin{verbatim}
  donuts [-h] [-H] [-v] [-l LEVEL] [-r RULEFILES] [-i IGNORELIST]
         [-C] [-c configfile] ZONEFILE DOMAINNAME...
\end{verbatim}

{\bf DESCRIPTION}

DoNutS is a DNS Lint application that examines DNS zone files looking for
particular problems.  This is especially important for zones making use of
DNSSEC security records, since many subtle problems can occur.

{\bf OPTIONS}

\begin{description}

\item [-h]\verb" "

Displays a help message.

\item [-v]\verb" "

Turns on more verbose output.

\item [-q]\verb" "

Turns on more quiet output.

\item [-l {\it LEVEL}]\verb" "

Sets the level of errors to be displayed.  The default is level 5.
The maximum value is level 9, which displays many debugging results.
You probably want to run no higher than level 8.

\item [-r {\it RULEFILES}]\verb" "

A comma-separated list of rule files to load.  The strings will be
passed to {\bf glob()} so * wildcards can be used to specify multiple files.

\item [-i {\it IGNORELIST}]\verb" "

A comma-separated list of regex patterns which are checked against rule names
to determine if some should be ignored.  Run with {\it -v} to figure out rule
names if you're not sure which rule is generating errors you don't wish to see.

\item [-L]\verb" "

Include rules that require live queries of data.  Generally, these
rules are ones that concentrate on pulling remote DNS data to test;
for example, parent/child zone relationships.

\item [-c {\it CONFIGFILE}]\verb" "

Parse a configuration file to change constraints specified by rules.
This defaults to {\bf \$HOME/.donuts.conf}.

\item [-C]\verb" "

Don't read user configuration files at all, such as those specified by
the {\it -c} option or the {\bf \$HOME/.donuts.conf} file.

\item [-t {\it INTERFACE}]\verb" "

Specifies that {\it tcpdump} should be started on {\it INTERFACE} (e.g.,
"eth0") just before donuts begins its run of rules for each domain
and will stop it just after it has processed the rules.  This is
useful when you wish to capture the traffic generated by the {\it --live}
option, described above.

\item [-T {\it FILTER}]\verb" "

When {\it tcpdump} is run, this {\it FILTER} is passed to it for purposes of
filtering traffic.  By default, this is set to {\it port 53 || ip[6:2] \&
0x1fff != 0} which limits the traffic to traffic destined to port 53
(DNS) or fragmented packets.

\item [o {\it FILE}]\verb" "

Saves the {\it tcpdump} captured packets to {\it FILE}.  The following
special fields can be used to help generate unique file names:

\begin{description}

\item [\%d]\verb" "

This is replaced with the current domain name being analyzed (e.g.,
"example.com").

\item [\%t]\verb" "

This is replaced with the current epoch time (i.e., the number of
seconds since Jan 1, 1970).

\end{description}

This field defaults to {\it \%d.\%t.pcap}.

\item [-H]\verb" "

Displays the personal configuration file rules and tokens that are
acceptable to be found in a configuration file.  The output will
consist of a rule name, a token, and a description of its meaning.

Your configuration file (e.g., {\bf \$HOME/.donuts.conf}) may have lines in it
that look like this:

\begin{verbatim}
  # change the default minimum number of legal NS records from 2 to 1
  name: DNS_MULTIPLE_NS
  minnsrecords: 1

  # change the level of the following rule from 8 to 5
  name: DNS_REASONABLE_TTLS
  level: 5
\end{verbatim}

This allows you to override certain aspects of how rules are executed.

\item [-R]\verb" "

Displays a list of all known rules along with their description (if available).

\end{description}

{\bf SEE ALSO}

{\bf Net::DNS::SEC::Tools::Donuts::Rule}, {\bf Net::DNS}

\url{http://dnssec-tools.sourceforge.net}

