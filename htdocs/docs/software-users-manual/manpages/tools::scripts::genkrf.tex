\clearpage

\subsubsection{\bf genkrf}

{\bf NAME}

\cmd{genkrf} - Generate a {\it keyrec} file from Key Signing Key (KSK)
and/or Zone Signing Key (ZSK) files.

{\bf SYNOPSIS}

\begin{verbatim}
    genkrf [options] <zone-file> [<signed-zone-file>]
\end{verbatim}

{\bf DESCRIPTION}

{\it genkrf} generates a {\it keyrec} file from KSK and/or ZSK files.  It
generates new KSK and ZSK keys if needed.

The name of the {\it keyrec} file to be generated is given by the
{\it -krfile} option.  If this option is not specified, \path{zone-name.krf}
is used as the name of the {\it keyrec} file.  If the {\it keyrec} file
already exists, it will be overwritten with new {\it keyrec} definitions.

The {\it zone-file} argument is required.  It specifies the name of the
zone file from which the signed zone file was created.  The optional
{\it signed-zone-file} argument specifies the name of the signed zone file.
If it is not given, then it defaults to \path{zone-file.signed}.

{\bf OPTIONS}

\cmd{genkrf} has a number of options that assist in creation of the {\it
keyrec} file.  These options will be set to the first value found from this
search path:

\begin{verbatim}
    command line options
    DNSSEC-Tools configuration file
    DNSSEC-Tools defaults
\end{verbatim}

See \perlmod{tooloptions.pm(3)} for more details.
Exceptions to this are given in the option descriptions below.

The \cmd{genkrf} options are described below.

{\bf General \cmd{genkrf} Options}

\begin{description}

\item {\it -zone zone-name}\verb" "

This option specifies the name of the zone.  If it is not given then
{\it zone-file} will be used as the name of the zone.

\item {\it -krfile keyrec-file}\verb" "

This option specifies the name of the {\it keyrec} file to be generated.
If it is not given, then \path{zone-name.krf} will be used.

\item {\it -algorithm algorithm}\verb" "

This option specifies the algorithm used to generate encryption keys.

\item {\it -endtime endtime}\verb" "

This option specifies the time that the signature on the zone expires,
measured in seconds.

\item {\it -random random-device}\verb" "

Source of randomness used to generate the zone's keys. See the man
page for \cmd{dnssec-signzone} for the valid format of this field.

\item {\it -verbose}\verb" "

Display additional messages during processing.  If this option is given at
least once, then a message will be displayed indicating the successful
generation of the {\it keyrec} file.  If it is given twice, then the values
of all options will also be displayed.

\item {\it -help}\verb" "

Display a usage message.

\end{description}

{\bf KSK-related Options}

\begin{description}

\item {\it -ksk KSK-name}\verb" "

This option specifies the KSK's key file being used to sign the zone.  If this
option is not given, a new KSK will be created.

\item {\it -kskdir KSK-directory}\verb" "

This option specifies the absolute or relative path of the directory
where the KSK resides.  If this option is not given, it defaults to
the current directory ``.''.

\item {\it -ksklength KSK-length}\verb" "

This option specifies the length of the KSK encryption key.

\item {\it -ksklife KSK-lifespan}\verb" "

This option specifies the lifespan of the KSK encryption key.  This lifespan
is {\bf not} inherent to the key itself.  It is {\bf only} used to determine
when the KSK must be rolled over.

\end{description}

{\bf ZSK-related Options}

\begin{description}

\item {\it -zskcur ZSK-name}\verb" "

This option specifies the current ZSK being used to sign the zone.
If this option is not given, a new ZSK will be created.

\item {\it -zskpub ZSK-name}\verb" "

This option specifies the published ZSK for the zone.  If this option
is not given, a new ZSK will be created.

\item {\it -zskcount ZSK-count}\verb" "

This option specifies the number of current and published ZSK keys that will
be generated.  If this option is not given, the default given in the
DNSSEC-Tools configuration file will be used.

\item {\it -zskdir ZSK-directory}\verb" "

This option specifies the absolute or relative path of the directory
where the ZSKs reside.  If this option is not given, it defaults to
the current directory ``.''.

\item {\it -zsklength ZSK-length}\verb" "

This option specifies the length of the ZSK encryption key.

\item {\it -zsklife ZSK-lifespan}\verb" "

This option specifies the lifespan of the ZSK encryption key.  This lifespan
is {\bf not} inherent to the key itself.  It is {\bf only} used to determine
when the ZSK must be rolled over.

\end{description}

{\bf SEE ALSO}

\cmd{dnssec-keygen(8)},
\cmd{dnssec-signzone(8)},
\cmd{zonesigner(8)}

\perlmod{Net::DNS::SEC::Tools::conf.pm(3)},
\perlmod{Net::DNS::SEC::Tools::defaults.pm(3)}, \\
\perlmod{Net::DNS::SEC::Tools::keyrec.pm(3)}

\path{conf(5)},
\path{keyrec(5)}
