\clearpage

\subsubsection{Rollrec Files}

{\bf NAME}

\struct{rollrec} - Rollover-related zone data used by DNSSEC-Tools programs.

{\bf DESCRIPTION}

\struct{rollrec} files contain data used by the DNSSEC-Tools to manage key
rollover.  A \struct{rollrec} file is organized in sets of \struct{rollrec}
records.  Each \struct{rollrec} record describes the rollover state of a
single zone and must be either of {\it roll} type or {\it skip} type.  Zone
\struct{rollrec}s record information about currently rolling zones.  Skip
\struct{rollrec}s record information about zones that are not being rolled.
A \struct{rollrec} consists of a set of keyword/value entries.

The DNSSEC-Tools \perlmod{rollrec.pm} module manipulates the contents of
a \struct{rollrec} file.  Module interfaces exist for looking up
\struct{rollrec} records, creating new records, and modifying existing
records.

Comment lines and blank lines are ignored by the DNSSEC-Tools programs.
Comment lines start with either a `\#' character or a `;' character.

A \struct{rollrec}'s name may consist of alphabetic characters, numbers, and
several special characters.  The special characters are the minus sign, the
plus sign, the underscore, the comma, the period, the colon, the
forward-slash, the space, and the tab.

The values in a \struct{rollrec}'s entries may consist of alphabetic
characters, numbers, and several special characters.  The special characters
are the minus sign, the plus sign, the underscore, the comma, the period, the
colon, the forward-slash, the space, and the tab.

{\bf FIELDS}

The fields in a \struct{rollrec} record are:

\begin{description}

\item {\it administrator}\verb" "

This is the email address for the zone's administrative user.  If it is not
set, the default from the DNSSEC-Tools configuration file will be used.

\item {\it directory}\verb" "

This field contains the name of the directory in which {\bf rollerd} will
execute for the \struct{rollrec}'s zone.  If it is not specified, then the
normal {\bf rollerd} execution directory will be used.

\item {\it display}\verb" "

This boolean field indicates whether or not the zone should be displayed by
the {\bf blinkenlights} program.

\item {\it keyrec}\verb" "

The zone's \struct{keyrec} file.

\item {\it kskphase}\verb" "

The zone's current KSK rollover phase.  A value of zero indicates that the
zone is not in rollover, but is in normal operation.  A numeric value of 1-7
indicates that the zone is in that phase of KSK rollover.

\item {\it ksk\_rolldate}\verb" "

The time at which the zone's last KSK rollover completed.  This is only used
to provide a human-readable format of the timestamp.  It is derived from the
{\it ksk\_rollsecs} field.

\item {\it ksk\_rollsecs}\verb" "

The time at which the zone's last KSK rollover completed.  This value is used
to derive the {\it ksk\_rolldate} field.

\item {\it loglevel}\verb" "

The {\bf rollerd} logging level for this zone.

\item {\it maxttl}\verb" "

The maximum time-to-live for the zone.  This is measured in seconds.

\item {\it phasestart}\verb" "

The time-stamp of the beginning of the zone's current phase.

\item {\it zonefile}\verb" "

The zone's zone file.

\item {\it zskphase}\verb" "

The zone's current ZSK rollover phase.  A value of zero indicates that the zone
is not in rollover, but is in normal operation.  A value of 1, 2, 3, 4
indicates that the zone is in that phase of ZSK rollover.

\item {\it zsk\_rolldate}\verb" "

The time at which the zone's last ZSK rollover completed.  This is only used
to provide a human-readable format of the timestamp.  It is derived from the
{\it ksk\_rollsecs} field.

\item {\it zsk\_rollsecs}\verb" "

The time at which the zone's last ZSK rollover completed.  This value is used
to derive the {\it ksk\_rolldate} field.

\end{description}

{\bf EXAMPLES}

The following is an example of a roll \struct{rollrec} record:

\begin{verbatim}

    roll "example.com"
            zonefile        "example.signed"
            keyrec          "example.krf"
            kskphase        "1"
            zskphase        "0"
            administrator   "bob@bobbox.example.com"
            loglevel        "info"
            maxttl          "60"
            display         "1"
            ksk_rollsecs    "1172614842"
            ksk_rolldate    "Tue Feb 27 22:20:42 2007"
            zsk_rollsecs    "1172615087"
            zsk_rolldate    "Tue Feb 27 22:24:47 2007"
            phasestart      "Mon Feb 20 12:34:56 2007"

\end{verbatim}

The following is an example of a skip \struct{rollrec} record:

\begin{verbatim}

    skip "test.com"
            zonefile        "test.com.signed"
            keyrec          "test.com.krf"
            kskphase        "0"
            zskphase        "2"
            administrator   "tess@test.com"
            loglevel        "info"
            maxttl          "60"
            display         "1"
            ksk_rollsecs    "1172614800"
            ksk_rolldate    "Tue Feb 27 22:20:00 2007"
            zsk_rollsecs    "1172615070"
            zsk_rolldate    "Tue Feb 27 22:24:30 2007"
            phasestart      "Mon Feb 20 12:34:56 2007"

\end{verbatim}

{\bf SEE ALSO}

lsroll(1),

blinkenlights(8),
rollerd(8),
zonesigner(8)

Net::DNS::SEC::Tools::keyrec(3),
Net::DNS::SEC::Tools::rollrec(3)

keyrec(5)

