\clearpage

\subsubsection{\bf expchk}

{\bf NAME}

\cmd{expchk} - Check a DNSSEC-Tools {\it keyrec} file for expired zones.

{\bf SYNOPSIS}

\begin{verbatim}
    expchk [options] keyrec_files
\end{verbatim}

{\bf DESCRIPTION}

\cmd{expchk} checks a set of {\it keyrec} files to determine if the zone
{\it keyrec}s are valid or expired.  The type of zones displayed depends
on the options chosen; if no options are given the expired zones will be
listed.

{\bf OPTIONS}

\begin{description}

\item {\it -all}\verb" "

Display expiration information on all zones, expired or valid, in the
specified {\it keyrec} files.

\item {\it -expired}\verb" "

Display expiration information on the expired zones in the specified
{\it keyrec} files.  This is the default action.

\item {\it -valid}\verb" "

Display expiration information on the valid zones in the specified
{\it keyrec} files.

\item {\it -warn numdays}\verb" "

A warning will be given for each valid zone that will expire in {\it numdays}
days.  This option has no effect on expired zones.

\item {\it -zone zonename}\verb" "

Display expiration information on the zone specified in {\it zonename}.

\item {\it -count}\verb" "

Only the count of matching zones (valid or expired) will be given.  If both
types of zones are selected, then the count will be the number of zones in the
specified {\it keyrec} files.

\item {\it -help}\verb" "

Display a usage message.

\end{description}

{\bf SEE ALSO}

\cmd{zonesigner(3)}

\perlmod{Net::DNS::SEC::Tools::keyrec.pm(3)}

\path{keyrec(5)}
