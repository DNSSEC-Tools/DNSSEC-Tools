\clearpage

\subsubsection{\bf dtinitconf}

{\bf NAME}

\cmd{dtinitconf} - Creates a DNSSEC-Tools configuration file.

{\bf SYNOPSIS}

\begin{verbatim}
    dtinitconf [options]
\end{verbatim}

{\bf DESCRIPTION}

The \cmd{dtinitconf} program initializes the DNSSEC-Tools configuration file.
By default, the actual configuration file will be created, though the created
file can be specified by the user.  Existing files, whether the default or one
specified by the user, will not be overwritten unless specifically directed
by the user.

Each configuration field can be individually specified on the command line.
The user will also be prompted for the fields, with default values taken from
the DNSSEC-Tools \perlmod{defaults.pm} module.  If the {\it -noprompt} option is
given, then a default configuration file (modulo command-line arguments) will
be created.

Configuration entries are created for several BIND programs.  Several
locations on the system are searched to find the locations of these programs. 
First, the directories in the path environment variable are checked; the
names of any directories that contain the BIND programs are saved.  Next,
several common locations for BIND programs are checked; again, the names of
directories that contain the BIND programs are saved.  After collecting these
directories, the user is presented with this list and may choose to use
whichever set is desired.  If no directories are found that contain the BIND
programs, the user is prompted for the proper location.

If the configuration file's parent directory does not exist, then an attempt
is made to create the directory.  The new directory's ownership will be set to
{\it root} for the owner and {\it dnssec} for the group, assuming the {\it
dnssec} group exists.

{\bf OPTIONS}

\cmd{dtinitconf} takes options that control the contents of the newly generated
DNSSEC-Tools configuration file.  Each configuration file entry has a
corresponding command-line option.  The options, described below, are ordered
in logical groups.

{\bf Key-related Options}

These options deal with different aspects of creating and managing
encryption keys.

\begin{description}

\item {\it -algorithm algorithm}\verb" "

Selects the cryptographic algorithm. The value of algorithm must be one that
is recognized by \cmd{dnssec-keygen}.

\item {\it -ksklength keylen}\verb" "

The default KSK key length to be passed to \cmd{dnssec-keygen}.

\item {\it -ksklife lifespan}\verb" "

The default length of time between KSK roll-overs.  This is measured in   
seconds.

This value is {\bf only} used for key roll-over.  Keys do not have a life-time
in any other sense.

\item {\it -zskcount ZSK-count}\verb" "

The default number of ZSK keys that will be created for a zone.

\item {\it -zsklength keylen}\verb" "

The default ZSK key length to be passed to {\it dnssec-keygen}.

\item {\it -zsklife lifespan}\verb" "

The default length of time between ZSK roll-overs.  This is measured in   
seconds.

This value is {\bf only} used for key roll-over.  Keys do not have a life-time
in any other sense.

\item {\it -random randomdev}\verb" "

The random device generator to be passed to \cmd{dnssec-keygen}.

\end{description}

{\bf Zone-related Options}

These options deal with different aspects of zone signing.

\begin{description}

\item {\it -endtime endtime}\verb" "

The zone default expiration time to be passed to \cmd{dnssec-signzone}.

\end{description}

{\bf DNSSEC-Tools Options}

These options deal specifically with functionality provided by DNSSEC-Tools.

\begin{description}

\item {\it -archivedir directory}\verb" "

{\it directory} is the archived-key directory.  Old encryption keys are moved
to this directory, but only if they are to be saved and not deleted.

\item {\it -binddir directory}\verb" "

{\it directory} is the directory holding the BIND programs.

\item {\it -entropy\_msg}\verb" "

A flag indicating that \cmd{zonesigner} should display a message about entropy
generation.  This is primarily dependent on the implementation of a system's
random number generation.

\item {\it -noentropy\_msg}\verb" "

A flag indicating that \cmd{zonesigner} should not display a message about
entropy generation.  This is primarily dependent on the implementation of
a system's random number generation.

\item {\it -roll-logfile logfile}\verb" "

{\it logfile} is the logfile for the \cmd{rollerd} daemon.

\item {\it -roll-loglevel loglevel}\verb" "

{\it loglevel} is the logging level for the \cmd{rollerd} daemon.

\item {\it -roll-sleep sleep-time}\verb" "

{\it sleep-time} is the sleep-time for the \cmd{rollerd} daemon.

\item {\it -savekeys}\verb" "

A flag indicating that old keys should be moved to the archive directory.

\item {\it -nosavekeys}\verb" "

A flag indicating that old keys should not be moved to the archive directory
but will instead be left in place.

\item {\it -usegui}\verb" "

A flag indicating that the GUI for specifying command options may be used.

\item {\it -nousegui}\verb" "

A flag indicating that the GUI for specifying command options should not be
used.

\end{description}

{\bf \cmd{dtinitconf} Options}

These options deal specifically with \cmd{dtinitconf}.

\begin{description}

\item {\it -outfile conffile}\verb" "

The configuration file will be written to {\it conffile}.  If this
is not given, then the default configuration file (as returned by
\func{Net::DNS::SEC::Tools::conf::getconffile()}) will be used.

\item {\it -overwrite}\verb" "

If {\it -overwrite} is specified, existing output files may be overwritten.
Without {\it -overwrite}, if the output file is found to exist then
\cmd{dtinitconf} will give an error message and exit.

\item {\it -noprompt}\verb" "

If {\it -noprompt} is specified, the user will not be prompted for any input.
The configuration file will be created from command-line options and
DNSSEC-Tools defaults.

\item {\it -edit}\verb" "

If {\it -edit} is specified, the output file will be edited after it has been
created.  The EDITOR environment variable is consulted for the editor to
use.  If the EDITOR environment variable isn't defined, then the \cmd{vi}
editor will be used.

\item {\it -verbose}\verb" "

Provide verbose output.

\item {\it -help}\verb" "

Display a usage message and exit.

\end{description}

{\bf SEE ALSO}

\cmd{dnssec-keygen(8)},
\cmd{dnssec-signzone(8)},
\cmd{named-checkzone(8)}, \\
\cmd{rollerd(8)},
\cmd{zonesigner(8)}

\perlmod{Net::DNS::SEC::Tools::conf.pm(3)},
\perlmod{Net::DNS::SEC::Tools::defaults.pm(3)}, \\
\perlmod{Net::DNS::SEC::Tools::tooloptions.pm(3)},
\perlmod{QWizard.pm(3)}

\path{dnssec-tools.conf(5)}
