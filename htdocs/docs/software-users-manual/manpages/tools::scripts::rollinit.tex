\clearpage

\subsubsection{\bf rollinit}

{\bf NAME}

\cmd{rollinit} - Create new {\it rollrec} records for a DNSSEC-Tools {\it rollrec} file.

{\bf SYNOPSIS}

\begin{verbatim}
    rollinit [options] <zonename1> ... <zonenameN>
\end{verbatim}

{\bf DESCRIPTION}

\cmd{rollinit} creates new {\it rollrec} entries for a {\it rollrec} file.
This {\it rollrec} file will be used by \cmd{rollerd} to manage key rollover
for the named domains.

A {\it rollrec} entry has this format:

\begin{verbatim}
    roll "example.com"
        zonefile        "example.com.signed"
        keyrec          "example.com.krf"
        curphase        "0"
        maxttl          "604800"
        display         "1"
        phasestart      "Mon Jan 9 16:00:00 2006"
\end{verbatim}

The {\it zonefile} and {\it keyrec} fields are set according to command-line
options and arguments.  The manner of generating the {\it rollrec}'s actual
values is a little complex and is described in the {\bf ZONEFILE And KEYREC
FIELDS} section below.  The {\it curphase} field is set to 0 to indicate that
the zone is in normal operation (non-rollover.)  The {\it display} field is
set to indicate that \cmd{blinkenlights} should display the record.  The {\it
maxttl} and {\it phasestart} fields are set to dummy values.

The keywords {\bf roll} and {\bf skip} indicate whether \cmd{rollerd} should
process or ignore a particular {\it rollrec} entry.  {\bf roll} records are
created by default; {\bf skip} entries are created if the {\it -skip} option
is specified.

The newly generated {\it rollrec} entries are written to standard output,
unless the {\it -out} option is specified.

{\bf ZONEFILE And KEYREC FIELDS}

The {\it zonefile} and {\it keyrec} fields may be given by using the
{\it -zone} and {\it -keyrec} options, or default values may be used.

The default values use the {\it rollrec}'s zone name, taken from the command
line, as a base.  {\bf .signed} is appended to the domain name for the zone
file; {\bf .krf} is appended to the domain name for the {\it keyrec} file.

If {\it -zone} or {\it -keyrec} are specified, then the options values are
used in one of two ways:

\begin{enumerate}

\item A single domain name is given on the command line.\verb" "

The option values for {\it -zone} and/or {\it -keyrec} are used for the actual
{\it rollrec} fields.

\item Multiple domain names are given on the command line.\verb" "

The option values for {\it -zone} and/or {\it -keyrec} are used as templates
for the actual {\it rollrec} fields.  The option values must contain 
the string {\bf =}.  This string is replaced by the domain whose {\it rollrec}
is being created.

\end{enumerate}

See the {\bf EXAMPLES} section for examples of how options are used by
\cmd{rollinit}.

{\bf OPTIONS}

\cmd{rollinit} may be given the following options:

\begin{description}

\item {\it -zone zonefile}\verb" "

This specifies the value of the {\it zonefile} field.
See the {\bf ZONEFILE And KEYREC FIELDS} and {\bf EXAMPLES} sections
for more details.

\item {\it -keyrec keyrec-file}\verb" "

This specifies the value of the {\it keyrec} field.
See the {\bf ZONEFILE And KEYREC FIELDS} and {\bf EXAMPLES} sections
for more details.

\item {\it -skip}\verb" "

By default, {\bf roll} records are generated.  If this option is given, then
{\bf skip} records will be generated instead.

\item {\it -out output-file}\verb" "

The new {\it rollrec} entries will be appended to {\it output-file}.
The file will be created if it does not exist.

If this option is not given, the new {\it rollrec} entries will be written
to standard output.

\item {\it -help}\verb" "

Display a usage message.

\end{description}

{\bf EXAMPLES}

The following options should make clear how \cmd{rollinit} deals with
options and the new {\it rollrec}s.  Example 1 will show the complete new
{\it rollrec} record.  For the sake of brevity, the remaining examples
will only show the newly created {\it zonefile} and {\it keyrec} records.

{\bf Example 1.  One domain, no options}

This example shows the {\it rollrec} generated by giving \cmd{rollinit} a
single domain, without any options.

\begin{verbatim}
    $ rollinit example.com
        roll    "example.com"
            zonefile        "example.com.signed"
            keyrec          "example.com.krf"
            curphase        "0"
            maxttl          "0"
            display         "1"
            phasestart      "new"
\end{verbatim}

{\bf Example 2.  One domain, -zone option}

This example shows the {\it rollrec} generated by giving \cmd{rollinit} a
single domain, with the {\it -zone} option.

\begin{verbatim}
    $ rollinit -zone signed-example example.com
        roll    "example.com"
            zonefile        "signed-example"
            keyrec          "example.com.krf"
\end{verbatim}

{\bf Example 3.  One domain, -keyrec option}

This example shows the {\it rollrec} generated by giving \cmd{rollinit} a
single domain, with the {\it -keyrec} option.

\begin{verbatim}
    $ rollinit -keyrec x-rrf example.com
        roll    "example.com"
            zonefile        "example.com.signed"
            keyrec          "x-rrf"
\end{verbatim}

{\bf Example 4.  One domain, -zone and -keyrec options}

This example shows the {\it rollrec} generated by giving \cmd{rollinit} a
single domain, with the {\it -zone} and {\it -keyrec} options.

\begin{verbatim}
    $ rollinit -zone signed-example -keyrec example.rrf example.com
        roll    "example.com"
            zonefile        "signed-example"
            keyrec          "xkrf"
\end{verbatim}

{\bf Example 5.  One domain, -skip option}

This example shows the {\it rollrec} generated by giving \cmd{rollinit} a
single domain, with the {\it -zone} and {\it -keyrec} options.

\begin{verbatim}
    $ rollinit -skip example.com
        skip    "example.com"
            zonefile        "example.com.signed"
            keyrec          "example.com.krf"
\end{verbatim}

{\bf Example 6.  Multiple domains, no options}

This example shows the {\it rollrec}s generated by giving \cmd{rollinit}
several domains, without any options.

\begin{verbatim}
    $ rollinit example1.com example2.com
        roll    "example1.com"
                zonefile        "example1.com.signed"
                keyrec          "example1.com.krf"

        roll    "example2.com"
                zonefile        "example2.com.signed"
                keyrec          "example2.com.krf"
\end{verbatim}

{\bf Example 7.  Multiple domains, -zone option}

This example shows the {\it rollrec}s generated by giving \cmd{rollinit}
several domains, with the {\it -zone} option.

\begin{verbatim}
    $ rollinit -zone =-signed example1.com example2.com
        roll    "example1.com"
                zonefile        "example1.com-signed"
                keyrec          "example1.com.krf"

        roll    "example2.com"
                zonefile        "example2.com-signed"
                keyrec          "example2.com.krf"
\end{verbatim}

{\bf Example 8.  Multiple domains, -keyrec option}

This example shows the {\it rollrec}s generated by giving \cmd{rollinit}
several domains, with the {\it -keyrec} option.

\begin{verbatim}
    $ rollinit -keyrec zone-=-keyrec example1.com example2.com
        roll    "example1.com"
                zonefile        "example1.com.signed"
                keyrec          "zone-example1.com-keyrec"

        roll    "example2.com"
                zonefile        "example2.com.signed"
                keyrec          "zone-example2.com-keyrec"
\end{verbatim}

{\bf Example 9.  Multiple domains, -zone and -keyrec options}

This example shows the {\it rollrec}s generated by giving \cmd{rollinit}
several domains, with the {\it -zone} and {\it -keyrec} options.

\begin{verbatim}
    $ rollinit -zone Z-= -keyrec =K example1.com example2.com
        roll    "example1.com"
                zonefile        "Z-example1.com"
                keyrec          "example1.comK"

        roll    "example2.com"
                zonefile        "Z-example2.com"
                keyrec          "example2.comK"
\end{verbatim}

{\bf Example 10.  Single domain, -zone and -keyrec options with template}

This example shows the {\it rollrec} generated by giving \cmd{rollinit} a
single domain, with the {\it -zone} and {\it -keyrec} options.  The options
use the multi-domain {\bf =} template.

\begin{verbatim}
    $ rollinit -zone Z-= -keyrec =.K example.com
        roll    "example.com"
                zonefile        "Z-="
                keyrec          "=.K"
\end{verbatim}

This is probably not what is wanted, since it results in the {\it zonefile}
and {\it keyrec} field values containing the {\bf =}.

{\bf Example 11.  Multiple domains, -zone and -keyrec options without template}

This example shows the {\it rollrec}s generated by giving \cmd{rollinit}
several domains, with the {\it -zone} and {\it -keyrec} options.  The options
do not use the multi-domain {\bf =} template.

\begin{verbatim}
    $ rollinit -zone ex.zone -keyrec ex.krf example1.com example2.com
        roll    "example1.com"
                zonefile        "ex.zone"
                keyrec          "ex.krf"

        roll    "example2.com"
                zonefile        "ex.zone"
                keyrec          "ex.krf"
\end{verbatim}

This may not be what is wanted, since it results in the same {\it zonefile}
and {\it keyrec} fields values for each {\it rollrec}.

{\bf SEE ALSO}

\cmd{lsroll(1)},
\cmd{rollerd(8)},
\cmd{rollchk(8)},
\cmd{zonesigner(8)}

\perlmod{Net::DNS::SEC::Tools::keyrec.pm(3)},
\perlmod{Net::DNS::SEC::Tools::rollrec.pm(3)}

\path{keyrec(5)},
\path{rollrec(5)}
