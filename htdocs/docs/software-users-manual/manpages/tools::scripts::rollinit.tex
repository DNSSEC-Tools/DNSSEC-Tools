\clearpage

\subsubsection{rollinit}

{\bf NAME}

\cmd{rollinit} - Create new \struct{rollrec} records for a DNSSEC-Tools
\struct{rollrec} file

{\bf SYNOPSIS}

\begin{verbatim}
  rollinit [options] <zonename1> ... <zonenameN>
\end{verbatim}

{\bf DESCRIPTION}

\cmd{rollinit} creates new \struct{rollrec} entries for a \struct{rollrec}
file.  This \struct{rollrec} file will be used by \cmd{rollerd} to manage key
rollover for the named domains.

A \struct{rollrec} entry has this format:

\begin{verbatim}
    roll "example.com"
        zonefile        "example.com.signed"
        keyrec          "example.com.krf"
        kskphase        "0"
        zskphase        "0"
        administrator   "bob@bobhost.example.com"
        directory       "/var/dns/zones/example.com"
        loglevel        "phase"
        ksk_rolldate    " "
        ksk_rollsecs    "0"
        zsk_rolldate    " "
        zsk_rollsecs    "0"
        maxttl          "604800"
        display         "1"
        phasestart      "Mon Jan 9 16:00:00 2006"
\end{verbatim}

The {\it zonefile} and \struct{keyrec} fields are set according to command-line
options and arguments.  The manner of generating the \struct{rollrec}'s actual
values is a little complex and is described in the ZONEFILE And KEYREC FIELDS
section below.

The {\it administrator} field is set to ``bob\@bobhost.example.com'' to indicate
that the email messages to the zone's administrator should be sent to
``bob\@bobhost.example.com''.

The {\it directory} field is set to ``/var/dns/zones/example.com''
to indicate that the files for this zone should be found in
\path{/var/dns/zones/example.com}.  This includes the zone file,
the signed zone file, and the \struct{keyrec} file.

The {\it loglevel} field is set to ``phase'' to indicate that \cmd{rollerd}
should only log phase-level (and greater) log messages for this zone.

The {\it kskphase} field is set to 0 to indicate that the zone is in normal
operation (non-rollover) for KSK keys.  The {\it zskphase} field is set to 0
to indicate that the zone is in normal operation (non-rollover) for ZSK keys.

The {\it ksk\_rolldate} and {\it ksk\_rollsecs} fields are set to indicate
that the zone has not yet undergone KSK rollover.

The {\it zsk\_rolldate} and {\it zsk\_rollsecs} fields are set to indicate
that the zone has not yet undergone ZSK rollover.

The {\it display} field is set to indicate that \cmd{blinkenlights} should
display the record.  The {\it maxttl} and {\it phasestart} fields are set to
dummy values.

The keywords {\bf roll} and {\bf skip} indicate whether \cmd{rollerd} should
process or ignore a particular \struct{rollrec} entry.  {\bf roll} records are
created by default; {\bf skip} entries are created if the {\it -skip} option
is specified.

The newly generated \struct{rollrec} entries are written to standard output,
unless the {\it -out} option is specified.

{\bf ZONEFILE and KEYREC FIELDS}

The {\it zonefile} and \struct{keyrec} fields may be given by using the {\it
-zone} and {\it -keyrec} options, or default values may be used.

The default values use the \struct{rollrec}'s zone name, taken from the
command line, as a base.  \path{.signed} is appended to the domain name
for the zone file; \path{.krf} is appended to the domain name for the
\struct{keyrec} file.

If {\it -zone} or {\it -keyrec} are specified, then the options values are
used in one of two ways:

\begin{itemize}

\item A single domain name is given on the command line.\verb" "

The option values for {\it -zone} and/or {\it -keyrec} are used for the actual
\struct{rollrec} fields.

\item Multiple domain names are given on the command line.\verb" "

The option values for {\it -zone} and/or {\it -keyrec} are used as templates
for the actual \struct{rollrec} fields.  The option values must contain the
string {\bf =}.  This string is replaced by the domain whose \struct{rollrec}
is being created.

\end{itemize}

See the EXAMPLES section for examples of how options are used by \cmd{rollinit}.

{\bf OPTIONS}

\cmd{rollinit} may be given the following options:

\begin{description}

\item {\bf -zone zonefile}\verb" "

This specifies the value of the {\it zonefile} field.
See the ZONEFILE And KEYREC FIELDS and EXAMPLES sections for more details.

\item {\bf -keyrec keyrec-file}\verb" "

This specifies the value of the \struct{keyrec} field.
See the ZONEFILE And KEYREC FIELDS and EXAMPLES sections for more details.

\item {\bf -admin}\verb" "

This specifies the value of the {\it administrator} field.  If it is not given,
an {\it administrator} field will not be included for the record.

\item {\bf -directory}\verb" "

This specifies the value of the {\it directory} field.  If it is not given,
a {\it directory} field will not be included for the record.

\item {\bf -loglevel}\verb" "

This specifies the value of the {\it loglevel} field.  If it is not given, a
{\it loglevel} field will not be included for the record.

\item {\bf -skip}\verb" "

By default, {\bf roll} records are generated.  If this option is given, then
{\bf skip} records will be generated instead.

\item {\bf -out output-file}\verb" "

The new \struct{rollrec} entries will be appended to {\it output-file}.
The file will be created if it does not exist.

If this option is not given, the new \struct{rollrec} entries will be written
to standard output.

\item {\bf -help}\verb" "

Display a usage message.

\item {\bf -Version}\verb" "

Display version information for \cmd{rollinit} and DNSSEC-Tools.

\end{description}

{\bf EXAMPLES}

The following options should make clear how \cmd{rollinit} deals with options
and the new \struct{rollrec}s.  Example 1 will show the complete new
\struct{rollrec} record.  For the sake of brevity, the remaining examples will
only show the newly created {\it zonefile} and \struct{keyrec} records.

{\bf Example 1.  One domain, no options}

This example shows the \struct{rollrec} generated by giving \cmd{rollinit} a
single domain, without any options.

\begin{verbatim}
    $ rollinit example.com
        roll    "example.com"
            zonefile        "example.com.signed"
            keyrec          "example.com.krf"
            kskphase        "0"
            zskphase        "0"
            ksk_rolldate    " "
            ksk_rollsecs    "0"
            zsk_rolldate    " "
            zsk_rollsecs    "0"
            maxttl          "0"
            display         "1"
            phasestart      "new"
\end{verbatim}

{\bf Example 2.  One domain, -zone option}

This example shows the \struct{rollrec} generated by giving \cmd{rollinit} a
single domain, with the {\it -zone} option.

\begin{verbatim}
    $ rollinit -zone signed-example example.com
        roll    "example.com"
            zonefile        "signed-example"
            keyrec          "example.com.krf"
\end{verbatim}

{\bf Example 3.  One domain, -keyrec option}

This example shows the \struct{rollrec} generated by giving \cmd{rollinit} a
single domain, with the {\it -keyrec} option.

\begin{verbatim}
    $ rollinit -keyrec x-rrf example.com
        roll    "example.com"
            zonefile        "example.com.signed"
            keyrec          "x-rrf"
\end{verbatim}

{\bf Example 4.  One domain, -zone and -keyrec options}

This example shows the \struct{rollrec} generated by giving \cmd{rollinit} a
single domain, with the {\it -zone} and {\it -keyrec} options.

\begin{verbatim}
    $ rollinit -zone signed-example -keyrec example.rrf example.com
        roll    "example.com"
            zonefile        "signed-example"
            keyrec          "example.rrf"
\end{verbatim}

{\bf Example 5.  One domain, -skip option}

This example shows the \struct{rollrec} generated by giving \cmd{rollinit} a
single domain, with the {\it -zone} and {\it -keyrec} options.

\begin{verbatim}
    $ rollinit -skip example.com
        skip    "example.com"
            zonefile        "example.com.signed"
            keyrec          "example.com.krf"
\end{verbatim}

{\bf Example 6.  Multiple domains, no options}

This example shows the \struct{rollrec}s generated by giving \cmd{rollinit}
several domains, without any options.

\begin{verbatim}
    $ rollinit example1.com example2.com
        roll    "example1.com"
                zonefile        "example1.com.signed"
                keyrec          "example1.com.krf"

        roll    "example2.com"
                zonefile        "example2.com.signed"
                keyrec          "example2.com.krf"
\end{verbatim}

{\bf Example 7.  Multiple domains, -zone option}

This example shows the \struct{rollrec}s generated by giving \cmd{rollinit}
several domains, with the {\it -zone} option.

\begin{verbatim}
    $ rollinit -zone =-signed example1.com example2.com
        roll    "example1.com"
                zonefile        "example1.com-signed"
                keyrec          "example1.com.krf"

        roll    "example2.com"
                zonefile        "example2.com-signed"
                keyrec          "example2.com.krf"
\end{verbatim}

{\bf Example 8.  Multiple domains, -keyrec option}

This example shows the \struct{rollrec}s generated by giving \cmd{rollinit}
several domains, with the {\it -keyrec} option.

\begin{verbatim}
    $ rollinit -keyrec zone-=-keyrec example1.com example2.com
        roll    "example1.com"
                zonefile        "example1.com.signed"
                keyrec          "zone-example1.com-keyrec"

        roll    "example2.com"
                zonefile        "example2.com.signed"
                keyrec          "zone-example2.com-keyrec"
\end{verbatim}

{\bf Example 9.  Multiple domains, -zone and -keyrec options}

This example shows the \struct{rollrec}s generated by giving \cmd{rollinit}
several domains, with the {\it -zone} and {\it -keyrec} options.

\begin{verbatim}
    $ rollinit -zone Z-= -keyrec =K example1.com example2.com
        roll    "example1.com"
                zonefile        "Z-example1.com"
                keyrec          "example1.comK"

        roll    "example2.com"
                zonefile        "Z-example2.com"
                keyrec          "example2.comK"
\end{verbatim}

{\bf Example 10.  Single domain, -zone and -keyrec options with template}

This example shows the \struct{rollrec} generated by giving \cmd{rollinit} a
single domain, with the {\it -zone} and {\it -keyrec} options.  The options
use the multi-domain {\bf =} template.

\begin{verbatim}

    $ rollinit -zone Z-= -keyrec =.K example.com
        roll    "example.com"
                zonefile        "Z-="
                keyrec          "=.K"

\end{verbatim}

This is probably not what is wanted, since it results in the {\it zonefile}
and \struct{keyrec} field values containing the {\bf =}.

{\bf Example 11.  Multiple domains, -zone and -keyrec options without template}

This example shows the \struct{rollrec}s generated by giving \cmd{rollinit}
several domains, with the {\it -zone} and {\it -keyrec} options.  The options
do not use the multi-domain {\bf =} template.

\begin{verbatim}
    $ rollinit -zone ex.zone -keyrec ex.krf example1.com example2.com
        roll    "example1.com"
                zonefile        "ex.zone"
                keyrec          "ex.krf"

        roll    "example2.com"
                zonefile        "ex.zone"
                keyrec          "ex.krf"
\end{verbatim}

This may not be what is wanted, since it results in the same {\it zonefile}
and \struct{keyrec} fields values for each \struct{rollrec}.

{\bf SEE ALSO}

lsroll(1)

rollerd(8),
rollchk(8),
zonesigner(8)

Net::DNS::SEC::Tools::keyrec.pm(3),
Net::DNS::SEC::Tools::rollrec.pm(3)

Net::DNS::SEC::Tools::file-keyrec.pm(5),
Net::DNS::SEC::Tools::file-rollrec.pm(5)

