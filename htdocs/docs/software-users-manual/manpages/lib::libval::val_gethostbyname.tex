\clearpage

\subsection{\bf val\_gethostbyname()}

{\bf NAME}

\func{val\_gethostbyname()}, \func{val\_gethostbyname2()},
\func{val\_gethostbyname\_r()}, \func{val\_gethostbyname2\_r()} -
get DNSSEC-validated network host entry

{\bf SYNOPSIS}

\begin{verbatim}
    #include <validator.h>

    extern int h_errno;
    struct hostent *val_gethostbyname(const val_context_t *ctx,
                                      const char *name,
                                      val_status_t *val_status);

    struct hostent *val_gethostbyname2(const val_context_t *ctx,
                                       const char *name,
                                       int af,
                                       val_status_t *val_status);

    int val_gethostbyname_r(const val_context_t *ctx,
                            const char *name,
                            struct hostent *ret,
                            char *buf,
                            size_t buflen,
                            struct hostent **result,
                            int *h_errnop,
                            val_status_t *val_status);

    int val_gethostbyname2_r(const val_context_t *ctx,
                             const char *name,
                             int af,
                             struct hostent *ret,
                             char *buf,
                             size_t buflen,
                             struct hostent **result,
                             int *h_errnop,
                             val_status_t *val_status);
\end{verbatim}

{\bf DESCRIPTION}

\func{val\_gethostbyname()}, \func{val\_gethostbyname2()},
\func{val\_gethostbyname\_r()} and \func{val\_gethostbyname2\_r()}
perform DNSSEC validation of DNS queries.  They return a network host entry
value of type \struct{struct hostent} and are DNSSEC-aware versions of the
\func{gethostbyname(3)}, \func{gethostbyname2(3)}, \func{gethostbyname\_r()}
and \func{gethostbyname2\_r()} functions, respectively.  (See
\func{gethostbyname(3)} for more information on type \struct{struct hostent}).

\func{val\_gethostbyname()} and \func{val\_gethostbyname\_r()}
support only IPv4 addresses.  IPv4 and IPv6 addresses are supported by
\func{val\_gethostbyname2()} and \func{val\_gethostbyname2\_r()}.

The \func{val\_gethostbyname\_r()} and \func{val\_gethostbyname2\_r()} are
reentrant versions and can be safely used in multi-threaded applications.

The \var{ctx} parameter specifies the validation context, which can be set to
NULL for default values.  The caller can use \var{ctx} to control the resolver
and validator policies.  Using a non-NULL validator context over multiple
calls can provide some optimization.  (See \lib{libval(3)} for
information on creating a validation context.)

\func{val\_gethostbyname()} and \func{val\_gethostbyname2()} set the global
\var{h\_errno} variable to return the resolver error code.  The reentrant
versions \func{val\_gethostbyname\_r()} and \func{val\_gethostbyname2\_r()}
use the \var{h\_errnop} parameter to return this value.  This ensures thread
safety, by avoiding the global \var{h\_errno} variable.  \var{h\_errnop} must
not be NULL.  (See the man page for \func{gethostbyname(3)} for possible
values of \var{h\_errno}.)

The \var{name}, \var{af}, \var{ret}, \var{buf}, \var{buflen}, and \var{result}
parameters have the same meaning and syntax as the corresponding parameters
for the original \func{gethostbyname*()} functions.  See the manual page for
\func{gethostbyname(3)} for more details about these parameters.

The \var{val\_status} parameter is used to return the validator error code.
\func{val\_istrusted()} determines whether this validation status represents a
trusted value and \func{p\_val\_status()} displays this value to the user in a
useful manner (See {\bf libval(3)} more for information).  \var{val\_status}
must not be NULL.

{\bf RETURN VALUE}

The \func{val\_gethostbyname()} and \func{val\_gethostbyname2()} functions
return a pointer to a \struct{hostent} structure when they can resolve the
given host name (with or without DNSSEC validation), and NULL on error.  The
memory for the returned value may be statically allocated by these two
functions.  Hence, the caller must not free the memory for the returned value.

The \func{val\_gethostbyname\_r()} and \func{val\_gethostbyname2\_r()}
functions return 0 when they can resolve the given host name (with or
without DNSSEC validation), and a non-zero error-code on failure.

{\bf EXAMPLE}

\begin{verbatim}
    #include <stdio.h>
    #include <stdlib.h>
    #include <validator.h>

    int main(int argc, char *argv[])
    {
             int val_status;
             struct hostent *h = NULL;

             if (argc < 2) {
                     printf("Usage: %s <hostname>\n", argv[0]);
                     exit(1);
             }
    
             h = val_gethostbyname(NULL, argv[1], &val_status);
             printf("h_errno = %d [%s]\n", h_errno,
                    hstrerror(h_errno));
             if (h) {
                     printf("Validation Status = %d [%s]\n", val_status,
                            p_val_status(val_status));
             }

             return 0;
    }
\end{verbatim}

{\bf NOTES}

These functions do not currently read the order of lookup from
\path{/etc/hosts.conf}.  This functionality will be provided in
future versions.  At present, the default order is set to consult
the \path{/etc/hosts} file first and then query DNS.

The current versions of these functions do not support NIS lookups.

{\bf SEE ALSO}

\func{gethostbyname(3)}, \func{gethostbyname2(3)}, \func{gethostbyname\_r(3)},
\func{gethostbyname2\_r(3)} \\
\func{get\_context(3)}, \func{val\_getaddrinfo(3)}, \func{val\_freeaddrinfo(3)},
\func{val\_query(3)} 

\lib{libval(3)}
