\clearpage

\subsection{{\bf val\_gethostbyname()} DNSSEC-Validated Network Host Entry}


{\bf NAME}

val\_gethostbyname, val\_x\_gethostbyname, val\_get\_hostent\_dnssec\_status, val\_duphostent, val\_freehostent - get DNSSEC-validated network host entry

{\bf SYNOPSIS}

\begin{verbatim}
  #include <validator.h>

  struct hostent *val_gethostbyname(const char *name, int *h_errnop);

  struct hostent *val_x_gethostbyname(const val_context_t *ctx,
                                      const char *name, int *h_errnop);

  int val_get_hostent_dnssec_status(const struct hostent *hentry);

  struct hostent* val_duphostent(const struct hostent *hentry);

  void val_freehostent(struct hostent *hentry);
\end{verbatim}

{\bf DESCRIPTION}

{\bf val\_gethostbyname()} is a DNSSEC-aware, thread-safe version of {\bf
gethostbyname(3)}.  It performs DNSSEC validation of DNS queries.  It returns
a network host entry value of type {\it hostent}.  (See {\bf gethostbyname(3)}
for more information on type {\it hostent}).

{\bf val\_x\_gethostbyname()} performs the same function as {\bf
val\_gethostbyname()}, but it optimized for multiple calls.  The two routines
take the same parameters, but {\bf val\_x\_gethostbyname()} takes an
additional parameter {\it ctx}, of type {\it val\_context}, which passes the
validation context for use in call optimization.  The {\it ctx} parameter
also gives the caller more control over the resolver and validator policies.
If a {\bf NULL} value is passed for the {\it ctx} parameter, the default
validation context is used.  (See {\bf get\_context()} for information on
creating a validation context.) {\bf val\_gethostbyname()} is equivalent to
calling {\bf val\_x\_gethostbyname()} with a {\bf NULL} {\it ctx} parameter.

{\it h\_errnop} returns the {\it h\_errno} value from these functions.  This
ensures thread safety, by avoiding the global {\it h\_errno} variable.  {\it
h\_errnop} must not be {\bf NULL}.  (See the man page for {\bf
gethostbyname(3)} for possible values of {\it h\_errno}.)

{\bf val\_duphostent()} provides a way to duplicate the {\it hostent}
structure and its auxiliary data.  It performs a deep copy; i.e., the internal
strings and arrays are also copied.

{\bf val\_freehostent()} is used to free the hostent structure returned by
the {\bf val\_gethostbyname()}, {\bf val\_x\_gethostbyname()} and {\bf
val\_duphostent()} functions.

{\bf val\_get\_hostent\_dnssec\_status()} is used to extract the DNSSEC
validation status from the returned {\it hostent} structure.  This function
must be called only for the values returned from {\bf val\_gethostbyname()},
{\bf val\_x\_gethostbyname()}, and {\bf val\_duphostent()} functions.

{\bf RETURN VALUE}

The {\bf val\_gethostbyname()} and {\bf val\_x\_gethostbyname()} functions
return a value of type {\it hostent} on success, and {\bf NULL} on error.
The memory for the returned value is dynamically allocated by these two
functions.  Hence, the caller must only call the {\bf val\_freehostent()}
function on the returned value in order to avoid memory leaks.

The {\bf val\_get\_hostent\_dnssec\_status()} function returns the result of
the DNSSEC validation.  The possible values for the DNSSEC status are given
in {\bf val\_errors.h}.

The {\bf val\_duphostent()} function returns a copy of specified the {\it
hostent} structure.  The returned value must be freed using {\bf
val\_freehostent()} to avoid memory leaks.

{\bf EXAMPLE}

\begin{verbatim}
 #include <stdio.h>
 #include <validator.h>

 int main(int argc, char *argv[])
 {
          int dnssec_status = ERROR;
          int val_h_errno = NETDB_INTERNAL;
          struct hostent *h = NULL;

          if (argc < 2) {
                  printf("Usage: %s <hostname>\n", argv[0]);
                  exit(1);
          }

          h = val_gethostbyname(argv[1], &val_h_errno);
          printf("h_errno = %d [%s]\n", val_h_errno,
                 hstrerror(val_h_errno));
          if (h) {
                  dnssec_status = val_get_hostent_dnssec_status(h);

                  printf("DNSSEC Status = %d [%s]\n", dnssec_status,
                         p_val_error(dnssec_status));
                  val_freehostent(h);
          }

          return 0;
 }
\end{verbatim}

{\bf NOTES}

This version of {\bf val\_x\_gethostbyname()} hence {\bf val\_gethostbyname()}
does not read the order-of-look-up method from {\bf /etc/hosts.conf}.  This
functionality will be provided in future versions.  At present, the default
order is set to consult the {\bf /etc/hosts} file first, and then query DNS.

This version of {\bf val\_x\_gethostbyname()} and {\bf val\_gethostbyname()}
does not support NIS lookups.

{\bf SEE ALSO}

{\bf getaddrinfo}(3)

{\bf get\_context(3)}, {\bf val\_dupaddrinfo(3)}, {\bf val\_freeaddrinfo(3)},\\
{\bf val\_getaddrinfo(3)}, {\bf val\_query(3)}, {\bf val\_x\_getaddrinfo(3)},\\
{\bf val\_x\_query(3)}

{\it p\_val\_error}

http://dnssec-tools.sourceforge.net

