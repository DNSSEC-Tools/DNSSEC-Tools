\clearpage

\subsubsection{BootStrap.pm}

{\bf NAME}

\perlmod{Net::DNS::SEC::Tools::BootStrap} - Optional loading of Perl modules

{\bf SYNOPSIS}

\begin{verbatim}
  use Net::DNS::SEC::Tools::BootStrap;

  dnssec_tools_load_mods(PerlModule => 'Additional help/error text');
\end{verbatim}

{\bf DESCRIPTION}

The DNSSEC-Tools package requires a number of Perl modules that are only
needed by some of the tools.  This module helps determine at run-time, rather
than at installation time, if the right tools are available on the system.  If
any module fails to load, \func{dnssec\_tools\_load\_mods()} will display an
error message and calls \func{exit()}.  The error message describes how to
install a module via CPAN.

The arguments to \func{dnssec\_tools\_load\_mods()} are given in pairs.  Each
pair is a module to try to load (and import) and a supplemental error message.
If the module fails to load, the supplemental error message will be displayed
along with the installation-via-CPAN message.  If the error message consists
of the string ``noerror'', then no error message will be displayed before the
function exits.

{\bf CAVEATS}

The module will try to import any exported subroutines from the module into
the \var{main} namespace.  This means that the \perlmod{BootStrap.pm} module
is likely to not be useful for importing symbols into other modules.
Work-arounds for this are:

\begin{description}

\item - import the symbols by hand\verb" "

\begin{verbatim}

  dnssec_tools_load_mods(PerlModule => 'Additional help/error text');

  import PerlModule qw(func1 func2);

  func1(arg1, arg2);

\end{verbatim}

\item - call the fully qualified function name\verb" "

\begin{verbatim}

  dnssec_tools_load_mods(PerlModule => 'Additional help/error text');

  PerlModule::func1(arg1, arg2);

\end{verbatim}

\end{description}

