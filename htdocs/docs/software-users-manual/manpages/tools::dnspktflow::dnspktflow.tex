\clearpage

\subsubsection{\cmd{dnspktflow}}

{\bf NAME}

\cmd{dnspktflow} - Analyze and draw DNS flow diagrams from a \cmd{tcpdump} file

{\bf SYNOPSIS}

\begin{verbatim}
    dnspktflow -o output.png file.tcpdump

    dnspktflow -o output.png -x -a -t -q file.tcpdump
\end{verbatim}

{\bf DESCRIPTION}

The \cmd{dnspktflow} application takes a tcpdump network traffic dump
file, passes it through the \cmd{tshark} application and then displays the
resulting DNS packet flows in a ``flow-diagram'' image.  \cmd{dnspktflow}
can output a single image or a series of images which can then be
shown in sequence as an animation.

\cmd{dnspktflow} was written as a debugging utility to help trace DNS
queries and responses, especially as they apply to DNSSEC-enabled lookups.

{\bf REQUIREMENTS}

This application requires the following Perl modules and software
components to work:

\begin{itemize}
\item \cmd{graphviz} - \url{http://www.graphviz.org}
\item \perlmod{GraphViz} - Perl module
\item \cmd{tshark}   - \url{http://www.wireshark.org}
\end{itemize}

The following is required for outputting screen presentations:

\begin{itemize}
\item \perlmod{MagicPoint} - \url{http://member.wide.ad.jp/wg/mgp}
\end{itemize}

If the following modules are installed, a GUI interface will be enabled for
communication with \cmd{dnspktflow}:

\begin{itemize}
\item \perlmod{QWizard} - Perl module
\item \perlmod{Getopt::GUI::Long} - Perl module
\end{itemize}

{\bf OPTIONS}

\cmd{dnspktflow} takes a wide variety of command-line options.  These options   
are described below in the following functional groups:  input packet 
selection, output file options, output visualization options, graphical 
options, and debugging.

{\bf Input Packet Selection}

These options determine the packets that will be selected by
\cmd{dnspktflow}.

\begin{description}

\item {\it -i STRING $|$ --ignore-hosts=STRING}\verb" "

A regular expression of host names to ignore in the query/response fields.

\item {\it -r STRING $|$ --only-hosts=STRING}\verb" "

A regular expression of host names to analyze in the query/response fields.

\item {\it -f $|$ --show-frame-num}\verb" "

Display the packet frame numbers.

\item {\it -b INTEGER $|$ --begin-frame=INTEGER}\verb" "

Begin at packet frame NUMBER.

\end{description}

{\bf Output File Options}

These options determine the type and location of \cmd{dnspktflow}'s output.

\begin{description}

\item {\it -o STRING $|$ --output-file=STRING}\verb" "

Output file name (default: out\%03d.png as PNG format.)

\item {\it --fig}\verb" "

Output format should be fig.

\item {\it -O STRING $|$ --tshark-out=STRING}\verb" "

Save \cmd{tshark} output to this file.

\item {\it -m $|$ --multiple-outputs}\verb" "

One picture per request (use \%03d in the filename).

\item {\it -M STRING $|$ --magic-point=STRING}\verb" "

Saves a MagicPoint presentation for the output.

\end{description}

{\bf Output Visualization Options:}

These options determine specifics of \cmd{dnspktflow}'s output.

\begin{description}

\item {\it -L $|$ --last-line-labels-only}\verb" "

Only show data on the last line drawn.

\item {\it -z INTEGER $|$ --most-lines=INTEGER}\verb" "

Only show at most INTEGER connections.

\item {\it -T $|$ --input-is-tshark-out}\verb" "

The input file is already processed by \cmd{tshark}.

\end{description}

{\bf Graphical Options:}

These options determine fields included in \cmd{dnspktflow}'s output.

\begin{description}

\item {\it -t $|$ --show-type}\verb" "

Shows message type in result image.

\item {\it -q $|$ --show-queries}\verb" "

Shows query questions in result image.

\item {\it -a $|$ --show-answers}\verb" "

Shows query answers in result image.

\item {\it -A $|$ --show-authoritative}\verb" "

Shows authoritative information in result image.

\item {\it -x $|$ --show-additional}\verb" "

Shows additional information in result image.

\item {\it -l $|$ --show-label-lines}\verb" "

Shows lines attaching labels to lines.

\item {\it --fontsize=INTEGER}\verb" "

Font Size

\end{description}

{\bf Debugging:}

These options may assist in debugging \cmd{dnspktflow}.

\begin{description}

\item {\it -d $|$ --dump-pkts}\verb" "

Dump data collected from the packets.

\item {\it -h $|$ --help}\verb" "

Show help for command line options.

\end{description}

{\bf SEE ALSO}

\perlmod{Getopt::GUI::Long(3)},
\perlmod{Net::DNS(3)},
\perlmod{QWizard.pm(3)}
