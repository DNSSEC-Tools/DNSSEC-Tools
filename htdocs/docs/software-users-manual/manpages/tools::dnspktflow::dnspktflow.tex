\clearpage

\subsubsection{dnspktflow}

{\bf NAME}

\cmd{dnspktflow} - Analyze and draw DNS flow diagrams from a \path{tcpdump} file

{\bf SYNOPSIS}

\begin{verbatim}
    dnspktflow -o output.png file.tcpdump

    dnspktflow -o output.png -x -a -t -q file.tcpdump
\end{verbatim}

{\bf DESCRIPTION}

The \cmd{dnspktflow} application takes a \cmd{tcpdump} network traffic dump
file, passes it through the \cmd{tshark} application and then displays the
resulting DNS packet flows in a ``flow-diagram'' image.  \cmd{dnspktflow}
can output a single image or a series of images which can then be
shown in sequence as an animation.

\cmd{dnspktflow} was written as a debugging utility to help trace DNS
queries and responses, especially as they apply to DNSSEC-enabled lookups.

{\bf REQUIREMENTS}

This application requires the following Perl modules and software
components to work:

\begin{verbatim}
     graphviz                  http://www.graphviz.org/
     GraphViz                  Perl module
     tshark                    http://www.wireshark.org/
\end{verbatim}

The following is required for outputting screen presentations:

\begin{verbatim}
    MagicPoint                 http://member.wide.ad.jp/wg/mgp/
\end{verbatim}

If the following modules are installed, a GUI interface will be enabled for
communication with \cmd{dnspktflow}:

\begin{verbatim}
     QWizard                   Perl module
     Getopt::GUI::Long         Perl module
\end{verbatim}

{\bf OPTIONS}

\cmd{dnspktflow} takes a wide variety of command-line options.  These options
are described below in the following functional groups:  input packet
selection, output file options, output visualization options, graphical
options, and debugging.

{\bf Input Packet Selection}

These options determine the packets that will be selected by \cmd{dnspktflow}.
Short versions of the options are given in parentheses.

\begin{description}

\item {\bf --ignore-hosts=STRING (-i)}\verb" "

A regular expression of host names to ignore in the query/response fields.

\item {\bf --only-hosts=STRING (-r)}\verb" "

A regular expression of host names to analyze in the query/response fields.

\item {\bf --show-frame-num (-f)}\verb" "

Display the packet frame numbers.

\item {\bf --begin-frame=INTEGER (-b)}\verb" "

Begin at packet frame NUMBER.

\end{description}

{\bf Output File Options}

These options determine the type and location of \cmd{dnspktflow}'s output.

\begin{description}

\item {\bf --output-file=STRING (-o)}\verb" "

Output file name (default: out%03d.png as PNG format.)

\item {\bf --fig}\verb" "

Output format should be fig.

\item {\bf --tshark-out=STRING (-O)}\verb" "

Save \cmd{tshark} output to this file.

\item {\bf --multiple-outputs (-m)}\verb" "

One picture per request (use %03d in the filename.)

\item {\bf --magic-point=STRING (-M)}\verb" "

Saves a MagicPoint presentation for the output.

\end{description}

{\bf Output Visualization Options:}

These options determine specifics of \cmd{dnspktflow}'s output.

\begin{description}

\item {\bf --last-line-labels-only (-L)}\verb" "

Only show data on the last line drawn.

\item {\bf --most-lines=INTEGER (-z)}\verb" "

Only show at most INTEGER connections.

\item {\bf --input-is-tshark-out (-T)}\verb" "

The input file is already processed by \cmd{tshark}.

\end{description}

{\bf Graphical Options:}

These options determine fields included in \cmd{dnspktflow}'s output.

\begin{description}

\item {\bf --show-type (-t)}\verb" "

Shows message type in result image.

\item {\bf --show-queries (-q)}\verb" "

Shows query questions in result image.

\item {\bf --show-answers (-a)}\verb" "

Shows query answers in result image.

\item {\bf --show-authoritative (-A)}\verb" "

Shows authoritative information in result image.

\item {\bf --show-additional (-x)}\verb" "

Shows additional information in result image.

\item {\bf --show-label-lines (-l)}\verb" "

Shows lines attaching labels to lines.

\item {\bf --fontsize=INTEGER}\verb" "

Font Size

\end{description}

{\bf Debugging:}

These options may assist in debugging \cmd{dnspktflow}.

\begin{description}

\item {\bf --dump-pkts (-d)}\verb" "

Dump data collected from the packets.

\item {\bf --help (-h)}\verb" "

Show help for command line options.

\end{description}

{\bf SEE ALSO}

Getopt::GUI::Long(3)
Net::DNS(3)
QWizard.pm(3)

