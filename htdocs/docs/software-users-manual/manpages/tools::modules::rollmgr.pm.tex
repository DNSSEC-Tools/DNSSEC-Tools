\clearpage

\subsection{\bf rollmgr.pm Module}

{\bf NAME}

\perlmod{Net::DNS::SEC::Tools::rollmgr.pm} - Communicate with the DNSSEC-Tools rollover manager.

{\bf SYNOPSIS}

\begin{verbatim}
    use Net::DNS::SEC::Tools::rollmgr;

    $dir = rollmgr_dir();

    $idfile = rollmgr_idfile();

    $id = rollmgr_getid();

    rollmgr_dropid();

    rollmgr_rmid();

    rollmgr_cmdint();

    rollmgr_halt();

    $curlevel = rollmgr_loglevel();
    $oldlevel = rollmgr_loglevel("info");
    $oldlevel = rollmgr_loglevel(LOG_ERR,1);

    $curlogfile = rollmgr_logfile();
    $oldlogfile = rollmgr_logfile("-");
    $oldlogfile = rollmgr_logfile("/var/log/roll.log",1);

    $loglevelstr = rollmgr_logstr(8)
    $loglevelstr = rollmgr_logstr("info")

    rollmgr_log(LOG_INFO,"example.com","zone is valid");

    rollmgr_channel(1);
    ($cmd,$data) = rollmgr_getcmd();
    $ret = rollmgr_verifycmd($cmd);

    rollmgr_sendcmd(CHANNEL_CLOSE,ROLLCMD_ROLLZONE,"example.com");

    rollmgr_sendcmd(CHANNEL_WAIT,ROLLCMD_ROLLZONE,"example.com");
    ($retcode, $respmsg) = rollmgr_getresp();
\end{verbatim}

\clearpage
{\bf DESCRIPTION}

The \perlmod{Net::DNS::SEC::Tools::rollmgr} module provides standard,
platform-independent methods for a program to communicate with DNSSEC-Tools'
\cmd{rollerd} rollover manager.  There are three interface classes described
here:  general interfaces, logging interfaces, and communications interfaces.

{\bf GENERAL INTERFACES}

The interfaces to the \perlmod{Net::DNS::SEC::Tools::rollmgr} module are given
below.

\begin{description}

\item \func{rollmgr\_dir()}\verb" "

This routine returns \cmd{rollerd}'s directory.

\item \func{rollmgr\_idfile()}\verb" "

This routine returns \cmd{rollerd}'s id file.

\item \func{rollmgr\_getid()}\verb" "

This routine returns \cmd{rollerd}'s process id.  If a non-zero value
is passed as an argument, the id file will be left open and accessible through
the PIDFILE file handle.  See the WARNINGS section below.

Return Values:

\begin{verbatim}
    On success, the first portion of the file contents
        (up to 80 characters) is returned.
    -1 is returned if the id file does not exist.
\end{verbatim}

\item \func{rollmgr\_dropid()}\verb" "

This interface ensures that another instance of \cmd{rollerd} is not
running and then creates a id file for future reference.

Return Values:

\begin{verbatim}
    1 - the id file was successfully created for this process
    0 - another process is already acting as rollerd
\end{verbatim}

\item \func{rollmgr\_rmid()}\verb" "

This interface deletes \cmd{rollerd}'s id file.

Return Values:

\begin{verbatim}
    1 - the id file was successfully deleted
    0 - no id file exists
   -1 - the calling process is not rollerd
   -2 - unable to delete the id file
\end{verbatim}

\item \func{rollmgr\_cmdint()}\verb" "

This routine informs \cmd{rollerd} that a command has been sent via
\func{rollmgr\_sendcmd()}.

Return Values:

\begin{verbatim}
    -1 - an invalid process id was found for rollerd
    Anything else indicates the number of processes that were signaled.
    (This should only ever be 1.)
\end{verbatim}

\item \func{rollmgr\_halt()}\verb" "

This routine informs \cmd{rollerd} to shut down.

In the current implementation, the return code from the {\bf kill()} command is
returned.

\begin{verbatim}
    -1 - an invalid process id was found for rollerd
    Anything else indicates the number of processes that were signaled.
    (This should only ever be 1.)
\end{verbatim}

\end{description}

{\bf LOGGING INTERFACES}

\begin{description}

\item \func{rollmgr\_loglevel(newlevel,useflag)}\verb" "

This routine sets and retrieves the logging level for \cmd{rollerd}.
The {\it newlevel} argument specifies the new logging level to be set.  The
valid levels are:

\begin{table}[ht]
\begin{center}
\begin{tabular}{lcl}
text & numeric & meaning	\\
\hline				\\
tmi       & 1 & The highest level -- all log messages are saved.	\\
expire    & 3 & A verbose countdown of zone expiration is given.	\\
info      & 4 & Many informational messages are recorded.		\\
curphase  & 6 & Each zone's current rollover phase is given.		\\
err       & 8 & Errors are recorded.					\\
fatal     & 9 & Fatal errors are saved.					\\
\end{tabular} 
\end{center}
\end{table}

{\it newlevel} may be given in either text or numeric form.  The levels
include all numerically higher levels.  For example, if the log level is set
to {\bf curphase}, then {\bf err} and {\bf fatal} messages will also be
recorded.

The {\it useflag} argument is a boolean that indicates whether or not to give
a descriptive message and exit if an invalid logging level is given.  If {\it
useflag} is true, the message is given and the process exits; if false, -1 is
returned.

If given with no arguments, the current logging level is returned.  In fact,
the current level is always returned unless an error is found.  -1 is returned
on error.

\item \func{rollmgr\_logfile(newfile,useflag)}\verb" "

This routine sets and retrieves the log file for \cmd{rollerd}.  The {\it
newfile} argument specifies the new log file to be set.  If {\it newfile}
exists, it must be a regular file.

The {\it useflag} argument is a boolean that indicates whether or not to give
a descriptive message if an invalid logging level is given.  If {\it useflag}
is true, the message is given and the process exits; if false, no message is
given.  For any error condition, an empty string is returned.

\item \func{rollmgr\_logstr(loglevel)}\verb" "

This routine translates a log level (given in {\it loglevel}) into the
associated text log level.  The text log level is returned to the caller.

If {\it loglevel} is a text string, it is checked to ensure it is a valid log
level.  Case is irrelevant when checking {\it loglevel}.

If {\it loglevel} is numeric, it is must be in the valid range of log levels.
{\it undef} is returned if {\it loglevel} is invalid.

\item \func{rollmgr\_log(level,group,message)}\verb" "

The \func{rollmgr\_log()} interface writes a message to the log file.  Log
messages have this format:

\begin{verbatim}        timestamp: group: message\end{verbatim}

The {\it level} argument is the message's logging level.  It will only be
written to the log file if the current log level is numerically equal to or
less than {\it level}.

{\it group} allows messages to be associated together.  It is currently used
by \cmd{rollerd} to group messages by the zone to which the message applies.

The {\it message} argument is the log message itself.  Trailing newlines are
removed.

\end{description}

{\bf ROLLERD COMMUNICATIONS INTERFACES}

\begin{description}

\item \func{rollmgr\_channel(serverflag)}\verb" "

This interface sets up a persistent channel for communications with
\cmd{rollerd}.  If {\it serverflag} is true, then the server's side of the
channel is created.  If {\it serverflag} is false, then the client's side of
the channel is created.

Currently, the connection may only be made to the localhost.  This may be
changed to allow remote connections, if this is found to be needed.

\item \func{rollmgr\_getcmd()}\verb" "

\func{rollmgr\_getcmd()} retrieves a command sent over \cmd{rollerd}'s
communications channel by a client program.  The command and the command's
data are sent in each message.

The command and the command's data are returned to the caller.

\item \func{rollmgr\_sendcmd(closeflag,cmd,data)}\verb" "

\func{rollmgr\_sendcmd()} sends a command to \cmd{rollerd}.  The command must
be one of the commands from the table below.  This interface creates a
communications channel to \cmd{rollerd} and sends the message.  The channel is
not closed, in case the caller wants to receive a response from \cmd{rollerd}.

The available commands and their required data are:

\begin{table}[ht]
\begin{center}
\begin{tabular}{lll}

command & data & purpose	\\
\hline				\\
\const{ROLLCMD\_DISPLAY}   & 1/0          & start/stop \cmd{rollerd}'s graphical display \\
\const{ROLLCMD\_LOGFILE}   & log-file     & set \cmd{rollerd}'s log filename \\
\const{ROLLCMD\_LOGLEVEL}  & log-level    & set \cmd{rollerd}'s logging level \\
\const{ROLLCMD\_ROLLALL}   & none         & force all zones to start rollover \\
\const{ROLLCMD\_ROLLREC}   & {\it rollrec}-name & change \cmd{rollerd}'s {\it rollrec} file \\
\const{ROLLCMD\_ROLLZONE}  & zone-name    & force a zone to start rollover \\
\const{ROLLCMD\_RUNQUEUE}  & none     & \cmd{rollerd} runs through its queue \\
\const{ROLLCMD\_SHUTDOWN}  & none         & stop \cmd{rollerd} \\
\const{ROLLCMD\_SLEEPTIME} & seconds-count& set \cmd{rollerd}'s sleep time \\
\const{ROLLCMD\_STATUS}    & none         & get \cmd{rollerd}'s status \\
\end{tabular} 
\end{center}
\end{table}

The data aren't checked for validity by \func{rollmgr\_sendcmd()}; validity
checking is a responsibility of \cmd{rollerd}.

If the caller does not need a response from \cmd{rollerd}, then {\it
closeflag} should be set to \const{CHANNEL\_CLOSE}; if a response is required
then {\it closeflag} should be \const{CHANNEL\_WAIT}.  These values are
boolean values, and the constants aren't required.

On success, 1 is returned.  If an invalid command is given, 0 is returned.

\item \func{rollmgr\_getresp()}\verb" "

After executing a client command sent via \func{rollmgr\_sendcmd()},
\cmd{rollerd} will send a response to the client.  \func{rollmgr\_getresp()}
allows the client to retrieve the response.

A return code and a response string are returned, in that order.  Both are
specific to the command sent.

\item \func{rollmgr\_verifycmd(cmd)}\verb" "

\func{rollmgr\_verifycmd()} verifies that {\it cmd} is a valid command for
\cmd{rollerd}.  1 is returned for a valid command; 0 is returned for an
invalid command.

\end{description}

{\bf WARNINGS}

\begin{enumerate}

\item \func{rollmgr\_getid()} attempts to exclusively lock the id file.
Set a timer if this matters to you.

\item \func{rollmgr\_getid()} has a nice little race condition.  We should lock
the file prior to opening it, but we can't do so without it being open.

\end{enumerate}

{\bf SEE ALSO}

\cmd{rollctl(1)},
\cmd{rollerd(8)}

\perlmod{Net::DNS::SEC::Tools::keyrec.pm(3)},
\perlmod{Net::DNS::SEC::Tools::rollrec.pm(3)}
