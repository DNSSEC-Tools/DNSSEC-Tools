\clearpage

\subsubsection{rollmgr.pm}

{\bf NAME}

\perlmod{Net::DNS::SEC::Tools::rollmgr} - Communicate with the DNSSEC-Tools
rollover manager.

{\bf SYNOPSIS}

\begin{verbatim}
  use Net::DNS::SEC::Tools::rollmgr;

  $dir = rollmgr_dir();

  $idfile = rollmgr_idfile();

  $id = rollmgr_getid();

  rollmgr_dropid();

  rollmgr_rmid();

  rollmgr_cmdint();

  rollmgr_halt();

  rollmgr_channel(1);
  ($cmd,$data) = rollmgr_getcmd();
  $ret = rollmgr_verifycmd($cmd);

  rollmgr_sendcmd(CHANNEL_CLOSE,ROLLCMD_ROLLZONE,"example.com");

  rollmgr_sendcmd(CHANNEL_WAIT,ROLLCMD_ROLLZONE,"example.com");
  ($retcode, $respmsg) = rollmgr_getresp();
\end{verbatim}

{\bf DESCRIPTION}

The \perlmod{Net::DNS::SEC::Tools::rollmgr} module provides standard,
platform-independent methods for a program to communicate with DNSSEC-Tools'
\cmd{rollerd} rollover manager.  There are two interface classes described
here:  general interfaces and communications interfaces.

{\bf GENERAL INTERFACES}

The interfaces to the \perlmod{Net::DNS::SEC::Tools::rollmgr} module are given
below.

\begin{description}

\item \cmd{rollmgr\_dir()}\verb" "

This routine returns \cmd{rollerd}'s directory.

\item \cmd{rollmgr\_idfile()}\verb" "

This routine returns \cmd{rollerd}'s id file.

\item \cmd{rollmgr\_getid()}\verb" "

This routine returns \cmd{rollerd}'s process id.  If a non-zero value is
passed as an argument, the id file will be left open and accessible through
the PIDFILE file handle.  See the WARNINGS section below.

Return Values:

\begin{description}
\item On success, the first portion of the file contents (up to 80 characters)
is returned.
\item -1 is returned if the id file does not exist.
\end{description}

\item \cmd{rollmgr\_dropid()}\verb" "

This interface ensures that another instance of \cmd{rollerd} is not
running and then creates a id file for future reference.

Return Values:

\begin{description}
\item 1 - the id file was successfully created for this process
\item 0 - another process is already acting as \cmd{rollerd}
\end{description}

\item \cmd{rollmgr\_rmid()}\verb" "

This interface deletes \cmd{rollerd}'s id file.

Return Values:

\begin{description}
\item  1 - the id file was successfully deleted
\item  0 - no id file exists
\item -1 - the calling process is not \cmd{rollerd}
\item -2 - unable to delete the id file
\end{description}

\item \cmd{rollmgr\_cmdint()}\verb" "

This routine informs \cmd{rollerd} that a command has been sent via
\cmd{rollmgr\_sendcmd()}.

Return Values:

\begin{description}
\item -1 - an invalid process id was found for \cmd{rollerd}
\item Anything else indicates the number of processes that were signaled.\\
(This should only ever be 1.)
\end{description}

\item \cmd{rollmgr\_halt()}\verb" "

This routine informs \cmd{rollerd} to shut down.

In the current implementation, the return code from the \func{kill()} command
is returned.

\begin{description}
\item -1 - an invalid process id was found for \cmd{rollerd}
\item Anything else indicates the number of processes that were signaled.\\
(This should only ever be 1.)
\end{description}

\end{description}

{\bf ROLLERD COMMUNICATIONS INTERFACES}

\begin{description}

\item \cmd{rollmgr\_channel(serverflag)}\verb" "

This interface sets up a persistent channel for communications with
\cmd{rollerd}.  If \var{serverflag} is true, then the server's side of the
channel is created.  If \var{serverflag} is false, then the client's side of
the channel is created.

Currently, the connection may only be made to the localhost.  This may be
changed to allow remote connections, if this is found to be needed.

\item \cmd{rollmgr\_getcmd()}\verb" "

\cmd{rollmgr\_getcmd()} retrieves a command sent over \cmd{rollerd}'s
communications channel by a client program.  The command and the command's
data are sent in each message.

The command and the command's data are returned to the caller.

\item \cmd{rollmgr\_sendcmd(closeflag,cmd,data)}\verb" "

\cmd{rollmgr\_sendcmd()} sends a command to \cmd{rollerd}.  The command must
be one of the commands from the table below.  This interface creates a
communications channel to \cmd{rollerd} and sends the message.  The channel is
not closed, in case the caller wants to receive a response from \cmd{rollerd}.

The available commands and their required data are:

\begin{table}[h]
\begin{center}
\begin{tabular}{|l|c|l|}
\hline
{\bf Command} & {\bf Data} & {\bf Purpose} \\
\hline
\const{ROLLCMD\_DISPLAY}   & 1/0           & start/stop \cmd{rollerd}'s graphical display \\
\const{ROLLCMD\_DSPUB}     & zone-name     & a DS record has been published \\
\const{ROLLCMD\_DSPUBALL}  & none          & DS records published for all zones \\
                           &               & in KSK rollover phase 6 \\
\const{ROLLCMD\_ROLLALL}   & none          & force all zones to start ZSK rollover \\
\const{ROLLCMD\_ROLLKSK}   & zone-name     & force a zone to start KSK rollover \\
\const{ROLLCMD\_ROLLREC}   & rollrec-name  & change \cmd{rollerd}'s \struct{rollrec} file \\
\const{ROLLCMD\_ROLLZONE}  & zone-name     & force a zone to start ZSK rollover \\
\const{ROLLCMD\_RUNQUEUE}  & none          & \cmd{rollerd} runs through its queue \\
\const{ROLLCMD\_SHUTDOWN}  & none          & stop \cmd{rollerd} \\
\const{ROLLCMD\_SLEEPTIME} & seconds-count & set \cmd{rollerd}'s sleep time \\
\const{ROLLCMD\_STATUS}    & none          & get \cmd{rollerd}'s status \\
\hline
\end{tabular}
\end{center}
\caption{\cmd{rollerd} Commands}
\end{table}

The data aren't checked for validity by \cmd{rollmgr\_sendcmd()}; validity
checking is a responsibility of \cmd{rollerd}.

If the caller does not need a response from \cmd{rollerd}, then
\var{closeflag} should be set to \const{CHANNEL\_CLOSE}; if a response is
required then \var{closeflag} should be \const{CHANNEL\_WAIT}.  These values
are boolean values, and the constants aren't required.

Return Values:

\begin{description}
\item 1 is returned on success.
\item 0 is returned if an invalid command is given.
\end{description}

\item \cmd{rollmgr\_getresp()}\verb" "

After executing a client command sent via \cmd{rollmgr\_sendcmd()},
\cmd{rollerd} will send a response to the client.  \cmd{rollmgr\_getresp()}
allows the client to retrieve the response.

A return code and a response string are returned, in that order.  Both are
specific to the command sent.

\item \cmd{rollmgr\_verifycmd(cmd)}\verb" "

\cmd{rollmgr\_verifycmd()} verifies that \var{cmd} is a valid command for
\cmd{rollerd}.

Return Values:

\begin{description}
\item 1 is returned for a valid command.
\item 0 is returned for an invalid command.
\end{description}

\end{description}

{\bf WARNINGS}

1.  \cmd{rollmgr\_getid()} attempts to exclusively lock the id file.
Set a timer if this matters to you.

2.  \cmd{rollmgr\_getid()} has a nice little race condition.  We should lock
the file prior to opening it, but we can't do so without it being open.

{\bf SEE ALSO}

rollctl(1)

Net::DNS::SEC::Tools::keyrec.pm(3),
Net::DNS::SEC::Tools::rolllog.pm(3), \\
Net::DNS::SEC::Tools::rollrec.pm(3)

rollerd(8)

