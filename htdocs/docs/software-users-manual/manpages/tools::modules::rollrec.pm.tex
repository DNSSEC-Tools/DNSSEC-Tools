\clearpage

\subsubsection{rollrec.pm}

{\bf NAME}

\perlmod{Net::DNS::SEC::Tools::rollrec} - Manipulate a DNSSEC-Tools
\struct{rollrec} file.

{\bf SYNOPSIS}

\begin{verbatim}
  use Net::DNS::SEC::Tools::rollrec;

  rollrec_lock();
  rollrec_read("localhost.rollrec");

  @rrnames = rollrec_names();

  $flag = rollrec_exists("example.com");

  $rrec = rollrec_fullrec("example.com");
  %rrhash = %$rrec;
  $zname = $rrhash{"maxttl"};

  $val = rollrec_recval("example.com","zonefile");

  rollrec_add("roll","example.com",\%rollfields);
  rollrec_add("skip","example.com",\%rollfields);

  rollrec_del("example.com");

  rollrec_type("example.com","roll");
  rollrec_type("example.com","skip");

  rollrec_setval("example.com","zonefile","db.example.com");

  rollrec_delfield("example.com","directory");

  rollrec_settime("example.com");
  rollrec_settime("example.com",0);

  @rollrecfields = rollrec_fields();

  $default_file = rollrec_default();

  rollrec_write();
  rollrec_close();
  rollrec_discard();

  rollrec_unlock();
\end{verbatim}

{\bf DESCRIPTION}

The \perlmod{Net::DNS::SEC::Tools::rollrec} module manipulates the contents of
a DNSSEC-Tools \struct{rollrec} file.  \struct{rollrec} files describe the
status of a zone rollover process, as performed by the DNSSEC-Tools programs.
Module interfaces exist for looking up \struct{rollrec} records, creating new
records, and modifying existing records.

A \struct{rollrec} file is organized in sets of \struct{rollrec} records.
\struct{rollrec}s describe the state of a rollover operation.  A
\struct{rollrec} consists of a set of keyword/value entries.  The following is
an example of a \struct{rollrec}:

\begin{verbatim}
    roll "example.com"
        zonefile           "/dnssec-tools/zones/db.example.com"
        keyrec             "/dnssec-tools/keyrec/example.keyrec"
        directory          "/dnssec-tools/dir-example.com"
        kskphase           "0"
        zskphase           "2"
        maxttl             "86400"
        administrator      "bob@bobhost.example.com"
        phasestart         "Wed Mar 09 21:49:22 2005"
        display            "0"
        loglevel           "info"
        rollrec_rollsecs   "1115923362"
        rollrec_rolldate   "Tue Mar 09 19:12:54 2005"
\end{verbatim}

The first step in using this module must be to read the \struct{rollrec} file.
The \func{rollrec\_read()} interface reads the file and parses it into an
internal format.  The file's records are copied into a hash table (for easy
reference by the \perlmod{rollrec.pm} routines) and in an array (for
preserving formatting and comments.)

After the file has been read, the contents are referenced using
\func{rollrec\_fullrec()} and \func{rollrec\_recval()}.  The
\func{rollrec\_add()}, \func{rollrec\_setval()}, and \func{rollrec\_settime()}
interfaces are used to modify the contents of a \struct{rollrec} record.

If the \struct{rollrec} file has been modified, it must be explicitly written
or the changes will not saved.  \func{rollrec\_write()} saves the new contents
to disk.  \func{rollrec\_close()} saves the file and close the Perl file
handle to the \struct{rollrec} file.  If a \struct{rollrec} file is no longer
wanted to be open, yet the contents should not be saved,
\func{rollrec\_discard()} gets rid of the data closes and the file handle {\bf
without} saving any modified data.

{\bf ROLLREC LOCKING}

This module includes interfaces for synchronizing access to the
\struct{rollrec} files.  This synchronization is very simple and relies upon
locking and unlocking a single lock file for all \struct{rollrec} files.

\struct{rollrec} locking is not required before using this module, but it is
recommended.  The expected use of these facilities follows:

\begin{verbatim}
    rollrec_lock() || die "unable to lock rollrec file\n";
    rollrec_read();
    ... perform other rollrec operations ...
    rollrec_close();
    rollrec_unlock();
\end{verbatim}

Synchronization is performed in this manner due to the way the module's
functionality is implemented, as well as providing flexibility to users of the
module.  It also provides a clear delineation in callers' code as to where and
when \struct{rollrec} locking is performed.

This synchronization method has the disadvantage of having a single lockfile
as a bottleneck to all \struct{rollrec} file access.  However, it reduces
complexity in the locking interfaces and cuts back on the potential number of
required lockfiles.

Using a single synchronization file may not be practical in large
installations.  If that is found to be the case, then this will be reworked.

{\bf ROLLREC INTERFACES}

The interfaces to the \perlmod{rollrec.pm} module are given below.

\begin{description}

\item \func{rollrec\_add(rollrec\_type,rollrec\_name,fields)}\verb" "

This routine adds a new \struct{rollrec} to the \struct{rollrec} file and the
internal representation of the file contents.  The \struct{rollrec} is added
to both the \var{\%rollrecs} hash table and the \var{\@rollreclines} array.
Entries are only added if they are defined for \struct{rollrec}s.

\var{rollrec\_type} is the type of the \struct{rollrec}.  This must be either
``roll'' or ``skip''.  \var{rollrec\_name} is the name of the \struct{rollrec}.
{\it fields} is a reference to a hash table that contains the name/value
\struct{rollrec} fields.  The keys of the hash table are always converted to
lowercase, but the entry values are left as given.

Timestamp fields are added at the end of the \struct{rollrec}.  These fields
have the key values \var{rollrec\_gensecs} and \var{rollrec\_gendate}.

A blank line is added after the final line of the new \struct{rollrec}.  The
\struct{rollrec} file is not written after \func{rollrec\_add()}, though it is
internally marked as having been modified.

\item \func{rollrec\_del(rollrec\_name)}\verb" "

This routine deletes a \struct{rollrec} from the \struct{rollrec} file and the
internal representation of the file contents.  The \struct{rollrec} is deleted
from both the \var{\%rollrecs} hash table and the \var{\@rollreclines} array.

Only the \struct{rollrec} itself is deleted from the file.  Any associated
comments and blank lines surrounding it are left intact.  The \struct{rollrec}
file is not written after \func{rollrec\_del()}, though it is internally
marked as having been modified.

Return values are:

\begin{description}
\item 0 successful rollrec deletion
\item -1 unknown name
\end{description}

\item \func{rollrec\_close()}\verb" "

This interface saves the internal version of the \struct{rollrec} file (opened
with \func{rollrec\_read()}) and closes the file handle.

\item \func{rollrec\_delfield(rollrec\_name,field)}\verb" "

Deletes the given field from the specified \struct{rollrec}.  The file is {\bf
not} written after updating the value, but the internal file-modified flag is
set.  The value is saved in both \var{\%rollrecs} and in \var{\@rollreclines}.

Return values:

\begin{description}
\item 0 - failure (rollrec not found or rollrec does not contain the field)
\item 1 - success
\end{description}

\item \func{rollrec\_discard()}\verb" "

This routine removes a \struct{rollrec} file from use by a program.  The
internally stored data are deleted and the \struct{rollrec} file handle is
closed.  However, modified data are not saved prior to closing the file
handle.  Thus, modified and new data will be lost.

\item \func{rollrec\_exists(rollrec\_name)}\verb" "

This routine returns a boolean flag indicating if the \struct{rollrec} named
in \var{rollrec\_name} exists.

\item \func{rollrec\_fullrec(rollrec\_name)}\verb" "

\func{rollrec\_fullrec()} returns a reference to the \struct{rollrec}
specified in \var{rollrec\_name}.

\item \func{rollrec\_lock()}\verb" "

\func{rollrec\_lock()} locks the \struct{rollrec} lockfile.  An exclusive lock
is requested, so the execution will suspend until the lock is available.  If
the \struct{rollrec} synchronization file does not exist, it will be created.
If the process can't create the synchronization file, an error will be
returned.  Success or failure is returned.

\item \func{rollrec\_names()}\verb" "

This routine returns a list of the \struct{rollrec} names from the file.

\item \func{rollrec\_read(rollrec\_file)}\verb" "

This interface reads the specified \struct{rollrec} file and parses it into a
\struct{rollrec} hash table and a file contents array.  \func{rollrec\_read()}
{\bf must} be called prior to any of the other \perlmod{rollrec.pm} calls.  If
another \struct{rollrec} is already open, then it is saved and closed prior to
opening the new \struct{rollrec}.

\func{rollrec\_read()} attempts to open the \struct{rollrec} file for reading
and writing.  If this fails, then it attempts to open the file for reading
only.

Upon success, \func{rollrec\_read()} returns the number of \struct{rollrec}s
read from the file.

Failure return values:

\begin{description}
\item -1 specified rollrec file doesn't exit
\item -2 unable to open rollrec file
\item -3 duplicate rollrec names in file
\end{description}

\item \func{rollrec\_rectype(rollrec\_name,rectype)}\verb" "

Set the type of the specified \struct{rollrec} record.  The file is {\bf not}
written after updating the value, but the internal file-modified flag is set.
The value is saved in both \var{\%rollrecs} and in \var{\@rollreclines}.

\var{rollrec\_name} is the name of the \struct{rollrec} that will be modified.
{\it rectype} is the new type of the \struct{rollrec}, which must be either
``roll'' or ``skip''.

Return values:

\begin{description}
\item 0 - failure (invalid record type or rollrec not found)
\item 1 - success
\end{description}

\item \func{rollrec\_recval(rollrec\_name,rollrec\_field)}\verb" "

This routine returns the value of a specified field in a given
\struct{rollrec}.  \var{rollrec\_name} is the name of the particular
\struct{rollrec} to consult.  \var{rollrec\_field} is the field name within
that \struct{rollrec}.

For example, the current \struct{rollrec} file contains the following
\struct{rollrec}.

\begin{verbatim}
    roll        "example.com"
                zonefile        "db.example.com"
\end{verbatim}

The call:

\begin{verbatim}
    rollrec_recval("example.com","zonefile")
\end{verbatim}

will return the value ``db.example.com''.

\item \func{rollrec\_settime(rollrec\_name,val)}\verb" "

Set the phase-start timestamp in the \struct{rollrec} specified by
\var{rollrec\_name} to the current time.  If the optional \var{val} parameter
is given and it is zero, then the phase-start timestamp is set to a null
value.

The file is {\bf not} written after updating the value.

\item \func{rollrec\_setval(rollrec\_name,field,value)}\verb" "

Set the value of a name/field pair in a specified \struct{rollrec}.  The file
is {\bf not} written after updating the value, but the internal file-modified
flag is set.  The value is saved in both \var{\%rollrecs} and in
\var{\@rollreclines}.

\var{rollrec\_name} is the name of the \struct{rollrec} that will be modified.
If the named \struct{rollrec} does not exist, it will be created as a
``roll''-type \struct{rollrec}.  \var{field} is the \struct{rollrec} field which
will be modified.  \var{value} is the new value for the field.

\item \func{rollrec\_unlock()}\verb" "

\func{rollrec\_unlock()} unlocks the \struct{rollrec} synchronization file.

\item \func{rollrec\_write()}\verb" "

This interface saves the internal version of the \struct{rollrec} file (opened
with \func{rollrec\_read()}).  It does not close the file handle.  As an
efficiency measure, an internal modification flag is checked prior to writing
the file.  If the program has not modified the contents of the
\struct{rollrec} file, it is not rewritten.

\end{description}

{\bf ROLLREC INTERNAL INTERFACES}

The interfaces described in this section are intended for internal use by the
\perlmod{rollrec.pm} module.  However, there are situations where external
entities may have need of them.  Use with caution, as misuse may result in
damaged or lost \struct{rollrec} files.

\begin{description}

\item \func{rollrec\_init()}\verb" "

This routine initializes the internal \struct{rollrec} data.  Pending changes
will be lost.  An open \struct{rollrec} file handle will remain open, though
the data are no longer held internally.  A new \struct{rollrec} file must be
read in order to use the \perlmod{rollrec.pm} interfaces again.

\item \func{rollrec\_newrec(type,name)}\verb" "

This interface creates a new \struct{rollrec}.  The \var{rollrec\_name} field
in the \struct{rollrec} is set to the values of the {\it name} parameter.  The
{\it type} parameter must be either ``roll'' or ``skip''.

\item \func{rollrec\_default()}\verb" "

This routine returns the name of the default \struct{rollrec} file.

\end{description}

{\bf ROLLREC DEBUGGING INTERFACES}

The following interfaces display information about the currently parsed
\struct{rollrec} file.  They are intended to be used for debugging and
testing, but may be useful at other times.

\begin{description}

\item \func{rollrec\_dump\_hash()}\verb" "

This routine prints the \struct{rollrec} file as it is stored internally in a
hash table.  The \struct{rollrec}s are printed in alphabetical order, with the
fields alphabetized for each \struct{rollrec}.  New \struct{rollrec}s and
\struct{rollrec} fields are alphabetized along with current \struct{rollrec}s
and fields.  Comments from the \struct{rollrec} file are not included with the
hash table.

\item \func{rollrec\_dump\_array()}\verb" "

This routine prints the \struct{rollrec} file as it is stored internally in an
array.  The \struct{rollrec}s are printed in the order given in the file, with
the fields ordered in the same manner.  New \struct{rollrec}s are appended to
the end of the array.  \struct{rollrec} fields added to existing
\struct{rollrec}s are added at the beginning of the \struct{rollrec} entry.
Comments and vertical whitespace are preserved as given in the
\struct{rollrec} file.

\end{description}

{\bf SEE ALSO}

lsroll(1)

rollchk(8),
rollinit(8)

Net::DNS::SEC::Tools::keyrec(3),
Net::DNS::SEC::Tools::keyrec(5)

