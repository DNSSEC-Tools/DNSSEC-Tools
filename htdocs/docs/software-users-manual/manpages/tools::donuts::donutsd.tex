\clearpage

\subsubsection{donutsd}

{\bf NAME}

\cmd{donutsd} - Run the \cmd{donuts} syntax checker periodically and report
the results to an administrator

{\bf SYNOPSIS}

\begin{verbatim}
    donutsd [-z FREQ] [-t TMPDIR] [-f FROM] [-s SMTPSERVER] [-a DONUTSARGS]
            [-x] [-v] [-i zonelistfile] [ZONEFILE ZONENAME ZONECONTACT]
\end{verbatim}

{\bf DESCRIPTION}

\cmd{donutsd} runs \cmd{donuts} on a set of zone files every so often (the
frequency is specified by the {\it -z} flag which defaults to 24 hours) and
watches for changes in the results.  These changes may be due to the
time-sensitive nature of DNSSEC-related records (e.g., RRSIG validity
periods) or because parent/child relationships have changed.  If any
changes have occurred in the output since the last run of \cmd{donuts} on a
particular zone file, the results are emailed to the specified zone
administrator's email address.

{\bf OPTIONS}

\begin{description}

\item -v\verb" "

Turns on more verbose output.

\item -o\verb" "

Run once and quit, as opposed to sleeping or re-running forever.

\item -a ARGUMENTS\verb" "

Passes arguments to command line arguments of \cmd{donuts} runs.

\item -z TIME\verb" "

Sleeps TIME seconds between calls to \cmd{donuts}.

\item -e ADDRESS\verb" "

Mail ADDRESS with a summary of the results from all the files.
These are the last few lines of the \cmd{donuts} output for each zone that
details the number of errors found.

\item -s SMTPSERVER\verb" "

When sending mail, send it to the SMTPSERVER specified.  The default
is {\it localhost}.

\item -f FROMADDR\verb" "

When sending mail, use FROMADDR for the From: address.

\item -x\verb" "

Send the \cmd{diff} output in the email message as well as the \cmd{donuts}
output.

\item -t TMPDIR\verb" "

Store temporary files in TMPDIR.

\item -i INPUTZONES\verb" "

See the next section details.

\end{description}

{\bf ZONE ARGUMENTS}

The rest of the arguments to \cmd{donutsd} should be triplets of the
following information:

\begin{description}

\item ZONEFILE\verb" "

The zone file to examine.

\item ZONENAME\verb" "

The zonename that file is supposed to be defining.

\item ZONECONTACT\verb" "

An email address of the zone administrator (or a comma-separated
list of addresses.)  The results will be sent to this email address.

\end{description}

Additionally, instead of listing all the zones you wish to monitor on
the command line, you can use the {\it -i} flag which specifies a
file to be read listing the TRIPLES instead.  Each line in this file
should contain one triple with white-space separating the arguments.

Example:

\begin{verbatim}
    db.zonefile1.com   zone1.com   admin@zone1.com
    db.zonefile2.com   zone2.com   admin@zone2.com,admin2@zone2.com
\end{verbatim}

For even more control, you can specify an XML file (whose name must end in
\path{.xml}) that describes the same information.  This also allows for
per-zone customization of the \cmd{donuts} arguments.  The \perlmod{XML::Smart}
Perl module must be installed in order to use this feature.

\begin{verbatim}
    <donutsd>
       <zones>
          <zone>
              <file>db.example.com</file>
              <name>example.com</name>
              <contact>admin@example.com</contact>
              <!-- this is not a signed zone therefore we'll
                   add these args so we don't display DNSSEC errors -->
              <donutsargs>-i DNSSEC</donutsargs>
          </zone>
       </zones>
    </donutsd>
\end{verbatim}

The \cmd{donutsd} tree may also contain a {\it configs} section where
command-line flags can be specified:

\begin{verbatim}
    <donutsd>
        <configs>
            <config><flag>a</flag><value>--live --level 8</value></config>
            <config><flag>e</flag><value>wes@example.com</value></config>
        </configs>
        <zones>
             ...
        </zones>
    </donutsd>
\end{verbatim}

Real command line flags will be used in preference to those specified
in the \path{.xml} file, however.

{\bf EXAMPLE}

\begin{verbatim}
    donutsd -a "--live --level 8" -f root@somewhere.com $\$
       db.example.com example.com admin@example.com
\end{verbatim}

{\bf SEE ALSO}

donuts(8)

