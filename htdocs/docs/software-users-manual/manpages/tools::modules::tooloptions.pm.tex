\clearpage

\subsection{\perlmod{tooloptions.pm}}


{\bf NAME}

Net::DNS::SEC::Tools::tooloptions - DNSSEC-Tools option routines.

{\bf SYNOPSIS}

\begin{verbatim}
  use Net::DNS::SEC::Tools::tooloptions;

  $keyrec_file = "example.keyrec";
  $keyrec_name = "Kexample.com.+005+10988";
  @specopts = ("propagate+", "waittime=i");

  $optsref = tooloptions($keyrec_file,$keyrec_name);
  %options = %$optsref;

  $optsref = tooloptions($keyrec_file,$keyrec_name,@specopts);
  %options = %$optsref;

  $optsref = tooloptions("",@specopts);
  %options = %$optsref;

  ($krfile,$krname,$optsref) = opts_krfile($keyrec_file,"");
  %options = %$optsref;

  ($krfile,$krname,$optsref) = opts_krfile("",$keyrec_name,@specopts);
  %options = %$optsref;

  ($krfile,$krname,$optsref) = opts_krfile("","");
  %options = %$optsref;

  $key_ref = opts_keykr();
  %key_kr  = %$key_ref;

  $optsref = opts_keykr($keyrec_file,$keyrec_name);
  %options = %$optsref;

  $zoneref = opts_zonekr();
  %zone_kr = %$zoneref;

  $zoneref = opts_zonekr($keyrec_file,$keyrec_name);
  %zone_kr = %$zoneref;

  opts_setcsopts(@specopts);

  opts_createkrf();

  opts_suspend();

  opts_restore();

  opts_drop();

  opts_reset();
\end{verbatim}


{\bf DESCRIPTION}

DNSSEC-Tools supports a set of options common to all the tools in the suite.
These options may have defaults set in the {\bf dnssec-tools.conf}
configuration file, in a {\it keyrec} file, from command-line options, or
from any combination of the three.  In order to enforce a common sequence of
option interpretation, all DNSSEC-Tools should use the {\bf tooloptions()}
routine to initialize its options.

The {\it keyrec\_file} argument specifies a {\it keyrec} file that will be
consulted.  The {\it keyrec} named by the {\it keyrec\_name} argument will
be loaded.  If no {\it keyrec} file should be used, then {\it keyrec\_file}
should be an empty string and the {\it keyrec\_name} parameter not included.
The {\it @specopts} array contains command-specific arguments; the arguments
must be in the format prescribed by the {\bf Getopt::Long} Perl module.

{\bf tooloptions()} combines data from these three option sources into a hash
table.  The hash table is returned to the caller, which will then use the
options as needed.

The command-line options are saved between calls, so a command may call {\bf
tooloptions()} multiple times and still have the command-line options included
in the final hash table.  This is useful for examining multiple {\it keyrec}s
in a single command.  Inclusion of command-line options may be suspended and
restored using the {\bf opts\_suspend()} and {\bf opts\_restore()} calls.
Options may be discarded entirely by calling {\bf opts\_drop()}; once dropped,
command-line options may never be restored.  Suspension, restoration, and
dropping of command-line options are only effective after the initial {\bf
tooloptions()} call.

The options sources are combined in this manner:

\begin{description}

\item [1.  {\bf dnssec-tools.conf}]\verb" "

The system-wide configuration file is read and these option values are used
as the defaults.  These options are put into a hash table, with the option
names as the hash key.

\item [2. {\it keyrec} File]\verb" "

If a {\it keyrec} file was specified, then the {\it keyrec} named by {\it
keyrec\_name} will be retrieved.  The {\it keyrec}'s fields are added to the
hash table.  Any field whose keyword matches an existing hash key will
override the existing value.

\item [3. Command-line Options]\verb" "

The command-line options, specified in {\it @specopts}, are parsed using {\bf
Getoptions()} from the {\bf Getopt::Long} Perl module.  These options are
folded into the hash table; again possibly overriding existing hash values.
The options given in {\it @specopts} must be in the format required by {\bf
Getoptions()}.

\end{description}

A reference to the hash table created in these three steps is returned to the
caller.


{\bf EXAMPLE}

{\bf dnssec-tools.conf} has these entries:

\begin{verbatim}
    ksklength      1024
    zsklength      512
\end{verbatim}

{\bf example.keyrec} has this entry:

\begin{verbatim}
    key         "Kexample.com.+005+10988"
        zsklength        "1024"
\end{verbatim}

{\it zonesigner} is executed with this command line:

\begin{verbatim}
    zonesigner -ksklength 512 -zsklength 4096 -wait 600 ...  example.com
\end{verbatim}

{\bf tooloptions("example.keyrec","Kexample.com.+005+10988",("wait=i"))}
will read each option source in turn, ending up with:
\begin{verbatim}
    I<ksklength>           512
    I<zsklength>          4096
    I<wait>                600
\end{verbatim}


{\bf TOOL OPTION ARGUMENTS}

Many of the DNSSEC-Tools option interfaces take the same set of arguments:
{\it \$keyrec\_file}, {\it \$keyrec\_name}, and {\it @csopts}.  These arguments
are used similarly by most of the interfaces; differences are noted in the
interface descriptions in the next section.

\begin{description}

\item [{\it \$keyrec\_file}] Name of the {\it keyrec} file to be searched.

\item [{\it \$keyrec\_name}] Name of the {\it keyrec} that is being sought

\item [{\it @csopts}] Command-specific options.

\end{description}

The {\it keyrec} named in {\it \$keyrec\_name} is selected from the {\it
keyrec} file given in {\it \$keyrec\_file}.  If either {\it \$keyrec\_file}
or {\it \$keyrec\_name} are given as empty strings, their values will be taken
from the {\it -krfile} and {\it -keyrec} command line options.

A set of command-specific options may be specified in {\it @csopts}.  These
options are in the format required by the {\bf Getopt::Long} Perl module.  If
{\it @csopts} is left off the call, then no command-specific options will be
included in the final option hash.  The {\it @csopts} array may be passed
directly to several interfaces or it may be saved in a call to {\it
opts\_setcsopts()}.


{\bf TOOL OPTION INTERFACES}

\begin{description}

\item [{\bf tooloptions(\$keyrec\_file,\$keyrec\_name,@csopts)}]\verb" "

This {\bf tooloptions()} call builds an option hash from the system
configuration file, a {\it keyrec}, and a set of command-specific options.
A reference to this option hash is returned to the caller.

If {\it \$keyrec\_file} is given as an empty string, then no {\it keyrec}
file will be consulted.  In this case, it is assumed that {\it \$keyrec\_name}
will be left out altogether.

If a non-existent {\it \$keyrec\_file} is given and {\bf opts\_createkrf()}
has been called, then the named {\it keyrec} file will be created.  {\it
opts\_createkrf()} must be called for each {\it keyrec} file that must be
created, as the {\bf tooloptions} {\it keyrec}-creation state is reset after
{\bf tooloptions()} has completed.

\item [{\bf opts\_krfile(\$keyrec\_file,\$keyrec\_name,@csopts)}]\verb" "

The {\bf opts\_krfile()} routine looks up the {\it keyrec} file and {\it
keyrec} name and uses those fields to help build an options hash.  References
to the {\it keyrec} file name, {\it keyrec} name, and the option hash table
are returned to the caller.

The {\it \$keyrec\_file} and {\it \$keyrec\_name} arguments are required
parameters.  They may be given as empty strings, but they {\bf must} be given.

If the {\it \$keyrec\_file} file and {\it \$keyrec\_name} name are both
specified by the caller, then this routine will have the same effect as
directly calling {\bf tooloptions()}.


\item [{\bf opts\_getkeys(\$keyrec\_file,\$keyrec\_name,@csopts)}]\verb" "

This routine returns references to the KSK and ZSK {\it keyrec}s associated
with a specified {\it keyrec} entry.  This gives an easy way to get a zone's
{\it keyrec} entries in a single step.

This routine acts as a front-end to the {\bf opts\_krfile()} routine.
Arguments to {\bf opts\_getkeys()} conform to those of {\bf opts\_krfile()}.

If {\bf opts\_getkeys()} isn't passed any arguments, it will act as if both
{\it \$keyrec\_file} and {\it \$keyrec\_name} were given as empty strings.  In
this case, their values will be taken from the {\it -krfile} and {\it -keyrec}
command line options.


\item [{\bf opts\_keykr(\$keyrec\_file,\$keyrec\_name,@csopts)}]\verb" "

This routine returns a reference to the key {\it keyrec} named by
{\it \$keyrec\_name}.  It ensures that the named {\it keyrec} is a
key {\it keyrec}; if it isn't, {\it undef} is returned.

This routine acts as a front-end to the {\bf opts\_krfile()} routine.
{\bf opts\_keykr()}'s arguments conform to those of {\bf opts\_krfile()}.

If {\bf opts\_keykr()} isn't passed any arguments, it will act as if both
{\it \$keyrec\_file} and {\it \$keyrec\_name} were given as empty strings.
In this case, their values will be taken from the {\it -krfile} and {\it
-keyrec} command line options.


\item [{\bf opts\_zonekr(\$keyrec\_file,\$keyrec\_name,@csopts)}]\verb" "

This routine returns a reference to the zone {\it keyrec} named by
{\it \$keyrec\_name}.  The {\it keyrec} fields from the zone's KSK and ZSK
are folded in as well, but the key's {\it keyrec\_} fields are excluded.
This call ensures that the named {\it keyrec} is a zone {\it keyrec};
if it isn't, {\it undef} is returned.

This routine acts as a front-end to the {\bf opts\_krfile()} routine.
{\bf opts\_zonekr()}'s arguments conform to those of {\bf opts\_krfile()}.

If {\bf opts\_zonekr()} isn't passed any arguments, it will act as if both
{\it \$keyrec\_file} and {\it \$keyrec\_name} were given as empty strings.
In this case, their values will be taken from the {\it -krfile} and {\it
-keyrec} command line options.

\item [{\bf opts\_setcsopts(@csopts)}]\verb" "

This routine saves a copy of the command-specific options given in {\it
@csopts}.  This collection of options is added to the {\it @csopts} array
that may be passed to {\bf tooloptions()}.

\item [{\bf opts\_createkrf()}]\verb" "

Force creation of an empty {\it keyrec} file if the specified file does not
exist.  This may happen on calls to {\bf tooloptions()}, {\bf opts\_getkeys()},
{\bf opts\_krfile()}, and {\bf opts\_zonekr()}.

\item [{\bf opts\_suspend()}]\verb" "

Suspend inclusion of the command-line options in building the final hash
table of responses.

\item [{\bf opts\_restore()}]\verb" "

Restore inclusion of the command-line options in building the final hash
table of responses.

\item [{\bf opts\_drop()}]\verb" "

Discard the command-line options.  They will no longer be available for
inclusion in building the final hash table of responses for this execution
of the command.

\item [{\bf opts\_reset()}]\verb" "

Reset an internal flag so that the command-line arguments may be
re-examined.  This is usually only useful if the arguments have been
modified by the calling program itself.

\end{description}

{\bf SEE ALSO}

{\bf zonesigner(8)}

{\bf Getopt::Long(3)}

{\bf Net::DNS::SEC::Tools::conf(3)}, {\bf Net::DNS::SEC::Tools::keyrec(3)},

{\bf Net::DNS::SEC::Tools::keyrec(5)}

