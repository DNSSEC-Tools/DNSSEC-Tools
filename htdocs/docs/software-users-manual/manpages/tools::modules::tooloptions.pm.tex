\clearpage

\subsubsection{tooloptions.pm}

{\bf NAME}

\perlmod{Net::DNS::SEC::Tools::tooloptions} - DNSSEC-Tools option routines

{\bf SYNOPSIS}

\begin{verbatim}
  use Net::DNS::SEC::Tools::tooloptions;

  @specopts = ("propagate+", "waittime=i");

  $optsref = opts_cmdopts(@specopts);
  %options = %$optsref;

  $zoneref = opts_zonekr($keyrec_file,$keyrec_name,@specopts);
  %zone_kr = %$zoneref;

  opts_setcsopts(@specopts);

  opts_createkrf();

  opts_suspend();

  opts_restore();

  opts_drop();

  opts_reset();

  opts_gui();

  opts_nogui();

  $oldaction = opts_onerr(1);
  opts_onerr(0);
\end{verbatim}

{\bf DESCRIPTION}

DNSSEC-Tools supports a set of options common to all the tools in the suite.
These options may be set from DNSSEC-Tools defaults, values set in the
\path{dnssec-tools.conf} configuration file, in a \struct{keyrec} file, from
command-specific options, from command-line options, or from any combination
of the five.  In order to enforce a common sequence of option interpretation,
all DNSSEC-Tools should use the \perlmod{tooloptions.pm} routines to initialize
their options.

\perlmod{tooloptions.pm} routines combine data from the aforementioned option
sources into a hash table.  The hash table is returned to the caller, which
will then use the options as needed.

The command-line options are saved between calls.  This allows a command may
call \perlmod{tooloptions.pm} routines multiple times and still have the
command-line options included in the final hash table.  This is useful for
examining multiple \struct{keyrec}s in a single command.  Inclusion of
command-line options may be suspended and restored using the
\func{opts\_suspend()} and \func{opts\_restore()} calls.  Options may be
discarded entirely by calling \func{opts\_drop()}; once dropped, command-line
options may never be restored.  Suspension, restoration, and dropping of
command-line options are only effective after the initial
\perlmod{tooloptions.pm} call.

The options sources are combined in this order:

\begin{itemize}

\item DNSSEC-Tools Defaults\verb" "

The DNSSEC-Tools defaults, as defined in \perlmod{conf.pm} are put into a hash
table, with the option names as the hash key.

\item DNSSEC-Tools Configuration File\verb" "

The system-wide DNSSEC-Tools configuration file is read and these option
values are added to the option collection.  Again, the option names are used
as the hash key.

\item \struct{keyrec} File\verb" "

If a \struct{keyrec} file was specified, then the \struct{keyrec} named by
\var{keyrec\_name} will be retrieved.  The \struct{keyrec}'s fields are added
to the hash table.  Any field whose keyword matches an existing hash key will
override any existing values.

\item Command-Specific Options\verb" "

Options specific to the invoking commands may be specified in
\var{\@specopts}.  This array is parsed by \func{Getoptions()} from the
\perlmod{Getopt::Long} Perl module.  These options are folded into the hash
table; possibly overriding existing hash values.  The options given in
\var{\@specopts} must be in the format required by \func{Getoptions()}.

\item Command-Line Options\verb" "

The command-line options are parsed using \func{Getoptions()} from the
\perlmod{Getopt::Long} Perl module.  These options are folded into the hash
table; again, possibly overriding existing hash values.  The options given in
\var{\@specopts} must be in the format required by \func{Getoptions()}.

\end{itemize}

A reference to the hash table created in these steps is returned to the caller.

{\bf EXAMPLE}

\path{dnssec-tools.conf} has these entries:

\begin{verbatim}
    ksklength      2048
    zsklength      512
\end{verbatim}

\path{example.keyrec} has this entry:

\begin{verbatim}
    key         "Kexample.com.+005+12345"
            zsklength        "1024"
\end{verbatim}

\cmd{zonesigner} is executed with this command line:

\begin{verbatim}
    zonesigner -zsklength 4096 -wait 3600 ...  example.com
\end{verbatim}

\func{opts\_zonekr(``example.keyrec'',``Kexample.com.+005+12345'',(``wait=i''))}
will read each option source in turn, ending up with:
\begin{verbatim}
    I<ksklength>           512
    I<zsklength>          4096
    I<wait>                600
\end{verbatim}

{\bf TOOLOPTIONS INTERFACES}

\begin{description}

\item \func{opts\_cmdopts(\@csopts)}\verb" "

This \func{opts\_cmdopts()} call builds an option hash from the system
configuration file, a \struct{keyrec}, and a set of command-specific options.
A reference to this option hash is returned to the caller.

If \var{\$keyrec\_file} is given as an empty string, then no \struct{keyrec}
file will be consulted.  In this case, it is assumed that \var{\$keyrec\_name}
will be left out altogether.

If a non-existent \var{\$keyrec\_file} is given and \func{opts\_createkrf()}
has been called, then the named \struct{keyrec} file will be created.
\func{opts\_createkrf()} must be called for each \struct{keyrec} file that
must be created, as the \perlmod{tooloptions} \struct{keyrec}-creation state
is reset after \func{tooloptions()} has completed.

\item \func{opts\_zonekr($keyrec\_file,$keyrec\_name,\@csopts)}\verb" "

This routine returns a reference to options gathered from the basic option
sources and from the zone \struct{keyrec} named by \var{\$keyrec\_name}, which
is found in \var{\$keyrec\_file}.  The \struct{keyrec} fields from the zone's
KSK and ZSK are folded in as well, but the key's \var{keyrec\_} fields are
excluded.  This call ensures that the named \struct{keyrec} is a zone
\struct{keyrec}; if it isn't, {\it undef} is returned.

The \var{\$keyrec\_file} argument specifies a \struct{keyrec} file that will
be consulted.  The \struct{keyrec} named by the \var{\$keyrec\_name} argument
will be loaded.  If a \struct{keyrec} file is found and
\func{opts\_createkrf()} has been previously called, then the \struct{keyrec}
file will be created if it doesn't exist.

If \var{\$keyrec\_file} is given as ``'', then the command-line options are
searched for a {\it -krfile} option.  If \var{\$keyrec\_name} is given as ``'',
then the name is taken from \var{\$ARGV[0]}.

The \var{\@specopts} array contains command-specific arguments; the arguments
must be in the format prescribed by the \perlmod{Getopt::Long} Perl module.

\item \func{opts\_setcsopts(\@csopts)}\verb" "

This routine saves a copy of the command-specific options given in
\var{\@csopts}.  This collection of options is added to the \var{\@csopts}
array that may be passed to \perlmod{tooloptions.pm} routines.

\item \func{opts\_createkrf()}\verb" "

Force creation of an empty \struct{keyrec} file if the specified file does not
exist.  This may happen on calls to \func{opts\_zonekr()}.

\item \func{opts\_suspend()}\verb" "

Suspend inclusion of the command-line options in building the final hash table
of responses.

\item \func{opts\_restore()}\verb" "

Restore inclusion of the command-line options in building the final hash
table of responses.

\item \func{opts\_drop()}\verb" "

Discard the command-line options.  They will no longer be available for
inclusion in building the final hash table of responses for this execution
of the command.

\item \func{opts\_reset()}\verb" "

Reset an internal flag so that the command-line arguments may be
re-examined.  This is usually only useful if the arguments have been
modified by the calling program itself.

\item \func{opts\_gui()}\verb" "

Set an internal flag so that command arguments may be specified with a GUI.
GUI usage requires that \perlmod{Getopt::Long::GUI} is available.  If it
isn't, then \perlmod{Getopt::Long} will be used.

\item \func{opts\_nogui()}\verb" "

Set an internal flag so that the GUI will not be used for specifying
command arguments.

\item \func{opts\_onerr(exitflag)}\verb" "

Set an internal flag indicating what should happen if an invalid option is
specified on the command line.  If \var{exitflag} is non-zero, then the process
will exit on an invalid option; if it is zero, then the process will not
exit.  The default action is to report an error without exiting.

The old exit action is returned.

\end{description}

{\bf SEE ALSO}

zonesigner(8)

Getopt::Long(3)

Net::DNS::SEC::Tools::conf(3),
Net::DNS::SEC::Tools::defaults(3), \\
Net::DNS::SEC::Tools::keyrec(3)

Net::DNS::SEC::Tools::keyrec(5)

