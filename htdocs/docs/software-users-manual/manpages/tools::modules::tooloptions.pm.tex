\clearpage

\subsection{\bf tooloptions.pm Module}

{\bf NAME}

\perlmod{Net::DNS::SEC::Tools::tooloptions.pm} - DNSSEC-Tools option routines.

{\bf SYNOPSIS}

\begin{verbatim}
    use Net::DNS::SEC::Tools::tooloptions;

    $keyrec_name = "Kexample.com.+005+10988";
    @specopts = ("propagate+", "waittime=i");

    $optsref = tooloptions($keyrec_file,$keyrec_name);
    %options = %$optsref;

    $optsref = tooloptions($keyrec_file,$keyrec_name,@specopts);
    %options = %$optsref;

    $optsref = tooloptions("",@specopts);
    %options = %$optsref;

    ($krfile,$krname,$optsref) = opts_krfile($keyrec_file,"");
    %options = %$optsref;

    ($krfile,$krname,$optsref) = opts_krfile("",$keyrec_name,@specopts);
    %options = %$optsref;

    ($krfile,$krname,$optsref) = opts_krfile("","");
    %options = %$optsref;

    $key_ref = opts_keykr();
    %key_kr  = %$key_ref;

    $optsref = opts_keykr($keyrec_file,$keyrec_name);
    %options = %$optsref;

    $zoneref = opts_zonekr();
    %zone_kr = %$zoneref;

    $zoneref = opts_zonekr($keyrec_file,$keyrec_name);
    %zone_kr = %$zoneref;

    opts_setcsopts(@specopts);

    opts_createkrf();

    opts_suspend();

    opts_restore();

    opts_drop();

    opts_reset();

    opts_gui();

    opts_nogui();
\end{verbatim}

{\bf DESCRIPTION}

DNSSEC-Tools supports a set of options common to all the tools in the suite.
These options may have defaults set in the \path{dnssec-tools.conf}
configuration file, in a {\it keyrec} file, from command-line options, or from
any combination of the three.  In order to enforce a common sequence of option
interpretation, all DNSSEC-Tools should use the \func{tooloptions()} routine to
initialize its options.

The {\it keyrec\_file} argument specifies a {\it keyrec} file that will be
consulted.  The {\it keyrec} named by the {\it keyrec\_name} argument will be
loaded.  If no {\it keyrec} file should be used, then {\it keyrec\_file} should
be an empty string and the {\it keyrec\_name} parameter not included.  The {\it
\@specopts} array contains command-specific arguments; the arguments must be in
the format prescribed by the \perlmod{Getopt::Long} Perl module.

\func{tooloptions()} combines data from these three option sources into a hash
table.  The hash table is returned to the caller, which will then use the
options as needed.

The command-line options are saved between calls, so a command may call
\func{tooloptions()} multiple times and still have the command-line options
included in the final hash table.  This is useful for examining multiple {\it
keyrec}s in a single command.  Inclusion of command-line options may be
suspended and restored using the \func{opts\_suspend()} and
\func{opts\_restore()} calls.  Options may be discarded entirely by calling
\func{opts\_drop()}; once dropped, command-line options may never be restored.
Suspension, restoration, and dropping of command-line options are only
effective after the initial \func{tooloptions()} call.

The options sources are combined in this manner:

\begin{enumerate}

\item \path{dnssec-tools.conf}\verb" "

The system-wide configuration file is read and these option values are used
as the defaults.  These options are put into a hash table, with the option
names as the hash key.

\item {\it keyrec} File\verb" "

If a {\it keyrec} file was specified, then the {\it keyrec} named by {\it
keyrec\_name} will be retrieved.  The {\it keyrec}'s fields are added to the
hash table.  Any field whose keyword matches an existing hash key will
override the existing value.

\item Command-line Options\verb" "

The command-line options, specified in {\it \@specopts}, are parsed using
\func{Getoptions()} from the \perlmod{Getopt::Long} Perl module.  These
options are folded into the hash table; again possibly overriding existing
hash values.  The options given in {\it \@specopts} must be in the format
required by \func{Getoptions()}.

\end{enumerate}

A reference to the hash table created in these three steps is returned to the
caller.


{\bf EXAMPLE}

\path{dnssec-tools.conf} has these entries:

\begin{verbatim}
    ksklength      1024
    zsklength      512
\end{verbatim}

{\bf example.keyrec} has this entry:

\begin{verbatim}
    key         "Kexample.com.+005+10988"
        zsklength        "1024"
\end{verbatim}

\cmd{zonesigner} is executed with this command line:

\begin{verbatim}
    zonesigner -ksklength 512 -zsklength 4096 -wait 600 ...  example.com
\end{verbatim}

\func{tooloptions(``example.keyrec'',``Kexample.com.+005+10988'',(``wait=i''))}
will read each option source in turn, ending up with:

\begin{verbatim}
    ksklength           512
    zsklength          4096
    wait                600
\end{verbatim}

{\bf TOOLOPTIONS ARGUMENTS}

Many of the DNSSEC-Tools option interfaces take the same set of arguments:
{\it \$keyrec\_file}, {\it \$keyrec\_name}, and {\it \@csopts}.  These arguments
are used similarly by most of the interfaces; differences are noted in the
interface descriptions in the next section.

\begin{description}

\item {\it \$keyrec\_file}\verb" "

Name of the {\it keyrec} file to be searched.

\item {\it \$keyrec\_name}\verb" "

Name of the {\it keyrec} that is being sought.

\item {\it @csopts}\verb" "

Command-specific options.

\end{description}

The {\it keyrec} named in {\it \$keyrec\_name} is selected from the {\it keyrec}
file given in {\it \$keyrec\_file}.  If either {\it \$keyrec\_file} or {\it
\$keyrec\_name} are given as empty strings, their values will be taken from the
{\it -krfile} and {\it -keyrec} command line options.

A set of command-specific options may be specified in {\it \@csopts}.  These
options are in the format required by the \perlmod{Getopt::Long} Perl module.
If {\it \@csopts} is left off the call, then no command-specific options will
be included in the final option hash.  The {\it \@csopts} array may be passed
directly to several interfaces or it may be saved in a call to
\func{opts\_setcsopts()}.

{\bf TOOLOPTION INTERFACES}

\begin{description}

\item \func{tooloptions(\$keyrec\_file,\$keyrec\_name,\@csopts)}\verb" "

This \func{tooloptions()} call builds an option hash from the system
configuration file, a {\it keyrec}, and a set of command-specific options.
A reference to this option hash is returned to the caller.

If {\it \$keyrec\_file} is given as an empty string, then no {\it keyrec} file
will be consulted.  In this case, it is assumed that {\it \$keyrec\_name} will
be left out altogether.

If a non-existent {\it \$keyrec\_file} is given and \func{opts\_createkrf()} has
been called, then the named {\it keyrec} file will be created.
\func{opts\_createkrf()} must be called for each {\it keyrec} file that must be
created, as the \func{tooloptions} {\it keyrec}-creation state is reset after
\func{tooloptions()} has completed.

\item \func{opts\_krfile(\$keyrec\_file,\$keyrec\_name,\@csopts)}\verb" "

The \func{opts\_krfile()} routine looks up the {\it keyrec} file and {\it
keyrec} name and uses those fields to help build an options hash.  References
to the {\it keyrec} file name, {\it keyrec} name, and the option hash table
are returned to the caller.

The {\it \$keyrec\_file} and {\it \$keyrec\_name} arguments are required
parameters.  They may be given as empty strings, but they {\bf must} be given.

If the {\it \$keyrec\_file} file and {\it \$keyrec\_name} name are both
specified by the caller, then this routine will have the same effect as
directly calling \func{tooloptions()}.

\item \func{opts\_getkeys(\$keyrec\_file,\$keyrec\_name,\@csopts)}\verb" "

This routine returns references to the KSK and ZSK {\it keyrec}s associated
with a specified {\it keyrec} entry.  This gives an easy way to get a zone's
{\it keyrec} entries in a single step.

This routine acts as a front-end to the \func{opts\_krfile()} routine.
Arguments to \func{opts\_getkeys()} conform to those of \func{opts\_krfile()}.

If \func{opts\_getkeys()} isn't passed any arguments, it will act as if both
{\it \$keyrec\_file} and {\it \$keyrec\_name} were given as empty strings.  In
this case, their values will be taken from the {\it -krfile} and {\it -keyrec}
command line options.

\item \func{opts\_keykr(\$keyrec\_file,\$keyrec\_name,\@csopts)}\verb" "

This routine returns a reference to the key {\it keyrec} named by {\it
\$keyrec\_name}.  It ensures that the named {\it keyrec} is a key {\it keyrec};
if it isn't, {\it undef} is returned.

This routine acts as a front-end to the \func{opts\_krfile()} routine.
\func{opts\_keykr()}'s arguments conform to those of \func{opts\_krfile()}.

If \func{opts\_keykr()} isn't passed any arguments, it will act as if both {\it
\$keyrec\_file} and {\it \$keyrec\_name} were given as empty strings.  In this
case, their values will be taken from the {\it -krfile} and {\it -keyrec}
command line options.

\item \func{opts\_zonekr(\$keyrec\_file,\$keyrec\_name,\@csopts)}\verb" "

This routine returns a reference to the zone {\it keyrec} named by
{\it \$keyrec\_name}.  The {\it keyrec} fields from the zone's KSK and ZSK are
folded in as well, but the key's {\it keyrec\_} fields are excluded.  This
call ensures that the named {\it keyrec} is a zone {\it keyrec}; if it isn't,
{\it undef} is returned.

This routine acts as a front-end to the \func{opts\_krfile()} routine.
\func{opts\_zonekr()}'s arguments conform to those of \func{opts\_krfile()}.

If \func{opts\_zonekr()} isn't passed any arguments, it will act as if both
{\it \$keyrec\_file} and {\it \$keyrec\_name} were given as empty strings.  In
this case, their values will be taken from the {\it -krfile} and {\it -keyrec}
command line options.

\item \func{opts\_setcsopts(\@csopts)}\verb" "

This routine saves a copy of the command-specific options given in {\it
\@csopts}.  This collection of options is added to the {\it \@csopts} array
that may be passed to \func{tooloptions()}.

\item \func{opts\_createkrf()}\verb" "

Force creation of an empty {\it keyrec} file if the specified file does not
exist.  This may happen on calls to \func{tooloptions()},
\func{opts\_getkeys()}, \func{opts\_krfile()}, and \func{opts\_zonekr()}.

\item \func{opts\_suspend()}\verb" "

Suspend inclusion of the command-line options in building the final hash
table of responses.

\item \func{opts\_restore()}\verb" "

Restore inclusion of the command-line options in building the final hash
table of responses.

\item \func{opts\_drop()}\verb" "

Discard the command-line options.  They will no longer be available for
inclusion in building the final hash table of responses for this execution
of the command.

\item \func{opts\_reset()}\verb" "

Reset an internal flag so that the command-line arguments may be
re-examined.  This is usually only useful if the arguments have been
modified by the calling program itself.

\item \func{opts\_gui()}\verb" "

Set an internal flag so that command arguments may be specified with a GUI.
GUI usage requires that \perlmod{Getopt::Long::GUI} is available.  If it
isn't, then \perlmod{Getopt::Long} will be used.

\item \func{opts\_nogui()}\verb" "

Set an internal flag so that the GUI will not be used for specifying
command arguments.

\end{description}

{\bf SEE ALSO}

\perlmod{Getopt::Long(3)}

\perlmod{Net::DNS::SEC::Tools::conf.pm(3)},
\perlmod{Net::DNS::SEC::Tools::keyrec.pm(3)},

\path{Net::DNS::SEC::Tools::keyrec(5)}
