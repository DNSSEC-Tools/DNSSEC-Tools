\clearpage

\subsubsection{\bf fixkrf}


{\bf NAME}

\cmd{fixkrf} - Fixes DNSSEC-Tools {\it keyrec} files whose encryption
key files have been moved.

{\bf SYNOPSIS}

\begin{verbatim}
    fixkrf [options] <keyrec-file> <dir 1> ... <dir N>
\end{verbatim}

{\bf DESCRIPTION}

\cmd{fixkrf} checks a specified {\it keyrec} file to ensure that the
referenced encryption key files exist where listed.  If a key is not
where the {\it keyrec} specifies it should be, then {\it fixkrf} will
search the given directories for those keys and adjust the {\it keyrec}
to match reality.  If a key of a particular filename is found in multiple
places, a warning will be printed and the {\it keyrec} file will not be
changed for that key.

{\bf OPTIONS}

\begin{description}

\item {\it -list}\verb" "

Display output about missing keys, but don't fix the {\it keyrec} file.

\item {\it -verbose}\verb" "

Display output about found keys as well as missing keys.

\item {\it -help}\verb" "

Display a usage message.

\end{description}

{\bf SEE ALSO}

\cmd{cleankrf(8)},
\cmd{genkrf(8)},
\cmd{lskrf(1)},
\cmd{zonesigner(8)}

\perlmod{Net::DNS::SEC::Tools::keyrec.pm(3)}

\path{keyrec(5)}
