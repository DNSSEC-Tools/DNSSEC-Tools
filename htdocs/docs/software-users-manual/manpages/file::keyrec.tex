\clearpage

\subsection{\bf keyrec}

{\bf NAME}

\path{keyrec} - Zone and key data used by DNSSEC-Tools programs.

{\bf DESCRIPTION}

{\it keyrec} files contain data about zones signed by and keys generated by
the DNSSEC-Tools programs.  A {\it keyrec} file is organized in sets of
{\it keyrec} records.  Each {\it keyrec} must be either of {\it zone} type,
{\it set} type, or {\it key} type.
Zone {\it keyrec}s describe how the zones were signed.
Set {\it keyrec}s describe sets of key {\it keyrec}s.
Key {\it keyrec}s describe how encryption keys were generated.
A {\it keyrec} consists of a set of keyword/value entries.

The DNSSEC-Tools \perlmod{keyrec.pm} module manipulates the contents of a
{\it keyrec} file.  Module interfaces exist for looking up {\it keyrec}
records, creating new records, and modifying existing records.

Comment lines and blank lines are ignored by the DNSSEC-Tools programs.
Comment lines start with either a `\#' character or a `$;$' character.

{\bf FIELDS}

The fields in a {\it keyrec} record are described in this section.  The fields
in each type of record (zone, set, key) are described in their own subsection.

{\bf Zone Keyrec Fields}

\begin{itemize}

\item {\it zonefile}

The name of the zone file for this zone.

\item {\it signedfile}

The name of the signed zone file for this zone.

\item {\it endtime}

The time when the zone's SIG records expire.  This field is passed to
\cmd{dnssec-signzone} as the argument to the {\it -e} option.

\item {\it kskkey}

The name of the zone's KSK key.  This is used as the name of the KSK key's
{\it keyrec} field.

\item {\it kskpath}

The path to the zone's KSK key.  This may be an absolute or relative path,
but it should be one which \cmd{zonesigner} may use (in conjunction with other
{\it keyrec} fields to find the key.

\item {\it kskdirectory}

The directory that holds the KSK key.

\item {\it zskcur}

The name of the signing set for the current ZSK keys.
This is the name of the signing set's set {\it keyrec}.

\item {\it zskpub}

The name of the signing set for the current ZSK keys.
This is the name of the signing set's set {\it keyrec}.

\item {\it zsknew}

The name of the signing set for the current ZSK keys.
This is the name of the signing set's set {\it keyrec}.

\item {\it keyrec\_signsecs}

The numeric timestamp of the zone {\it keyrec}'s last update.
This is measured in seconds since the epoch.

\item {\it keyrec\_signdate}

The textual timestamp of the zone {\it keyrec}'s last update.
This is a translation of the {\it keyrec\_signsecs} field.

\end{itemize}

{\bf Set Keyrec Fields}

\begin{itemize}

\item {\it zonename}

The name of the zone for which this signing set was generated.

\item {\it keys}

The list of keys in this signing set.  Each key listed should have a
corresponding key {\it keyrec} whose name matches the key name.

\item {\it keyrec\_setsecs}

The numeric timestamp of the signing set's creation.
This is measured in seconds since the epoch.

\item {\it keyrec\_setdate}

The textual timestamp of the signing set's creation.
This is a translation of the {\it keyrec\_setsecs} field.

\end{itemize}

{\bf Key Keyrec Fields}

\begin{itemize}

\item {\it zonename}

The name of the zone for which this key was generated.

\item {\it algorithm}

The encryption algorithm used to generate this key.

\item {\it random}

The random number generator used to generate this key.

\item {\it keypath}

The path to the key.  This may be an absolute or relative path, but it should
be one which {\bf zonesigner} may use (in conjunction with other {\it keyrec}
fields to find the key.

\item {\it ksklength}

The length of a KSK key.  This is only included in {\it keyrec}s for KSK keys.

\item {\it zsklength}

The length of a ZSK key.  This is only included in {\it keyrec}s for ZSK keys.

\item {\it keyrec\_gensecs}

The numeric timestamp of the key's creation.  This is measured in seconds
since the epoch.

\item {\it keyrec\_gendate}

The textual timestamp of the key's creation.  This is a translation of
the {\it keyrec\_gensecs} field.

\end{itemize}

{\bf EXAMPLES}

The following is an example of a zone {\it keyrec}:

\begin{verbatim}
    zone        "example.com"
            zonefile        "db.example.com"
            signedfile      "db.example.com.signed"
            endtime         "+604800"
            kskkey          "Kexample.com.+005+33333"
            kskpath         "keydir/Kexample.com.+005+33333"
            kskdirectory    "keydir"
            zskcur          "signing-set-42"
            zskpub          "signing-set-43"
            zsknew          "signing-set-44"
            keyrec_signsecs "1123771721"
            keyrec_signdate "Thu Aug 11 14:48:41 2005"
\end{verbatim}

The following is an example of a set {\it keyrec}:

\begin{verbatim}
    set        "signing-set-42"
            zonename        "example.com"
            keys            "Kexample.com.+005+88888"
            keyrec_setsecs  "1123771350"
            keyrec_setdate  "Thu Aug 11 14:42:30 2005"
\end{verbatim}

The following is an example of a key {\it keyrec}:

\begin{verbatim}
    key        "Kexample.com.+005+88888"
            zonename        "example.com"
            algorithm       "rsasha1"
            random          "/dev/urandom"
            keypath         "./Kexample.com.+005+88888.key"
            ksklength       "1024"
            keyrec_gensecs  "1123771354"
            keyrec_gendate  "Thu Aug 11 14:42:34 2005"
\end{verbatim}

{\bf SEE ALSO}

\cmd{lskrf(1)}

\cmd{dnssec-signzone(8)},
\cmd{signset-editor(8)},
\cmd{zonesigner(8)}

\perlmod{Net::DNS::SEC::Tools::keyrec.pm(3)}

