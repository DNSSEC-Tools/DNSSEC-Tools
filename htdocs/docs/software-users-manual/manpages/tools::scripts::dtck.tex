\clearpage

\subsubsection{dtck}

{\bf NAME}

\cmd{dtck} - Check the DNSSEC-Tools data files for sanity

{\bf SYNOPSIS}

\begin{verbatim}
  dtck [options] [dtck_config_file]
\end{verbatim}

{\bf DESCRIPTION}

\cmd{dtck} checks DNSSEC-Tools data files to determine if the entries are
valid.  \cmd{dtck} checks the validity of DNSSEC-Tools configuration files,
\struct{rollrec} files, and \struct{keyrec} files.  It does not perform the
file checking itself, but runs checking programs specific to each type of
data file.

A \cmd{dtck} configuration file is consulted to determine the files to check.
This file lists the DNSSEC-Tools data files and their types.  If a \cmd{dtck}
configuration file is not given on the command line, \cmd{dtck} will only
check the DNSSEC-Tools configuration file.  This is equivalent to running
\cmd{dtconfchk} directly.

{\bf DTCK CONFIGURATION FILE}

A \cmd{dtck} configuration file contains a list of the files to be checked by
\cmd{dtck}.  Except for comments, each line has the following format:

\begin{verbatim}
    keyword file directory
\end{verbatim}

{\it keyword} is one of ``config'', ``rollrec'', or ``keyrec''.
{\it file} is the pathname of the file to be checked.
{\it directory} is the name of the directory that holds {\it file} and is
optional.

The \cmd{dtck} configuration file contains the following types of records:

\begin{description}

\item {\it config}\verb" "

These lines define the DNSSEC-Tools configuration files that will be checked.
The \cmd{dtconfchk} program will be used to verify these files.

\item {\it rollrec}\verb" "

These lines define the \struct{rollrec} files that will be checked.
The \cmd{rollchk} program will be used to verify these files.

\item {\it keyrec}\verb" "

These lines define the \struct{keyrec} files that will be checked.
The \cmd{krfcheck} program will be used to verify these files.

\item comments\verb" "

Any lines starting with an octothorpe (\#) are comment lines and are ignored.

\end{description}

{\bf OPTIONS}

\cmd{dtck} takes two types of options.  Options of the first type are handled 
directly by \cmd{dtck}, controlling its output and processing.  Options of the
second type are passed to the file-checking programs and are not further
handled by \cmd{dtck}.

{\bf Options Handled by \cmd{dtck}}

\begin{description}

\item {\bf -defcon}\verb" "

This option directs \cmd{dtck} to add the default DNSSEC-Tools configuration
file to the list of configuration files to be checked.

\item {\bf -list}\verb" "

The names of the files will be listed as they are checked.

\item {\bf -pretty}\verb" "

Clarifying output is added to the output from \cmd{dtck} and the
file-checking programs.

\item {\bf -help}\verb" "

Display a usage message.

\end{description}

{\bf Options Not Handled by \cmd{dtck}}

\begin{description}

\item {\bf -count}\verb" "

The file-checking programs will display a final error count.

\item {\bf -quiet}\verb" "

No output will be given by the file-checking program.

\item {\bf -verbose}\verb" "

Verbose output will be given by the file-checking program.

\end{description}

{\bf SEE ALSO}

dtconfchk(8),
krfcheck(8),
rollchk(8)

dnssec-tools.conf(5),
keyrec(5),
rollrec(5)

