\clearpage

\subsection{\bf dnsval.conf}

{\bf NAME}

\path{/etc/dnsval.conf} - Configuration policy for the DNSSEC validator
library \lib{libval(3)}.

{\bf DESCRIPTION}

The validator library reads configuration information from three files,
\path{/etc/resolv.conf}, \path{/etc/root.hints}, and \path{/etc/dnsval.conf}.

\begin{description}

\item [/etc/resolv.conf]\verb" "

Only the {\it nameserver} option is supported in the \path{resolv.conf} file.
This option is used to specify the IP address of the name server to which
queries must be sent by default.  For example,

\begin{verbatim}
    nameserver 10.0.0.1
\end{verbatim}

The \path{/etc/resolv.conf} file may be empty in which case the validator in
\lib{libval(3)} tries to recursively answer the query using information
present in \path{/etc/root.hints}.

\item [/etc/root.hints]\verb" "

The \path{/etc/root.hints} file contains bootstrapping information for the
resolver while it attempts to recursively answer queries.  The contents of
this file may be generated by the following command:

\begin{verbatim}
    dig @e.root-servers.net . ns > root.hints
\end{verbatim}

\item [/etc/dnsval.conf]\verb" "

The \path{/etc/dnsval.conf} file contains a sequence of the following
``policy-fragments'':

\begin{verbatim}
    <label> <KEYWORD> <additional-data>; 
\end{verbatim}

{\it label} identifies the policy fragment 
and {\it KEYWORD} is the specific policy component that is 
configured.  The format of additional-data depends on the 
keyword specified.

If multiple policy fragments are defined for the same label and keyword
combination then the last definition in the file is used.

Currently two different keywords are specified:

\begin{description}

\item [trust-anchor]\verb" "

Specifies the trust anchors for a sequence of zones.  The additional
data portion for this keyword is a sequence of the zone name and a 
quoted string containing the RDATA portion for the trust anchor's 
DNSKEY.

\item [zone-security-expectation]\verb" "

Specifies the local security expectation for a zone.  The additional
data portion for this keyword is a sequence of the zone name and 
its trust status - {\it ignore}, {\it validate}, {\it trusted}, or
{\it untrusted}.

\end{description}
\end{description}

{\bf EXAMPLE}

The \path{/etc/dnsval.conf} configuration file might appear as follows:

\begin{verbatim}
    mozilla trust-anchor]
        dnssec-tools.org.
            "257 3 5 AQO8XS4y9r77X9SHBmrx-
            MoJf1Pf9AT9Mr/L5BBGtO9/e9f/zl4FFgM2l
            B6M2XEm6mp6mit4tzpB/sAEQw1McYz6bJdKkTiqtuWTCfDmgQhI6/Ha0
            EfGPNSqnY 99FmbSeWNIRaa4fgSCVFhvbrYq1nXkNVyQPeEVHkoDNCAlr
            qOA3lw=="]
        netsec.tislabs.com.
            "257 3 5 AQO8XS4y9r77X9SHBmrx-
            MoJf1Pf9AT9Mr/L5BBGtO9/e9f/zl4FFgM2l
            B6M2XEm6mp6mit4tzpB/sAEQw1McYz6bJdKkTiqtuWTCfDmgQhI6/Ha0
            EfGPNSqnY 99FmbSeWNIRaa4fgSCVFhvbrYq1nXkNVyQPeEVHkoDNCAlr
            qOA3lw==" ;]

    : zone-security-expectation
        org ignore ]
        net ignore]
        dnssec-tools.org validate]
        com ignore;]
\end{verbatim}

{\bf FILES}

\path{/etc/dnsval.conf(5)}

\path{/etc/resolv.conf(5)}

\path{/etc/root.hints(5)}

{\bf SEE ALSO}

\lib{libval(3)}
