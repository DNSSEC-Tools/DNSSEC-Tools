\clearpage

\subsubsection{dnsval.conf}

{\bf NAME}

\path{/etc/dnsval.conf}, \path{/etc/resolv.conf}, \path{/etc/root.hints} -
Configuration policy for the DNSSEC validator library \lib{libval(3)}\\
\func{val\_add\_valpolicy()} - Dynamically add a new policy to the validator
context

\func{val\_remove\_valpolicy()} - Remove a dynamically added policy from the
validator context

{\bf SYNOPSIS}

\begin{verbatim}
    int val_add_valpolicy(val_context_t *context,
			  const char *keyword,
                          char *zone,
                          char *value,
                          long ttl,
                          val_policy_entry_t **pol);

    int val_remove_valpolicy(val_context_t *context,
			     val_policy_entry_t *pol);
\end{verbatim}

{\bf DESCRIPTION}

Applications can use local policy to influence the validation outcome.
Examples of local policy elements include trust anchors for different zones
and untrusted algorithms for cryptographic keys and hashes.  Local policy
may vary for different applications and operating scenarios.

The \func{val\_add\_valpolicy()} function can be used to dynamically add a new
policy for a given context.  The {\it keyword}, {\it zone} and {\it value}
arguments are identical to {\it KEYWORD}, {\it zone} and {\it additional-data}
defined below for \path{/etc/dnsval.conf}.  {\it ttl} specifies the duration in
seconds for which the policy is kept in effect.  A value of {\bf -1} adds to
policy to the context indefinitely.  A handle to the newly added policy is
returned in {\it *pol}.  This structure is opaque to the applications;
applications must not modify the contents of the memory returned in {\it
*pol}.

Applications may also revoke the effects of a newly added policy, {\it pol}, 
before the expiry of its timeout interval using the 
\func{val\_remove\_valpolicy()} policy.

The validator library reads configuration information from three
files: \path{/etc/resolv.conf}, \path{/etc/root.hints}, and
\path{/etc/dnsval.conf}.

\begin{description}

\item /etc/resolv.conf\verb" "

The {\it nameserver} and {\it search} options are supported in the
\path{resolv.conf} file.

This {\it nameserver} option is used to specify the IP address of the name
server to which queries must be sent by default.  For example,

\begin{verbatim}
    nameserver 10.0.0.1
\end{verbatim}

This {\it search} option is used to specify the search path for issuing
queries.  For example,

\begin{verbatim} 
    search test.dnssec-tools.org dnssec-tools.org
\end{verbatim}

If the \path{/etc/resolv.conf} file contains no name servers, the validator 
tries to recursively answer the query using information present
in \path{/etc/root.hints}.

\item /etc/root.hints\verb" "

The \path{/etc/root.hints} file contains bootstrapping information for the
resolver while it attempts to recursively answer queries.  The contents of
this file may be generated by the following command:

\begin{verbatim}
    dig @e.root-servers.net . ns > root.hints
\end{verbatim}

\item /etc/dnsval.conf\verb" "

The \path{/etc/dnsval.conf} file contains the validator policy.  It consists
of a sequence of the following ``policy-fragments'':

\begin{verbatim}
<label> <KEYWORD> <zone> <additional-data> [<zone> <additional-data> ];
\end{verbatim}

Policies are identified by simple text strings called labels, which must be
unique within the configuration system.  For example, ``browser'' could be
used as the label that defines the validator policy for all web-browsers in a
system.  A label value of ``:'' identifies the default policy, the policy that
is used when a NULL context is specified as the {\it ctx} parameter for
interfaces listed in \func{libval(3)}, \func{val\_getaddrinfo(3)}, and
\func{val\_gethostbyname(3)}.  The default policy is unique within the
configuration system.

{\it KEYWORD} is the specific policy component that is specified within the 
policy fragment.  The format of {\it additional-data} depends on the 
keyword specified.

If multiple policy fragments are defined for the same label and keyword
combination then the last definition in the file is used.

The following keywords are defined for \path{dnsval.conf}:

\begin{description}

\item trust-anchor\verb" "

Specifies the trust anchors for a sequence of zones.  The additional
data portion for this keyword is a quoted string containing the 
RDATA portion for the trust anchor's DNSKEY.

\item zone-security-expectation\verb" "

Specifies the local security expectation for a zone.  The additional
data portion for this keyword is the zone's trust status - 
ignore, validate, trusted, or untrusted.  The default zone security
expectation is validate.

\item provably-unsecure-status \verb" "

Specifies if the provably unsecure condition must be considered as 
trusted or not.  The additional data portion for this keyword is the 
trust status for the provably unsecure condition for a given zone - 
trusted, or untrusted.  The default provably unsecure status is trusted.

\item clock-skew \verb" "

Specifies how many seconds of clock skew is acceptable when verifying 
signatures for data from a given zone. The additional data portion 
for this keyword is the number of seconds of clock skew that is
acceptable.  A value of -1 completely ignores inception and expiration times 
on signatures for data from a given zone. The default clock skew is 0.

\item nsec3-max-iter [only if LIBVAL\_NSEC3 is enabled]\verb" "

Specifies the maximum number of iterations allowable while computing
a zone's NSEC3 hash.  A value of -1 does not place a maximum limit on
the number of iterations.  This is also the default setting for a zone.

\item dlv-trust-points [only if LIBVAL\_DLV is enabled]\verb" "

Specifies the DLV tree for the target zone. 

\end{description}

\end{description}

{\bf EXAMPLE}

The \path{/etc/dnsval.conf} configuration file might appear as follows:

\begin{description}

\item : trust-anchor\verb" "

\begin{description}

\item dnssec-tools.org.\verb" "

\begin{description}
\item ``257 3 5 AQO8XS4y9r77X9SHBmrxMoJf1Pf9AT9Mr/L5BBGtO9/e9f/zl4FFgM2l B6M2XEm6mp6mit4tzpB/sAEQw1McYz6bJdKkTiqtuWTCfDmgQhI6/Ha0 EfGPNSqnY 99FmbSeWNIRaa4fgSCVFhvbrYq1nXkNVyQPeEVHkoDNCAlr qOA3lw==''\verb" "
\end{description}

\item netsec.tislabs.com.\verb" "

\begin{description}
\item ``257 3 5 AQO8XS4y9r77X9SHBmrxMoJf1Pf9AT9Mr/L5BBGtO9/e9f/zl4FFgM2l B6M2XEm6mp6mit4tzpB/sAEQw1McYz6bJdKkTiqtuWTCfDmgQhI6/Ha0 EfGPNSqnY 99FmbSeWNIRaa4fgSCVFhvbrYq1nXkNVyQPeEVHkoDNCAlr qOA3lw=='' \verb" "
\end{description}

\item ;\verb" "

\end{description}

\item browser zone-security-expectation    \verb" "

\begin{description}

\item org ignore \verb" "

\item net ignore\verb" "

\item dnssec-tools.org validate\verb" "

\item com ignore\verb" "

\item ;\verb" "

\end{description}

\item : provably-unsecure-status \verb" "

\begin{description}

\item . trusted \verb" "

\item net untrusted\verb" "

\item ;\verb" "

\end{description}

\item mta clock-skew \verb" "

\begin{description}

\item . 0 \verb" "

\item fruits.netsec.tislabs.com. -1\verb" "

\item ;\verb" "

\end{description}

\item : nsec3-max-iter \verb" "

\begin{description}

\item . 30\verb" "

\item ;\verb" "

\end{description}

\item browser dlv-trust-points \verb" "

\begin{description}

\item . dlv.isc.org\verb" "

\item ;\verb" "

\end{description}

\end{description}

{\bf FILES}

\path{/etc/resolv.conf},
\path{/etc/root.hints},
\path{/etc/dnsval.conf}

{\bf SEE ALSO}

libval(3)

