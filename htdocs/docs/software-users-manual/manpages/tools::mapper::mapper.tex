\clearpage

\subsection{\it mapper}


{\bf NAME}

mapper - Create graphical maps of DNS zone data

{\bf SYNOPSIS}

\begin{verbatim}  mapper [options] zonefile1 ... zonefileN\end{verbatim}

{\bf DESCRIPTION}

This application creates a graphical map of one or more zone files.  The
output gives a graphical representation of a DNS zone or zones.  The output
is written in the PNG format.  The result can be useful for getting a more
intuitive view of a zone or set of zones.  It is extremely useful for
visualizing DNSSEC deployment within a given zone as well as to help discover
problem spots.

{\bf OPTIONS}

\begin{description}

\item [-h]

Help summary.

\item [-o OUTFILE.png]\verb" "

Saves the results to a given filename.  If this option is not given, the map
will be saved to {\bf map.png}.

\item [-r]

Lists resources records assigned to each node within the map.

\item [-t TYPE,TYPE...]\verb" "

Adds the data portion of a resource record to the displayed node
information.  Data types passed will be automatically upper-cased for
ease of use.  Example usage: {\it -t A} will add IPv4 addresses to
all displayed nodes that have A records.

\item [-L]

Adds a legend to the map.

\item [-l (neato|dot|twopi|circo|fdp)]\verb" "

Selects a layout format.  The default is {\it neato} which is circular in
pattern.  See the documentation on the {\it graphviz} package and the
{\bf GraphViz} Perl module for further details.

\item [-a]

Allows overlapping of nodes.  This makes much tighter maps with the
downside being that they are somewhat cluttered.  Maps of extremely
large zones will be difficult to decipher if this option is not used.

\item [-e WEIGHT]\verb" "

Assigns an edge weight to edges.  In theory $>$1 means shorter, $<$1 means
longer although I don't believe as implemented it will have any
effect.  Should work better in the future.

\item [-f INTEGER]\verb" "

Uses the INTEGER value for the font size to print node names with.
Defaults to 10.

\item [-w WARNTIME]\verb" "

How far in advance that warnings are turned on for signed
resource records that are about to expire.  The default is 7 days.
The warning time is measured in seconds.

\item [-i REGEX]\verb" "

Ignore record types matching a {\it REGEX} regular expression.

\item [-s TYPE,TYPE...]\verb" "

Skips a list of record types from being analyzed or displayed in the map.
By default, this is set to NSEC and CNAME in order to reduce clutter.
Setting it to ``'' will display these results again.

\item [-T TYPE,TYPE...]\verb" "

Only process record types of type {\it TYPE}.  This is the converse of the
{\it -s} option.  It is not meaningful to use both {\it -s} and {\it -t}
in the same invocation.  They will both work at once, however, so if
{\it -T} specifies a type which {\it -s} excludes, it will not be shown.

\item [-g]

Attempts to cluster nodes around the domain name.  For `dot' layouts,
this actually means drawing a box around the cluster.  For the other
types, it makes very little, if any, difference.

\item [-q]

Don't output warnings or errors about records that have DNSSEC signatures
that are near or beyond their signature lifetimes.

\end{description}

{\bf EXAMPLE INVOCATIONS}

\begin{description}

\item [{\it mapper -s cname,nsec -i dhcp -L zonefile zone.com}]\verb" "

Writes to the default file ({\bf map.png}) of a {\it zone.com} zone
stored in {\it zonefile}.  It excludes any hosts with a name containing
{\it dhcp} and ignores any record of type {\it CNAME} or {\it NSEC}.  It also
includes a legend in the output.

\item [{\it mapper -s txt,hinfo,cname,nsec,a,aaaa,mx,rrsig -L zonefile zone.com zonefile2 sub.zone.com ...}]\verb" "

Removes a lot of records from the display in order to display mostly a
map of a zone hierarchy.

\item [{\it mapper -l dot -s txt,hinfo,cname,nsec,a,aaaa,mx,rrsig -L zonefile zone.com zonefile2 sub.zone.com ...}]\verb" "

The same thing, but draws a more vertical tree-style graph of the
zone.  This works well for fairly deep but narrow hierarchies.  Tree-style
diagrams rarely look as nice for full zones.

\end{description}

{\bf SEE ALSO}

\perlmod{Net::DNS}

\url{http://dnssec-tools.sourceforge.net}

