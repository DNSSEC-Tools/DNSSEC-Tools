\clearpage

\subsubsection{blinkenlights}

{\bf NAME}

\cmd{blinkenlights} - DNSSEC-Tools rollerd GUI

{\bf SYNOPSIS}

\begin{verbatim}

  blinkenlights <rollrec-file>

\end{verbatim}

{\bf DESCRIPTION}

\cmd{blinkenlights} is a GUI tool for use with monitoring and controlling the
DNSSEC-Tools \cmd{rollerd} program.  It displays information on the current
state of the zones \cmd{rollerd} is managing.  The user may control some
aspects of \cmd{rollerd}'s execution using \cmd{blinkenlights} menu commands.

\cmd{blinkenlights} creates a window in which to display information about
each zone \cmd{rollerd} is managing.  (These zones are those in
\cmd{rollerd}'s current \struct{rollrec} file.)  As a zone's rollover status
changes, \cmd{blinkenlights} will update its display for that zone.  Skipped
zones, zones listed in the \struct{rollrec} file but which are not in rollover
or normal operation, are displayed but have very little useful information to
display.

The user may also select a set of zones to hide from the display.  These
zones, if in the rolling state, will continue to roll; however, their zone
information will not be displayed.  Display state for each zone will persist
across \cmd{blinkenlights} executions.

Menu commands are available for controlling \cmd{rollerd}.  The commands which
operate on a single zone may be executed by keyboard shortcuts.  The zone may
be selected either by clicking in its ``zone stripe'' or by choosing from a
dialog box.  Display and execution options for \cmd{blinkenlights} are also
available through menu commands.  More information about the menu commands is
available in the MENU COMMANDS section.

\cmd{blinkenlights} is only intended to be started by \cmd{rollerd}, not
directly by a user.  There are two ways to have \cmd{rollerd} start
\cmd{blinkenlights}.  First, \cmd{rollctl} may be given the {\it -display}
option.  Second, the {\it -display} option may be given on \cmd{rollerd}'s
command line.

{\bf SCREEN LAYOUT}

The \cmd{blinkenlights} window is laid out as a series of ``stripes''.  The
top stripe contains status information about \cmd{rollerd}, the second stripe
contains column headers, and the bulk of the window consists of zone stripes.
The list below provides more detail on the contents of each stripe.

See the WINDOW COLORS section for a discussion of the colors used for the
zone stripes.

\begin{itemize}

\item \cmd{rollerd} information stripe\verb" "

The information stripe contains four pieces of information:  \cmd{rollerd}'s
current \struct{rollrec} file, the count of rolling zones, the count of
skipped zones, and the amount of time \cmd{rollerd} waits between processing
its queue.  Coincidentally, that last datum is also the amount of time between
\cmd{blinkenlights} screen updates.

\item column headers stripe\verb" "

This stripe contains the column headers for the columns of each zone stripe.

\item zone stripes\verb" "

Each zone managed by \cmd{rollerd} (i.e., every zone in the current
\struct{rollrec} file) will have a zone stripe which describes that zone's
current state.  The stripe is divided into four sections:  the zone name,
the current rollover state, and the zone's DNSSEC keys.

The zone name section just contains the name of the zone.

The rollover state section contains the rollover phase number, a text
explanation of the phase, and the amount of time remaining in that rollover
phase.  The phase explanation is ``normal operation'' when the zone isn't
currently in rollover.

The DNSSEC key section contains two subsections, one for the zone's ZSK keys
and another for the zone's KSK keys.  Each subsection contains the names of
the signing sets active for the zone.  The ZSK subsection lists the Current,
Published, and New ZSK keys; the KSK subsection lists the Current and
Published.

See the WINDOW COLORS section for a discussion of the colors used for the
zone stripes.

\end{itemize}

{\bf WINDOW COLORS}

The default \cmd{blinkenlights} configuration uses window coloring to provide
visual cues and to aid in easily distinguishing zone information.  The default
window coloring behavior gives each zone stripe has its own color and the
rollover state section of each zone stripe is shaded to show the zone's phase.
Window coloring can be turned off (and on) with configuration options and menu
commands.

{\bf Color Usage}

The two window coloring behaviors are discussed more fully below:

\begin{itemize}

\item zone stripe colors\verb" "

Each rolling zone's stripe is given one of three colors:  blue, red, or green.
The color is assigned on a top-down basis and the colors wrap if there are
more than three zones.  So, the first zone is always blue, the second zone
red, the third zone green, the fourth zone blue, etc.

The colors do not stay with a particular zone.  If a rolling zone becomes a
skipped zone, the zone stripes will be reassigned new colors to account for
that skipped zone.

Skipped zones are not colored with these three colors.  Stripes for skipped
zones are colored either grey or a color set in the configuration file.  If
you choose to use a non-standard color for skipped zones your should ensure
that it is {\bf not} one of the colors used for rolling zones' stripes.
Modifying the {\it skipcolor} configuration field allows the skipped-zone color
to be changing.

The {\it colors} configuration field can be used to turn on or off the use of
colors for zone stripes.  If stripe coloring is turned off, then every stripe
will be displayed using the {\it skipcolor} color.

\item rollover-state shading\verb" "

The only portion of a zone stripe that changes color is the status column; the
color of the rest of the zone stripe stays constant.  Before a zone enters
rollover, the status column is the same color as the rest of the stripe.  When
the zone enters rollover, the status column's color is changed to a very light
shade of the stripe's normal color.  As the rollover phases progress towards
rollover completion, the status column's shade darkens.  Once rollover
completes, the status column returns again to the same shade as the rest of
that stripe.

The {\it shading} configuration field can be used to turn on or off the use of
shading in the rollover-state column.  If shading is turned off, then the zone
stripe will be a solid color.

See the CONFIGURATION FILE section for information on setting the
configuration fields.

\end{itemize}

{\bf Colors Used}

The color names are taken from the X11 \path{rgb.txt} file (X11 1.1.3 -
XFree86 4.4.0 for MacOS X.)  If these aren't available in your \path{rgb.txt}
file, similar names should be selected.  The actual red/green/blue values used
are given below to assist in finding suitable replacements.  These values were
taken from the \path{rgb.txt} file.

Blue Shades:

\begin{table}[ht]
\begin{center}
\begin{tabular}{|l|r|r|r|}
\hline
{\bf Color Name} & {\bf Red Value} & {\bf Green Value} & {\bf Blue Value} \\
\hline
blue            &   0 &   0 & 255 \\
lightblue2      & 178 & 223 & 238 \\
darkslategray1  & 151 & 255 & 255 \\
skyblue1        & 135 & 206 & 255 \\
steelblue1      &  99 & 184 & 255 \\
turquoise1      &   0 & 245 & 255 \\
cornflower blue & 100 & 149 & 237 \\
dodger blue     &  30 & 144 & 255 \\
\hline
\end{tabular}
\end{center}
\caption{Blue Shades}
\end{table}

\eject

Red Shades:

\begin{table}[ht]
\begin{center}
\begin{tabular}{|l|r|r|r|}
\hline
{\bf Color Name} & {\bf Red Value} & {\bf Green Value} & {\bf Blue Value} \\
\hline
red          &    255 &   0 &   0 \\
pink         &    255 & 192 & 203 \\
lightsalmon1 &    255 & 160 & 122 \\
tomato       &    255 &  99 &  71 \\
indianred    &    205 &  92 &  92 \\
violetred1   &    255 &  62 & 150 \\
orangered1   &    255 &  69 &   0 \\
firebrick1   &    255 &  48 &  48 \\
\hline
\end{tabular}
\end{center}
\caption{Red Shades}
\end{table}

Green Shades:

\begin{table}[ht]
\begin{center}
\begin{tabular}{|l|r|r|r|}
\hline
{\bf Color Name} & {\bf Red Value} & {\bf Green Value} & {\bf Blue Value} \\
\hline
green           &   0 & 255 &   0 \\
darkseagreen1   & 193 & 255 & 193 \\
darkolivegreen1 & 202 & 255 & 112 \\
lightgreen      & 144 & 238 & 144 \\
seagreen1       &  84 & 255 & 159 \\
spring green    &   0 & 255 & 127 \\
greenyellow     & 173 & 255 &  47 \\
lawngreen       & 124 & 252 &   0 \\
\hline
\end{tabular}
\caption{Green Shades}
\end{center}
\end{table}

{\bf MENU COMMANDS}

A number of menu commands are available to control the behavior of
\cmd{blinkenlights} and to send commands to \cmd{rollerd}.  These
commands are discusses in this section.

{\bf File Menu}

The commands in this menu are basic GUI commands.

\begin{itemize}

\item Quit\verb" "

\cmd{blinkenlights} will stop execution.

\end{itemize}

{\bf Options Menu}

The commands in this menu control the appearance and behavior of
\cmd{blinkenlights}.

\begin{itemize}

\item Row Colors (toggle)\verb" "

This menu item is a toggle to turn on or off the coloring of zone stripes.
If row coloring is turned off, zone stripes will all be the same color.
If row coloring is turned on, zone stripes will be displayed in varying
colors.  See the WINDOW COLORS section for a discussion of row coloring.

\item Status Column Shading (toggle)\verb" "

This menu item is a toggle to turn on or off the shading of the zone status
column.  If shading is turned off, the zone stripes will present a solid,
unchanging band of color for each zone.  If shading is turned on, the color
of the zone status column will change according to the zone's rollover state.

\item Skipped Zones Display (toggle)\verb" "

This menu item is a toggle to turn on or off the display of skipped zones.  If
display is turned off, zone stripes for skipped zones will not be displayed.
If display is turned on, zone stripes for all zones will be displayed.

\item Modification Commands (toggle)\verb" "

In some situations, it may be desirable to turn off \cmd{blinkenlights}'
ability to send commands to \cmd{rollerd}.  This menu item is a toggle to turn
on or off this ability.  If the commands are turned off, then the ``Zone
Control'' menu and keyboard shortcuts are disabled.  If the commands are
turned on, then the ``Zone Control'' menu and keyboard shortcuts are enabled.

\item Font Size\verb" "

This menu item allows selection of font size of text displayed in the main
window.

Normally, changing the font size causes the window to grow and shrink as
required.  However, on Mac OS X there seems to be a problem when the size
selected increases the window size to be greater than will fit on the screen.
If the font size is subsequently reduced, the window size does not shrink in
response.

\end{itemize}

{\bf General Control Menu}

The commands in this menu are GUI interfaces for the \cmd{rollctl} commands
related to {\it general} zone management.

\begin{itemize}

\item Roll Selected Zone\verb" "

The selected zone will be moved to the rollover state.  This only has an
effect on skipped zones.  A zone may be selected by clicking on its zone
stripe.  If this command is selected without a zone having been selected,
a dialog box is displayed from which a currently skipped zone may be chosen.

\item Roll All Zones\verb" "

All zones will be moved to the rollover state.  This has no effect on
currently rolling zones.

\item Run the Queue\verb" "

\cmd{rollerd} is awoken and runs through its queue of zones.  The operation
required for each zone is then performed.

\item Skip Selected Zone\verb" "

The selected zone will be moved to the skipped state.  This only has an effect
on rolling zones.  A zone may be selected by clicking on its zone stripe.  If
this command is selected without a zone having been selected, a dialog box is
displayed from which a currently rolling zone may be chosen.

\item Skip All Zones\verb" "

All zones will be moved to the skipped state.  This has no effect on
currently skipped zones.

\item Halt Rollerd\verb" "

\cmd{rollerd}'s execution is halted.  As a result, \cmd{blinkenlights}'
execution will also be halted.

\end{itemize}

{\bf KSK Control Menu}

The commands in this menu are GUI interfaces for the \cmd{rollctl} commands
related to KSK-specific zone management.

\begin{itemize}

\item DS Published Selected Zone\verb" "

This command is used to indicate that the selected zone's parent has published
a new DS record for the zone.  It moves the zone from phase 6 to phase 7 of
KSK rollover.

\item DS Published All Zones\verb" "

This command is used to indicate that all the zones in KSK rollover phase 6
have new DS records published by their parents.  It moves all these zones from
phase 6 to phase 7 of KSK rollover.

\end{itemize}

{\bf Zone Display Menu}

The commands in this menu are GUI interfaces parts of the zone display.  There
are commands for displaying and hiding both zone stripes and key columns.  The
commands allow all, some, or none of the zone stripes and key columns to be
displayed.  Undisplayed rolling zones will continue to roll, but they will do
so without the \cmd{blinkenlights} window indicating this.

\begin{itemize}

\item Zone Selection\verb" "

A dialog box is created that holds a list of the zones currently managed by
\cmd{rollerd}.  The user may select which zones should be displayed by clicking
on the zone's checkbox.  Zones with a selected checkbox will be displayed;
zones without a selected checkbox will not be displayed.

\item Display All Zones\verb" "

All zones will be displayed in the \cmd{blinkenlights} window.

\item Hide All Zones\verb" "

No zones will be displayed in the \cmd{blinkenlights} window.

\item KSK Sets (toggle)\verb" "

This menu item is a toggle to turn on or off the display of KSK signing set
names.  If display is turned off, the columns holding the KSK signing set
names and labels will be removed from the display and the display window will
shrink.  If display is turned on, the columns holding the KSK signing set
names and labels will be restored to the display and the display window will
be expanded.

When displayed, KSK signing sets will always be the right-most columns.

\item ZSK Sets (toggle)\verb" "

This menu item is a toggle to turn on or off the display of ZSK signing set
names.  If display is turned off, the columns holding the ZSK signing set
names and labels will be removed from the display and the display window will
shrink.  If display is turned on, the columns holding the ZSK signing set
names and labels will be restored to the display and the display window will
be expanded.

When displayed, ZSK signing sets will always be immediately to the right of
the zone status column.

\item Hide All Keysets\verb" "

Turns off display of the KSK and ZSK signing set names.

\item Show All Keysets\verb" "

Turns on display of the KSK and ZSK signing set names.

\end{itemize}

{\bf Help Menu}

The commands in this menu provide assistance to the user.

\begin{itemize}

\item Help\verb" "

Display a window containing help information.

\end{itemize}

{\bf CONFIGURATION FILE}

Several aspects of \cmd{blinkenlights}' behavior may be controlled from
configuration files.  Configuration value may be specified in the DNSSEC-Tools
configuration file or in a more specific \path{rc.blinkenlights}.  The
system-wide \cmd{blinkenlights} configuration file is in the DNSSEC-Tools
configuration directory and is named \path{blinkenlights.conf}.  Multiple
\path{rc.blinkenlights} files may exist on a system, but only the one in the
directory in which \cmd{blinkenlights} is executed is used.

The following are the available configuration values:

\begin{table}[ht]
\begin{center}
\begin{tabular}{|l|c|l|}
\hline
{\bf Configuration Value} & {\bf Meaning} \\
colors    & Turn on/off use of colors on zone stripes \\
fontsize  & The size of the font in the output window \\
modify    & Turn on/off execution of rollerd modification commands \\
shading   & Turn on/off shading of the status columns \\
showskip  & Turn on/off display of skipped zones \\
skipcolor & The background color used for skipped zones \\
\hline
\end{tabular}
\end{center}
\caption{\cmd{blinkenlights} Configuration File Entries}
\end{table}

The \path{rc.blinkenlights} file is {\bf only} searched for in the directory
in which \cmd{blinkenlights} is executed.  The potential problems inherent in
this may cause these \cmd{blinkenlights}-specific configuration files to be
removed in the future.

This file is in the ``field value'' format, where {\it field} specifies the
output aspect and {\it value} defines the value for that field.  The following
are the recognized fields:

Empty lines and comments are ignored.  Comment lines are lines that start with
an octothorpe (`\#').

Spaces are not allowed in the configuration values.

Choose your skipcolors carefully.  The only foreground color used is black, so
your background colors must work well with black.

{\bf REQUIREMENTS}

\cmd{blinkenlights} is implemented in Perl/Tk, so both Perl and Perl/Tk must be
installed on your system.

{\bf WARNINGS}

\cmd{blinkenlights} has several potential problems that must be taken into
account.

\begin{description}

\item development environment\verb" "

\cmd{blinkenlights} was developed and tested on a single-user system running
X11.  While it works fine in this environment, it has not been run on a system
with many users or in a situation where the system console hasn't been in use
by the \cmd{blinkenlights} user.

\item long-term performance issues\verb" "

In early tests, the longer \cmd{blinkenlights} runs, the slower the updates
become.  This is {\it probably} a result of the Tk implementation or the way
Tk interfaces with X11.  This is pure supposition, though.

This performance impact is affected by a number of things, such as the number
of zones managed by \cmd{rollerd} and the length of \cmd{rollerd}'s sleep
interval.  Large numbers of zones or very short sleep intervals will increase
the possibility of \cmd{blinkenlights}' performance degrading.

This appears to have been resolved by periodically performing a complete
rebuild of the screen.  \cmd{blinkenlights} keeps track of the number of
screen updates it makes and rebuilds the screen when this count exceeds a
threshold.  The threshold is built into \cmd{blinkenlights} and stored in the
\var{\$PAINTMAX} variable.  This threshold may be adjusted if there are too
many screen rebuilds or if \cmd{blinkenlights}' performance slows too much.
Raising the number will reduce the screen rebuilds; lowering the number will
(may) increase performance.

\end{description}

{\bf SEE ALSO}

rollctl(8),
rollerd(8),
zonesigner(8)

Net::DNS::SEC::Tools::timetrans(3)

Net::DNS::SEC::Tools::keyrec(5),
Net::DNS::SEC::Tools::rollrec(5),

