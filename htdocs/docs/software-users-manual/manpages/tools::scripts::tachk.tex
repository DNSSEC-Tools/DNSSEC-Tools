\clearpage

\subsubsection{tachk}

{\bf NAME}

\cmd{tachk} - Check the validity of the trust anchors in a \path{named.conf}
file

{\bf SYNOPSIS}

\begin{verbatim}
  tachk [options] <named.conf>
\end{verbatim}

{\bf DESCRIPTION}

\cmd{tachk} checks the validity of the trust anchors in the specified
\path{named.conf} file.  The output given depends on the options selected.

Note:  This script may be removed in future releases.

{\bf OPTIONS}

\cmd{tachk} takes two types of options:  record-attribute options
and output-style options.  These option sets are detailed below.

{\bf Record-Attribute Options}

These options define which trust anchor records will be displayed.

\begin{description}

\item {\bf -valid}\verb" "

This option displays the valid trust anchors in a \path{named.conf} file.

\item {\bf -invalid}\verb" "

This option displays the invalid trust anchors in a \path{named.conf} file.

\end{description}

{\bf Output-Format Options}

These options define how the trust anchor information will be displayed.
Without any of these options, the zone name and key tag will be displayed
for each trust anchor.

\begin{description}

\item {\bf -count}\verb" "

The count of matching records will be displayed, but the matching records
will not be.

\item {\bf -long}\verb" "

The long form of output will be given.  The zone name and key tag will be
displayed for each trust anchor.

\item {\bf -terse}\verb" "

This option displays only the name of the zones selected by other options.

\item {\bf -help}\verb" "

Display a usage message.

\end{description}

{\bf SEE ALSO}

trustman(8)

