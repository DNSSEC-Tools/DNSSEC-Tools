\clearpage

\subsubsection{rollerd}

{\bf NAME}

\cmd{rollerd} - DNSSEC-Tools daemon to manage DNSSEC key rollover

{\bf SYNOPSIS}

\begin{verbatim}
  rollerd [-options] -rrfile <rollrec_file>
\end{verbatim}

{\bf DESCRIPTION}

The \cmd{rollerd} daemon manages key rollover for zones.  \cmd{rollerd} handles
both KSK and ZSK rollover, though only one rollover may take place at a time.
Initiation of KSK rollovers takes precedence over the initiation of ZSK
rollovers.  The Pre-Publish Method of key rollover is used for ZSK key
rollovers.  The Double Signature Method of key rollover is used for KSK
rollovers.  \cmd{rollerd} maintains zone rollover state in files called
\struct{rollrec} files.  The administrator may control \cmd{rollerd} with the
\cmd{rollctl} command.  These are described in their own sections below.

{\bf ZSK Rollover Using the Pre-Publish Method}

The Pre-Publish Method has four phases that are entered when it is time to
perform ZSK rollover:

\begin{enumerate}
\item wait for old zone data to expire from caches
\item sign the zone with the KSK and Published ZSK
\item wait for old zone data to expire from caches
\item adjust keys in keyrec and sign the zone with new Current ZSK
\end{enumerate}

\cmd{rollerd} uses the \cmd{zonesigner} command during ZSK rollover phases 2
and 4.  \cmd{zonesigner} will generate keys as required and sign the zone
during these two phases.

The Pre-Publish Method of key rollover is defined in the Step-by-Step DNS
Security Operator Guidance Document.  See that document for more detailed
information.

{\bf KSK Rollover Using the Double Signature Method}

The Double Signature Method has seven phases that are entered when it is time
to perform ZSK rollover:

\begin{enumerate}
\item wait for old zone data to expire from caches
\item generate a new (published) KSK
\item wait for the old DNSKEY RRset to expire from caches
\item roll the KSKs
\item transfer new keyset to the parent
\item wait for parent to publish the new DS record
\item reload the zone
\end{enumerate}

\cmd{rollerd} uses the \cmd{zonesigner} command during KSK rollover phases 2
and 4.  \cmd{zonesigner} will generate keys as required and sign the zone
during these two phases.

Currently, steps 5 and 6 are handled manually.  In step 5, \cmd{rollerd}
informs the administrator that the zone's keyset must be transferred to its
parent in order for rollover to continue.  In step 6, after the parent has
published a new DS record, the administrator uses \cmd{rollctl} to inform
\cmd{rollerd} that the DS record has been published and rollover may continue.

The Double Signature Method of key rollover is defined in the Step-by-Step
DNS Security Operator Guidance Document.  See that document for more detailed
information.

{\bf \struct{rollrec} Files}

The zones to be managed by \cmd{rollerd} are described in a \struct{rollrec}
file.  Each zone's entry contains data needed by \cmd{rollerd} and some data
useful to a user.  Below is a sample \struct{rollrec} entry:

\begin{verbatim}
        roll "example.com"
                zonefile        "example.com.signed"
                keyrec          "example.com.krf"
                directory        "dir-example.com"
                kskphase        "0"
                zskphase        "3"
                ksk_rollsecs    "1172614842"
                ksk_rolldate    "Tue Feb 27 22:20:42 2007"
                zsk_rollsecs    "1172615087"
                zsk_rolldate    "Tue Feb 27 22:24:47 2007"
                maxttl          "60"
                display         "1"
                phasestart      "Tue Feb 27 22:25:07 2007"
\end{verbatim}

The first line gives the \struct{rollrec} entry's name.  The following lines
give the zone's signed zone file, \struct{keyrec} file, the current rollover
phases, the rollover timestamps, and other information.

If either of the {\it zonefile} or \struct{keyrec} files do not exist, then a
``roll'' \struct{rollrec} will be changed into a ``skip'' \struct{rollrec}.
That record will not be processed.

A more detailed explanation may be found in \struct{rollrec(5)}.

{\bf Directories}

\cmd{rollerd}'s execution directory is either the directory in which it is
executed or the directory passed in the {\it -directory} command-line option.
Any files used by \cmd{rollerd} that were not specified with absolute paths use
this directory as their base.

The {\it directory} field informs \cmd{rollerd} where the zone's files may be
found.  For that zone, \cmd{rollerd} will move into that directory, then
return to its execution directory when it finishes rollover operations for
that zone.  If the {\it directory} value is a relative path, it will be
appended to \cmd{rollerd}'s execution directory.  If the {\it directory} value
is an absolute path, it will be used as is.

{\bf Controlling \cmd{rollerd} with \cmd{rollctl}}

The \cmd{rollctl} command is used to control the behavior of \cmd{rollerd}.
A number of commands are available, such as starting or stopping rollover for
a selected zone or all zones, turning on or off a GUI rollover display, and
halting \cmd{rollerd} execution.  The communications path between
\cmd{rollerd} and \cmd{rollctl} is operating system-dependent.  On Unix-like
systems, it is a Unix pipe that should {\bf only} be writable by root.
A more detailed explanation of \cmd{rollctl} may be found in \cmd{rollctl(8)}.

{\bf A Note About Files and Filenames}

There are a number of files and filenames used by \cmd{rollerd} and
\cmd{zonesigner}.  The user must be aware of the files used by these programs,
where the files are located, and where the programs are executed.

By default, \cmd{rollerd} will change directory to the DNSSEC-Tools directory,
though this may be changed by the {\it -directory} option.  Any programs
started by \cmd{rollerd}, most importantly \cmd{zonesigner}, will run in this
same directory.  If files and directories referenced by these programs are
named with relative paths, those paths must be relative to this directory.

The \struct{rollrec} entry name is used as a key to the \struct{rollrec} file
and to the zone's \struct{keyrec} file.  This entry does not have to be the
name of the entry's domain, but it is a very good idea to make it so.
Whatever is used for this entry name, the same name {\bf must} be used for the
zone \struct{keyrec} in that zone's \struct{keyrec} file.

It is probably easiest to store \struct{rollrec} files, \struct{keyrec} files,
zone files, and key files in a single directory.

{\bf INITIALIZATION AND USAGE}

The following steps must be taken to initialize and use \cmd{rollerd}.  This
assumes that zone files have been created, and that BIND and DNSSEC-Tools have
been installed.

\begin{description}

\item 0. sign zones\verb" "

The zones to be managed by \cmd{rollerd} must be signed.  Use \cmd{zonesigner}
to create the signed zone files and the \struct{keyrec} files needed by
\cmd{rollerd}.  The \struct{rollrec} file created in the next step {\bf must}
use the \struct{keyrec} file names and the signed zone file names created
here.

\item 1. create \struct{rollrec} file\verb" "

Before \cmd{rollerd} may be used, a \struct{rollrec} file must first be
created.  While this file may be built by hand, the \cmd{rollinit} command
was written specifically to build the file.

\item 2. select operational parameters\verb" "

A number of \cmd{rollerd}'s operational parameters are taken from the
DNSSEC-Tools configuration file.  However, these may be overridden
by command-line options.  See the OPTIONS section below for more details.
If non-standard parameters are desired to always be used, the appropriate
fields in the DNSSEC-Tools configuration file may be modified to use these
values.

\item 3. install the rollover configuration\verb" "

The complete rollover configuration -- \cmd{rollerd}, \struct{rollrec} file,
DNSSEC-Tools configuration file values, zone files -- should be installed.  
The appropriate places for these locations are both installation-dependent
and operating system-dependent.

\item 4. test the rollover configuration\verb" "

The complete rollover configuration should be tested.  

Edit the zone files so that their zones have short TTL values.  A minute TTL
should be sufficient.  Test rollovers of this speed should {\bf only} be done
in a test environment without the real signed zone.

Run the following command:

\begin{verbatim}
    rollerd -rrfile test.rollrec -logfile - -loglevel info -sleep 60
\end{verbatim}

This command assumes the test \struct{rollrec} file is \path{test.rollrec}.
It writes a fair amount of log messages to the terminal, and checks its queue
every 60 seconds.  Follow the messages to ensure that the appropriate actions,
as required by the Pre-Publish Method, are taking place.

\item 5. set \cmd{rollerd} to start at boot\verb" "

Once the configuration is found to work, \cmd{rollerd} should be set to start
at system boot.  The actual operations required for this step are operating
system-dependent.

\item 6. reboot and verify\verb" "

The system should be rebooted and the \cmd{rollerd} logfile checked to ensure
that \cmd{rollerd} is operating properly.

\end{description}

{\bf OPTIONS}

The following options are recognized:

\begin{description}

\item {\bf -rrfile rollrec\_file}\verb" "

Name of the \struct{rollrec} file to be processed.  This is the only required
``option''.

\item {\bf -directory dir}\verb" "

Sets the \cmd{rollerd} execution directory.  This must be a valid directory.

\item {\bf -logfile log\_file}\verb" "

Sets the \cmd{rollerd} log file to {\it log\_file}.  This must be a valid
logging file, meaning that if {\it log\_file} already exists, it must be a
regular file.  The only exceptions to this are if {\it logfile} is
\path{/dev/stdout}, \path{/dev/tty}, \path{-}.  Of these three, using a {\it
log\_file} of \path{-} is preferable since Perl will properly convert the
\path{-} to the process' standard output.

\item {\bf -loglevel level}\verb" "

Sets \cmd{rollerd}'s logging level to {\it level}.  \perlmod{rollmgr.pm(3)}
contains a list of the valid logging levels.

\item {\bf -sleep sleeptime}\verb" "

Sets \cmd{rollerd}'s sleep time to {\it sleeptime}.  The sleep time is the
amount of time (in seconds) \cmd{rollerd} waits between processing its
\struct{rollrec}-based queue.

\item {\bf -parameters}\verb" "

Prints a set of \cmd{rollerd} parameters and then exits.

\item {\bf -display}\verb" "

Starts the \cmd{blinkenlights} graphical display program to show the status of
zones managed by \cmd{rollerd}.

\item {\bf -help}\verb" "

Display a usage message.

\item {\bf -verbose}\verb" "

Verbose output will be given.

\end{description}

{\bf ASSUMPTIONS}

\cmd{rollerd} uses the \cmd{rndc} command to communicate with the BIND
\cmd{named} daemon.  Therefore, it assumes that appropriate measures have
been taken so that this communication is possible.

{\bf KNOWN PROBLEMS}

The following problems (or potential problems) are known:

\begin{itemize}

\item Any process that can write to the rollover socket can send commands to
\cmd{rollerd}.  This is probably not a Good Thing.

\item Very little testing was done with zone files and key files not in the
process' directory.

\end{itemize}

{\bf POSSIBLE ENHANCEMENTS}

The following potential enhancements may be made:

\begin{itemize}

\item It'd be good to base \cmd{rollerd}'s sleep time on when the next
operation must take place, rather than a simple seconds count.

\end{itemize}

{\bf SEE ALSO}

blinkenlights(8),
named(8),
rndc(8),
rollchk(8),
rollctl(8),
rollinit(8),
zonesigner(8)

Net::DNS::SEC::Tools::conf.pm(3),
Net::DNS::SEC::Tools::defaults.pm(3),\\
Net::DNS::SEC::Tools::keyrec.pm(3),
Net::DNS::SEC::Tools::rolllog.pm(3),\\
Net::DNS::SEC::Tools::rollmgr.pm(3),
Net::DNS::SEC::Tools::rollrec.pm(3)

rollrec(5)

