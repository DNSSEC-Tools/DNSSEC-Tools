\clearpage

\subsubsection{\bf rollerd}

{\bf NAME}

\cmd{rollerd} - DNSSEC-Tools daemon to manage DNSSEC key rollover.

{\bf SYNOPSIS}

\begin{verbatim}
    rollerd [-options] -rrfile <rollrec_file>
\end{verbatim}

{\bf DESCRIPTION}

The \cmd{rollerd} daemon manages key rollover for zones.  The Pre-Publish
Method of key rollover is used for ZSK key rollovers.  (Currently,
\cmd{rollerd} only handles ZSK rollover.) This method has four phases that are
entered when it is time to perform the ZSK rollover:

\begin{enumerate}
\item wait for old zone data to expire from caches
\item sign the zone with the KSK and Published ZSK
\item wait for old zone data to expire from caches
\item adjust keys in keyrec and sign the zone with new Current ZSK
\end{enumerate}

\cmd{rollerd} uses the \cmd{zonesigner} command during rollover phases 2 and 4.
\cmd{zonesigner} will generate keys as required and sign the zone during these
two phases.

The Pre-Publish Method of key rollover is defined in the Step-by-Step DNS
Security Operator Guidance Document.  See that document for more detailed
information.

The zones to be managed by \cmd{rollerd} are defined in a {\it rollrec} file.
Each zone's entry contains data needed by \cmd{rollerd} and some data useful
to a user.  Below is a sample {\it rollrec} entry:

\begin{verbatim}
        roll "example.com"
                zonefile        "example.com.signed"
                keyrec          "example.com.krf"
                curphase        "3"
                maxttl          "2400"
                phasestart      "Thu May  4 19:19:21 2006"
\end{verbatim}

The first line gives the {\it rollrec} entry's name.  The ``roll'' keyword
indicates that \cmd{rollerd} should include the zone in its roll queue.  Using
``skip'' in place of ``roll'' allows a zone to be stored in the {\it rollrec}
file, but it will not be included in rollover processing.  The first three
fields tell \cmd{rollerd} where to find example.com's signed zone file and
{\it keyrec} file and the zone's current rollover phase.  The last two are
for reference by the user.  The {\it maxttl} field is derived from the signed
zone file.

If either of the {\it zonefile} or {\it keyrec} files do not exist, then a
``roll'' {\it rollrec} will be changed into a ``skip'' {\it rollrec}.  That
record will not be processed.

The \cmd{rollctl} command is used to control the behavior of \cmd{rollerd}.
A number of commands are available, such as starting or stopping rollover for
a selected zone or all zones, turning on or off a GUI rollover display, and
halting \cmd{rollerd} execution.  The communications path between
\cmd{rollerd} and \cmd{rollctl} is operating system-dependent.  On Unix-like
systems, it is a Unix pipe that should {\bf only} be writable by root.

\clearpage
{\bf A Note About Files and Filenames}

There are a number of files and filenames used by \cmd{rollerd} and
\cmd{zonesigner}.  The user must be aware of the files used by these programs,
where the files are located, and where the programs are executed.

By default, \cmd{rollerd} will change directory to the DNSSEC-Tools directory,
though this may be changed by the {\it -directory} option.  Any programs
started by \cmd{rollerd}, most importantly \cmd{zonesigner}, will run in this
same directory.  If files and directories referenced by these programs are
named with relative paths, those paths must be relative to this directory.

The {\it rollrec} entry name is used as a key to the {\it rollrec} file and to
the zone's {\it keyrec} file.  This entry does not have to be the name of the
entry's domain, but it is a very good idea to make it so.  Whatever is used
for this entry name, the same name {\bf must} be used for the zone
{\it keyrec} in that zone's {\it keyrec} file.

It is probably easiest to store {\it rollrec} files, {\it keyrec} files, zone
files, and key files in a single directory.

{\bf INITIALIZATION AND USAGE}

The following steps must be taken to initialize and use \cmd{rollerd}.  This
assumes that zone files have been created, and that BIND and DNSSEC-Tools
have been installed.

\begin{enumerate}

\item sign zones\verb" "

The zones to be managed by \cmd{rollerd} must be signed.  Use \cmd{zonesigner}
to create the signed zone files and the {\it keyrec} files needed by
\cmd{rollerd}.  The {\it rollrec} file created in the next step {\bf must} use
the {\it keyrec} file names and the signed zone file names created here.

\item create {\it rollrec} file\verb" "

Before \cmd{rollerd} may be used, a {\it rollrec} file must first be created.
While this file may be built by hand, the \cmd{rollinit} command was
written specifically to build the file.

\item select operational parameters\verb" "

A number of \cmd{rollerd}'s operational parameters are taken from the
DNSSEC-Tools configuration file.  However, these may be overridden
by command-line options.  See the {\bf OPTIONS} section below for more details.
If non-standard parameters are desired to always be used, the appropriate
fields in the DNSSEC-Tools configuration file may be modified to use these
values.

\item install the rollover configuration\verb" "

The complete rollover configuration -- \cmd{rollerd}, {\it rollrec} file,
DNSSEC-Tools configuration file values, zone files -- should be installed.  
The appropriate places for these locations are both installation-dependent
and operating system-dependent.

\item test the rollover configuration\verb" "

The complete rollover configuration should be tested.  

Edit the zone files so that their zones have short TTL values.  A one-minute
TTL should be sufficient.  Test rollovers of this speed should {\bf only} be
done in a test environment without the real signed zone.

Run the following command:

\begin{verbatim}
    rollerd -rrfile test.rollrec -logfile - -loglevel info -sleep 60
\end{verbatim}

This command assumes the test {\it rollrec} file is \path{test.rollrec}.  It
writes a fair amount of log messages to the terminal, and checks its queue
every 60 seconds.  Follow the messages to ensure that the appropriate actions,
as required by the Pre-Publish Method, are taking place.

\item set \cmd{rollerd} to start at boot\verb" "

Once the configuration is found to work, \cmd{rollerd} should be set to start
at system boot.  The actual operations required for this step are operating
system-dependent.

\item reboot and verify\verb" "

The system should be rebooted and the \cmd{rollerd} logfile checked to ensure
that \cmd{rollerd} is operating properly.

\end{enumerate}

{\bf OPTIONS}

The following options are recognized:

\begin{description}

\item {\it -rrfile rollrec\_file}\verb" "

Name of the {\it rollrec} file to be processed.  This is the only required
``option''.

\item {\it -directory dir}\verb" "

Sets the \cmd{rollerd} execution directory.  This must be a valid directory.

\item {\it -logfile log\_file}\verb" "

Sets the \cmd{rollerd} log file to {\it log\_file}.  This must be a valid
logging file, meaning that if {\it logfile} already exists, it must be a
regular file.  The only exceptions to this are if {\it logfile} is
\path{/dev/stdout}, \path{/dev/tty}, and \path{-}.
Of these three, using a {\it logfile} of \path{-} is preferable since Perl
will properly convert the \path{-} to the process' standard output.


\item {\it -loglevel level}\verb" "

Sets \cmd{rollerd}'s logging level to {\it level}.  \perlmod{rollmgr.pm(3)}
contains a list of valid logging levels.

\item {\it -sleep sleeptime}\verb" "

Sets \cmd{rollerd}'s sleep time to {\it sleeptime}.  The sleep time is the
amount of time \cmd{rollerd} waits between processing its {\it rollrec}-based
queue.

\item {\it -parameters}\verb" "

Prints a set of \cmd{rollerd} parameters and then exits.

\item {\it -display}\verb" "

Starts the \cmd{blinkenlights} graphical display program to show the status of
zones managed by \cmd{rollerd}.

\item -{\it help}\verb" "

Display a usage message.

\item {\it -verbose}\verb" "

Verbose output will be given.

\end{description}

{\bf ASSUMPTIONS}

\cmd{rollerd} uses the \cmd{rndc} command to communicate with the BIND
\cmd{named} daemon.  Therefore, it assumes that appropriate measure have been
taken so that this communication is possible.

{\bf KNOWN PROBLEMS}

The following problems (or potential problems) are known:

\begin{description}

\item - Only deals with ZSK rollover.

\item - Any process that can write to the rollover socket can send commands to
\cmd{rollerd}.  This is probably not a Good Thing.

\item - No testing with zone files and key files not in the process' directory.

\end{description}

{\bf POSSIBLE ENHANCEMENTS}

The following potential enhancements may be made:

\begin{description}

\item - It'd be good to base \cmd{rollerd}'s sleep time on when the next
operation must take place, rather than a simple seconds count.

\item - It'd be nice to allow each {\it rollrec} entry to specify its own
logging level.

\end{description}

{\bf SEE ALSO}

\cmd{blinkenlights(8)},
\cmd{named(8)},
\cmd{rndc(8)},
\cmd{rollchk(8)},
\cmd{rollctl(8)},	\\
\cmd{rollinit(8)},
\cmd{zonesigner(8)}

\perlmod{Net::DNS::SEC::Tools::conf.pm(3)},
\perlmod{Net::DNS::SEC::Tools::defaults.pm(3)}, \\
\perlmod{Net::DNS::SEC::Tools::keyrec.pm(3)},
\perlmod{Net::DNS::SEC::Tools::rollmgr.pm(3)}, \\
\perlmod{Net::DNS::SEC::Tools::rollrec.pm(3)}

\path{rollrec(5)}
