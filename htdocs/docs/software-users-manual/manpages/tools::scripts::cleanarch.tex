\clearpage

\subsubsection{cleanarch}

{\bf NAME}

\cmd{cleanarch} - Clean a DNSSEC-Tools key archive of old keys

{\bf SYNOPSIS}

\begin{verbatim}
  cleanarch [options] <keyrec-file | rollrec-file>
\end{verbatim}

{\bf DESCRIPTION}

\cmd{cleanarch} deletes old keys from a DNSSEC-Tools key archive.  Key ``age''
and archives are determined by options and arguments.

Command line options and arguments allow selection of archives, keys to
delete, amount of output to provide.  The options are divided into three
groups:  archive selection, key selection, and output format.  Complete
information on options is provided in the OPTIONS section.

\cmd{cleanarch} takes a single argument (as distinguished from an option.)
This argument may be either a \struct{keyrec} file or a \struct{rollrec} file.
If the file is a \struct{keyrec} file, the archive directory for its zone
\struct{keyrec}s are added to the list of archives to clean.  If the file is a
\struct{rollrec} file, \struct{keyrec} files for its zones are searched for
zone's the archive directory, and those directories are added to the list of
archives to clean.  If a zone does not have an archive directory explicitly
defined, then the DNSSEC-Tools default will be cleaned.  The archives
specified by this argument may be modified by archive-selection options.

The archive-selection options combine with the \struct{keyrec} or
\struct{rollrec} file to select a set of archive directories to clean.
(Some options can take the place of the file argument.)

The key-selection options allow the set of keys to be deleted to contain an
entire archive, a particular zone's keys, or all the keys prior to a certain
date.

The output-format options sets how much output will be given.  Without any
options selected, the names of keys will be printed as they are deleted.  If
the {\it -verbose} option is given, then the directories selected for
searching and the keys selected for deletion will be printed.  If the {\it
-dirlist} option is given, then the directories selected for searching will be
printed and no other action will be taken.  If the {\it -list} option is
given, then the keys selected for deletion will be printed and no other action
will be taken.

{\bf OPTIONS}

{\bf Archive-Selection Options}

The following options allow the user to select the archives to be cleaned.

\begin{description}

\item {\bf -archive directory}\verb" "

This option specifies an archive directory to be cleaned.

\item {\bf -defarch}\verb" "

This option indicates that the default archive directory (named in the
DNSSEC-Tools configuration file) should be cleaned. 

\item {\bf -zone zone}\verb" "

This option indicates that {\it zone} is the only zone whose archive will be
cleaned.  If the archive directory is shared by other zones then their keys
may also be deleted.

\end{description}

{\bf Key-Selection Options}

The following options allow the user to select the keys to be deleted.

\begin{description}

\item {\bf -all}\verb" "

Deletes all keys in the selected archives.
This option may not be used with any other key-selection options.

\item {\bf -days days}\verb" "

Deletes all keys except those whose modification date is within the
{\it days} full days preceding the current day.

\item {\bf -onezone zone}\verb" "

Only keys with {\it zone} in the key's filename are deleted.
This is intended for use in cleaning a multi-zone key archive.

This does not validate that {\it zone} is an actual zone.  {\bf Any} string
can be used here.  For example, using ``private'' will select old private key
files for deletion and using ``com'' will select any filename that contains
``com''.

\end{description}

{\bf Options for Output Control}

The following options allow the user to control \cmd{cleanarch}'s output.

\begin{description}

\item {\bf -dirlist}\verb" "

This option lists the selected archive directories.  No other action is taken.

\item {\bf -list}\verb" "

This option lists the selected keys.  No other action is taken.

\item {\bf -quiet}\verb" "

Display no output.

\item {\bf -verbose}\verb" "

Display verbose output.

\item {\bf -Version}\verb" "

Display the \cmd{cleanarch} and DNSSEC-Tools versions.

\item {\bf -help}\verb" "

Display a usage message and exit.

\end{description}

{\bf WARNINGS}

The user is advised to invest a bit of time testing this tool {\bf prior}
to putting it into production use.  Once a key is deleted, it is {\bf gone}.
Some may find this to be detrimental to the health of their DNSSEC-Tools
installation.

{\bf SEE ALSO}

cleankrf(8),
lskrf(8),
zonesigner(8)

Net::DNS::SEC::Tools::keyrec.pm(3),
Net::DNS::SEC::Tools::rollrec.pm(3)

dnssec-tools.conf(5),
keyrec(5),
rollrec(5)

