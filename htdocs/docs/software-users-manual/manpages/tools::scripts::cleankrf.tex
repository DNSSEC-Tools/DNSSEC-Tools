\clearpage

\subsubsection{\bf cleankrf}

{\bf NAME}

\cmd{cleankrf} - Clean a DNSSEC-Tools {\it keyrec} files of old data.

{\bf SYNOPSIS}

\begin{verbatim}
    cleankrf [options] <keyrec-files>
\end{verbatim}

{\bf DESCRIPTION}

\cmd{cleankrf} cleans old data out of a set of DNSSEC-Tools {\it keyrec} files.
The old data are orphaned signing sets, orphaned keys, and obsolete keys.

Orphaned signing sets are set {\it keyrec}s unreferenced by a zone {\it keyrec}.

Orphaned keys are KSK key {\it keyrec}s unreferenced by a zone {\it keyrec}
and ZSK key {\it keyrec}s unreferenced by any set {\it keyrec}s.

Obsolete keys are ZSK key {\it keyrec}s with a {\it keyrec\_type} of {\bf
zskobs}.

\cmd{cleankrf}'s exit code is the count of orphaned and obsolete {\it keyrec}s
found.

{\bf OPTIONS}

\begin{description}

\item {\it -count}\verb" "

Display a final count of old {\it keyrec}s found in the {\it keyrec} files.
This option allows the count to be displayed even if the {\it -quiet} option
is given.

\item {\it -list}\verb" "

The key {\it keyrec}s are checked for old {\it keyrec}s, but they are not
removed from the {\it keyrec} file.  The names of the old {\it keyrec}s are
displayed.

\item {\it -rm}\verb" "

Delete the key files, both {\bf .key} and {\bf .private}, from orphaned and
expired {\it keyrec}s.

\item {\it -quiet}\verb" "

Display no output.

\item {\it -verbose}\verb" "

Display output about referenced keys and unreferenced keys.

\item {\it -help}\verb" "

Display a usage message.

\end{description}

{\bf SEE ALSO}

\cmd{fixkrf(8)},
\cmd{lskrf(8)},
\cmd{zonesigner(8)}

\perlmod{Net::DNS::SEC::Tools::keyrec.pm(3)}

\path{keyrec(5)}
