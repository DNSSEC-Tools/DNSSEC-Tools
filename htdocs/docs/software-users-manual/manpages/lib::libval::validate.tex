\clearpage

\subsubsection{\bf validate}

{\bf NAME}

\cmd{validate} - Query the Domain Name System and display results of the
DNSSEC validation process

{\bf SYNOPSIS}

\begin{verbatim}
    validate

    validate [options] DOMAIN_NAME
\end{verbatim}

{\bf DESCRIPTION}

\cmd{validate} is a diagnostic tool built on top of the DNSSEC validator.  If
given a domain name argument (as {\it DOMAIN\_NAME}), \cmd{validate} queries
the DNS for that domain name.  It outputs the series of responses that were
received from the DNS and the DNSSEC validation results for each domain name.
An examination of the queries and validation results can help an administrator
uncover errors in DNSSEC configuration of DNS zones.

If no options are specified and no {\it DOMAIN\_NAME} argument is given,
\cmd{validate} will perform a series of pre-defined test queries against the
{\it test.dnssec-tools.org} zone.  This serves as a test-suite for the
validator.  If any options are specified (e.g., configuration file locations),
{\it -s} or {\it --selftest} must be specified to run the test-suite.

{\bf OPTIONS}

\begin{description}

\item {\it -c CLASS, --class=CLASS}\verb" "

This option can be used to specify the DNS class of the Resource Record
queried.  If this option is not given, the default class {\bf IN} is used.

\item {\it -h, --help}\verb" "

Display the help and exit.

\item {\it -m, --merge}\verb" "

When this option is given, \cmd{validate} will merge different RRsets in the
response into a single answer.  If this option is not given, each RRset is
output as a separate response.  This option makes sense only when used with
the {\it -p} option.

\item {\it -p, --print}\verb" "

Print the answers and validation results.  By default, \cmd{validate} just
outputs a series of responses and their validation results on {\it stderr}.
When the {\it -p} option is used, \cmd{validate} will also output the final
result on {\it stdout}.

\item {\it -t TYPE, --type=TYPE}\verb" "

This option can be used to specify the DNS type of the Resource Record
queried.  If this option is not given, \cmd{validate} will query for the
{\bf A} record for the given {\it DOMAIN\_NAME}.

\item {\it -v FILE, --dnsval-conf=FILE}\verb" "

This option can be used to specify the location of the \path{dnsval.conf}
configuration file.

\item {\it -r FILE, --resolv-conf=FILE}\verb" "

This option can be used to specify the location of the \path{resolv.conf}
configuration file containing the name servers to use for lookups.

\item {\it -i FILE, --root-hints=FILE}\verb" "

This option can be used to specify the location of the \path{root.hints}
configuration file, containing the root name servers. This is only used when
no name server is found, and \cmd{validate} must do recursive lookups itself.

\item {\it -s, --selftest}\verb" "

This option can be used to specify that the application should perform its
test-suite against the {\it dnssec-tools.org} test domain. If the name servers
configured in the system \path{resolv.conf} do not support DNSSEC, use the
{\it -r} and {\it -i} options to enable \cmd{validate} to use its own internal
recursive resolver.

\item {\it -T testcase number>}\verb" "

This option can be used to run a specific test from the test-suite.

\item {\it -o, --output=debug-level:dest-type[:dest-options]}\verb" "

\var{debug-level} is 1-7, corresponding to syslog levels ALERT-DEBUG. \\
\var{dest-type} is one of {\it file}, {\it net}, {\it syslog},.
{\it stderr}, {\it stdout}. \\
\var{dest-options} depends on \var{dest-type}:
\begin{verbatim}
    file:<file-name>              (opened in append mode)
    net[:<host-name>:<host-port>] (127.0.0.1:1053)
    syslog[:facility]             (0-23 (default 1 USER))
\end{verbatim}

\end{description}

{\bf PRE-REQUISITES}

\lib{libval(3)}

{\bf SEE ALSO}

\cmd{drawvalmap(1)}

\lib{libval(3)}, \lib{val\_query(3)}
