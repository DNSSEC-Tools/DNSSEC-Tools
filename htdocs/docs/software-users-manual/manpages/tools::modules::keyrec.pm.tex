\clearpage

\subsection{\bf keyrec.pm Module}

{\bf NAME}

\perlmod{Net::DNS::SEC::Tools::keyrec.pm} - DNSSEC-Tools {\it keyrec} file operations

{\bf SYNOPSIS}

\begin{verbatim}
    use Net::DNS::SEC::Tools::keyrec;

    keyrec_creat("localzone.keyrec");
    keyrec_open("localzone.keyrec");
    keyrec_read("localzone.keyrec");

    @krnames = keyrec_names();

    $krec = keyrec_fullrec("example.com");
    %keyhash = %$krec;
    $zname = $keyhash{"algorithm"};

    $val = keyrec_recval("example.com","zonefile");

    keyrec_add("zone","example.com",\%zone_krfields);
    keyrec_add("key","Kexample.com.+005+12345",\%keydata);

    keyrec_del("example.com");
    keyrec_del("Kexample.com.+005+12345");

    keyrec_setval("zone","example.com","zonefile","db.example.com");

    $setname = keyrec_signset_newname("example.com");

    keyrec_signset_new("zone","example-keys");

    keyrec_signset_addkey("example-keys","Kexample.com+005+12345",
                                            "Kexample.com+005+54321");
    keyrec_signset_addkey("example-keys",@keylist);

    keyrec_signset_delkey("example-keys","Kexample.com+005+12345");

    $flag = keyrec_signset_haskey("example-keys","Kexample.com+005+12345");

    keyrec_signset_clear("example-keys","Kexample.com+005+12345");

    @signset = keyrec_signsets();

    keyrec_settime("zone","example.com");
    keyrec_settime("set","signing-set-42");
    keyrec_settime("key","Kexample.com.+005+76543");

    @keyfields = keyrec_keyfields();
    @zonefields = keyrec_zonefields();

    keyrec_write();
    keyrec_saveas("filecopy.krf);
    keyrec_close();
    keyrec_discard();
\end{verbatim}

{\bf DESCRIPTION}

The \perlmod{Net::DNS::SEC::Tools::keyrec} module manipulates the contents of
a DNSSEC-Tools {\it keyrec} file.  {\it keyrec} files contain data about
zones signed by and keys generated by the DNSSEC-Tools programs.  Module
interfaces exist for looking up {\it keyrec} records, creating new
records, and modifying existing records.

A {\it keyrec} file is organized in sets of {\it keyrec} records.  Each {\it
keyrec} must be either of {\it zone} type, {\it set}, or {\it key} type.  Zone
{\it keyrec}s describe how zones were signed.  Set {\it keyrec}s gather key
{\it keyrec}s into groups.  Key {\it keyrec}s describe how encryption keys
were generated.  A {\it keyrec} consists of a set of keyword/value entries.
The following is an example of a key {\it keyrec}:

\begin{verbatim}
    key     "Kexample.com.+005+30485"
          zonename        "example.com"
          keyrec_type     "ksk"
          algorithm       "rsasha1"
          random          "/dev/urandom"
          ksklength       "512"
          ksklife         "15768000"
          keyrec_gensecs  "1101183727"
          keyrec_gendate  "Tue Nov 23 04:22:07 2004"
\end{verbatim}

The first step in using this module {\bf must} be to create a new {\it keyrec}
file or open and read an existing one.  The \func{keyrec\_creat()} interface
creates a {\it keyrec} file if it does not exist and opens it.  The
\func{keyrec\_open()} interface opens an existing {\it keyrec} file.  The
\func{keyrec\_read()} interface reads the file and parses it into an internal
format.  The file's records are copied into a hash table (for easy reference
by the \perlmod{Net::DNS::SEC::Tools::keyrec} routines) and in an array (for
preserving formatting and comments.)

After the file has been read, the contents are referenced using
\func{keyrec\_fullrec()} and \func{keyrec\_recval()}.  The contents
are modified using \func{keyrec\_add()}, and \func{keyrec\_setval()}.
A {\it keyrec}'s timestamp may be updated to the current time with
\func{keyrec\_settime()}.  {\it keyrec}s may be deleted with the
\func{keyrec\_del()} interface.

If the {\it keyrec} file has been modified, it must be explicitly written or
the changes are not saved.  \func{keyrec\_write()} saves the new contents to
disk.  \func{keyrec\_saveas()} saves the in-memory {\it keyrec} contents to the
specified file name, without affecting the original file.

\func{keyrec\_close()} saves the file and close the Perl file handle to the
{\it keyrec} file.  If a {\it keyrec} file is no longer wanted to be open, yet
the contents should not be saved, \func{keyrec\_discard()} gets rid of the
data, and closes the file handle {\bf without} saving any modified data.

{\bf KEYREC INTERFACES}

The interfaces to the \perlmod{Net::DNS::SEC::Tools::keyrec} module are given
below.

\begin{description}

\item \func{keyrec\_add(keyrec\_type,keyrec\_name,fields)}

This routine adds a new {\it keyrec} to the {\it keyrec} file and the internal
representation of the file contents.  The {\it keyrec} is added to both the
{\it \%keyrecs} hash table and the {\it \@keyreclines} array.

{\it keyrec\_type} specifies the type of the {\it keyrec} -- ``key'' or
``zone''.  {\it keyrec\_name} is the name of the {\it keyrec}.  {\it fields}
is a reference to a hash table that contains the name/value {\it keyrec}
fields.  The keys of the hash table are always converted to lowercase, but the
entry values are left as given.

The {\it ksklength} entry is only added if the value of the {\it keyrec\_type}
field is ``ksk''.

The {\it zsklength} entry is only added if the value of the {\it keyrec\_type}
field is ``zsk'', ``zskcur'', ``zskpub'', or ``zsknew''.

Timestamp fields are added at the end of the {\it keyrec}.  For key {\it
keyrec}s, the {\it keyrec\_gensecs} and {\it keyrec\_gendate} timestamp fields
are added.  For zone {\it keyrec}s, the {\it keyrec\_signsecs} and {\it
keyrec\_signdate} timestamp fields are added.

If a specified field isn't defined for the {\it keyrec} type, the entry isn't
added.  This prevents zone {\it keyrec} data from getting mingled with key
{\it keyrec} data.

A blank line is added after the final line of the new {\it keyrec}.  After
adding all new {\it keyrec} entries, the {\it keyrec} file is written but is
not closed.

Return values are:

\begin{verbatim}
    0 success
    -1 invalid krtype
\end{verbatim}

\item \func{keyrec\_close()}

This interface saves the internal version of the {\it keyrec} file (opened
with \func{keyrec\_creat()}, \func{keyrec\_open()} or \func{keyrec\_read()})
and closes the file handle.

\item \func{keyrec\_creat(keyrec\_file)}

This interface creates a {\it keyrec} file if it does not exist, and truncates
the file if it already exists.  It leaves the file in the open state.

\func{keyrec\_creat()} returns 1 if the file was created successfully.
It returns 0 if there was an error in creating the file.

\item \func{keyrec\_del(keyrec\_name)}

This routine deletes a {\it keyrec} from the {\it keyrec} file and the internal
representation of the file contents.  The {\it keyrec} is deleted from both
the {\it \%keyrecs} hash table and the {\it \@keyreclines} array.

Only the {\it keyrec} itself is deleted from the file.  Any associated comments
and blank lines surrounding it are left intact.

Return values are:

\begin{verbatim}
    0 successful keyrec deletion
    -1 invalid krtype (empty string or unknown name)
\end{verbatim}

\item \func{keyrec\_discard()}

This routine removes a {\it keyrec} file from use by a program.  The internally
stored data are deleted and the {\it keyrec} file handle is closed.  However,
modified data are not saved prior to closing the file handle.  Thus, modified
and new data will be lost.

\item \func{keyrec\_fullrec(keyrec\_name)}

\func{keyrec\_fullrec()} returns a reference to the {\it keyrec} specified in
{\it keyrec\_name}.

\item \func{keyrec\_keyfields()}

This routine returns a list of the recognized fields for a key {\it keyrec}.

\item \func{keyrec\_names()}

This routine returns a list of the {\it keyrec} names from the file.

\item \func{keyrec\_open(keyrec\_file)}

This interface opens an existing {\it keyrec} file.

\func{keyrec\_open()} returns 1 if the file was opened successfully.  It
returns 0 if the file does not exists or if there was an error in opening
the file.

\item \func{keyrec\_read(keyrec\_file)}

This interface reads the specified {\it keyrec} file and parses it into
a {\it keyrec} hash table and a file contents array.  \func{keyrec\_read()}
{\bf must} be called prior to any of the other
\perlmod{Net::DNS::SEC::Tools::keyrec} calls.  If another {\it keyrec} is
already open, then it is saved and closed prior to opening the new {\it
keyrec}.

Upon success, \func{keyrec\_read()} returns the number of {\it keyrec}s read
from the file.

Failure return values:

\begin{verbatim}
    -1 specified keyrec file doesn't exit
    -2 unable to open keyrec file
    -3 duplicate keyrec names in file
\end{verbatim}

\item \func{keyrec\_recval(keyrec\_name,keyrec\_field)}

This routine returns the value of a specified field in a given {\it keyrec}.
{\it keyrec\_name} is the name of the particular {\it keyrec} to consult.
{\it keyrec\_field} is the field name within that {\it keyrec}.

For example, the current {\it keyrec} file contains the following {\it keyrec}:

\begin{verbatim}
    zone        "example.com"
        zonefile    "db.example.com"
\end{verbatim}

The call:

\begin{verbatim}
    keyrec_recval("example.com","zonefile")
\end{verbatim}

will return the value ``db.example.com''.

\item \func{keyrec\_saveas(keyrec\_file\_copy)}

This interface saves the internal version of the {\it keyrec} file (opened
with \func{keyrec\_creat()}, \func{keyrec\_open()} or \func{keyrec\_read()})
to the file named in the {\it keyrec\_file\_copy} parameter.  The new file's
file handle is closed, but the original file and the file handle to the
original file are not affected.

\item \func{keyrec\_setval(keyrec\_type,keyrec\_name,field,value)}

Set the value of a {\it name/field} pair in a specified {\it keyrec}.  The
file is {\bf not} written after updating the value.  The value is saved in
both {\it \%keyrecs} and in {\it \@keyreclines}, and the file-modified flag
is set.

{\it keyrec\_type} specifies the type of the {\it keyrec}.  This is only used
if a new {\it keyrec} is being created by this call.  {\it keyrec\_name} is
the name of the {\it keyrec} that will be modified.  {\it field} is the {\it
keyrec} field which will be modified.  {\it value} is the new value for the
field.

Return values are:

\begin{verbatim}
    0 if the creation succeeded
    -1 invalid type was given
\end{verbatim}

\item \func{keyrec\_settime(keyrec\_type,keyrec\_name)}

Set the timestamp of a specified {\it keyrec}.  The file is {\bf not} written
after updating the value.  The value is saved in both {\it \%keyrecs} and in
{\it \@keyreclines}, and the file-modified flag is set.  The {\it keyrec}'s
{\it keyrec\_signdate} and {\it keyrec\_signsecs} fields are modified.

\item \func{keyrec\_write()}

This interface saves the internal version of the {\it keyrec} file (opened
with \func{keyrec\_creat()}, \func{keyrec\_open()} or \func{keyrec\_read()}).
It does not close the file handle.  As an efficiency measure, an internal
modification flag is checked prior to writing the file.  If the program has
not modified the contents of the {\it keyrec} file, it is not rewritten.

\item \func{keyrec\_zonefields()}

This routine returns a list of the recognized fields for a zone {\it keyrec}.

\end{description}

{\bf KEYREC SIGNING-SET INTERFACES}

Signing Sets are collections of encryption keys, defined by inclusion in a
particular ``set'' {\it keyrec}.  The names of the keys are in the {\it
keyrec}'s {\it keys} record, which contains the names of the key {\it
keyrec}s.  Due to the way key names are handled, the names in a Signing Set
must not contain spaces.

The Signing-Set-specific interfaces are given below.

\begin{description}

\item \func{keyrec\_signset\_newname(zone\_name)}

\func{keyrec\_signset\_newname()} creates a name for a new Signing Set.  The
name will be generated by referencing the {\it lastset} field in the {\it
keyrec} for zone {\it zone\_name}, if the {\it keyrec} has such a field.  The
set index number (described below) will be incremented and the {\it lastset}
with the new index number will be returned as the new Signing Set name.  If
the zone {\it keyrec} does not have a {\it lastset} field, then the default
set name of {\it signing-set-0} will be used.

The set index number is the first number found in the {\it lastset} field.  It
doesn't matter where in the field it is found, the first number will be
considered to be the Signing Set index.  The examples below show how this is
determined:

\begin{table}[ht]
\begin{center}
\begin{tabular}{lc}
\var{lastset} field	 & index	\\
\hline					\\
signing-set-0		 & 0		\\
signing-0-set		 & 0		\\
1-signing-0-set		 & 1		\\
signing-88-set-1	 & 88		\\
signingset4321		 & 4321		\\
\end{tabular} 
\end{center}
\end{table}

\item \func{keyrec\_signset\_new(signing\_set\_name)}

\func{keyrec\_signset\_new()} creates the Signing Set named by {\it
signing\_set\_name}.  It returns 1 if the call is successful; 0 if it is not.

\item \func{keyrec\_signset\_addkey(signing\_set\_name,key\_list)}

\func{keyrec\_signset\_addkey()} adds the keys listed in {\it key\_list} to
the Signing Set named by {\it signing\_set\_name}.  {\it key\_list} may either
be an array or a set or arguments to the routine.  The {\it keyrec} is created
if it does not already exist.  It returns 1 if the call is successful; 0 if it
is not.

\item \func{keyrec\_signset\_delkey(signing\_set\_name,key\_name)}

\func{keyrec\_signset\_delkey()} deletes the key given in {\it key\_name} to
the Signing Set named by {\it signing\_set\_name}.  It returns 1 if the call
is successful; 0 if it is not.

\item \func{keyrec\_signset\_haskey(signing\_set\_name,key\_name)}

\func{keyrec\_signset\_delkey()} returns a flag indicating if the key
specified in {\it key\_name} is one of the keys in the Signing Set named by
{\it signing\_set\_name}.  It returns 1 if the signing set has the key; 0 if
it does not.

\item \func{keyrec\_signset\_clear(keyrec\_name)}

\func{keyrec\_signset\_clear()} clears the entire signing set from the {\it
keyrec} named by {\it keyrec\_name}.  It returns 1 if the call is successful;
0 if it is not.

\item \func{keyrec\_signsets()}

\func{keyrec\_signsets()} returns the names of the signing sets in the {\it
keyrec} file.  These names are returned in an array.

\end{description}

{\bf KEYREC INTERNAL INTERFACES}

The interfaces described in this section are intended for internal use by the
\perlmod{keyrec.pm} module.  However, there are situations where external
entities may have need of them.  Use with caution, as misuse may result in
damaged or lost {\it keyrec} files.

\begin{description}

\item \func{keyrec\_init()}

This routine initializes the internal {\it keyrec} data.  Pending changes will
be lost.  An open {\it keyrec} file handle will remain open, though the data
are no longer held internally.  A new {\it keyrec} file must be read in order
to use the \perlmod{keyrec.pm} interfaces again.

\item \func{keyrec\_newkeyrec(kr\_name,kr\_type)}

This interface creates a new {\it keyrec}.  The {\it keyrec\_name} and {\it
keyrec\_hash} fields in the {\it keyrec} are set to the values of the {\it
kr\_name} and {\it kr\_type} parameters.  {\it kr\_type} must be either
``key'' or ``zone''.

Return values are:

\begin{verbatim}
    0 if the creation succeeded
    -1 if an invalid keyrec type was given
\end{verbatim}

\end{description}

{\bf KEYREC DEBUGGING INTERFACES}

The following interfaces display information about the currently parsed
{\it keyrec} file.  They are intended to be used for debugging and testing,
but may be useful at other times.

\begin{description}

\item \func{keyrec\_dump\_hash()}

This routine prints the {\it keyrec} file as it is stored internally in a hash
table.  The {\it keyrec}s are printed in alphabetical order, with the fields
alphabetized for each {\it keyrec}.  New {\it keyrec}s and {\it keyrec} fields
are alphabetized along with current {\it keyrec}s and fields.  Comments from
the {\it keyrec} file are not included with the hash table.

\item \func{keyrec\_dump\_array()}

This routine prints the {\it keyrec} file as it is stored internally in an
array.  The {\it keyrec}s are printed in the order given in the file, with the
fields ordered in the same manner.  New {\it keyrec}s are appended to the end
of the array.  {\it keyrec} fields added to existing {\it keyrec}s are added
at the beginning of the {\it keyrec} entry.  Comments and vertical whitespace
are preserved as given in the {\it keyrec} file.

\end{description}

{\bf SEE ALSO}

\perlmod{Net::DNS::SEC::Tools::keyrec.pm(3)}

\path{keyrec(5)}
