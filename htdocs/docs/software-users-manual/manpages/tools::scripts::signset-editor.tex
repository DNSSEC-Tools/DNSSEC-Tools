\clearpage

\subsubsection{signset-editor}

{\bf NAME}

\cmd{signset-editor} - DNSSEC-Tools Signing Set GUI Editor

{\bf SYNOPSIS}

\begin{verbatim}
  signset-editor <keyrec-file>
\end{verbatim}

{\bf DESCRIPTION}

\cmd{signset-editor} provides the capability for easy management of signing
sets in a GUI.  A signing set contains zero or more names of key
\struct{keyrec}s.  These sets are used by other DNSSEC-Tools utilities for
signing zones.  The signing sets found in the given \struct{keyrec} file are
displayed in a new window.  New signing sets may be created and existing
signing sets may be modified or deleted from \cmd{signset-editor}.

\cmd{signset-editor} has two display modes.  The Signing Set Display shows the
names of all the set \struct{keyrec}s in the given \struct{keyrec} file.  The
Keyrec Display shows the names of all the key \struct{keyrec}s in the given
\struct{keyrec} file.  \cmd{signset-editor} starts in Signing Set Display
mode, but the mode can be toggled back and forth as needed.

An additional toggle controls the display of additional data.  If the Extended
Data toggle is turned on, then the Signing Set Display shows the names of the
key \struct{keyrec}s in each signing set and the Keyrec Display shows the
names of each signing set each key \struct{keyrec} is in.  If the Extended
Data toggle is turned off, then the Signing Set Display only shows the names
of the set \struct{keyrec}s and the Keyrec Display only shows the names key
\struct{keyrec}s.

\cmd{signset-editor} has a small number of commands.  These commands are all
available through the menus, and most have a keyboard accelerator.  The
commands are described in the next section.

Management of signing sets may be handled using a normal text editor.
However, \cmd{signset-editor} provides a nice GUI that {\bf only} manipulates
signing sets without the potential visual clutter of the rest of the
\struct{keyrec} entries.

{\bf UNDOING MODIFICATIONS}

\cmd{signset-editor} has the ability to reverse modifications it has made to a
\struct{keyrec} file.  This historical restoration will only work for
modifications made during a particular execution of \cmd{signset-editor};
modifications made during a previous execution may not be undone.

When undoing modifications, \cmd{signset-editor} does not necessarily restore
name-ordering within a \struct{keyrec}'s {\bf signing\_set} field.  However,
the signing-set data are maintained.  This means that an ``undone''
\struct{keyrec} file may not be exactly the same, byte-for-byte, as the
original file, but the proper meaning of the data is kept.

After a ``Save'' operation, the data required for reversing modifications are
deleted.  This is not the case for the ``Save As'' operation.

{\bf COMMANDS}

\cmd{signset-editor} provides the following commands, organized by menus:

\begin{itemize}

\item {\bf Open} (File menu)\verb" "

Open a new \struct{keyrec} file.  If the specified file does not exist, the
user will be prompted for the action to take.  If the user chooses the
``Continue'' action, then \cmd{signset-editor} will continue editing the current
\struct{keyrec} file.  If the ``Quit'' action is selected, then
\cmd{signset-editor} will exit.

\item {\bf Save} (File menu)\verb" "

Save the current \struct{keyrec} file.  The data for the ``Undo Changes''
command are purged, so this file will appear to be unmodified.

Nothing will happen if no changes have been made.

\item {\bf Save As} (File menu)\verb" "

Save the current \struct{keyrec} file to a name selected by the user.

\item {\bf Quit} (File menu)\verb" "

Exit \cmd{signset-editor}.

\item {\bf Undo Changes} (Edit menu)\verb" "

Reverse modifications made to the signing sets and keyrecs.  This is {\bf only}
for the in-memory version of the \struct{keyrec} file.

\item {\bf New Signing Set} (Commands menu)\verb" "

Create a new signing set.   The user is given the option of adding key
\struct{keyrec}s to the new set.

This command is available from both viewing modes.

\item {\bf Delete Signing Set/Key} (Commands menu)\verb" "

Delete the selected signing set or key.

This command is available from both viewing modes.  If used from the Signing
Set Display mode, then all the keys in the selected signing set will be
removed from that set.  If used from the Keyrec Display mode, then the
selected key will no longer be part of any signing set.

\item {\bf Modify Signing Set/Key} (Commands menu)\verb" "

Modify the selected signing set or key.

This command is available from both viewing modes.  If used from the Signing
Set Display mode, then the selected signing set may be modified by adding keys
to that set or deleting them from that set.  If used from the Keyrec Display
mode, then the selected key may be added to or deleted from any of the defined
signing sets.

\item {\bf View Signing Sets} (Display menu)\verb" "

The main window will display the \struct{keyrec} file's signing sets.  If
Extended Data are to be displayed, then each key \struct{keyrec} in the
signing set will also be shown.  If Extended data are not to be displayed,
then only the signing set names will be shown.

This command is a toggle that switches between View Signing Sets mode and View
Keyrecs mode.

\item {\bf View Keyrecs} (Display menu)\verb" "

The main window will display the names of the \struct{keyrec} file's key
\struct{keyrec}s.  If Extended Data are to be displayed, then the name of each
signing set of the \struct{keyrec} will also be shown.  If Extended data are
not to be shown, then only the \struct{keyrec} names will be displayed.

This command is a toggle that switches between View Keyrecs mode and View
Signing Sets mode.

\item {\bf Display Extended Data} (Display menu)\verb" "

Additional data will be shown in the main window.  For Signing Sets Display
mode, the names of the signing set and their constituent key \struct{keyrec}s
will be displayed.  For Keyrec Display mode, the names of the key
\struct{keyrec}s and the Signing Sets it is in will be displayed.

This command is a toggle that switches between Extended Data display and No
Extended Data display.

\item {\bf Do Not Display Extended Data} (Display menu)\verb" "

No additional data will be shown in the main window.  For Signing Sets Display
mode, only the names of the Signing Sets will be displayed.  For Keyrec
Display mode, only the names of the \struct{keyrec}s will be displayed.

This command is a toggle that switches between No Extended Data display and
Extended Data display.

\item {\bf Help} (Help menu)\verb" "

Display a help window.

\end{itemize}

\eject

{\bf KEYBOARD ACCELERATORS}

Below are the keyboard accelerators for the \cmd{signset-editor} commands:

\begin{table}[ht]
\begin{center}
\begin{tabular}{|c|l|}
\hline
{\bf Accelerator} & {\bf Function} \\
\hline
Ctrl-D & Delete Signing Set \\
Ctrl-E & Display Extended Data / Do Not Display Extended Data \\
Ctrl-H & Help \\
Ctrl-M & Modify Signing Set \\
Ctrl-N & New Signing Set \\
Ctrl-O & Open \\
Ctrl-Q & Quit \\
Ctrl-S & Save \\
Ctrl-U & Undo Changes \\
Ctrl-V & View Signing Sets / View Keyrecs \\
Ctrl-W & Close Window (New Signing Set, Modify Signing Set, Help) \\
\hline
\end{tabular}
\end{center}
\caption{Keyboard Accelerators for \cmd{signset-editor}}
\end{table}

These accelerators are all lowercase letters.

{\bf REQUIREMENTS}

\cmd{signset-editor} is implemented in Perl/Tk, so both Perl and Perl/Tk must
be installed on your system.

{\bf SEE ALSO}

cleankrf(8),
fixkrf(8),
genkrf(8),
krfcheck(8),
lskrf(1),
zonesigner(8)

Net::DNS::SEC::Tools::keyrec(3)

file-keyrec(5)

