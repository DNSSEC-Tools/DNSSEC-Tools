\clearpage

\subsection{{\bf val\_getaddrinfo()} DNSSEC-validated address translation}


{\bf NAME}

val\_getaddrinfo, val\_x\_getaddrinfo, val\_get\_addrinfo\_dnssec\_status,
val\_dupaddrinfo, val\_freeaddrinfo - get DNSSEC-validated network address
and service translation

{\bf SYNOPSIS}

\begin{verbatim}
  #include <validator.h>

  int val_getaddrinfo(const char *nodename, const char *servname,
                    const struct addrinfo *hints,
                    struct addrinfo **res);

  int val_x_getaddrinfo(const struct val_context *ctx,
                    const char *nodename, const char *servname,
                    const struct addrinfo *hints,
                    struct addrinfo **res);

  int val_get_addrinfo_dnssec_status(const struct addrinfo *ainfo);

  struct addrinfo* val_dupaddrinfo(const struct addrinfo *ainfo);

  void val_freeaddrinfo(struct addrinfo *ainfo);
\end{verbatim}

{\bf DESCRIPTION}

{\bf val\_getaddrinfo()} is a DNSSEC-aware version of {\bf getaddrinfo(3)}.  It
performs DNSSEC validation of DNS queries.  It returns a network address value
of type {\it addrinfo}.  (See {\bf getaddrinfo(3)} for more information on
{\it addrinfo}.)

{\bf val\_x\_getaddrinfo()} performs the same function as {\bf
val\_getaddrinfo()}, but is optimized for multiple calls.  The two routines
take the same parameters, but {\bf val\_x\_getaddrinfo()} takes an additional
parameter {\it ctx}, of type {\it val\_context}, which passes the validation
context for use in call optimization.  The {\it ctx} parameter also gives the
caller more control over the resolver and validator policies.  If a {\bf NULL}
value is passed for the {\it ctx} parameter, the default validation context
is used.  (See {\bf get\_context(3)} for information on creating a validation
context.) {\bf val\_getaddrinfo()} is equivalent to calling {\bf
val\_x\_getaddrinfo()} with a {\bf NULL} {\it ctx} parameter.

{\bf val\_dupaddrinfo()} duplicates the {\it addrinfo} structure and its
auxiliary data.  It performs a deep copy; i.e., the internal strings, arrays,
and other structures are also copied.

{\bf val\_freeaddrinfo()} frees a {\it addrinfo} structure, such as those
returned by the {\bf val\_getaddrinfo()}, {\bf val\_x\_getaddrinfo()} and {\bf
val\_dupaddrinfo()} functions.

{\bf val\_get\_addrinfo\_dnssec\_status()} extracts the DNSSEC validation
status from the returned {\it addrinfo} structure.  This function must be
called only for the values returned from {\bf val\_getaddrinfo()}, {\bf
val\_x\_getaddrinfo()}, and {\bf val\_dupaddrinfo()} functions.

{\bf RETURN VALUES}

The {\bf val\_getaddrinfo()} and {\bf val\_x\_getaddrinfo()} functions return
a value of type {\it addrinfo} on success, and {\bf NULL} on error.  The
memory for the returned value is dynamically allocated by these functions.
Hence, the caller must only call the {\bf val\_freeaddrinfo()} function on
the returned value in order to avoid memory leaks.

The {\bf val\_get\_addrinfo\_dnssec\_status()} function returns the result
of the DNSSEC validation.  The possible values for the DNSSEC status are given
in {\bf val\_errors.h}.

The {\bf val\_dupaddrinfo()} function returns a copy of the specified {\it
addrinfo} structure.  The returned value must be freed using {\bf
val\_freeaddrinfo()} to avoid memory leaks.

{\bf EXAMPLE}

\begin{verbatim}
 #include <stdio.h>
 #include <validator.h>

 int main(int argc, char *argv[])
 {
          int dnssec_status = ERROR;
          struct addrinfo *ainfo = NULL;

          if (argc < 2) {
                  printf("Usage: %s <hostname>\n", argv[0]);
                  exit(1);
          }
 
          ainfo = val_getaddrinfo(argv[1]);

          if (ainfo) {
                  dnssec_status = val_get_addrinfo_dnssec_status(h);

                  printf("DNSSEC Status = %d [%s]\n", dnssec_status,
                         p_val_error(dnssec_status));
                  val_freeaddrinfo(h);
          }

          return 0;
  }
\end{verbatim}

{\bf SEE ALSO}

{\bf gethostbyname}(3)

{\bf get\_context(3)}, {\bf val\_duphostent(3)}, {\bf val\_freehostent(3)},\\
{\bf val\_gethostbyname(3)}, {\bf val\_query(3)},\\
{\bf val\_x\_gethostbyname(3)}, {\bf val\_x\_query(3)}

{\it p\_val\_error}

\url{http://dnssec-tools.sourceforge.net}

