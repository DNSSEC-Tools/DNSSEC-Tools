\clearpage

\subsection{\bf val\_getaddrinfo()}

{\bf NAME}

\func{val\_getaddrinfo()}, \func{val\_freeaddrinfo()} - get DNSSEC-validated
network address and service translation

{\bf SYNOPSIS}

\begin{verbatim}
    #include <validator.h>

    int val_getaddrinfo(const struct val_context *ctx,
                        const char *nodename,
                        const char *servname,
                        const struct addrinfo *hints,
                        struct val_addrinfo **res);

    void val_freeaddrinfo(struct val_addrinfo *ainfo);
\end{verbatim}

{\bf DESCRIPTION}

\func{val\_getaddrinfo()} performs DNSSEC validation of DNS queries.
It is intended to be a DNSSEC-aware replacement for \func{getaddrinfo(3)}.
It returns a network address value of type \struct{struct val\_addrinfo}
in the \var{res} parameter.

\begin{verbatim}
    struct val_addrinfo
    {
          int                  ai_flags;
          int                  ai_family;
          int                  ai_socktype;
          int                  ai_protocol;
          size_t               ai_addrlen;
          struct sockaddr     *ai_addr;
          char                *ai_canonname;
          struct val_addrinfo *ai_next;
          val_status_t         ai_val_status;
    };
\end{verbatim}

The \struct{val\_addrinfo} structure is similar to the \struct{addrinfo}
structure (described in \func{getaddrinfo(3)}.)  \struct{val\_addrinfo} has an
additional field \var{ai\_val\_status} that represents the validation status
for that particular record.  \func{val\_istrusted()} determines whether this
validation status represents a trusted value and \func{p\_val\_status()}
displays this value to the user in a useful manner.  (See \lib{libval(3)}
for more information).

The \var{ctx} parameter of type \var{val\_context} represents the validation
context, which can be NULL for default value.  The caller uses this parameter
to control the resolver and validator policies used during validation.
(See \lib{libval(3)} for information on creating a validation context.)

The \var{nodename}, \var{servname}, and \var{hints} parameters are similar
in meaning and syntax to the corresponding parameters for the original
\func{getaddrinfo()} function.  The \var{res} parameter is similar to the
\var{res} parameter for \func{getaddrinfo()}, except that it is of type
\struct{struct val\_addrinfo} instead of \struct{struct addrinfo}.  (See
the manual page for \func{getaddrinfo(3)} for more details about these
parameters.)

\func{val\_freeaddrinfo()} frees the memory allocated for a
\struct{val\_addrinfo} linked list.

{\bf RETURN VALUES}

The \func{val\_getaddrinfo()} function returns 0 on success and a non-zero
error code on failure.  \var{*res} will point to a dynamically allocated
linked list of \struct{val\_addrinfo} structures on success and will be
NULL on error.  The memory for the value returned in \var{*res} can be
released using \func{val\_freeaddrinfo()}.

{\bf EXAMPLE}

\begin{verbatim}
    #include <stdio.h>
    #include <stdlib.h>
    #include <validator.h>

    int main(int argc, char *argv[])
    {
             struct val_addrinfo *ainfo = NULL;
             int retval;

             if (argc < 2) {
                     printf("Usage: %s <hostname>\n", argv[0]);
                     exit(1);
             }
    
             retval = val_getaddrinfo(NULL, argv[1], NULL, NULL, &ainfo);

             if ((retval == 0) && (ainfo != NULL)) {

                     printf("Validation Status = %d [%s]\n",
                            ainfo->ai_val_status,
                            p_val_status(ainfo->ai_val_status));

                     val_freeaddrinfo(ainfo);
             }

             return 0;
    }
\end{verbatim}

{\bf SEE ALSO}

\func{getaddrinfo(3)}, \func{val\_gethostbyname(3)}, \func{val\_query(3)}

\lib{libval(3)}
