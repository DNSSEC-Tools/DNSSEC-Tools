\clearpage

\subsection{\perlmod{timetrans.pm}}

{\bf NAME}

Net::DNS::SEC::Tools::timetrans - Convert an integer seconds count into text units.

{\bf SYNOPSIS}

\begin{verbatim}
  use Net::DNS::SEC::Tools::timetrans;

  $timestring = timetrans(86488);
\end{verbatim}

{\bf DESCRIPTION}

The {\bf timetrans}() interface in the {\bf
Net::\-DNS::\-SEC::\-Tools::\-timetrans} converts an integer seconds count
into the equivalent number of weeks, days, hours, and minutes.  The time
converted is a relative time, {\bf not} an absolute time.  The returned time
is given in terms of weeks, days, hours, minutes, and seconds, as required
to express the seconds count appropriately.

{\bf EXAMPLES}

{\bf timetrans(400)} returns {\it 6 minutes, 40 seconds}

{\bf timetrans(420)} returns {\it 7 minutes}

{\bf timetrans(888)} returns {\it 14 minutes, 48 seconds}

{\bf timetrans(86400)} returns {\it 1 day}

{\bf timetrans(86488)} returns {\it 1 day, 28 seconds}

{\bf timetrans(715000)} returns {\it 1 week, 1 day, 6 hours, 36 minutes, 40 second}

{\bf timetrans(720000)} returns {\it 1 week, 1 day, 8 hours}

{\bf INTERFACES}

The interfaces to the {\bf Net::\-DNS::\-SEC::\-Tools::\-timetrans} module
are given below.

{\bf {\bf timetrans()}}

This routine converts an integer seconds count into the equivalent number of
weeks, days, hours, and minutes.  This converted seconds count is returned
as a text string.  The seconds count must be greater than zero or an error
will be returned.

Return Values:

\begin{description}
\item If a valid seconds count was given, the count converted into the
appropriate text string will be returned.

\item An empty string is returned if the no seconds count was given or if
the seconds count is less than one.
\end{description}

{\bf SEE ALSO}

{\it timetrans(1)}

