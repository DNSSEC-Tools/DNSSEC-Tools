\clearpage

\subsubsection{\bf lsroll}

{\bf NAME}

\cmd{lsroll} - List the {\it rollrec}s in a DNSSEC-Tools {\it rollrec} file.

{\bf SYNOPSIS}

\begin{verbatim}
    lsroll [options] <rollrec-files>
\end{verbatim}

{\bf DESCRIPTION}

This script lists the contents of the specified {\it rollrec} files.  All
{\it rollrec} files are loaded before the output is displayed.  If any
{\it rollrec}s have duplicated names, whether within one file or across
multiple files, the later {\it rollrec} will be the one whose data are
displayed.

Each record's name is always included in the output.  Additional output
depends on the options selected.

{\bf OPTIONS}

There are three types of options recognized by \cmd{lsroll}:  record-selection
options, attribute-selection options, and output-format options.  Each type
is described in the sections below.

{\bf Record-selection Options}

These options select the records that will be displayed by \cmd{lsroll}.

\begin{description}

\item {\it -all}\verb" "

List all records in the {\it rollrec} file.

\item {\it -roll}\verb" "

List all ``roll'' records in the {\it rollrec} file.

\item {\it -skip}\verb" "

List all ``skip'' records in the {\it rollrec} file.

\end{description}

{\bf Attribute-selection Options}

These options select the attributes of the records that will be displayed
by \cmd{lsroll}.

\begin{description}

\item {\it -type}\verb" "

Include each {\it rollrec} record's type in the output.  The type will be
either ``roll'' or ``skip''.  The type is given parenthetically.

\item {\it -zone}\verb" "

The record's zonefile is included in the output.  This field is part
of the default output.

\item {\it -keyrec}\verb" "

The record's {\it keyrec} file is included in the output.  This field is part
of the default output.

\item {\it -phase}\verb" "

The record's rollover phase is included in the output.  This field is part of
the default output.

\item {\it -ttl}\verb" "

The record's TTL value is included in the output.

\item {\it -display}\verb" "

The record's display flag, used by \cmd{blinkenlights}, is included in the
output.

\item {\it -phstart}\verb" "

The record's rollover phase is included in the output.

\end{description}

{\bf Output-format Options}

These options select the type of output that will be given by \cmd{lsroll}.

\begin{description}

\item {\it -count}\verb" "

Only a count of matching keyrecs in the {\it rollrec} file is given.

\item {\it -terse}\verb" "

Terse output is given.  Only the record name and any other fields specifically
selected are included in the output.

\item -help\verb" "

Display a usage message.

\end{description}

{\bf SEE ALSO}

\cmd{blinkenlights(8)},
\cmd{rollchk(8)},
\cmd{rollinit(8)},
\cmd{rollerd(8)}

\perlmod{Net::DNS::SEC::Tools::rollrec.pm(3)}

\path{rollrec(5)}
