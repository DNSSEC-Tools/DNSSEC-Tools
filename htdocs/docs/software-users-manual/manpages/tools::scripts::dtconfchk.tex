\clearpage

\subsubsection{dtconfchk}

{\bf NAME}

\cmd{dtconfchk} - Check a DNSSEC-Tools configuration file for sanity

{\bf SYNOPSIS}

\begin{verbatim}

  dtconfchk [options] [config_file]

\end{verbatim}

{\bf DESCRIPTION}

\cmd{dtconfchk} checks a DNSSEC-Tools configuration file to determine if the
entries are valid.  If a configuration file isn't specified, the system
configuration file will be verified.

Without any display options, \cmd{dtconfchk} displays error messages for
problems found, followed by a summary line.  Display options will increase or
decrease the amount of detail about the configuration file's sanity.  In all
cases, the exit code is the count of errors found in the file.

The tests are divided into five groups:  key-related checks, zone-related
checks, path checks, rollover checks, and miscellaneous checks.  The checks
in each of these self-explanatory groups are described below.

The {\it default\_keyrec} configuration entry is not checked.  This entry
specifies the default \struct{keyrec} file name and isn't necessarily expected
to exist in any particular place.

{\bf Key-related Checks}

The following key-related checks are performed:

\begin{itemize}

\item {\it algorithm}\verb" "

Ensure the {\it algorithm} field is valid.  The acceptable values may be found
in the \cmd{dnssec-keygen} man page.

\item {\it ksklength}\verb" "

Ensure the {\it ksklength} field is valid.  The acceptable values may be found
in the \cmd{dnssec-keygen} man page.

\item {\it ksklife}\verb" "

Ensure the {\it ksklife} field is valid.  The acceptable values may be found
in the \perlmod{defaults.pm} man page.

\item {\it zskcount}\verb" "

Ensure the {\it zskcount} field is valid.  The ZSK count must be positive.

\item {\it zsklength}\verb" "

Ensure the {\it zsklength} field is valid.  The acceptable values may be found
in the \cmd{dnssec-keygen} man page.

\item {\it zsklife}\verb" "

Ensure the {\it zsklife} field is valid.  The acceptable values may be found
in the \perlmod{defaults.pm} man page.

\item {\it random}\verb" "

Ensure the {\it random} field is valid.  This file must be a character
device file.

\end{itemize}

{\bf Zone-related Checks}

The following zone-related checks are performed:

\begin{description}

\item {\it endtime}\verb" "

Ensure the {\it endtime} field is valid.  This value is assumed to be in the
``+NNNNNN" format.  There is a lower limit of two hours.  (This is an
artificial limit under which it {\it may} not make sense to have an end-time.)

\end{description}

{\bf Path Checks}

The following path checks are performed:

\begin{itemize}

\item {\it keygen}\verb" "

Ensure the {\it keygen} field is valid.  If the filename starts with a `/',
the file must be a regular executable file.

\item {\it viewimage}\verb" "

Ensure the {\it viewimage} field is valid.  If the filename starts with a `/',
the file must be a regular executable file.

\item {\it zonecheck}\verb" "

Ensure the {\it zonecheck} field is valid.  If the filename starts with a `/',
the file must be a regular executable file.

\item {\it zonesign}\verb" "

Ensure the {\it zonesign} field is valid.  If the filename starts with a `/',
the file must be a regular executable file.

\end{itemize}

{\bf Rollover Daemon Checks}

The following checks are performed for \cmd{rollerd} values:

\begin{itemize}

\item {\it roll\_logfile}\verb" "

Ensure that the log file for the \cmd{rollerd} is valid.  If the file
exists, it must be a regular file.

\item {\it roll\_loglevel}\verb" "

Ensure that the logging level for the \cmd{rollerd} is reasonable.  The
log level must be one of the following text or numeric values:

\begin{table}[ht]
\begin{center}
\begin{tabular}{|l|c|l|}
\hline
{\bf Textual Level} & {\bf Numeric Level} & {\bf Meaning} \\
\hline
{\bf tmi}    & 1 & Overly verbose informational messages.		\\
{\bf expire} & 3 & A verbose countdown of zone expiration is given.	\\
{\bf info}   & 4 & Informational messages.				\\
{\bf phase}  & 6 & Current state of zone.				\\
{\bf err}    & 8 & Errors messages.					\\
{\bf fatal}  & 9 & Fatal errors.					\\
\hline
\end{tabular}
\end{center}
\caption{Logging Levels}
\end{table}

Specifying a particular log level will causes messages of a higher numeric
value to also be displayed.

\item {\it roll\_sleeptime}\verb" "

Ensure that the \cmd{rollerd}'s sleep-time is reasonable.
\cmd{rollerd}'s sleep-time must be at least one minute.

\end{itemize}

{\bf Miscellaneous Checks}

The following miscellaneous checks are performed:

\begin{itemize}

\item {\it admin-email}\verb" "

Ensure that the {\it admin-email} field is defined and has a value.
\cmd{dtconfchk} does not try to validate the email address itself.

\item {\it archivedir}\verb" "

Ensure that the {\it archivedir} directory is actually a directory.
This check is only performed if the {\it savekeys} flag is set on.

\item {\it entropy\_msg}\verb" "

Ensure that the {\it entropy\_msg} flag is either 0 or 1.

\item {\it savekeys}\verb" "

Ensure that the {\it savekeys} flag is either 0 or 1.
If this flag is set to 1, then the {\it archivedir} field will also be checked.

\item {\it usegui}\verb" "

Ensure that the {\it usegui} flag is either 0 or 1.

\end{itemize}

{\bf OPTIONS}

\begin{description}

\item {\bf -expert}\verb" "

This option will bypass the following checks:

\begin{itemize}

\item KSK has a longer lifespan than the configuration file's default minimum
lifespan

\item KSK has a shorter lifespan than the configuration file's default maximum
lifespan

\item ZSKs have a longer lifespan than the configuration file's default
minimum lifespan

\item ZSKs have a shorter lifespan than the configuration file's default
maximum lifespan

\end{itemize}

\item {\bf -quiet}\verb" "

No output will be given.
The number of errors will be used as the exit code.

\item {\bf -summary}\verb" "

A final summary of success or failure will be printed.
The number of errors will be used as the exit code.

\item {\bf -verbose}\verb" "

Success or failure status of each check will be given.
A {\bf +} or {\bf -} prefix will be given for each valid and invalid entry.
The number of errors will be used as the exit code.

\item {\bf -help}\verb" "

Display a usage message.

\end{description}

{\bf SEE ALSO}

dtdefs(8),
dtinitconf(8),
rollerd(8),
zonesigner(8)

Net::DNS::SEC::Tools::conf.pm(3),
Net::DNS::SEC::Tools::defaults.pm(3)

dnssec-tools.conf(5)

