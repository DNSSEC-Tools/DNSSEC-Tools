\clearpage

\subsubsection{\bf dtconfchk}

{\bf NAME}

\cmd{dtconfchk} - Check a DNSSEC-Tools configuration file for sanity.

{\bf SYNOPSIS}

\begin{verbatim}
    dtconfchk [options] [config_file]
\end{verbatim}

{\bf DESCRIPTION}

\cmd{dtconfchk} checks a DNSSEC-Tools configuration file to determine if the
entries are valid.

The {\it default\_keyrec} configuration entry is not checked.  This entry
specifies the default {\it keyrec} file name and isn't necessarily expected
to exist in any particular place.

{\bf Key-related Checks}

The following key-related checks are performed:

\begin{description}

\item {\it algorithm}\verb" "

Ensure the {\it algorithm} field is valid.  The acceptable values may be found
in the \cmd{dnssec-keygen} man page.

\item {\it ksklength}\verb" "

Ensure the {\it ksklength} field is valid.  The acceptable values may be found
in the \cmd{dnssec-keygen} man page.

\item {\it ksklife}\verb" "

Ensure the {\it ksklife} field is valid.  The acceptable values may be found
in the \perlmod{defaults.pm(3)} man page.

\item {\it zskcount}\verb" "

Ensure the {\it zskcount} field is valid.  The ZSK count must be positive.

\item {\it zsklength}\verb" "

Ensure the {\it zsklength} field is valid.  The acceptable values may be found
in the \cmd{dnssec-keygen} man page.

\item {\it zsklife}\verb" "

Ensure the {\it zsklife} field is valid.  The acceptable values may be found
in the \perlmod{defaults.pm(3)} man page.

\item {\it random}\verb" "

Ensure the {\it random} field is valid.  This file must be a character
device file.

\end{description}

{\bf Zone-related Checks}

The following zone-related checks are performed:

\begin{description}

\item {\it endtime}\verb" "

Ensure the {\it endtime} field is valid.  This value is assumed to be in the
"+NNNNNN" format.  There is a lower limit of two hours.  (This is an
artificial limit under which it {\it may} not make sense to have an end-time.)

\end{description}

{\bf Path Checks}

The following path checks are performed:

\begin{description}

\item {\it checkzone}\verb" "

Ensure the {\it checkzone} field is valid.  If the filename starts with a '/',
the file must be a regular executable file.

\item {\it keygen}\verb" "

Ensure the {\it keygen} field is valid.  If the filename starts with a '/',
the file must be a regular executable file.

\item {\it signzone}\verb" "

Ensure the {\it signzone} field is valid.  If the filename starts with a '/',
the file must be a regular executable file.

\item {\it viewimage}\verb" "

Ensure the {\it viewimage} field is valid.  If the filename starts with a '/',
the file must be a regular executable file.

\end{description}

{\bf Roll-over Daemon Checks}

The following checks are performed for \cmd{rollerd} values:

\begin{description}

\item {\it roll\_logfile}\verb" "

Ensure that the log file for the \cmd{rollerd} is valid.  If the file
exists, it must be a regular file.

\item {\it roll\_loglevel}\verb" "

Ensure that the logging level for the \cmd{rollerd} is reasonable.  The
log level must be one of the following text or numeric values:

\begin{verbatim}
    tmi        1       (Overly verbose informational messages.)
    info       3       (Informational messages.)
    curphase   5       (Current state of zone.)
    err        7       (Error messages.)
    fatal      9       (Fatal errors.)
\end{verbatim}

Specifying a particular log level will causes messages of a higher numeric
value to also be displayed.

\item {\it roll\_sleeptime}\verb" "

Ensure that the \cmd{rollerd}'s sleep-time is reasonable.
\cmd{rollerd}'s sleep-time must be at least one minute.

\end{description}

{\bf Miscellaneous Checks}

The following miscellaneous checks are performed:

\begin{description}

\item {\it archivedir}\verb" "

Ensure that the {\it archivedir} directory is actually a directory.
This check is only performed if the {\it savekeys} flag is set on.

\item {\it entropy\_msg}\verb" "

Ensure that the {\it entropy\_msg} flag is either 0 or 1.

\item {\it savekeys}\verb" "

Ensure that the {\it savekeys} flag is either 0 or 1.
If this flag is set to 1, then the {\it archivedir} field will also be checked.

\item {\it usegui}\verb" "

Ensure that the {\it usegui} flag is either 0 or 1.

\end{description}

{\bf OPTIONS}

\begin{description}

\item {\it -expert}\verb" "

Only non-expert checks will be performed.  Currently, this option will prevent
the key lifespan checks from being run.

\item {\it -quiet}\verb" "

No output will be given.
The number of errors will be used as the exit code.

\item {\it -summary}\verb" "

A final summary of success or failure will be printed.
The number of errors will be used as the exit code.

\item {\it -verbose}\verb" "

Success or failure status of each check will be given.
A {\bf +} or {\bf -} prefix will be given for each valid and invalid entry.
The number of errors will be used as the exit code.

\item {\it -help}\verb" "

Display a usage message.

\end{description}

{\bf SEE ALSO}

\cmd{dtdefs(8)},
\cmd{dtinitconf(8)},
\cmd{rollerd(8)},
\cmd{zonesigner(8)}

\perlmod{Net::DNS::SEC::Tools::conf.pm(3)},
\perlmod{Net::DNS::SEC::Tools::defaults.pm(3)}

\path{dnssec-tools.conf(5)}
