\clearpage

\subsection{\perlmod{conf.pm}}


{\bf NAME}

Net::DNS::SEC::Tools::conf - DNSSEC tools configuration file routines.

{\bf SYNOPSIS}

\begin{verbatim}
  use Net::DNS::SEC::Tools::conf;

  %dtconf = parseconfig();
  %dtconf = parseconfig("localzone.keyrec");
\end{verbatim}

{\bf DESCRIPTION}

The DNSSEC tools have a configuration file for commonly used values.
These values are the defaults for a variety of things, such as
encryption algorithm and encryption key length.

{\bf /usr/local/etc/dnssec/dnssec-tools.conf} is the path for the DNSSEC tools
configuration file.  The {\bf Net::DNS::SEC::Tools::conf} module provides
methods for accessing the configuration data in this file.

The DNSSEC tools configuration file consists of a set of configuration
value entries, with only one entry per line.  Each entry has the
``keyword value'' format.  During parsing, the line is broken into
tokens, with tokens being separated by spaces and tabs.  The first
token in a line is taken to be the keyword.  All other tokens in that
line are concatenated into a single string, with a space separating
each token.  The untokenized string is added to a hash table, with the
keyword as the value's key.

Comments may be included by prefacing them with the `\#' or `;'
comment characters.  These comments can encompass an entire line or may
follow a configuration entry.  If a comment shares a line with an entry,
value tokenization stops just prior to the comment character.

An example configuration file follows:

\begin{verbatim}
    # Sample configuration entries.

    algorithm       rsasha1     # Encryption algorithm.
    ksk_length      1024        ; KSK key length.
\end{verbatim}

{\bf CONFIGURATION INTERFACES}

\begin{description}

\item [{\bf parseconfig()}]\verb" "

This routine reads and parses the system's DNSSEC tools configuration file.
The parsed contents are put into a hash table, which is returned to the caller.

\item [{\bf parseconfig(conffile)}]\verb" "

This routine reads and parses a caller-specified DNSSEC tools configuration
file.  The parsed contents are put into a hash table, which is returned to
the caller.  The routine quietly returns if the configuration file does not
exist. 

\end{description}

{\bf SEE ALSO}

{\it zonesigner(1)}

{\bf Net::DNS::SEC::Tools::keyrec(3)}

{\bf dnssec-tools.conf(5)}

