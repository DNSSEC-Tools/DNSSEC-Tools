\clearpage

\subsection{\bf conf.pm Module}

{\bf NAME}

\perlmod{Net::DNS::SEC::Tools::conf.pm} - DNSSEC-Tools configuration routines.

{\bf SYNOPSIS}

\begin{verbatim}
    use Net::DNS::SEC::Tools::conf;

    %dtconf = parseconfig();

    %dtconf = parseconfig("localzone.keyrec");

    bindcheck(\%options_hashref);

    $confdir = getconfdir();

    $conffile = getconffile();
\end{verbatim}

{\bf DESCRIPTION}

The routines in this module perform configuration operations.  Some routines
access the DNSSEC-Tools configuration file, while others validate the
execution environment.

The DNSSEC tools have a configuration file for commonly used values.  These
values are the defaults for a variety of things, such as encryption algorithm
and encryption key length.

\path{/usr/local/etc/dnssec/dnssec-tools.conf} is the path for the DNSSEC
tools configuration file.  The \perlmod{Net::DNS::SEC::Tools::conf} module
provides methods for accessing the configuration data in this file.

The DNSSEC tools configuration file consists of a set of configuration value
entries, with only one entry per line.  Each entry has the ``keyword value''
format.  During parsing, the line is broken into tokens, with tokens being
separated by spaces and tabs.  The first token in a line is taken to be the
keyword.  All other tokens in that line are concatenated into a single string,
with a space separating each token.  The untokenized string is added to a hash
table, with the keyword as the value's key.

Comments may be included by prefacing them with the `\#' or `;' comment
characters.  These comments can encompass an entire line or may follow a
configuration entry.  If a comment shares a line with an entry, value
tokenization stops just prior to the comment character.

An example fragment of a configuration file follows:

\begin{verbatim}
    # Sample configuration entries.

    algorithm       rsasha1     # Encryption algorithm.
    ksk_length      1024        ; KSK key length.
\end{verbatim}

{\bf INTERFACES}

\begin{description}

\item {\it parseconfig()}\verb" "

This routine reads and parses the system's DNSSEC tools configuration file.
The parsed contents are put into a hash table, which is returned to the caller.

\item {\it parseconfig(conffile)}\verb" "

This routine reads and parses a caller-specified DNSSEC tools configuration
file.  The parsed contents are put into a hash table, which is returned to
the caller.  The routine quietly returns if the configuration file does not
exist. 

\item {\it bindcheck(\%options\_hashref)}\verb" "

This routine ensures that the needed BIND commands are available and
executable.  If any of the commands either don't exist or aren't executable,
then an error message will be given and the process will exit.  If all is
well, everything will proceed quietly onwards.

The BIND commands currently checked are {\it checkzone}, {\it keygen}, and
{\it signzone}.  The pathnames for these commands are found in the given
options hash referenced by {\it \%options\_hashref}.  If the hash doesn't
contain an entry for one of those commands, it is not checked.

\item {\it getconfdir()}\verb" "

This routine returns the name of the DNSSEC-Tools configuration directory.

\item {\it getconffile()}\verb" "

This routine returns the name of the DNSSEC-Tools configuration file.

\end{description}

{\bf SEE ALSO}

\path{dnssec-tools.conf(5)}
