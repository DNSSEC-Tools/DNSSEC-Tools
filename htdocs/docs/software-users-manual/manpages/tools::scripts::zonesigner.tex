\clearpage

\subsection{{\it zonesigner}}


{\bf NAME}

zonesigner - Generates encryption keys and signs a DNS zone.

{\bf SYNOPSIS}

\begin{verbatim}  zonesigner [options] <zone-file> <zone-out>\end{verbatim}

{\bf DESCRIPTION}

This script combines into a single command many actions that are required to
sign a DNS zone.  It generates the required KSK and ZSK keys, adds the key
data to a zone record file, signs the zone file, and runs checks to ensure
that everything worked properly.

Using {\it keyrec} files, defined and maintained by DNSSEC-Tools,
{\it zonesigner} can automatically gather many of the options used to previously
sign and generate a zone and its keys.  This allows the zone to be maintained
using the same key lengths and expiration times, for example, without an
administrator needing to manually track these fields.

{\bf KEYREC FILES}

{\it keyrec} files retain information about previous key-generation and
zone-signing operations.  If a {\it keyrec} file is not specified (by way of
the {\it -krfile} option), then a default {\it keyrec} file is used.  If this
default in not specified in the system's DNSSEC-Tools configuration
file, the filename will be the zone name appended with {\bf .krf}.  If the
{\it -nokrfile} option is given, then no {\it keyrec} file will be consulted
or saved.

{\it keyrec} files contain two types of entries:  zone {\it keyrec}s and key
{\it keyrec}s.  Each {\it keyrec} contains a set of ``key/value'' entries,
one per line.  Example 4 below contains the contents of a sample {\it keyrec}
file.

{\bf USING ZONESIGNER}

{\it zonesigner} is used in this way:

\begin{verbatim}    zonesigner [options] <zone-file> <zone-out>\end{verbatim}

The {\it zone-file} and {\it zone-out} arguments are required.

{\it zone-file} is the name of the zone file from which a signed zone file
will be created.  If the {\it -zone} option is not given, then {\it zone-file}
will be used as the name of the zone that will be signed.  Generated keys are
given this name as their base.

The zone file is modified to have {\bf include} commands, which will include
the KSK and ZSK keys.  These lines are placed at the end of the file and
should not be modified by the user.  If the zone file already includes any
key files, those inclusions will be deleted.  These lines are distinguished
by starting with ``\$INCLUDE'' and end with ``.key''.  Only the actual include
lines are deleted; any related comment lines are left untouched.

{\it zone-out} is the name of the output zone file.  This file will be a copy
of the {\it zone-file} zone file, with some modifications.  Several \$INCLUDE
lines will be added at the end of the file and the SOA serial number will be
incremented.

The signed zone file will be this name, appended with {\it .signed}.

Unless the {\it -genkeys}, {\it -genksk}, or {\it -genzsk} options are
specified, the last keys generated for a particular zone will be used in
subsequent {\it zonesigner} executions.

{\bf ENTROPY}

On some systems, the implementation of the pseudo-random number generator
requires keyboard activity.  This keyboard activity is used to fill a buffer
in the system's random number generator.  If zonesigner appears hung, you may
have to add entropy to the random number generator by randomly striking keys
until the program completes.  Display of this message is controlled by the
{\bf entropy\_msg} configuration file parameter.

{\bf DETERMINING OPTION VALUES}

{\it zonesigner} checks four places in order to determine option values.  
In descending order of precedence, these places are:

\begin{verbatim}
    command line options

    keyrec file

    dnssec-tools configuration file

    zonesigner defaults
\end{verbatim}

Each is checked until a value is found.  That value is then used for that
{\it zonesigner} execution and the value is stored in the {\it keyrec} file.

{\bf Example}

For example, the KSK length has the following values:

\begin{verbatim}
    -ksklength command line option:         8192

    keyrec file:                            1024

    dnssec-tools configuration file:        2048

    zonesigner defaults:                    512
\end{verbatim}

If all are present, then the KSK length will be 8192.

If the {\it -ksklength} command line option wasn't given, the KSK length
will be 1024.

If the KSK length wasn't given in the configuration file, it will be 8192.

If the KSK length wasn't in the {\it keyrec} file or the configuration file,
the KSK length will be 8192.

If the {\it -ksklength} command line option wasn't given and the KSK length
wasn't in the configuration file, it'll be 1024.

If the command line option wasn't given, the KSK length wasn't in the
{\it keyrec} file, and it wasn't in the configuration file, then the KSK
length will be 512.

{\bf OPTIONS}

Three types of options may be given, based on the command for which they are
intended.  These commands are  {\it dnssec-keygen}, {\it dnssec-signzone}, and
{\it zonesigner}.

{\bf {\it zonesigner}-specific Options}

\begin{description}

\item [-nokrfile]\verb" "

No {\it keyrec} file will be consulted or created.

\item [-krfile]\verb" "

{\it keyrec} file to use in processing options.  See the man page for
{\bf Net::\-DNS::\-SEC::\-Tools::\-tooloptions} for more details about
this file.

\item [-genkeys]\verb" "

Generate a new KSK and ZSK for the zone.

\item [-genksk]\verb" "

Generate a new KSK for the zone.  By default, the last KSK generated for this
zone will be used.

\item [-genzsk]\verb" "

Generate a new ZSK for the zone.  By default, the last ZSK generated for this
zone will be used.

\item [-forceroll]\verb" "

Force a roll-over of the ZSK keys.  The {\it keyrec}s of the ZSK keys are
adjusted as follows:

\begin{verbatim}
    The current ZSK key is marked as obsolete.
    The published ZSK key is marked as current.
    The new ZSK key, if it exists, is marked as published.
    A new ZSK key is generated.
\end{verbatim}

This should only be used if you know what you're doing.

\item [-zone]\verb" "

Name of the zone that will be signed.  This zone name may be given with this
option or as the first non-option command line argument.

\item [-help]\verb" "

Display a usage message.

\item [-verbose]\verb" "

Verbose output will be given.  As more instances of {\it -verbose} are given on
the command line, additional levels of verbosity are achieved.

\begin{verbatim}
    level        output
    -----        ------
      1          operations being performed
                    (e.g., generating key files, signing zone) 
      2          details on operations and some operation results
                    (e.g., new key names, zone serial number)
      3          operations' parameters and additional details
                    (e.g., key lengths, encryption algorithm,
		     executed commands)
\end{verbatim}

Higher levels of verbosity are cumulative.  Specifying two instances of
{\it -verbose} will get the output from the first and second levels of output.

\end{description}

{\bf {\it dnssec-keygen}-specific Options}

\begin{description}

\item [-algorithm]\verb" "

Cryptographic algorithm used to generate the zone's keys.
The default value is RSA\-SHA1.

\item [-ksklength]\verb" "

Bit length of the zone's KSK key.
The default is 1024.

\item [-random]\verb" "

Source of randomness used to generate the zone's keys.	(/dev/urandom)

\item [-zsklength]\verb" "

Bit length of the zone's ZSK key.
The default is 512.

\item [-kgopts]\verb" "

Additional options for {\it dnssec-keygen} may be specified using this option.
The additional options are passed as a single string value as an argument to
the {\it -kgopts} option.

\end{description}

{\bf {\it dnssec-signzone}-specific Options}

\begin{description}

\item [-endtime]\verb" "

Time that the zone expires, measured in seconds.  See the man page for
{\it dnssec-signzone} for the valid format of this field.
The default value is 259200 seconds (30 days.)

\item [-gends]\verb" "

Force {\it dnssec-signzone} to generate DS records for the zone.  This option is
translated into {\it -g} when passed to {\it dnssec-signzone}.

\item [-ksdir]\verb" "

Specify a directory for storing keysets.  This is passed to {\it
dnssec-signzone} as the {\it -d} option.

\item [-szopts]\verb" "

Additional options for {\it dnssec-signzone} may be specified using this option.
The additional options are passed as a single string value as an argument to
the {\it -szopts} option.

\end{description}

{\bf Examples}

Example 1.

In the first example, an existing {\it keyrec} file is used to assist in
signing the example.com domain.  Zone data are stored in {\bf example.com},
and the keyrec is in {\bf example.krf}.  The output zone file and final signed
zone file will use {\bf db.example.com} as a base.  Using this execution:

\begin{verbatim}    # zonesigner -krfile example.krf example.com db.example.com\end{verbatim}

the following files are created:

\begin{verbatim}
    Kexample.com.+005+45842.private
    Kexample.com.+005+45842.key
    Kexample.com.+005+50186.private
    Kexample.com.+005+50186.key
    Kexample.com.+005+59143.key
    Kexample.com.+005+59143.private

    dsset-example.com.
    keyset-example.com.

    db.example.com
    db.example.com.signed
\end{verbatim}


The first six files are the KSK and ZSK keys required for the zone.  The next
two files are created by the zone-signing process.  The last two are the zone
file used as input to the zone-signing process and the final signed zone.

Example 2.

In the second example, an existing {\it keyrec} file is used to assist in
signing the example.com domain.  Zone data are stored in {\bf example.com},
and the keyrec is in {\bf example.krf}.  The generated keys, output zone file,
and final signed zone file will use {\bf example.com} as a base.  Using this
execution:

\begin{verbatim}    # zonesigner -krfile example.krf example.com db.example.com\end{verbatim}

the following files are created:

\begin{verbatim}
    Kdb.example.com.+005+12354.key
    Kdb.example.com.+005+12354.private
    Kdb.example.com.+005+82197.key
    Kdb.example.com.+005+82197.private
    Kdb.example.com.+005+55888.key
    Kdb.example.com.+005+55888.private

    dsset-db.example.com.
    keyset-db.example.com.

    db.example.com.signed
\end{verbatim}

The first six files are the KSK and ZSK keys required for the zone.  The next
two files are created by the zone-signing process.  The last file is file is
the final signed zone.

Example 3.

In the third example, no {\it keyrec} file is specified for the signing of
the example.com domain.  In addition to files created as shown in previous
examples, a new {\it keyrec} file is created.  The new {\it keyrec} file uses
the domain name as its base.  Using this execution:

\begin{verbatim}    # zonesigner example.com db.example.com\end{verbatim}

the following {\it keyrec} file is created:

\begin{verbatim}    example.com.krf\end{verbatim}

Example 4.

This example shows a {\it keyrec} file generated by {\it zonesigner}.

The command executed is:

\begin{verbatim}    # zonesigner example.com db.example.com\end{verbatim}

The generated {\it keyrec} file contains three {\it keyrec}s:  a zone {\it
keyrec}, one KSK {\it keyrec}, and two ZSK {\it keyrec}s.

\begin{verbatim}
    zone        "example.com"
        zskpubpath        "./Kexample.com.+005+13531.key"
        zskpub            "Kexample.com.+005+13531"
        zskdirectory      "."
        zskcurpath        "./Kexample.com.+005+53135.key"
        zskcur            "Kexample.com.+005+53135"
        signedfile        "db.example.com.signed"
        kskpath           "./Kexample.com.+005+24082.key"
        kskkey            "Kexample.com.+005+24082"
        kskdirectory      "."
        endtime           "+259200"
        zonefile          "db.example.com"
        keyrec_type       "zone"
        keyrec_signsecs   "1115166642"
        keyrec_signdate   "Wed May  4 00:30:42 2005"

    key                "Kexample.com.+005+24082"
        zonename          "example.com"
        keyrec_type       "ksk"
        algorithm         "rsasha1"
        random            "/dev/urandom"
        keypath           "./Kexample.com.+005+24082.key"
        ksklength         "1024"
        keyrec_gensecs    "1115166638"
        keyrec_gendate    "Wed May  4 00:30:38 2005"

    key                "Kexample.com.+005+53135"
        zonename         "example.com"
        keyrec_type       "zskcur"
        algorithm         "rsasha1"
        random            "/dev/urandom"
        keypath           "./Kexample.com.+005+53135.key"
        zsklength         "512"
        keyrec_gensecs    "1115166638"
        keyrec_gendate    "Wed May  4 00:30:38 2005"

    key                "Kexample.com.+005+13531"
        zonename          "example.com"
        keyrec_type       "zskpub"
        algorithm         "rsasha1"
        random            "/dev/urandom"
        keypath           "./Kexample.com.+005+13531.key"
        zsklength         "512"
        keyrec_gensecs    "1115166638"
        keyrec_gendate    "Wed May  4 00:30:38 2005"
\end{verbatim}


{\bf NOTES}

\begin{description}

\item [1.  SOA Serial Numbers]\verb" "

Serial numbers in SOA records are merely incremented in this version.  Future
plans are to allow for more flexible serial number manipulation.

\end{description}

{\bf SEE ALSO}

\perlmod{Net::DNS::SEC::Tools::conf.pm(3)},\\
\perlmod{Net::DNS::SEC::Tools::keyrec.pm(3)},\\
\perlmod{Net::DNS::SEC::Tools::tooloptions.pm(3)}

