\clearpage

\subsubsection{\bf zonesigner}

{\bf NAME}

\cmd{zonesigner} - Generates encryption keys and signs a DNS zone.

{\bf SYNOPSIS}

\begin{verbatim}
    zonesigner [options] <zone-file> <signed-zone>
\end{verbatim}

{\bf DESCRIPTION}

This script combines into a single command many actions that are required to
sign a DNS zone.  It generates the required KSK and ZSK keys, adds the key
data to a zone record file, signs the zone file, and runs checks to ensure
that everything worked properly.  It also keeps records about the keys and
how the zone was signed in order to facilitate re-signing of the zone in the
future.

The \cmd{zonesigner}-specific zone-signing records are kept in {\it keyrec}
files.  Using {\it keyrec} files, defined and maintained by DNSSEC-Tools,
\cmd{zonesigner} can automatically gather many of the options used to previously
sign and generate a zone and its keys.  This allows the zone to be maintained
using the same key lengths and expiration times, for example, without an
administrator needing to manually track these fields.

{\bf QUICK START}

The following are examples that will allow a quick start on using
\cmd{zonesigner}:

\begin{description}

\item {\it first run on example.com}\verb" "

The following command will generate keys and sign the zone file for
example.com, giving an expiration date 31 days in the future.  The zone
file is named \path{example.com} and the signed zone file will be named
\path{example.com.signed}.

\begin{verbatim}
    zonesigner -genkeys -endtime +2678400 example.com
\end{verbatim}

\item {\it subsequent runs on example.com}\verb" "

The following command will re-sign example.com's zone file, but will not
generate new keys.  The files and all key-generation and zone-signing
arguments will remain the same.

\begin{verbatim}
    zonesigner example.com
\end{verbatim}

\end{description}

{\bf USING ZONESIGNER}

\cmd{zonesigner} is used in this way:

\begin{verbatim}
    zonesigner [options] <zone-file> <signed-zone>
\end{verbatim}

The {\it zone-file} arguments is required.  If {\it signed-zone} is not
specified, then the signed zone file will be named {\it signed-zone.signed}.

{\it zone-file} is the name of the zone file from which a signed zone file
will be created.  If the {\it -zone} option is not given, then {\it zone-file}
will be used as the name of the zone that will be signed.  Generated keys are
given this name as their base.

The zone file is modified to have {\bf include} commands, which will include the
KSK and ZSK keys.  These lines are placed at the end of the file and should
not be modified by the user.  If the zone file already includes any key files,
those inclusions will be deleted.  These lines are distinguished by starting
with ``\$INCLUDE'' and end with ``.key''.  Only the actual include lines are
deleted; any related comment lines are left untouched.

An intermediate file is used in signing the zone.  {\it zone-file} is copied to
the intermediate file and is modified in preparation of signing the zone file.
Several \$INCLUDE lines will be added at the end of the file and the SOA serial
number will be incremented.

{\it signed-zone} is the name of the signed zone file.  If it is not given on
the command line, the default signed zone filename is the {\it zone-file}
appended with ``.signed''.  Thus, executing \cmd{zonesigner example.com} will
result in the signed zone being stored in {\it example.com.signed}.

Unless the {\it -genkeys}, {\it -genksk}, or {\it -genzsk} options are
specified, the last keys generated for a particular zone will be used in
subsequent \cmd{zonesigner} executions.

{\bf KEYREC FILES}

{\it keyrec} files retain information about previous key-generation and
zone-signing operations.  If a {\it keyrec} file is not specified (by way of
the {\it -krfile} option), then a default {\it keyrec} file is used.  If this
default is not specified in the system's DNSSEC-Tools configuration file, the
filename will be the zone name appended with {\bf .krf}.  If the {\it
-nokrfile} option is given, then no {\it keyrec} file will be consulted or
saved.

{\it keyrec} files contain three types of entries:  zone {\it keyrec}s, set
{\it keyrec}s, and key {\it keyrec}s.  Zone {\it keyrec}s contain information
specifically about the zone, such as the number of ZSKs used to sign the zone,
the end-time for the zone, and the key signing set names (names of set {\it
keyrecs}.)  Set {\it keyrec}s contain lists of keys names used for a specific
purpose, such as the current ZSK keys or the published ZSK keys.  Key {\it
keyrec}s contain information about the generated keys themselves, such as
encryption algorithm, key length, and key lifetime.

Each {\it keyrec} contains a set of ``key/value'' entries, one per line.
Example 4 below contains the contents of a sample {\it keyrec} file.

{\bf ENTROPY}

On some systems, the implementation of the pseudo-random number generator
requires keyboard activity.  This keyboard activity is used to fill a buffer
in the system's random number generator.  If \cmd{zonesigner} appears hung,
you may have to add entropy to the random number generator by randomly
striking keys until the program completes.  Display of this message is
controlled by the {\bf entropy\_msg} configuration file parameter.

{\bf DETERMINING OPTION VALUES}

\cmd{zonesigner} checks four places in order to determine option values.  
In descending order of precedence, these places are:

\begin{itemize}
\item command line options
\item keyrec file
\item DNSSEC-Tools configuration file
\item zonesigner defaults
\end{itemize}

Each is checked until a value is found.  That value is then used for that
\cmd{zonesigner} execution and the value is stored in the {\it keyrec} file.

{\bf Example}

For example, the KSK length has the following values:

\begin{verbatim}
    -ksklength command line option:         8192

    keyrec file:                            1024

    DNSSEC-Tools configuration file:        2048

    zonesigner defaults:                    512
\end{verbatim}

If all are present, then the KSK length will be 8192.

If the {\it -ksklength} command line option wasn't given, the KSK length
will be 1024.

If the KSK length wasn't given in the configuration file, it will be 8192.

If the KSK length wasn't in the {\it keyrec} file or the configuration file,
the KSK length will be 8192.

If the {\it -ksklength} command line option wasn't given and the KSK length
wasn't in the configuration file, it'll be 1024.

If the command line option wasn't given, the KSK length wasn't in the
{\it keyrec} file, and it wasn't in the configuration file, then the KSK
length will be 512.

{\bf OPTIONS}

Three types of options may be given, based on the command for which they are
intended.  These commands are  \cmd{dnssec-keygen}, \cmd{dnssec-signzone}, and
\cmd{zonesigner}.

{\bf \cmd{zonesigner}-specific Options}

\begin{description}

\item {\it -nokrfile}\verb" "

No {\it keyrec} file will be consulted or created.

\item {\it -krfile}\verb" "

{\it keyrec} file to use in processing options.  See the man page for
\perlmod{tooloptions.pm} for more details about this file.

\item {\it -genkeys}\verb" "

Generate a new KSK and ZSK for the zone.

\item {\it -genksk}\verb" "

Generate a new KSK for the zone.  By default, the last KSK generated for this
zone will be used.

\item {\it -genzsk}\verb" "

Generate a new ZSK for the zone.  By default, the last ZSK generated for this
zone will be used.

\item {\it -usepub}\verb" "

Use the existing Published ZSK to sign the zone.

\item {\it -archivedir}\verb" "

The key archive directory.

\item {\it -savekeys}\verb" "

Save obsolete keys to the key archive directory.

\item {\it -ksklife}\verb" "

The time between KSK rollovers.  This is measured in seconds.

\item {\it -zsklife}\verb" "

The time between ZSK rollovers.  This is measured in seconds.

\item {\it -zskcount}\verb" "

The number of ZSK keys to generate and with which to sign the zone.  The
default is to use a single ZSK key.

\item {\it -signset}\verb" "

The name of the ZSK signing set to use.  If the signing set does not exist,
then this must be used in conjunction with either {\it -genkeys} or {\it
-genzsk}.  The name may contain alphanumerics, underscores, hyphens, periods,
and commas.

The default signing set name is ``signing-set-{\it N}'', where {\it N} is a
number.  If {\it -signset} is not specified, then \cmd{zonesigner} will use
the default and increment the number for subsequent signing sets.

\item {\it -forceroll}\verb" "

Force a rollover of the ZSK keys.  The {\it keyrec}s of the ZSK keys are
adjusted as follows:

\begin{verbatim}
    The current ZSK key is marked as obsolete.
    The published ZSK key is marked as current.
    The new ZSK key, if it exists, is marked as published.
    A new ZSK key is generated.
    The published ZSK key's zsklife field is copied to the
    new ZSK key's keyrec.
\end{verbatim}

This should only be used if you know what you're doing.

\item {\it -intermediate}\verb" "

Filename to use for the temporary zone file.  The zone file will be copied to
this file and then the key names appended.

\item {\it -zone}\verb" "

Name of the zone that will be signed.  This zone name may be given with this
option or as the first non-option command line argument.

\item {\it -help}\verb" "

Display a usage message.

\item {\it -Version}\verb" "

Display the version information for zonesigner and the DNSSEC-Tools package.

\item {\it -verbose}\verb" "

Verbose output will be given.  As more instances of {\it -verbose} are given on
the command line, additional levels of verbosity are achieved.

\begin{table}[ht]
\begin{center}
\begin{tabular}{c|l}
level  & output							 \\
\hline
1 & operations being performed					 \\
  & (e.g., generating key files, signing zone)			 \\
2 & details on operations and some operation results		 \\
  & (e.g., new key names, zone serial number)			 \\
3 & operations' parameters and additional details		 \\
  & (e.g., key lengths, encryption algorithm, executed commands) \\
\end{tabular} 
\end{center}
\end{table}

Higher levels of verbosity are cumulative.  Specifying two instances of
{\it -verbose} will get the output from the first and second levels of output.

\end{description}

{\bf \cmd{dnssec-keygen}-specific Options}

\begin{description}

\item {\it -algorithm}\verb" "

Cryptographic algorithm used to generate the zone's keys.  The default value
is RSASHA1.  The option value is passed to \cmd{dnssec-keygen} as the the {\it
-a} flag.  Consult \cmd{dnssec-keygen}'s manual page for to determine legal
values.

\item {\it -ksklength}\verb" "

Bit length of the zone's KSK key.
The default is 1024.

\item {\it -random}\verb" "

Source of randomness used to generate the zone's keys.	(/dev/urandom)

\item {\it -zsklength}\verb" "

Bit length of the zone's ZSK key.
The default is 512.

\item {\it -kgopts}\verb" "

Additional options for \cmd{dnssec-keygen} may be specified using this option.
The additional options are passed as a single string value as an argument to
the {\it -kgopts} option.

\end{description}

{\bf \cmd{dnssec-signzone}-specific Options}

\begin{description}

\item {\it -endtime}\verb" "

Time that the zone expires, measured in seconds.  See the man page for
\cmd{dnssec-signzone} for the valid format of this field.
The default value is 2592000 seconds (30 days.)

\item {\it -gends}\verb" "

Force \cmd{dnssec-signzone} to generate DS records for the zone.  This option
is translated into {\it -g} when passed to \cmd{dnssec-signzone}.

\item {\it -ksdir}\verb" "

Specify a directory for storing keysets.  This is passed to
\cmd{dnssec-signzone} as the {\it -d} option.

\item {\it -szopts}\verb" "

Additional options for \cmd{dnssec-signzone} may be specified using this
option.  The additional options are passed as a single string value as an
argument to the {\it -szopts} option.

\end{description}

{\bf Examples}

Example 1.

In the first example, an existing {\it keyrec} file is used to assist in
signing the example.com domain.  Zone data are stored in \path{example.com},
and the keyrec is in \path{example.krf}.  The final signed zone file will be
\path{db.example.com}.  Using this execution:

\begin{verbatim}
   # zonesigner -krfile example.krf example.com db.example.com.signed
\end{verbatim}

the following files are created:

\begin{verbatim}
    Kexample.com.+005+45842.private
    Kexample.com.+005+45842.key
    Kexample.com.+005+50186.private
    Kexample.com.+005+50186.key
    Kexample.com.+005+59143.private
    Kexample.com.+005+59143.key

    dsset-example.com.
    keyset-example.com.

    db.example.com.signed
\end{verbatim}

The first six files are the KSK and ZSK keys required for the zone.  The next
two files are created by the zone-signing process.  The last file is the zone
the final signed zone file.

Example 2.

In the second example, an existing {\it keyrec} file is used to assist in
signing the example.com domain.  Zone data are stored in \path{example.com},
and the keyrec is in \path{example.krf}.  The generated keys, an intermediate
zone file, and final signed zone file will use \path{example.com} as a base.
Using this execution:

\begin{verbatim}
    # zonesigner -krfile example.krf -intermediate example.zs \
	    example.com db.example.com
\end{verbatim}

the following files are created:

\begin{verbatim}
    Kdb.example.com.+005+12354.key
    Kdb.example.com.+005+12354.private
    Kdb.example.com.+005+82197.key
    Kdb.example.com.+005+82197.private
    Kdb.example.com.+005+55888.key
    Kdb.example.com.+005+55888.private

    dsset-db.example.com.
    keyset-db.example.com.

    example.zs
    db.example.com
\end{verbatim}

The first six files are the KSK and ZSK keys required for the zone.  The next
two files are created by the zone-signing process.  The second last file is
an intermediate file that will be signed.  The last file is file is the final
signed zone.

Example 3.

In the third example, no {\it keyrec} file is specified for the signing of the
example.com domain.  In addition to files created as shown in previous
examples, a new {\it keyrec} file is created.  The new {\it keyrec} file uses
the domain name as its base.  Using this execution:

\begin{verbatim}
    # zonesigner example.com db.example.com
\end{verbatim}

the following {\it keyrec} file is created:

\begin{verbatim}
    example.com.krf
\end{verbatim}

The signed zone file is created in:

\begin{verbatim}
    db.example.com
\end{verbatim}

Example 4.

This example shows a {\it keyrec} file generated by \cmd{zonesigner}.

The command executed is:

\begin{verbatim}
    # zonesigner example.com db.example.com
\end{verbatim}

The generated {\it keyrec} file contains six {\it keyrec}s:  a zone
{\it keyrec}, two set {\it keyrec}s, one KSK {\it keyrec}, and two ZSK
{\it keyrec}s.

\begin{verbatim}
    zone        "example.com"
        zonefile         "db.example.com"
        signedzone       "db.example.com.signed"
        endtime          "+2592000"
        kskpath          "./Kexample.com.+005+24082.key"
        kskkey           "Kexample.com.+005+24082"
        kskdirectory     "."
        zskcur           "signing-set-42"
        zskpub           "signing-set-43"
        zskdirectory     "."
        keyrec_type      "zone"
        keyrec_signsecs  "1115166642"
        keyrec_signdate  "Wed May  4 00:30:42 2005"

    set                "signing-set-42"
        zonename         "example.com"
        keys             "Kexample.com.+005+53135"
        keyrec_setsecs   "1115166640"
        keyrec_setdate   "Wed May  4 00:30:40 2005"

    set                "signing-set-43"
        zonename         "example.com"
        keys             "Kexample.com.+005+13531"
        keyrec_setsecs   "1115166641"
        keyrec_setdate   "Wed May  4 00:30:41 2005"

    key                "Kexample.com.+005+24082"
        zonename         "example.com"
        keyrec_type      "ksk"
        algorithm        "rsasha1"
        random           "/dev/urandom"
        keypath          "./Kexample.com.+005+24082.key"
        ksklength        "1024"
        ksklife          "15768000"
        keyrec_gensecs   "1115166638"
        keyrec_gendate   "Wed May  4 00:30:38 2005"

    key                "Kexample.com.+005+53135"
        zonename         "example.com"
        keyrec_type      "zskcur"
        algorithm        "rsasha1"
        random           "/dev/urandom"
        keypath          "./Kexample.com.+005+53135.key"
        zsklength        "512"
        zsklife          "604800"
        keyrec_gensecs   "1115166638"
        keyrec_gendate   "Wed May  4 00:30:38 2005"

    key                "Kexample.com.+005+13531"
        zonename         "example.com"
        keyrec_type      "zskpub"
        algorithm        "rsasha1"
        random           "/dev/urandom"
        keypath          "./Kexample.com.+005+13531.key"
        zsklength        "512"
        zsklife          "604800"
        keyrec_gensecs   "1115166638"
        keyrec_gendate   "Wed May  4 00:30:38 2005"
\end{verbatim}


{\bf NOTES}

\begin{enumerate}

\item SOA Serial Numbers\verb" "

Serial numbers in SOA records are merely incremented in this version.  Future
plans are to allow for more flexible serial number manipulation.

\end{enumerate}

{\bf SEE ALSO}

\cmd{dnssec-keygen(8)},
\cmd{dnssec-signzone(8)}

\perlmod{Net::DNS::SEC::Tools::conf.pm(3)},
\perlmod{Net::DNS::SEC::Tools::defaults.pm(3)},	\\
\perlmod{Net::DNS::SEC::Tools::keyrec.pm(3)},
\perlmod{Net::DNS::SEC::Tools::tooloptions.pm(3)}

\path{keyrec(5)}
