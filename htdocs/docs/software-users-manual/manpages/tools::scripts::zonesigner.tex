\clearpage

\subsubsection{zonesigner}

{\bf NAME}

\cmd{zonesigner} - Generates encryption keys and signs a DNS zone

{\bf SYNOPSIS}

\begin{verbatim}
  zonesigner [options] <zone-file> <signed-zone-file>

  # get started immediately examples:

  # first run on a zone for example.com:
  zonesigner -genkeys -endtime +2678400 example.com

  # future runs before expiration time (reuses the same keys):
  zonesigner -endtime +2678400 example.com
\end{verbatim}

{\bf DESCRIPTION}

This script combines into a single command many actions that are required to
sign a DNS zone.  It generates the required KSK and ZSK keys, adds the key
data to a zone record file, signs the zone file, and runs checks to ensure
that everything worked properly.  It also keeps records about the keys and
how the zone was signed in order to facilitate re-signing of the zone in the
future.

The \cmd{zonesigner}-specific zone-signing records are kept in \struct{keyrec}
files.  Using \struct{keyrec} files, defined and maintained by DNSSEC-Tools,
\cmd{zonesigner} can automatically gather many of the options used to
previously sign and generate a zone and its keys.  This allows the zone to be
maintained using the same key lengths and expiration times, for example,
without an administrator needing to manually track these fields.

{\bf QUICK START}

The following are examples that will allow a quick start on using
\cmd{zonesigner}:

\begin{description}

\item first run on example.com\verb" "

The following command will generate keys and sign the zone file for
example.com, giving an expiration date 31 days in the future.  The
zone file is named \path{example.com} and the signed zone file will be
named \path{example.com.signed}.

\begin{verbatim}
    zonesigner -genkeys -endtime +2678400 example.com
\end{verbatim}

\item subsequent runs on example.com\verb" "

The following command will re-sign example.com's zone file, but will not
generate new keys.  The files and all key-generation and zone-signing
arguments will remain the same.

\begin{verbatim}
    zonesigner example.com
\end{verbatim}

\end{description}

{\bf USING ZONESIGNER}

\cmd{zonesigner} is used in this way:

\begin{verbatim}
    zonesigner [options] <zone-file> <signed-zone-file>
\end{verbatim}

The {\it zone-file} argument is required.

{\it zone-file} is the name of the zone file from which a signed zone file
will be created.  If the {\it -zone} option is not given, then {\it zone-file}
will be used as the name of the zone that will be signed.  Generated keys are
given this name as their base.

The zone file is modified to have {\bf include} commands, which will include
the KSK and ZSK keys.  These lines are placed at the end of the file and
should not be modified by the user.  If the zone file already includes any key
files, those inclusions will be deleted.  These lines are distinguished by
starting with ``\$INCLUDE'' and end with \path{.key}.  Only the actual include
lines are deleted; any related comment lines are left untouched.

An intermediate file is used in signing the zone.  {\it zone-file} is copied
to the intermediate file and is modified in preparation of signing the zone
file.  Several \$INCLUDE lines will be added at the end of the file and the
SOA serial number will be incremented.

{\it signed-zone} is the name of the signed zone file.  If it is not given on
the command line, the default signed zone filename is the {\it zone-file}
appended with \path{.signed}.  Thus, executing \xqt{zonesigner example.com}
will result in the signed zone being stored in \path{example.com.signed}.

Unless the {\it -genkeys}, {\it -genksk}, {\it -genzsk}, or {\it -newpubksk}
options are specified, the last keys generated for a particular zone will be
used in subsequent \cmd{zonesigner} executions.

{\bf KEYREC FILES}

\struct{keyrec} files retain information about previous key-generation and
zone-signing operations.  If a \struct{keyrec} file is not specified (by way
of the {\it -krfile} option), then a default \struct{keyrec} file is used.  If
this default is not specified in the system's DNSSEC-Tools configuration file,
the filename will be the zone name appended with \path{.krf}.  If the {\it
-nokrfile} option is given, then no \struct{keyrec} file will be consulted or
saved.

\struct{keyrec} files contain three types of entries:  zone \struct{keyrec}s,
set \struct{keyrec}s, and key \struct{keyrec}s.  Zone \struct{keyrec}s contain
information specifically about the zone, such as the number of ZSKs used to
sign the zone, the end-time for the zone, and the key signing set names (names
of set \struct{keyrec}s.) Set \struct{keyrec}s contain lists of key
\struct{keyrec} names used for a specific purpose, such as the current ZSK
keys or the published ZSK keys.  Key \struct{keyrec}s contain information
about the generated keys themselves, such as encryption algorithm, key length,
and key lifetime.

Each \struct{keyrec} contains a set of ``key/value'' entries, one per line.
Example 4 below contains the contents of a sample \struct{keyrec} file.

{\bf ENTROPY}

On some systems, the implementation of the pseudo-random number generator
requires keyboard activity.  This keyboard activity is used to fill a buffer
in the system's random number generator.  If \cmd{zonesigner} appears hung,
you may have to add entropy to the random number generator by randomly
striking keys until the program completes.  Display of this message is
controlled by the {\bf entropy\_msg} configuration file parameter.

{\bf DETERMINING OPTION VALUES}

\cmd{zonesigner} checks four places in order to determine option values.  
In descending order of precedence, these places are:

\begin{description}
\item command line options
\item keyrec file
\item DNSSEC-Tools configuration file
\item zonesigner defaults
\end{description}

Each is checked until a value is found.  That value is then used for that
\cmd{zonesigner} execution and the value is stored in the \struct{keyrec} file.

{\bf Example}

For example, the KSK length has the following values:

\begin{table}[ht]
\begin{tabular}{clr}
& -ksklength command line option  & 8192 \\
& keyrec file                     & 1024 \\
& DNSSEC-Tools configuration file & 2048 \\
& zonesigner defaults             & 512 \\
\end{tabular}
\end{table}

If all are present, then the KSK length will be 8192.

If the {\it -ksklength} command line option wasn't given, the KSK length
will be 1024.

If the KSK length wasn't given in the configuration file, it will be 8192.

If the KSK length wasn't in the \struct{keyrec} file or the configuration
file, the KSK length will be 8192.

If the {\it -ksklength} command line option wasn't given and the KSK length
wasn't in the configuration file, it'll be 1024.

If the command line option wasn't given, the KSK length wasn't in the
\struct{keyrec} file, and it wasn't in the configuration file, then the KSK
length will be 512.

{\bf OPTIONS}

Three types of options may be given, based on the command for which they are
intended.  These commands are  \cmd{dnssec-keygen}, \cmd{dnssec-signzone}, and
\cmd{zonesigner}.

{\bf \cmd{zonesigner}-specific Options}

\begin{description}

\item {\bf -nokrfile}\verb" "

No \struct{keyrec} file will be consulted or created.

\item {\bf -krfile}\verb" "

\struct{keyrec} file to use in processing options.  See the man page for the
DNSSEC-Tools \perlmod{tooloptions.pm} module for more details about this file.

\item {\bf -genkeys}\verb" "

Generate new KSKs and ZSKs for the zone.

\item {\bf -genksk}\verb" "

Generate new Current KSKs for the zone.  Any existing Current KSKs will be
marked as obsolete.  If this option is not given, the last KSKs generated for
this zone will be used.

\item {\bf -genzsk}\verb" "

Generate new ZSKs for the zone.  By default, the last ZSKs generated for this
zone will be used.

\item {\bf -newpubksk}\verb" "

Generate new Published KSKs for the zone.  Any existing Published KSKs will
be marked as obsolete.

\item {\bf -useboth}\verb" "

Use the existing Current {\bf and} Published ZSKs to sign the zone.

\item {\bf -usezskpub}\verb" "

Use the existing Published ZSKs to sign the zone.

\item {\bf -archivedir}\verb" "

The key archive directory.  If a key archive directory hasn't been specified
(on the command line or in the DNSSEC-Tools configuration file) and the {\it
-nosave} option was {\bf not} given, an error message will be displayed and
\cmd{zonesigner} will exit.

When the files are saved into the archive directory, the existing file names
are prepended with a timestamp.  The timestamp indicates when the files are
archived.

This directory {\bf may not} be the root directory.

\item {\bf -nosave}\verb" "

Do not save obsolete keys to the key archive directory.  The default behavior
is to save obsolete keys.

\item {\bf -kskcount}\verb" "

The number of KSK keys to generate and with which to sign the zone.  The
default is to use a single KSK key.

\item {\bf -ksklife}\verb" "

The time between KSK rollovers.  This is measured in seconds.

\item {\bf -ksignset}\verb" "

The name of the KSK signing set to use.  If the signing set does not exist,
then this must be used in conjunction with either {\it -genkeys} or {\it
-genksk}.  The name may contain alphanumerics, underscores, hyphens, periods,
and commas.

The default signing set name is ``signing-set-{\it N}'', where {\it N} is a
number.  If {\it -signset} is not specified, then \cmd{zonesigner} will use
the default and increment the number for subsequent signing sets.

\item {\bf -zsklife}\verb" "

The time between ZSK rollovers.  This is measured in seconds.

\item {\bf -zskcount}\verb" "

The number of ZSK keys to generate and with which to sign the zone.  The
default is to use a single ZSK key.

\item {\bf -signset}\verb" "

The name of the ZSK signing set to use as the Current ZSK signing set.  The
zone is signed and the given signing set becomes the zone's new Current ZSK
signing set.  If the signing set does not exist, then this must be used in
conjunction with either {\it -genkeys} or {\it -genzsk}.

The name may contain alphanumerics, underscores, hyphens, periods, and commas.
The default signing set name is ``signing-set-{\it N}'', where {\it N} is a
number.  If {\it -signset} is not specified, then \cmd{zonesigner} will use
the default and increment the number for subsequent signing sets.

\item {\bf -rollksk}\verb" "

Force a rollover of the KSK keys.  The Current KSK keys are marked as Obsolete
and the Published KSK keys are marked as Current.  The zone is then signed
with the new set of Current KSK keys.  If the zone's \struct{keyrec} does not
list a Current or Published KSK, an error message is printed and
\cmd{zonesigner} stops execution.

The zone's \struct{keyrec} file is updated to show the new key state.

The \struct{keyrec}s of the KSK keys are adjusted as follows:

\begin{enumerate}
\item The Current KSK keys are marked as Obsolete.
\item The Published KSK keys are marked as Current.
\item The obsolete KSK keys are moved to the archive directory.
\end{enumerate}

{\bf Warning}:  The timing of key-rolling is critical.  Great care must be
taken when using this option.  \cmd{rollerd} automates the KSK rollover
process and may be used to safely take care of this aspect of DNSSEC
management.

{\bf Warning}:  Using the {\it -rollksk} option should only be used if you
know what you're doing.

{\bf Warning}:  This is a {\it temporary} method of KSK rollover.  It {\it
may} be changed in the future.

\item {\bf -rollzsk}\verb" "

Force a rollover of the ZSK keys using the Pre-Publish Key Rollover method.
The rollover process adjusts the keys used to sign the specified zone,
generates new keys, signs the zone with the appropriate keys, and updates the
\struct{keyrec} file.  The Pre-Publish Key Rollover process is described in the
DNSSEC Operational Practices document.

Three sets of ZSK keys are used in the rollover process:  Current, Published,
and New.  Current ZSKs are those which are used to sign the zone.  Published
ZSKs are available in the zone data, and therefore in cached zone data, but
are not yet used to sign the zone.  New ZSKs are not available in zone data
nor yet used to sign the zone, but are waiting in the wings for future use.

The \struct{keyrec}s of the ZSK keys are adjusted as follows:

\begin{enumerate}
\item The Current ZSK keys are marked as obsolete.
\item The Published ZSK keys are marked as Current.
\item The New ZSK keys, if they exist, are marked as Published.
\item Another set of ZSK keys are generated, which will be
        marked as the New ZSK keys.
\item The Published ZSK keys' zsklife field is copied to the
        new ZSK keys' keyrecs.
\item The obsolete ZSK keys are moved to the archive directory.
\end{enumerate}

The quick summary of proper ZSK rolling (which \cmd{rollerd} does for you if
you use it):

\begin{enumerate}
\item wait 2 * max(TTL in zone)
\item run zonesigner using -usezskpub
\item wait 2 * max(TTL in zone)
\item run zonesigner using -rollzsk
\item wait 2 * max(TTL in zone)
\end{enumerate}

{\bf Warning}:  The timing of key-rolling is critical.  Great care must be taken
when using this option.  {\bf rollerd} automates the rollover process and may be
used to safely take care of this aspect of DNSSEC management.  Using the
{\it -rollzsk} option should only be used if you know what you're doing.

\item {\bf -intermediate}\verb" "

Filename to use for the temporary zone file.  The zone file will be copied to
this file and then the key names appended.

\item {\bf -zone}\verb" "

Name of the zone that will be signed.  This zone name may be given with this
option or as the first non-option command line argument.

\item {\bf -help}\verb" "

Display a usage message.

\item {\bf -Version}\verb" "

Display the version information for \cmd{zonesigner} and the DNSSEC-Tools
package.

\item {\bf -verbose}\verb" "

Verbose output will be given.  As more instances of {\it -verbose} are given
on the command line, additional levels of verbosity are achieved.

\begin{table}[ht]
\begin{center}
\begin{tabular}{|c|l|}
\hline
{\bf Verbosity Level} & {\bf Output} \\
\hline
1 & operations being performed \\
  & (e.g., generating key files, signing zone)  \\
2 & details on operations and some operation results \\
  & (e.g., new key names, zone serial number) \\
3 & operations' parameters and additional details \\
  & (e.g., key lengths, encryption algorithm, \\
  & executed commands) \\
\hline
\end{tabular}
\end{center}
\caption{\cmd{zonesigner} Verbosity Levels}
\end{table}

Higher levels of verbosity are cumulative.  Specifying two instances of
{\it -verbose} will get the output from the first and second levels of output.

\item {\bf -showkeycmd}\verb" "

Display the actual key-generation command (with options and arguments) that is
executed.  This is a small subset of verbose level 3 output.

\item {\bf -showsigncmd}\verb" "

Display the actual zone-signing command (with options and arguments) that is
executed.  This is a small subset of verbose level 3 output.

\end{description}

{\bf \cmd{dnssec-keygen}-specific Options}

\begin{description}

\item {\bf -algorithm}\verb" "

Cryptographic algorithm used to generate the zone's keys.  The default value
is RSASHA1.  The option value is passed to \cmd{dnssec-keygen} as the the {\it
-a} flag.  Consult \cmd{dnssec-keygen}'s manual page to determine legal values.

\item {\bf -ksklength}\verb" "

Bit length of the zone's KSK key.
The default is 1024.

\item {\bf -random}\verb" "

Source of randomness used to generate the zone's keys.  This is assumed to be
a file, for example \path{/dev/urandom}.

\item {\bf -zsklength}\verb" "

Bit length of the zone's ZSK key.
The default is 512.

\item {\bf -kgopts}\verb" "

Additional options for \cmd{dnssec-keygen} may be specified using this option.
The additional options are passed as a single string value as an argument to
the {\it -kgopts} option.

\end{description}

{\bf \cmd{dnssec-signzone}-specific Options}

\begin{description}

\item {\bf -endtime}\verb" "

Time that the zone expires, measured in seconds.  See the man page for
\cmd{dnssec-signzone} for the valid format of this field.
The default value is 2592000 seconds (30 days.)

\item {\bf -gends}\verb" "

Force \cmd{dnssec-signzone} to generate DS records for the zone.  This option
is translated into {\it -g} when passed to \cmd{dnssec-signzone}.

\item {\bf -ksdir}\verb" "

Specify a directory for storing keysets.  This is passed to
\cmd{dnssec-signzone} as the {\it -d} option.

\item {\bf -szopts}\verb" "

Additional options for \cmd{dnssec-signzone} may be specified using this
option.  The additional options are passed as a single string value as an
argument to the {\it -szopts} option.

\end{description}

{\bf Other Options}

\begin{description}

\item {\bf -zcopts}\verb" "

Additional options for \cmd{named-checkzone} may be specified using this option.
The additional options are passed as a single string value as an argument to
the {\it -zcopts} option.

\end{description}

{\bf Examples}

Example 1.

In the first example, an existing \struct{keyrec} file is used to assist in
signing the example.com domain.  Zone data are stored in \path{example.com},
and the keyrec is in \path{example.krf}.  The final signed zone file will be
\path{db.example.com.signed}.  Using this execution:

\begin{verbatim}
    # zonesigner -krfile example.krf example.com db.example.com.signed
\end{verbatim}

the following files are created:

\eject

\begin{table}[ht]
\begin{tabular}{cl}
 & \path{Kexample.com.+005+45842.private} \\
 & \path{Kexample.com.+005+45842.key} \\
 & \path{Kexample.com.+005+50186.private} \\
 & \path{Kexample.com.+005+50186.key} \\
 & \path{Kexample.com.+005+59143.private} \\
 & \path{Kexample.com.+005+59143.key} \\
 & \path{dsset-example.com.} \\
 & \path{keyset-example.com.} \\
 & \path{db.example.com.signed} \\
\end{tabular}
\end{table}

The first six files are the KSK and ZSK keys required for the zone.  The next
two files are created by the zone-signing process.  The last file is the 
final signed zone file.

Example 2.

In the second example, an existing \struct{keyrec} file is used to assist in
signing the example.com domain.  Zone data are stored in \path{example.com},
and the keyrec is in \path{example.krf}.  The generated keys, an intermediate
zone file, and final signed zone file will use \path{example.com} as a base.
Using this execution:

\begin{verbatim}
    # zonesigner -krfile example.krf -intermediate example.zs example.com
\end{verbatim}

the following files are created:

\begin{table}[ht]
\begin{tabular}{cl}
 & \path{Kdb.example.com.+005+12354.key} \\
 & \path{Kdb.example.com.+005+12354.private} \\
 & \path{Kdb.example.com.+005+82197.key} \\
 & \path{Kdb.example.com.+005+82197.private} \\
 & \path{Kdb.example.com.+005+55888.key} \\
 & \path{Kdb.example.com.+005+55888.private} \\
 & \path{dsset-db.example.com.} \\
 & \path{keyset-db.example.com.} \\
 & \path{example.zs} \\
 & \path{example.com.signed} \\
\end{tabular}
\end{table}

The first six files are the KSK and ZSK keys required for the zone.  The next
two files are created by the zone-signing process.  The second last file is
an intermediate file that will be signed.  The last file is file is the final
signed zone.

Example 3.

In the third example, no \struct{keyrec} file is specified for the signing of
the example.com domain.  In addition to files created as shown in previous
examples, a new \struct{keyrec} file is created.  The new \struct{keyrec} file
uses the domain name as its base.  Using this execution:

\begin{verbatim}
    # zonesigner example.com db.example.com
\end{verbatim}

The \struct{keyrec} file is created as \path{example.com.krf}.

The signed zone file is created in \path{db.example.com}.

Example 4.

This example shows a \struct{keyrec} file generated by \cmd{zonesigner}.

The command executed is:

\begin{verbatim}
    # zonesigner example.com db.example.com
\end{verbatim}

The generated \struct{keyrec} file contains six \struct{keyrec}s:  a zone
\struct{keyrec}, two set \struct{keyrec}s, one KSK \struct{keyrec}, and two
ZSK \struct{keyrec}s.

\begin{verbatim}
    zone        "example.com"
        zonefile        "example.com"
        signedzone      "db.example.com"
        endtime         "+2592000"
        kskcur          "signing-set-24"
        kskdirectory    "."
        zskcur          "signing-set-42"
        zskpub          "signing-set-43"
        zskdirectory    "."
        keyrec_type     "zone"
        keyrec_signsecs "1115166642"
        keyrec_signdate "Wed May  4 00:30:42 2005"

    set                "signing-set-24"
        zonename        "example.com"
        keys            "Kexample.com.+005+24082"
        keyrec_setsecs  "1110000042"
        keyrec_setdate  "Sat Mar  5 05:20:42 2005"

    set                "signing-set-42"
        zonename        "example.com"
        keys            "Kexample.com.+005+53135"
        keyrec_setsecs  "1115166640"
        keyrec_setdate  "Wed May  4 00:30:40 2005"

    set                "signing-set-43"
        zonename        "example.com"
        keys            "Kexample.com.+005+13531"
        keyrec_setsecs  "1115166641"
        keyrec_setdate  "Wed May  4 00:30:41 2005"

    key                "Kexample.com.+005+24082"
        zonename        "example.com"
        keyrec_type     "kskcur"
        algorithm       "rsasha1"
        random          "/dev/urandom"
        keypath         "./Kexample.com.+005+24082.key"
        ksklength       "1024"
        ksklife         "15768000"
        keyrec_gensecs  "1110000042"
        keyrec_gendate  "Sat Mar  5 05:20:42 2005"

    key                "Kexample.com.+005+53135"
        zonename        "example.com"
        keyrec_type     "zskcur"
        algorithm       "rsasha1"
        random          "/dev/urandom"
        keypath         "./Kexample.com.+005+53135.key"
        zsklength       "512"
        zsklife         "604800"
        keyrec_gensecs  "1115166638"
        keyrec_gendate  "Wed May  4 00:30:38 2005"

    key                "Kexample.com.+005+13531"
        zonename        "example.com"
        keyrec_type     "zskpub"
        algorithm       "rsasha1"
        random          "/dev/urandom"
        keypath         "./Kexample.com.+005+13531.key"
        zsklength       "512"
        zsklife         "604800"
        keyrec_gensecs  "1115166638"
        keyrec_gendate  "Wed May  4 00:30:38 2005"
\end{verbatim}

{\bf NOTES}

\begin{itemize}

\item One Zone in a \struct{keyrec} File\verb" "

There is a bug in the signing-set code that necessitates only storing one
zone in a \struct{keyrec} file.

\item SOA Serial Numbers\verb" "

Serial numbers in SOA records are merely incremented in this version.
Future plans are to allow for more flexible serial number manipulation.

\end{itemize}

{\bf SEE ALSO}

dnssec-keygen(8),
dnssec-signzone(8)

Net::DNS::SEC::Tools::conf.pm(3),
Net::DNS::SEC::Tools::defaults.pm(3),\\
Net::DNS::SEC::Tools::keyrec.pm(3),
Net::DNS::SEC::Tools::tooloptions.pm(3)

