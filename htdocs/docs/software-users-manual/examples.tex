
%
% The preamble begins here.
%

% Articles give "Section" as the header for \section lines.
% Reports give "Chapter" as the header for \section lines.
% Having "footer" in the arguments of the \documentstyle causes
% the footer.sty style to be used.  This causes the date to
% be printed at the bottom of each page.
%\documentstyle[footer,11pt]{report}
\documentstyle[footer,11pt]{article}

% This macro gives an itemized list, with one major exception.
% "itemized" lists have a blank line between each item; "packed"
% lists don't.
\newenvironment{packed}{\begin{list}{$\bullet$}{\parsep0in\itemsep0in}}{\end{list}}


% This command puts the paragraph header on its own line -- just like
% subsection, section, and subsubsection do.
\newcommand{\wgmparagraph}[1]{\paragraph{#1}\verb" "\\}

%
% This command allows a subsection header to be used that isn't added
% to the table of contents.
%
% WARNING:  This is a very dumb command.  It does NOT keep track of
%           section numbers -- those must be part of the heading.
%
\newcommand{\subsectionnotoc}[1]{\verb" "\\{\large\bf #1}\verb" "\\}


% These lines set a few formatting variables.  \textwidth gives you the
% width of the page.  \oddsidemargin gives the size of the margin on
% odd-numbered pages.  secnumdepth sets the maximum number of section
% numbers you'll see.  There's another counter -- tocdepth, I think --
% that lets you set the number of section numbers you'll get in a table
% of contents.
\setlength{\textwidth}{6.5in}
\setlength{\oddsidemargin}{0in}
\setcounter{secnumdepth}{8}

% Uncomment this if you don't want page numbers.
% \pagestyle{empty}

% Specify whatever title page you want here.  If you don't want a title
% page, remove this section.
\title {Report Title\\}
\author {Your Name\\
Draft \\
\\
Your Address
\\
\date {January 9, 1991}\\
\vspace{2.5in}\\
Copyright $\copyright$ 1991}


%
% The document begins here.
%

\begin{document}

% Uncomment the next two lines if you want a title page.
% \maketitle
% \clearpage

% Uncomment the next two lines if you want a table of contents.
% \tableofcontents
% \clearpage

% This puts a blank line between each paragraph.
\parskip0.65em

% This left-justifies each paragraph.
\parindent0em

\section{Test Section Header}

This line has some {\it italicized text.}

This line has some {\em emphasized text.}

This line has some bold {\bf {\it italics}} and {\bf {\em emphasized}} words.

\parindent2em
This is an indented line.
\parindent0em

\subsection{Test Subsection Header}

{\bf This whole line is bold.}

Part of this line is \underline {underlined}.

\subsubsection{Test Subsubsection Header}

This is an example of text \verb"quoted" within a line.

\begin{center}
This is a centered line.
\end{center}

Itemized list:
\begin{itemize}
\item Item Number 1
\item Item Number 2
\end{itemize}

Packed list:
\begin{packed}
\item Packed Item Number 1
\item Packed Item Number 2
\end{packed}

\section{Referencing Section}
This is how to refer to other sections:\\
See Section~\ref{myref} for a bit of reference information.

\section{Referenced Section}
\label{myref}
This section was referred to before.

To get the reference numbers to work out properly, you might have
to run latex on the source file a few times.  A general rule of
thumb is to latex the file two or three times if you have references
or a table of contents.

% This includes a postscript file.
% \special{psfile=file.ps voffset=-90 hoffset=-40 hscale=.85 vscale=.90}

% Use \cline{1-3} to draw a line across a few fields in a table.
% The numbers in the squigglies are a range of table fields.

% This is a sample of a table.
\begin{table}[h]
\caption{Sample Table}
\begin{center}
\begin{tabular}{|l|c|r|}
\hline
Field 1 & Field 2 & Field 3 \\
\hline
foo & \multicolumn{2}{|c|}{multicolumn field} \\
\hline
1 & 2 & 3 \\
\cline{1-2}
81 & 21 & 32 \\
\hline
\end{tabular}
\end{center}
\end{table}

% If you have multiple source files, you need to use a \input line
% to include the additional files.  \include also works and works
% with the \includeonly command to do something that I've forgotten.
% \input *may* be a texism that doesn't work under latex, but I
% don't think that's the case.  The file must have the .tex suffix
% and the line would like like this:
% \input{file}
%		or:
% \include{file}

{\tiny This line is really small.}

{\tt This is a teletype line.}

This line is normal.

{\large This line is bigger.}

{\Large This line is bigger still.}

{\huge These lines just keep expanding.}

{\Huge This line is the biggest of 'em all.}

\end{document}
