This section describes how one can use the interfaces provided by the validator 
library, \lib{libval(3)}, to build DNSSEC-aware programs.

The validator library allows applications to check the authenticity 
of data returned by the DNS by using the security extensions 
provided by DNSSEC.  The library exports a number of interfaces 
that applications may use in place of legacy resolver interfaces provided by the 
\lib{libres(3)} library. Additional interfaces that allow more visibility into the 
validation process are also available. This allows
for the creation of applications that are either only interested in a basic
$"validated"$ or $"not-validated"$ result or more sophisticated applications that
can look for specific errors as a sign of some network abnormality or attack.

The process of transitioning an application from using simple DNS towards being
DNSSEC-aware consists of three steps -- 
\begin{enumerate}
\item Instrumentation -- making applications capable of seeing DNSSEC results 
\item Action -- making applications act upon DNSSEC results
\item Policy -- allowing applications control over how DNSSEC status values are determined.
\end{enumerate}

Section ~\ref{instrumentation} describes the interfaces an application would use in order 
access the results of DNSSEC evaluation. Section ~\ref{getstatus} describes 
the mechanisms for obtaining the the DNSSEC status values from the above calls.
Section ~\ref{behaviour} discusses possible ways in which applications may
act based upon DNSSEC status values. Section ~\ref{configuration} introduces 
validator policy and describes the sytax of the validator configuration file. 
Section ~\ref{context} describes how applications can be made more flexible
in their choosing of validator policy at run time. Section ~\ref{detail} 
describes other validator interfaces that allow the application to obtain greater 
detail about the validation process. Section ~\ref{custom} describes how one 
could write their own custom resolver interfaces using the interfaces described in 
Section ~\ref{detail}.

%%%%%%%%%%%%%%%%%%%%%%%%%%%%%%%%%%%%%%%%%%%%%%%%%%%%%%%%%%%%%%%%%%%%

\subsection{Application Instrumentation for DNSSEC}
\label{instrumentation}
                                                                                                                             
Applications that only use \funct{res\_query()}, \funct{gethostbyname()} and
\funct{getaddrinfo()} from the \lib{libres(3)} for name resolution can be
easily made DNSSEC-aware by using similarly named interfaces provided by the
validator library -  \funct{val\_query()}, \funct{val\_gethostbyname()}
and  \funct{val\_getaddrinfo()}\footnote{We hope to eventually have most if not
all interfaces provided by libres mapped to an equivalent libval interface.}. 
These interfaces perform DNSSEC validation of DNS responses in addition to name
resolution.  It is possible to also make applications that use other interfaces 
for querying the DNS DNSSEC-aware, but this requires a little more effort as 
described in Section ~\ref{custom}.
                                                                                                                             
The prototypes of \funct{val\_query()}, \funct{val\_gethostbyname()}
and \funct{val\_getaddrinfo()} are given below:

\begin{verbatim}
    int val_query ( const char *domain_name, int class, int type,
        unsigned char *answer, int anslen, int flags,
        int *dnssec_status);

    struct hostent *val_gethostbyname(const char *name, int *h_errnop );

    int val_getaddrinfo ( const char *nodename, const char *servname,
        const struct addrinfo *hints, struct addrinfo **res );
\end{verbatim}
                                                                                                                             
\funct{val\_query()}, \funct{val\_gethostbyname} and \funct{val\_getaddrinfo()}
are semantically equivalent to \funct{res\_query()}, \funct{gethostbyname()} and
\funct{getaddrinfo()} respectively, with slightly differing syntaxes.

The essential difference between the interfaces provided by
\lib{libval(3)} and \lib{libres(3)} is that the former includes additional
parameters for controlling the behaviour and for returning the value of
the dnssec status, which the \lib{libres} interfaces do not provide.  Setting all
DNSSEC-relevant  value-parameters in interfaces exported by the libval library
to  \const{0}  provides the default validator behaviour using the default validator policy.
Configuration of the validator policy is described in section ~\ref{configuration}.
                                                                                                                             
Return values from each of the libval APIs are similar to their \lib{libres(3)}
counterparts and similar tests for success and failure must be made following invocation
of these calls. Also, the structures allocated by the \funct{val\_gethostbyname()}
and \funct{val\_getaddrinfo()} interfaces
must be free'd once they are no longer required using \funct{val\_freehostent()} and
\funct{val\_freeaddrinfo()} interfaces respectively.
                                                                                                                             
The following code snippet gives an example of how one might change code using exising
libres interfaces in place of the ones provided by \lib{libval(3)}.

\begin{verbatim}                      
#include <stdio.h>
#include <strings.h>
#include <arpa/nameser.h>
#include <validator.h>

#define BUFLEN 8096
                                                                                                                             
int use_val_query(int argc, char *argv[])
{
    int dnssec_status = ERROR;
    int anslen = NETDB_INTERNAL;
    int class = ns_c_in;
    int type = ns_t_a;
    char buf[BUFLEN];
                                                                                                                             
    bzero(buf, BUFLEN);
                                                                                                                             
    if (argc < 2) {
        printf("Usage: %s <domain-name>\n", argv[0]);
        exit(1);
    }
                                                                                                                             
    anslen = val_query(argv[1], class, type, buf, BUFLEN,
                                 0, &dnssec_status);
                                                                                                                             
    if (anslen > 0) {
        printf("DNSSEC Status = %d [%s]\n", dnssec_status,
                             p_val_error(dnssec_status));
    }
                                                                                                                             
    return 0;
}

int use_val_gethostbyname(int argc, char *argv[])
{
    int dnssec_status = ERROR;
    int val_h_errno = NETDB_INTERNAL;
    struct hostent *h = NULL;
                                                                                                                             
    if (argc < 2) {
        printf("Usage: %s <hostname>\n", argv[0]);
        exit(1);
    }
                                                                                                                             
    h = val_gethostbyname(argv[1], &val_h_errno);
    printf("h_errno = %d [%s]\n", val_h_errno,
    hstrerror(val_h_errno));
    if (h) {
        dnssec_status = val_get_hostent_dnssec_status(h);
        printf("DNSSEC Status = %d [%s]\n", dnssec_status,
                             p_val_error(dnssec_status));
        val_freehostent(h);
    }
                                                                                                                             
    return 0;
}

int use_val_getaddrinfo(int argc, char *argv[])
{
    int dnssec_status = ERROR;
    struct addrinfo *ainfo = NULL;
                                                                                                                             
    if (argc < 2) {
        printf("Usage: %s <hostname>\n", argv[0]);
        exit(1);
    }
                                                                                                                             
    ainfo = val_getaddrinfo(argv[1]);
                                                                                                                             
    if (ainfo) {
        dnssec_status = val_get_addrinfo_dnssec_status(h);
        printf("DNSSEC Status = %d [%s]\n", dnssec_status,
                             p_val_error(dnssec_status));
        val_freeaddrinfo(h);
    }
                                                                                                                             
    return 0;
}
\end{verbatim}
                                                                                                       
%%%%%%%%%%%%%%%%%%%%%%%%%%%%%%%%%%%%%%%%%%%%%%%%%%%%%%%%%%%%%%%%%%%%

\subsection{Obtaining DNSSEC Status}
\label{getstatus}
                                                                                                                             
The result from DNSSEC processing is available to the application
as an integer value. There are different ways to obtain the DNSSEC status 
for a response to a query based on the interface used. 
                                                                                                                             
\begin{itemize}
                                                                                                                             
\item The \param{dnssec\_status} result parameter in \funct{val\_query()} directly returns
      the DNSSEC status resulting from this call.
                                                                                                                             
\item \funct{val\_get\_hostent\_dnssec\_status()} returns the DNSSEC status obtained
      after a call to \funct{val\_gethostbyname()}. Its prototype is as follows:

\begin{verbatim}
    int val_get_hostent_dnssec_status(const struct hostent *hentry);
\end{verbatim}

\item \funct{val\_get\_addrinfo\_dnssec\_status()} returns the DNSSEC status obtained
      after a call to \funct{val\_getaddrinfo()}. Its prototype is as follows:
                                                                                                                             
\begin{verbatim}
    int val_get_addrinfo_dnssec_status(const struct addrinfo *ainfo);
\end{verbatim}
                                                                                                                             
\end{itemize}
                                                                                                                             
                                                                                                                             
The different DNSSEC status values that can be returned are listed below.

\begin{description}
\item[BARE\_RRSIG] No DNSSEC validation possible, query was for an RRSIG.
			 There is really no reason why applications would need to 
			 check for the validity of an RRSIG itself.
\item[VALIDATE\_SUCCESS] Answer received and validated successfully. 
\item[BOGUS\_UNPROVABLE] Could not build a chain of trust for a failure condition. 
\item[BOGUS\_PROVABLE] Failure condition that was provable. 
\item[VALIDATION\_ERROR] Did not have sufficient or relevant data to complete 
				  validation, or encountered some DNS error. 
\item[NONEXISTENT\_NAME] No name present, trusted, and proof present. 
\item[NONEXISTENT\_TYPE] No type exists for name, trusted, and proof present. 
\item[INCOMPLETE\_PROOF] Proof does not have all required components to prove 
				  non-existence. 
\item[BOGUS\_PROOF] Proof of non-existence cannot be validated. 
\item[INDETERMINATE\_DS] Can't prove that the DS is trusted. 
\item[INDETERMINATE\_PROOF]  Some intermediate Proof of non-existence obtained 
					 dont know if answer exists and proof is bogus 
					 or answer is bogus. 
\item[INDETERMINATE\_ERROR] Sequence of errors.
\item[INDETERMINATE\_TRUST] Don't know if trust is absent or answer is bogus. 
\item[INDETERMINATE\_ZONE]  Dont know if zone is unsigned or sigs have been stripped. 
\end{description}


Of the above set of status values only \const{VALIDATE\_SUCCESS}, \const{NONEXISTENT\_NAME} \\
and \const{NONEXISTENT\_TYPE} represent the successful conditions. \const{VALIDATE\_SUCCESS} \\
indicates that the answer received was valid while \const{NONEXISTENT\_NAME} and \\
\const{NONEXISTENT\_TYPE} represent that a particular name or type was provably absent.
                                                                                                                             
Other DNSSEC status values either represent validation failure,
indeterminate states, or other error conditions. In each of these cases
the answer returned by the DNS cannot be safely relied upon.
                                                                                                                             
DNSSEC status values can be displayed using the \funct{p\_val\_error()} interface,
which has the following prototype:

\begin{verbatim}
    char *p_val_error(int valerrno);
\end{verbatim}

%%%%%%%%%%%%%%%%%%%%%%%%%%%%%%%%%%%%%%%%%%%%%%%%%%%%%%%%%%%%%%%%%%%%
                                                                                                                             
\subsection{Application Behaviour}
\label{behaviour}
                                                                                                                             
The action that an application may take based on DNSSEC status values is
again based on policy. Two scenarios are discussed - Web Browsers and 
Mail Transfer Agents.
                                                                                                                             
\subsubsection{Web Browser}
\label{browser}
                                                                                                                             
TODO
                                                                                                                             
\subsubsection{Mail Transfer Agents (MTAs)}
\label{mta}
                                                                                                                             
TODO

%%%%%%%%%%%%%%%%%%%%%%%%%%%%%%%%%%%%%%%%%%%%%%%%%%%%%%%%%%%%%%%%%%%%
                                                                                                                             
\subsection{Validator Configuration}
\label{configuration}
                                                                                                                             
The validator library reads configuration information from two separate \\
files, \file{/etc/resolv.conf} and \file{/etc/dnsval.conf}.
                                                                                                                             
\begin{itemize}
                                                                                                                             
\item  \file{/etc/resolv.conf}
                                                                                                                             
Only the \filecontent{nameserver} option is supported in the \file{resolv.conf} file.
This option is used to specify the IP address of the name server to
which queries must be sent by default. For example,

\begin{verbatim}
    nameserver 10.0.0.1
\end{verbatim}
                                                                                                                             
\item \file{/etc/dnsval.conf}
                                                                                                                             
This file contains a sequence of the following policy-fragments:
                                                                                                                             
{\it $<label> <keyword> <additional-data>;$}
                                                                                                                             
{\it label} is the context to which this policy applies and {\it keyword} is the
specific policy component that is being configured. The format of
{\it additional-data} depends on the keyword specified.
                                                                                                                             
Policy fragments are indexed by the {\it label} and {\it keyword}. If multiple
policy fragments are defined for the same {\it label} and {\it keyword}
combination then the last definition in the file is used.

Currently two different values for {\it keyword} are defined:

\keyword{trust-anchor}

Specifies the trust anchors for a sequence of zones. The
additional data portion for this keyword is a sequence of the
zone name and a quoted string containing the RDATA portion for
the trust anchor's DNSKEY. An example is:

\begin{verbatim}
mozilla trust-anchor
        fruits.netsec.tislabs.com.
            "257 3 5
            AQO8XS4y9r77X9SHBmrxMoJf1Pf9AT9Mr/L5BBGtO9/e9f/zl4FFgM2l
            B6M2XEm6mp6mit4tzpB/sAEQw1McYz6bJdKkTiqtuWTCfDmgQhI6/Ha0
            EfGPNSqnY 99FmbSeWNIRaa4fgSCVFhvbrYq1nXkNVyQPeEVHkoDNCAlr
            qOA3lw=="
        netsec.tislabs.com.
           "257 3 5
            AQO8XS4y9r77X9SHBmrxMoJf1Pf9AT9Mr/L5BBGtO9/e9f/zl4FFgM2l
            B6M2XEm6mp6mit4tzpB/sAEQw1McYz6bJdKkTiqtuWTCfDmgQhI6/Ha0
            EfGPNSqnY 99FmbSeWNIRaa4fgSCVFhvbrYq1nXkNVyQPeEVHkoDNCAlr
            qOA3lw=="
        ;
\end{verbatim}
 
\keyword{zone-security-expectation}

Specifies the local security expection for a zone. The
additional data portion for this keyword is a sequence of the
zone name and its trust status - trusted or untrusted. An
example is:

\begin{verbatim}
mozilla zone-security-expectation wesh.fruits.netsec.tislabs.com
        untrusted;
\end{verbatim}

\end{itemize}
                                                                                                                             
%%%%%%%%%%%%%%%%%%%%%%%%%%%%%%%%%%%%%%%%%%%%%%%%%%%%%%%%%%%%%%%%%%%%

\subsection{Controlling Validator Behaviour Within Applications}
\label{context}
                                                                                                                             
The validator library allows multiple applications to operate
with differing validator policies at the same instance. 
A handle to each of the different instantiations of the validator
under different policies are available through the validator context.
                                                                                                                             
Applications can create a new validator context using the \funct{get\_context()}
method, which has the following prototype.

\begin{verbatim}
    int get_context(const char *label, val_context_t **newcontext);
\end{verbatim}
                                                                                                                             
This method parses the resolver and validator configuration
files and creates the handle \param{newcontext} to this parsed information.
Validator policy is applied hierarchically
based on the name used to identify a particular policy fragment.
As an example, $"browser"$ and $"mozilla.browser"$ form a hierarchical
ordering of policy, where the configuration under the identifier of
$"mozilla.browser"$ may be used to override generic browser policy with
mozilla-specific options.
                                                                                                                             
Applications may switch their default policy
to a policy that lies below ($"mozilla:browser"$ lies below $"browser"$) the
originally defined label using the \funct{switch\_effective\_policy()}
 method, which has the following prototype.

\begin{verbatim} 
    int switch_effective_policy(val_context_t *ctx, const char *label);
\end{verbatim}    
                                                                                                                         
The \param{label} parameter is a string that must match another string
that appears as the policy definition identifier in the validator configuration file.
It is a simple text string with the $":"$ character used as the delimiter
between two levels in the policy hierarchy. The $":"$ label by itself refers to
the default policy in the configuration file. \const{NULL} may also be used to
represent the label for the default policy.
                                                                                                                             
Once a context is available, it is also possible to access the
functionalities provided by \funct{val\_gethostbyname()},
\funct{val\_query()} and \funct{val\_getaddrinfo()} within this
context by using the \funct{val\_x\_query()}, \funct{val\_x\_gethostbyname()}
and \funct{val\_x\_getaddrinfo()} interfaces respectively. The prototypes
for these functions are given below.
                                                                      
\begin{verbatim}
    int val_x_query(const val_context_t *ctx,
        const char *domain_name, const u_int16_t class,
        const u_int16_t type, const u_int8_t flags,
        struct response_t *resp, int *resp_count);
                                                                                                                             
    struct hostent *val_x_gethostbyname(const val_context_t *ctx,
        const char *name, int *h_errnop);
                                                                                                                             
    int val_x_getaddrinfo(const struct val_context *ctx,
        const char *nodename, const char *servname,
        const struct addrinfo *hints,
        struct addrinfo **res);
\end{verbatim}                                                                                                                             
                                                                                                                             
If a \const{NULL} value is passed for the \param{ctx} parameter, the default validator
context is used in each instance.
                                                                                                                             
While \funct{val\_x\_gethostbyname()} and \funct{val\_x\_getaddrinfo()} are 
functionally identical to \\
 \funct{val\_gethostbyname()} and  \funct{val\_getaddrinfo()} respectively,
 \funct{val\_x\_query()} and \\
\funct{val\_query()} have some differences. 

\funct{val\_query()} gives applications that already use the \lib{res\_query(3)} inter-
face a simple way to transition towards being DNSSEC-aware; it does
not handle cases where multiple resource record sets may be
returned for a query, each with a potentially different DNSSEC status.
\funct{val\_x\_query()} on the other hand provides this functionality by 
encapsulating its results within the following structure:

\begin{verbatim}
    struct response_t {
        u_int8_t *response;
        int response_length;
        int validation_result;
    };
\end{verbatim}
                                                                                                                             
\param{response}, \param{response\_length} and \param{validation\_result} are functionally similar 
to \param{answer}, \param{anslen} and \param{dnssec\_status} respectively in \funct{val\_query()}.
\param{resp} in \funct{val\_query()} is an array and must be allocated by the user to be of sufficient
size to hold all the answers returned by the DNS name server.
The \funct{val\_x\_query()} function returns \const{NO\_ERROR} on success.
                                                                                                                             
The number of answers actually available is returned in the \param{resp\_count}
parameter.  In case \param{resp} is not large enough to hold all answers
returned from \funct{val\_x\_query()}, it returns an error code of 
\const{NO\_SPACE} and \param{resp\_count} is set to the total number of 
answers available.  Applications may re-query after reallocating \param{resp} 
to this size. For queries that are not of type \const{ns\_t\_any} it is generally sufficient
to allocate an array of three elements for \param{results}.
                                                                                                                             
Contexts that were created using the \funct{get\_context()} interface should be
destroyed when they are no longer needed, using the \funct{destroy\_context()}
method.

\begin{verbatim}
    void destroy_context(val_context_t *context);
\end{verbatim}
                                                                                                                             
%%%%%%%%%%%%%%%%%%%%%%%%%%%%%%%%%%%%%%%%%%%%%%%%%%%%%%%%%%%%%%%%%%%%

\subsection{Observing the Finer Details}
\label{detail}
                                                                                                                             
The \funct{resolve\_n\_check()} interface forms the core of the 
validator functionality and can be used to query a set of name servers 
for the $<*domain\_name\_n*, *type*, *class*>$ tuple and to verify and
validate the response.  Verification is the step of checking the RRSIGs
and validation includes performing verification up the chain of trust
all the way to a trust anchor.
All of the interfaces described above that are used to fetch and
validate DNS answers make use of this interface.

\begin{verbatim}
    int resolve_n_check( val_context_t *context, u_char *domain_name_n,
        const u_int16_t type, const u_int16_t class, const u_int8_t flags,
        struct query_chain **queries, struct assertion_chain **assertions,
        struct val_result **results);
\end{verbatim}
                                                                                                                             
The \param{domain\_name\_n} parameter is the queried name in DNS wire format.
The conversion from host format to DNS wire format can be done using
the \funct{ns\_name\_pton()} \lib{libres(3)} helper function that is also exported by
\lib{libval(3)}.
                                                                                                                             
\funct{resolve\_n\_check()} returns \const{NO\_ERROR} on success.
Answers returned by \funct{resolve\_n\_check()} are made available in the
\param{*results} array. Each answer is a distinct RRset; multiple RRs within
the RRset are treated as the same answer. Multiple answers are possible
when \param{type} is \const{ns\_t\_any} or when a proof of non-existence is returned
in which case RRsets of type \const{ns\_t\_nsec} and \const{ns\_t\_soa} are also
returned.

Individual elements in *\param{results} point to corresponding elements in the
\param{**assertions} linked list. The assertions contain the actual RRsets
returned by the name server in response to the query. The \param{**queries}
linked-list provides a handle to the query that returned the data
present within an assertion.
                                                                                                                             
\param{*queries}, \param{*assertions} and \param{*results} must be initially set to NULL.
The libval library internally allocates memory for these parameters so
they must be freed by the invoking application using the
\funct{free\_query\_chain()}, \funct{free\_assertion\_chain()} and \funct{free\_result\_chain()}
interfaces respectively.

\begin{verbatim}
        void free_query_chain(struct query_chain **queries);
        void free_assertion_chain(struct assertion_chain **assertions);
        void free_result_chain(struct val_result **results);
\end{verbatim}

The three dimensions of \param{results}, \param{assertions} and 
\param{queries} allow the invoking application to select to 
investigate the DNSSEC validation status of a particular response in much detail. Most
applications would only require \param{results} since this provides a single
error code for representing the authenticity of returned data. Other
more intrusive applications such as a DNSSEC troubleshooting utility may
look at the individual assertions to identify what particular component
in the chain-of-trust led to validation failure if any.
                                                                                                                             
Upon completion of the \funct{resolve\_n\_check()} call, 
the \param{ac\_state} element within the assertion structure contains the validation
status of a particular assertion. This generally contains a different than the
final DNSSEC status returned by higher level applications since it applies only to 
the given assertion. Successful DNSSEC Status values can only be generated if every 
component in the chain of trust was validated and culminates with a trusted record.
The \param{ac\_state} field may contain the following values for an assertion:


\begin{description}
\item[DNSKEY\_NOMATCH]  RRSIG was created by a DNSKEY that does not exist in the apex keyset. 
\item[WRONG\_LABEL\_COUNT]  The number of labels on the signature is greater than the the count given in the RRSIG RDATA. 
\item[SECURITY\_LAME] RRSIG created by a key that does not exist in the parent DS record set.
\item[NOT\_A\_ZONE\_KEY] The key used to verify the RRSIG is not a zone key, but some other key such as the public key used for TSIG. 
\item[RRSIG\_NOTYETACTIVE] The RRSIG's inception time is in the future. 
\item[RRSIG\_EXPIRED] The RRSIG has expired. 
\item[ALGO\_NOT\_SUPPORTED] Algorithm in DNSKEY or RRSIG or DS is not supported. 
\item[UNKNOWN\_ALGO] Unknown DNSKEY or RRSIG or DS algorithm 
\item[RRSIG\_VERIFIED] The RRSIG verified successfully. 
\item[RRSIG\_VERIFY\_FAILED] The RRSIG did not verify.
\item[NOT\_VERIFIED] Different RRSIGs failed for different reasons 
\item[KEY\_TOO\_LARGE] The zone is using a key size that is too large as per local policy. 
\item[KEY\_TOO\_SMALL] The zone is using a key size that is too small as per local policy 
\item[KEY\_NOT\_AUTHORIZED] The zone is using a key that is not authorized as per local policy.
\item[ALGO\_REFUSED] Algorithm in DNSKEY or RRSIG or DS is not allowed as per local policy 
\item[CLOCK\_SKEW] Verified but with clock skew taken into account 
\item[DUPLICATE\_KEYTAG] Two DNSKEYs have the same keytag 
\item[NO\_PREFERRED\_SEP] There is no DNSKEY in the parent DS set that our local policy allows us to traverse  
\item[WRONG\_RRSIG\_OWNER] The RRSIG and the data that it purportedly covers have differing notions of owner name 
\item[RRSIG\_ALGO\_MISMATCH] The DNSKEY and RRSIG pair have a mismatch in their algorithm.  
\item[KEYTAG\_MISMATCH] The DNSKEY and RRSIG pair have a mismatch in the key tags. 
\item[VERIFIED] The signature verified OK. 
\item[LOCAL\_ANSWER] Answer was obtained locally. This is currently not implemented. 
\item[TRUST\_KEY] The key present in this assertion is trusted. 
\item[TRUST\_ZONE] The zone in this assertion is trusted. 
\item[BARE\_RRSIG] Query was for an RRSIG. 
\end{description}
                                                                                                                             

In cases where data is insufficient to generate a validation result, \param{ac\_state} may \\ 
also contain the following status values. The final validation result in such circumstances \\
is \const{VALIDATION\_ERROR}.

                                                                                                                             
\begin{description}
\item[DATA\_MISSING]  No data was returned in the response. Based on what data was queried for this also covers cases where the DNSKEY or DS are missing. 
\item[RRSIG\_MISSING] Could not find an RRSIG that matched the query type. 
\item[NO\_TRUST\_ANCHOR] No trust anchor at current level and no possiblity for finding any up this tree.  
\item[UNTRUSTED\_ZONE] The zone has been configured as un-trusted. 
\item[IRRELEVANT\_PROOF] An NSEC received does not contribute towards proving non-existence.
\item[DNSSEC\_VERSION\_ERROR] DNSSEC version error. Not implemented. 
\item[TOO\_MANY\_LINKS] Too many links were traversed in going up the chain-of-trust.  Not implemented. 
\item[UNKNOWN\_DNSKEY\_PROTO] The protocol field in the DNSKEY is not set to 3 (DNSSEC). 
\item[FLOOD\_ATTACK\_DETECTED] Detected multiple (conflicting) answers for the same query.  Possible spoofing attack. Not implemented. 
\item[CONFLICTING\_ANSWERS] Multiple answers received for a query which conflict. 
\item[SR\_REFERRAL\_ERROR] Some error encountered while following referrals.
\item[SR\_MISSING\_GLUE] Glue was missing 
\item[Other DNS errors.] 
\end{description}


                                                                                                                             
Assertion status values may be displayed using the \funct{p\_val\_error()} interface.

%%%%%%%%%%%%%%%%%%%%%%%%%%%%%%%%%%%%%%%%%%%%%%%%%%%%%%%%%%%%%%%%%%%%
                                                                                                                             

\subsection{DNSSEC Instrumentation For Unsupported Resolver Calls}
\label{custom}
                                                                                                                             
                                                                                                                             
Applications that make use of resolver calls that are not directly exported by
the validator library would need to implement such wrapper functions themselves.
The \funct{resolve\_n\_check()} interface is generic enough to allow almost any 
name resolution functionality to be made DNSSEC-aware. The following code 
snippet illustrates how \funct{val\_x\_query()} is internally implemented 
using the \funct{resolve\_n\_check()} interface.

\begin{verbatim}
#include <stdio.h>
#include <stdlib.h>
#include <string.h>
#include <ctype.h>
#include <arpa/nameser.h>
                                                                                                                             
#include <resolver.h>
#include <validator.h>
#include "val_cache.h"
#include "val_support.h"

int compose_answer( const char *name_n,
            const u_int16_t type_h,
            const u_int16_t class_h,
            struct val_result *results,
            struct response_t *resp,
            int *resp_count,
            u_int8_t flags);

int val_x_query(const val_context_t *ctx,
            const char *domain_name,
            const u_int16_t class,
            const u_int16_t type,
            const u_int8_t flags,
            struct response_t *resp,
            int *resp_count)
{
    struct query_chain *queries = NULL;
    struct assertion_chain *assertions = NULL;
    struct val_result *results = NULL;
    int retval;
    val_context_t *context;
    u_char name_n[MAXCDNAME];
                                                                                                                             
    if(ctx == NULL) {
        if(NO_ERROR !=(retval = get_context(NULL, &context)))
            return retval;
    }
    else
        context = ctx;
                                                                                                                             
    if (ns_name_pton(domain_name, name_n, MAXCDNAME-1) == -1)
        return (BAD_ARGUMENT);
    if(NO_ERROR == (retval = resolve_n_check(context, name_n, 
                               type, class, flags,
                               &queries, &assertions, &results))) {

        /* Construct the answer response in response_t */
        retval = compose_answer(name_n, type, class, results, 
                                    resp, resp_count, flags);
        free_query_chain(&queries);
        free_assertion_chain(&assertions);
        free_result_chain(&results);
    }
    if((ctx == NULL)&& context)
        destroy_context(context);
    return retval;
}
\end{verbatim}
