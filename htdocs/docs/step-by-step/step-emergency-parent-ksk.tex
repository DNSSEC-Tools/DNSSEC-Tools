
\clearpage
\subsection{KSK Roll-Over --- Parent Action During KSK Compromise}
\label{roll-emergency-parent-ksk}

During a KSK compromise the child zone is no longer secure. This change in
status is performed by deleting the child's DS record from the parent zone.

See Figure~\ref{fig:zskroll-emerg}.

%%%%%%%%%%%%%%%%%%%%%%%%%%%%%%%%%%%%%%

\subsubsection{Ensure that the KSK Compromise Notification Came Over a Secure
Channel}

Authentication and communication between parent and child occurs out-of-band.

%%%%%%%%%%%%%%%%%%%%%%%%%%%%%%%%%%%%%%

\subsubsection{Delete the Child's Keyset File at the Parent}

The DS record for the child should not be created. This can be achieved by
not having the child's keyset file available to the \cmd{dnssec-signzone}
process.

%%%%%%%%%%%%%%%%%%%%%%%%%%%%%%%%%%%%%%

\subsubsection{Increase the Zone SOA Number}

Although there is no change made to the unsigned zone file, the DS record for
the child will no longer be present in the signed zone file.  This amounts to
an implicit change.  Thus, the zone serial number must be incremented.
\begin{tabbing}
\hspace{0.5in} \= 12345678 \= 12345678 \= 12345678 \= 12345678 \kill \\
\hspace{0.5in}\$ {\bf vi \underline{zonefile}} $[$ENTER$]$ \\
\hspace{0.5in}\underline{zone.name}        IN     SOA        servername contact ( \\
\hspace{3.5in}{\bf 2005092109} ; Increase current value by 1. \\
\hspace{4.4in};  This value may be different \\
\hspace{4.4in}; in your zone file. \\
\hspace{0.5in}\>           \>         ... \\
\hspace{0.5in}\>              ) \\
\end{tabbing}

%%%%%%%%%%%%%%%%%%%%%%%%%%%%%%%%%%%%%%


\subsubsection{Reload the Zone}

The \cmd{rndc} command will reload the name server configuration files and
the zone contents.  The name server process is assumed to be already running.

\begin{tabbing}
\hspace{0.5in}\$ {\bf rndc reload zone.name} $[$ENTER$]$ \\
\hspace{0.5in}\$ \\
\end{tabbing}

