
\subsubsection{Generate the Key}

Use the \cmd{dnssec-keygen} command to generate a key 512 bits long. 
\begin{tabbing}
\hspace{0.5in} \= 12345678 \= 12345678 \= 12345678 \= 12345678 \kill
\hspace{0.5in}\$ {\bf dnssec-keygen -a RSASHA1 -b 512 -n ZONE \underline{zone.name}} \\
\hspace{0.5in} [ENTER] \\
\hspace{0.5in} K\underline{zone.name}.+005+\underline{zsktag} \\
\hspace{0.5in}\$ \\
\end{tabbing}

The process may take a few minutes to return its results. If the process
appears to have stalled, run the command using a pseudo-random number
generator as follows:
\begin{tabbing}
\hspace{0.5in} \= 12345678 \= 12345678 \= 12345678 \= 12345678 \kill \\
\hspace{0.5in}\$ {\bf dnssec-keygen -r /dev/urandom -a RSASHA1 -b 512} \\
\hspace{0.5in} {\bf -n ZONE \underline{zone.name}} [ENTER] \\
\hspace{0.5in} K\underline{zone.name}.+005+\underline{zsktag} \\
\hspace{0.5in}\$ \\
\end{tabbing}

Two files are output by \cmd{dnssec-keygen}: \\
    - Private key contained in \path{K\underline{zone.name}.+005+\underline{zsktag}.private}\\
    - Public key contained in \path{K\underline{zone.name}.+005+\underline{zsktag}.key}

{\bf \underline{zone.name}} - the name of the zone (e.g., example.com)\\
{\bf \underline{zsktag}} - the key identifier (e.g., 57011) 

You must note this number in the key-tag table as you walk through this
document.  This number is automatically generated and should not be changed.
The key id will be the only field in the filename that changes as you rotate
keys, so it must be tracked.

This document uses RSASHA1 as the cryptographic algorithm, which is
represented by the ``005'' in the key name's algorithm field.

