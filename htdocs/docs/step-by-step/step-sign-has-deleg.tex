
\clearpage
\subsection{Signing a Zone that Has Delegations}
\label{signzone-hasdel}

A zone needs to be re-signed when \underline{any} change is made to it.
Steps~\ref{keygen-set-hasdel} and~\ref{update-zonefile-hasdel} must be
followed if the zone has not been previously signed.
Steps~\ref{check-zonefile-unsigned-deleg}~-~\ref{signzone-deleg-last} must
be followed when re-signing a zone file that contains delegations.

See Figure~\ref{fig:zonesign-unsecure}.

%%%%%%%%%%%%%%%%%%%%%%%%%%%%%%%%%%%%%%


\subsubsection{Generate Two ZSKs and One KSK}

Follow the steps in Section~\ref{genzsk} for generating the ZSKs and steps in
Section~\ref{genksk} for generating the KSK.

Designate one of the two ZSKs as the Current (C) zone signing key and use it
to sign the zone data; designate the other as the Published (P) key, which
is for future use following a ZSK roll-over.  Set the status of each of these
keys in the column marked S.

There is only one KSK and this is used to sign the zone apex keyset; mark its
status as Current (C).  The Published ZSK should be kept more
safely\footnote{It would be a good idea for an operator to apply increased
protection mechanisms (physical, file permissions and ownership, network,
etc.) to the Published ZSK than are used for the Current ZSK.}
than the Current ZSK.  The idea is that the Published ZSK can be easily rolled
in even if the Current ZSK is compromised (the Current ZSK may have to be
kept on-line in some circumstances).

If the KSK has been stored in a more secure location (off-line, more highly
protected directory, etc.) then it might be a good idea to store the
Published ZSK in the same secure location.

\begin{center}
\begin{tabular}{|c|c|c|c|c|c|c|c|c|c|}
\hline
{\bf Zone} &
\multicolumn{4}{c|}{{\bf ZSK}} &
\multicolumn{4}{c|}{{\bf KSK}} &
{\bf Exp} \\
\cline{2-9}

 & Tag & Size & Creat & S & Tag & Size & Creat & S & \\
\hline

\underline{zone.name}	&
\underline{zsktag-cur}	&
512			&
\underline{date}	&
C			&
\underline{ksktag}	&
1024			&
\underline{date}	&
C			& \\

\cline{2-9}

			&
\underline{zsktag-pub}	&
512			&
\underline{date}	&
P			&
& & & & \\

\hline
\end{tabular}
\end{center}
		\label{keygen-set-hasdel}

\subsubsection{Modify the Zone File}

The zone file must be modified to account for the new keys.  Add lines to
include the KSK and the ZSKs in the zone file.  Also, the SOA serial number
must be changed so that the zone file's new contents will be recognized.

\begin{tabbing}
\hspace{0.5in} \= 12345678 \= 12345678 \= 12345678 \= 12345678 \kill \\
\hspace{0.5in}\$ {\bf vi \underline{zonefile}} $[$ENTER$]$ \\
\hspace{0.5in}\underline{zone.name}        IN     SOA        servername contact ( \\
\hspace{3.5in}{\bf 2005092105} ; Increase current value by 1. \\
\hspace{4.4in};  This value may be different \\
\hspace{4.4in}; in your zone file. \\
\hspace{0.5in}\>           \>         ... \\
\hspace{0.5in}\>              ) \\
\hspace{0.5in}... \\
\hspace{0.5in}{\bf ;; ksk} \\
\hspace{0.5in}{\bf \$INCLUDE ``/path/to/K\underline{zone.name}.+005+\underline{ksktag}.key''} \\
\hspace{0.5in}{\bf ;; cur zsk} \\
\hspace{0.5in}{\bf \$INCLUDE ``/path/to/K\underline{zone.name}.+005+\underline{zsktag-cur}.key''} \\
\hspace{0.5in}{\bf ;; pub zsk} \\
\hspace{0.5in}{\bf \$INCLUDE ``/path/to/K\underline{zone.name}.+005+\underline{zsktag-pub}.key''} \\
\hspace{0.5in}... \\
\hspace{0.5in}\$ \\
\end{tabbing}
		\label{update-zonefile-hasdel}


\subsubsection{Check the Unsigned Zone File for Errors}

Ensure that the unsigned zone file was modified correctly.
\begin{tabbing}
\hspace{0.5in}\$ {\bf named-checkzone zone.name zonefile} [ENTER] \\
\hspace{0.5in} zone zone.name/IN: loaded serial SerialNumber \\
\hspace{0.5in} OK \\
\hspace{0.5in}\$ \\
\end{tabbing}
	\label{check-zonefile-unsigned-deleg}


\subsubsection{Check Permissions and Ownership on ZSK and KSK Files}

The key files must be readable by the \cmd{dnssec-signzone} tool and the
name server's user id.



%%%%%%%%%%%%%%%%%%%%%%%%%%%%%%%%%%%%%%

\subsubsection{Sign the Zone}
\label{signzone-has-deleg}

Use the \cmd{dnssec-signzone} command to sign the zone file.
This should not be executed unless the keysets from all child zones have been
received.  (See Section~\ref{delegation-parent}.)

\begin{tabbing}
\hspace{0.5in}\$ {\bf dnssec-signzone -g -k \underline{/path/to}/K\underline{zone.name}.+005+\underline{ksktag}.key} \\
\hspace{0.5in} {\bf -o \underline{zone.name} -e +2592000 -d \underline{keyset-dir} \underline{zonefile}} \\
\hspace{0.5in} {\bf \underline{/path/to}/K\underline{zone.name}.+005+\underline{zsktag-cur}.key} $[$ENTER$]$ \\
\hspace{0.5in} \underline{zonefile}.signed \\
\hspace{0.5in}\$ \\
\end{tabbing}

Signature generation may take a few minutes to complete, depending on the size
of the zone file. If the above operation appears to be unresponsive for an
unreasonable length of time, use pseudorandom numbers (using the \option{-p}
option) instead.

\begin{tabbing}
\hspace{0.5in}\$ {\bf dnssec-signzone -g -k \underline{/path/to}/K\underline{zone.name}.+005+\underline{ksktag}.key} \\
\hspace{0.5in} {\bf -o \underline{zone.name} -p -e +2592000 -d \underline{keyset-dir} \underline{zonefile}} \\
\hspace{0.5in} {\bf \underline{/path/to}/K\underline{zone.name}.+005+\underline{zsktag-cur}.key} $[$ENTER$]$ \\
\hspace{0.5in} \underline{zonefile}.signed \\
\hspace{0.5in}\$ \\
\end{tabbing}

The \option{-d} option specifies the directory in which the child zone's
keyset files have been stored.  It there are no keyset files available, run
the \cmd{dnssec-signzone} command without the \option{-d keyset-dir} option.

Three files are created by \cmd{dnssec-signzone}:
\begin{itemize}
\item Signed zone file in \underline{zonefile}.signed.
The {\it .signed} suffix is appended by default.

\item Keyset file in keyset-\underline{zone.name}.
This may have to be sent to the parent zone if this zone is also a child zone;
see Section~\ref{delegation-child}.

\item DS-set file in dsset-\underline{zone.name}.
Used to verify that the correct DS record was generated at the parent;
see Section~\ref{delegation-child}.

\end{itemize}

The \cmd{dnssec-signzone} command generates signatures for the records that
are valid for 30 days (2,592,000 seconds) from the current time.  This
is offset by -1 hour to account for clock skew between the name server and
DNSSEC validators.

%%%%%%%%%%%%%%%%%%%%%%%%%%%%%%%%%%%%%%


\subsubsection{Check the Signed Zone File for Errors}

Ensure that the signed zone file was modified correctly.
\begin{tabbing}
\hspace{0.5in}\$ {\bf named-checkzone zone.name zonefile} [ENTER] \\
\hspace{0.5in} zone zone.name/IN: loaded serial SerialNumber \\
\hspace{0.5in} OK \\
\hspace{0.5in}\$ \\
\end{tabbing}


\subsubsection{Record the Signature Expiry Time}

Update the key-tag file to hold the expiration date of the zone signature.

\begin{center}
\begin{tabular}{|c|c|c|c|c|c|c|c|c|c|}
\hline
{\bf Zone} &
\multicolumn{4}{c|}{{\bf ZSK}} &
\multicolumn{4}{c|}{{\bf KSK}} &
{\bf Exp} \\
\cline{2-9}

 & Tag & Size & Creat & S & Tag & Size & Creat & S & \\
\hline

\underline{zone.name}	&
\underline{zsktag-cur}	&
512			&
\underline{date}	&
C			&
\underline{ksktag}	&
1024			&
\underline{date}	&
C			&
\underline{date}	\\

\cline{2-10}

			&
\underline{zsktag-pub}	&
512			&
\underline{date}	&
P			&
& & & & \\

\hline
\end{tabular}
\end{center}


%%%%%%%%%%%%%%%%%%%%%%%%%%%%%%%%%%%%%%

\subsubsection{Confirm DS Record Creation in Signed Zone File}

There should be a DS record in the signed zone file ({\it zone.name.signed})
for every domain name from which a keyset was received.

%%%%%%%%%%%%%%%%%%%%%%%%%%%%%%%%%%%%%%

\label{signzone-deleg-last}

