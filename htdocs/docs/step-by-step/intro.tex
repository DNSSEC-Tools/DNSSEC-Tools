
DNS Security (DNSSEC) helps protect against DNS-spoofing attacks by providing
origin authentication and integrity protection of DNS information.  Proper
maintenance of a DNSSEC-enhanced DNS zone is essential to protecting the
domain's zone data.

% This Step-by-Step DNS Security Operator Guidance document is intended for
% operations using the BIND-9.3.0 (or later) distribution [4].  It will assist
% operators in gaining operational experience with DNSSEC. Some basic
% understanding of DNSSEC terms and concepts is required. A good introduction
% to DNSSEC can be found in [5].  Some basic understanding of public key
% cryptography is also required [7].

This Step-by-Step DNS Security Operator Guidance document is intended for
operations using the BIND-9.3.0 (or later) distribution.  It will assist
operators in gaining operational experience with DNSSEC. Some basic
understanding of DNSSEC terms and concepts is required.

This document is meant to be a learning aid and is not intended
to define policy in any form. Any implicit recommendations for key sizes,
signature validity periods, and command line parameters are for illustration
purposes ONLY and MUST NOT be used in production environments unless
due-diligence has been taken to ensure that these values are acceptable within
such environments.  See [1] for suggestions on determining appropriate
security characteristics.

This document was written as part of the DNSSEC-Tools project.  The goal of
this project is to create a set of documentation, tools, patches,
applications, wrappers, extensions, and plug-ins that will help ease the
deployment of DNSSEC-related technologies.  For more information about this
project and the tools that are being developed and provided, please see the
DNSSEC-Tools project web page at:

\url{http://www.dnssec-tools.org}

%%%%%%%%%%%%%%%%%%%%%%%%%%%%%%%%%%%%%%%%%%%%%%%%

\subsection{Organization of this Document}

The following operations are described in this document:

\begin{description}

\item [Section~\ref{preliminaries}] {Essential Preliminaries}\\
This section contains essential instructions that must be followed before
continuing with the rest of the document.

\item [Section~\ref{genzsk}] {Zone-Signing Key (ZSK) Generation}\\
This section describes the procedure for creating new Zone Signing Keys; i.e.,
the keys used for signing zone data.

\item [Section~\ref{genksk}] {Key-Signing Key (KSK) Generation}\\
This section describes the procedure for creating new Key Signing Keys; i.e.,
keys that are used to sign the ZSKs in the apex keyset.

\item [Section~\ref{config-n-serve}] {Configuring and Serving a Signed Zone}\\
This section describes the procedure for serving a signed zone file.

\item [Section~\ref{roll-curzsk}] {Current ZSK Roll-Over}\\
This section describes the procedure for rolling over an old ZSK. These steps
should be used only if the older ZSK is known to have not been compromised.

\item [Section~\ref{roll-ksk}] {KSK Roll-Over}\\
This section describes the procedure for rolling over an old KSK. These steps
should be used only if the older KSK is known to have not been compromised.

\item [Section~\ref{signzone-nodel}] {Signing a Zone with No Delegations}\\
This section describes the procedure for signing a zone that has no
delegations (no non-authoritative NS records) present in the zone file.

\item [Section~\ref{delegation-child}] {Creating a Signed Delegation in a Child Zone}\\
This section describes the activities that a child zone must perform
in order to facilitate the creation of a signed delegation at the parent.

\item [Section~\ref{signzone-hasdel}] {Signing a Zone that Has Delegations}\\
This section describes the procedure for signing a zone that has delegations
(non-authoritative NS records) present in the zone file. The difference
between this and section 3 lies in the additional communication involved
between the parent and the child, as well as creation of the DS record in the
parent zone, when delegations are present.

\item [Section~\ref{delegation-parent}] {Creating a Signed Delegation in a Parent Zone}\\
This section describes the activities that a parent zone must perform
in order to facilitate the creation of a signed delegation at the parent.

\item [Section~\ref{roll-emergency-ksk}] {KSK Roll-Over -- KSK Compromise}\\
This section describes the procedure for performing an emergency roll-over
of the KSK when it is suspected or known to be compromised.

\item [Section~\ref{roll-emergency-curzsk}] {ZSK Roll-Over -- Current ZSK Compromise}\\
This section describes the procedure for performing an emergency roll-over
of the Current ZSK, when it is suspected or known to be compromised.

\item [Section~\ref{roll-emergency-pubzsk}] {ZSK Roll-Over -- Published ZSK Compromise}\\
This section describes the procedure for performing an emergency roll-over
of the Published ZSK, when it is suspected or known to be compromised.

\item [Section~\ref{roll-emergency-zsks}] {ZSK Roll-Over -- Published and Current ZSK Compromise}\\
This section describes the procedure for performing an emergency roll-over
of the Published and Current ZSKs, when both are suspected or known to be
compromised.

\item [Section~\ref{roll-emergency-parent-ksk}] {KSK Roll-Over -- Parent Action During KSK Compromise}\\
This section describes the actions that the parent zone must perform when it
receives notification from the child about a KSK compromise, and before it
publishes a DS value that points to the new KSK in the child.

\end{description}

These sections are followed by several appendices.  These appendices contain
useful information, such as checklists for the operations and pictorial
illustrations for each of the operations described in this guide.

%%%%%%%%%%%%%%%%%%%%%%%%%%%%%%%%%%%%%%%%%%%%%%%%

\subsection{Identifying Relevant Steps}

The following table summarizes the list of steps relevant to different kinds
of zones. The columns marked with an 'X' for any row correspond to those
operations with which the zone operator for that type of zone must be
familiar.

\begin{center}
\begin{tabular}{|l|c|c|c|c|c|c|c|c|c|c|c|c|c|c|}
\hline
{\bf Zone Profile} & \multicolumn{14}{c|}{{\bf Steps}} \\
\cline{2-15}
 & 1 & 2 & 3 & 4 & 5 & 6 & 7 & 8 & 9 & 10 & 11 & 12 & 13 & 14 \\
\hline
No signed delegations, & X & X & X &   & X &   &   & X & X & X & X & X & X & \\
parent is not signed   &   &   &   &   &   &   &   &   &   &   &   &   &   & \\
\hline
No signed delegations, & X & X & X &   & X & X &   & X & X & X & X & X & X & \\
parent is signed       &   &   &   &   &   &   &   &   &   &   &   &   &   & \\
\hline
Signed delegations,    & X & X &   & X & X &   & X & X & X & X & X & X & X & X\\
parent is not signed   &   &   &   &   &   &   &   &   &   &   &   &   &   &  \\
\hline
Signed delegations,    & X & X &   & X & X & X & X & X & X & X & X & X & X & X\\
parent is signed       &   &   &   &   &   &   &   &   &   &   &   &   &   &  \\
\hline
\end{tabular}
\end{center}

%%%%%%%%%%%%%%%%%%%%%%%%%%%%%%%%%%%%%%%%%%%%%%%%

\subsection{Key Concepts}
\begin{description}
\item{ZSK} - Zone-Signing Key\\
The key used to sign zone data.

\item{KSK} - Key-Signing Key\\
The key used to sign the ZSKs in the apex keyset.

\item{Current ZSK}\\
The Current ZSK is the key that is currently used to sign zone data.

\item{Published ZSK}\\
The Published ZSK is the key that is pre-published in the zone file as the
future ZSK.

\item{New ZSK}\\
The New ZSK is the key that is scheduled to become the Published ZSK.

\end{description}

%%%%%%%%%%%%%%%%%%%%%%%%%%%%%%%%%%%%%%%%%%%%%%%%

\subsection{Conventions Used in this Document}

One of the goals of this document is to self-contain DNS Security
operations within sections and prevent constant cross-referencing between
sections.  Consequently, certain parts of the text are repeated throughout
the document.

In particular, one might notice that zone SOA serial numbers may not change
between sections.  This should {\bf not} be taken as an indication that the
serial numbers do not need to change when the guide states that they should.

Text marked in bold represents text or commands entered by users
within a given procedural step.

Underlined text, which can also be in bold, is a place-holder for actual
run-time values.  These values are either automatically generated or are
values that are known to the user from some other step.

Additionally, the following typographical conventions are used in this
document.

\begin{tabular}{lll}
\cmd{command}		& & Command names\\
\path{path}		& & File and path names\\
\url{URL}		& & Web URLs\\
\xqt{execution}		& & Simple command executions\\
\end{tabular}

Longer sets of command sequences are given in this format:
\begin{tabbing}
\hspace{0.5in}\$ {\bf cd /tmp} $[$ENTER$]$ \\
\hspace{0.5in}\$ {\bf ls} $[$ENTER$]$ \\
\hspace{0.5in}\$ {\bf rm -fr *} $[$ENTER$]$ \\
\hspace{0.5in}\$ \\
\end{tabbing}

In most cases, output will not be displayed for given command sequences.

%%%%%%%%%%%%%%%%%%%%%%%%%%%%%%%%%%%%%%%%%%%%%%%%

\subsection{Acknowledgments}

This document builds upon the procedures described in [1] for key
roll-over techniques; the step-by-step instructions described in
Sections~\ref{roll-curzsk} and~\ref{roll-ksk} are meant to closely follow
the recommendations given by that document. Early versions of this guide
were reviewed and critiqued by SAIC, Inc./Quotient, Inc., including Rip
Loomis and Rob Payne.  Their contributions are much appreciated.

%%%%%%%%%%%%%%%%%%%%%%%%%%%%%%%%%%%%%%%%%%%%%%%%

\subsection{Comments}

Please send any comments and corrections to developers@dnssec-tools.org.

